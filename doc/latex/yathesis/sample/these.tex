% Document de classe yathesis, en 12 points et interligne un et demi
\documentclass[12pt,space=onehalf]{yathesis}
%
% Chargement manuel de packages (pas déjà chargés par la classe yathesis)
\usepackage[utf8]{inputenc}
\usepackage[T1]{fontenc}
\usepackage{kpfonts}
\usepackage{varioref}
\usepackage{booktabs}
\usepackage{siunitx}
\usepackage{pgfplots}
\usepackage{listings}
\usepackage{floatrow}
\usepackage{lipsum} % À proscrire dans un vrai mémoire de thèse !
\usepackage{caption}
\usepackage{microtype}
\usepackage[autostyle]{csquotes}
\usepackage[backend=biber,safeinputenc]{biblatex}
\usepackage[xindy,acronyms,symbols,toc]{glossaries}
%
% Configuration des styles du glossaire et de la liste d'acronymes
\setglossarystyle{indexhypergroup}
\setacronymstyle{long-sc-short}
%
% Génération du glossaire
\makeglossaries
%
% Spécification de ressource(s) terminologique(s)
\loadglsentries{configuration/glossaire}
\loadglsentries{configuration/acronymes}
\loadglsentries{configuration/symboles}
%
% Génération de l'index
\makeindex
%
% Spécification des ressource(s) bibliographique(s)
\addbibresource{configuration/bibliographie.bib}
\addbibresource{biblatex-examples.bib} % Fournie par biblatex
%
% Commande permettant de lister toutes les références des ressources
% bibliographiques ci-dessus, même si elles ne sont pas citées
% explicitement (à proscrire dans un vrai mémoire de thèse !)
\nocite{*}
%
%%%%%%%%%%%%%%%%%%%%%%%%%%%%%%%%%%%%%%%%%%%%%%%%%%%%%%%%%%%%%%%%%%%%%%
%%%%%%%%%%%%%%%%%%%%%%%%%%%%%%%%%%%%%%%%%%%%%%%%%%%%%%%%%%%%%%%%%%%%%%
% Début du document
%%%%%%%%%%%%%%%%%%%%%%%%%%%%%%%%%%%%%%%%%%%%%%%%%%%%%%%%%%%%%%%%%%%%%%
%%%%%%%%%%%%%%%%%%%%%%%%%%%%%%%%%%%%%%%%%%%%%%%%%%%%%%%%%%%%%%%%%%%%%%
\begin{document}
%
%%%%%%%%%%%%%%%%%%%%%%%%%%%%%%%%%%%%%%%%%%%%%%%%%%%%%%%%%%%%%%%%%%%%%%
% Page de couverture et page(s) de titre
%%%%%%%%%%%%%%%%%%%%%%%%%%%%%%%%%%%%%%%%%%%%%%%%%%%%%%%%%%%%%%%%%%%%%%
%
% Préparation des pages de couverture et de titre
% Les propriétés de la thèse ont été saisies dans le fichier properties.tex
% (dans le dossier configuration/)
%
% Production des pages de couverture et de titre
\maketitle
%
%%%%%%%%%%%%%%%%%%%%%%%%%%%%%%%%%%%%%%%%%%%%%%%%%%%%%%%%%%%%%%%%%%%%%%
% Début de la partie liminaire de la thèse
%%%%%%%%%%%%%%%%%%%%%%%%%%%%%%%%%%%%%%%%%%%%%%%%%%%%%%%%%%%%%%%%%%%%%%
%
% Production de la page de clause de non-responsabilité, si souhaitée. Sinon,
% il suffit de commenter ou de supprimer la ligne suivante.
\makedisclaimer
%
% Production de la page de mots clés, si souhaitée. Sinon, il suffit de
% commenter ou de supprimer la ligne suivante.
\makekeywords
%
% Production de la page affichant les logo, nom et coordonnées du ou des
% laboratoires (ou unités de recherche) où la thèse a été préparée, si
% souhaitée. Sinon, il suffit de commenter ou de supprimer la ligne suivante.
\makelaboratory
%
% Dédicace(s), si souhaitée(s). Sinon, il suffit de commenter ou supprimer les
% lignes suivantes.
\dedication{Je dédie ce travail\\à tous ceux qui le méritent}
\dedication{À mon directeur bien-aimé !}
\dedication{À mon co-directeur bien-co-aimé !}
% Production de la page de dédicace(s), si souhaitée. Sinon, il suffit de
% commenter ou supprimer la ligne suivante.
\makededications
%
% Épigraphe(s), si souhaitée(s). Sinon, il suffit de commenter ou supprimer les
% lignes suivantes.
\frontepigraph{Science sans conscience n'est que ruine de l'âme.}{François Rabelais}
\frontepigraph[english]{I can resist everything, except temptation!}{Oscar Wilde}
\frontepigraph{Il est plus facile de désintégrer un atome qu'un préjugé.}{Albert Einstein}
% Production de la page d'épigraphe(s), si souhaitée. Sinon, il suffit de commenter ou supprimer
% la ligne suivante.
\makefrontepigraphs
%
% Remerciements, si souhaités. Sinon, il suffit de commenter ou de supprimer
% les lignes suivantes.
\chapter{Remerciements}
\section{Une section de remerciements}
\lipsum[1]
\section{Une autre section de remerciements}
\lipsum[2-9]
%
% Avertissement, si souhaité. Sinon, il suffit de commenter ou de supprimer les
% lignes suivantes qui s'y rapportent.
\chapter{Attention !}
Thèse hilarante, comme le gaz du même nom !
%
% Résumés (de 1700 caractères maximum, espaces compris) dans la
% langue principale (1re occurrence de l'environnement « abstract »)
% et, facultativement, dans la langue secondaire (2e occurrence de
% l'environnement « abstract »).
\begin{abstract}
  \lipsum[1-2]
\end{abstract}
\begin{abstract}
  \lipsum[3-4]
\end{abstract}
% Production de la page de résumés.
\makeabstract
%
% Liste des acronymes, si souhaitée. Sinon, il suffit de commenter ou de
% supprimer la ligne suivante.
\printacronyms
%
% Liste des symboles, si souhaitée. Sinon, il suffit de commenter ou de
% supprimer la ligne suivante.
\printsymbols
%
% Avant-propos, si souhaité.  Sinon, il suffit de commenter ou de supprimer les
% lignes suivantes.
\chapter{Avant-propos}
\section{Une section d'avant-propos}
\lipsum[30-45]
\section{Une autre section d'avant-propos}
\lipsum[30-35]
%
% Sommaire
\tableofcontents[depth=chapter,name=Sommaire]
%
% Liste des tableaux, si souhaitée. Sinon, il suffit de commenter ou de
% supprimer la ligne suivante.
\listoftables
%
% Table des figures, si souhaitée. Sinon, il suffit de commenter ou de
% supprimer la ligne suivante.
\listoffigures
%
% Table des listings, si souhaitée. Sinon, il suffit de commenter ou de
% supprimer la ligne suivante.
% \lstlistoflistings
%
%%%%%%%%%%%%%%%%%%%%%%%%%%%%%%%%%%%%%%%%%%%%%%%%%%%%%%%%%%%%%%%%%%%%%%
% Début de la partie principale (du « corps ») de la thèse
%%%%%%%%%%%%%%%%%%%%%%%%%%%%%%%%%%%%%%%%%%%%%%%%%%%%%%%%%%%%%%%%%%%%%%
\mainmatter
%
% Chapitre d'introduction générale
\chapter*{Introduction générale}
\lipsum[26-27]
\section{Une section d'introduction}
\lipsum[28-29]
\subsection{Une sous-section d'introduction}
\lipsum[29-31]
\subsubsection{Une sous-sous-section d'introduction}
\lipsum[31-35]
\paragraph{Un paragraphe d'introduction}
\lipsum[36-38]
\subparagraph{Un sous-paragraphe d'introduction}
\lipsum[39-41]
\subparagraph{Un autre sous-paragraphe d'introduction}
\lipsum[39-41]
\paragraph{Un autre paragraphe d'introduction}
\lipsum[36-38]
\subsubsection{Une autre sous-sous-section d'introduction}
\lipsum[31-37]
\subsection{Une autre sous-section d'introduction}
\lipsum[29-31]
\section{Une autre section d'introduction}
\lipsum[28-43]

%
% Chapitres ordinaires (avec parties éventuelles)
%
% Première partie
\part{Le chaos du rire}
% Insertion du premier chapitre
\chapter{Contexte du chaos du rire}
\epigraphhead[30]{\selectlanguage{english}\epigraph{I have not failed. I've
    just found 10.000 ways that won't work.}{Thomas A. Edison}}

Je peux citer\index{citation}\index{citation!formelle} des extraits de
documents. J'ai par exemple reproduit ci-dessous, par extraits, le discours de
\citeauthor{cond}, prononcé à l'Assemblée Nationale au nom de l'Académie des
Sciences.

Voici tout d'abord un extrait composé automatiquement\footnote{Car il dépasse
  trois lignes.} \og hors-texte\fg{}\index{citation!longue}.
%
\blockcquote[pp.~508-509]{cond}{%
  Messieurs,

  Vous avez daigné nous associer en quelque sorte à vos nobles
  travaux; et, en nous permettant de concourir au succès de vos vues
  bienfaisantes, vous avez montré que les sages représentants d'une
  nation éclairée ne pouvaient méconnaître ni le prix des sciences, ni
  l'utilité des compagnies occupées d'en accélérer le progrès et d'en
  multiplier l'application.

  Depuis son institution, l'Académie a toujours saisi et même
  recherché les occasions d'employer pour le bien des hommes, les
  connaissances acquises par la méditation, ou par l'étude de la
  nature : c'est dans son sein qu'un étranger
  illustre\footnote{Huygens}, à qui une théorie profonde avait révélé
  le moyen d'obtenir une unité de longueur naturelle et invariable,
  forma le premier le plan d'y rapporter toutes les mesures pour les
  rendre par là uniformes et inaltérables.
}

Ensuite, je peux citer un long extrait en étant certain qu'il ne
sera pas \og hors-texte\fg{}\index{citation!courte} :
\textcquote[p.~509]{cond}{L'Académie s'est toujours honorée dans ses
  annales d'un préjugé détruit, d'un établissement public
  perfectionné, d'un procédé économique ou salutaire introduit dans
  les arts, que d'une découverte difficile ou brillante; et son
  zèle, encouragé par votre confiance, va doubler d'activité et de
  force.}. Il est bien entendu possible de citer de courts extraits
de manière informelle\index{citation!informelle} : \citeauthor{cond}
insiste sur \enquote{la loi de la nature\textins{, qui} a voulu que
  l'homme fût éclairé pour qu'il pût être juste, et libre pour qu'il
  pût être heureux} de façon à ce que les membres de l'assemblée
nationale n'oublient pas l'importance de l'instruction.

Un petit aparté : il est possible de citer des
citations\index{citation!imbriquée}. Ainsi, le zygomaticien
s'exclama : \enquote{Pierre Dac ne disait-il pas : \enquote{Je suis
    pour tout ce qui est contre et contre tout ce qui est pour!}?
  Si, si, il le disait\ldots{}}, ce qui ne nous surprît guère.

Et maintenant la suite (et fin) du discours, dans un extrait dont je
suis certain qu'il sera \og hors-texte\fg{} :
%
\begin{displaycquote}[pp.~509-511]{cond}
  Et comment pourrions nous oublier jamais que les premiers honneurs
  publics, décernés par vous, l'ont été à la mémoire d'un de nos
  confrères? Ne nous est-il permis de croire que les sciences ont eu
  droit aussi quelque part à ces marques glorieuses de votre estime
  pour un sage qui, célèbre dans les deux mondes par de grandes
  découvertes, n'a jamais chéri dans l'éclat de sa renommée que le
  moyen d'appeler ses concitoyens à l'indépendance d'une voix plus
  imposante, et de rallier en Europe, à une si noble cause, tout ce
  que son génie lui avait mérité de disciples et d'admirateurs?

  Chacun de nous, comme homme, comme citoyen, vous doit une éternelle
  reconnaissance pour le bienfait d'une constitution égale et libre,
  bienfait dont aucune grande nation de l'Europe n'avait encore joui;
  et pour celui de cette déclaration des droits, qui, enchaînant les
  législateurs eux-mêmes par les principes de la justesse universelle,
  rend l'homme indépendant de l'homme, et ne soumet sa volonté qu'à
  l'empire de sa raison.  Mais des citoyens voués par état à la
  recherche de la vérité, instruits par l'expérience, et ce que
  peuvent les lumières pour la félicité générale, et de tout ce que
  les préjugés y opposent d'obstacles, en égarant ou en dégradant les
  esprits, doivent porter plus loin leurs regards, et, sans doute, ont
  le droit de vous remercier au nom de l'humanité, comme au nom de la
  patrie.

  Ils sentent combien, en ordonnant que les hommes ne seraient plus
  rien par des qualités étrangères, et tout par leurs par leurs
  qualités personnelles, vous avez assuré le progrès de l'espèce
  humaine, puisque vous avez forcé l'ambition et la vanité même à ne
  plus attendre les distinctions ou le pouvoir que du talent et des
  lumières; puisque le soin de fortifier sa raison, de cultiver son
  esprit, d'étendre ses connaissances, est devenu le seul moyen
  d'obtenir une considération indépendante et une supériorité réelle.

  Ils savent que vous n'avez pas moins fait pour le bonheur des
  générations futures, en rétablissant l'esprit humain dans son
  indépendance naturelle, que pour celui de la génération présente, en
  mettant les propriétés et la vie des hommes à l'abri des attentats
  du despotisme.

  Ils voient, dans les commissions dont vous les avez chargé, avec
  quelle profondeur de vues vous avez voulu simplifier toutes les
  opérations nécessaires dans les conventions, dans les échanges, dans
  les actions de la vie commune, de peur que l'ignorance ne rendit
  esclave celui que vous aviez déclaré libre, et ne réduisit l'égalité
  prononcée par vos lois à n'être jamais qu'un vain nom.

  Pourraient-ils enfin ne pas apercevoir qu'en établissant pour la
  première fois, le système entier de la société sur des bases
  immuables de la vérité et de la justice, en attachant ainsi par une
  chaîne éternelle les progrès de l'art social au progrès de la
  raison, vous avez étendu vos bienfaits à tous les pays, à tous les
  siècles, et dévoué toutes les erreurs, comme toutes les tyrannies à
  une destruction rapide?

  Ainsi, grâce à la générosité, à la pureté de vos principes, la
  force, l'avarice, ou la séduction, cesseront bientôt de contrarier,
  par des institutions arbitraires, la loi de la nature, qui a voulu
  que l'homme fût éclairé pour qu'il pût être juste, et libre pour
  qu'il pût être heureux. Ainsi, vous jouirez à la fois et du bien que
  vous faites, et du bien que vous préparez, et vous achèverez votre
  ouvrage au milieu des bénédictions de la foule des opprimés dont
  vous avez brisé les fers, et des acclamations des hommes éclairés
  dont vous avez surpassé les espérances.
\end{displaycquote}


% Insertion du deuxième chapitre
\chapter{Développement du chaos du rire}
\label{chap-developpement}

\epigraphhead[30]{\selectlanguage{english}\epigraph{I have not failed. I've
    just found \num{10000} ways that won't work.}{Thomas A. Edison}}

Dans ce chapitre, nous développons notre travail. Nous citons une
référence\index{référence!bibliographique|see{bibliographie}}
bibliographique\index{bibliographie!référence} \autocite{relativite}
car, en effet, nous nous appuierons dans cette partie sur des
résultats fondamentaux qu'on y trouve
\autocite[chap.~3]{relativite}.

Nous ne manquerons pas de causer de \glspl{vrnc}, d'\gls{af} et de \gls{mf},
termes définis dans le glossaire\index{glossaire}. Nous recourrons également aux
symboles de l'\gls{ohm}, du \gls{exists} et de la \gls{planck} définis dans la
liste des symboles. Vous noterez que notre travail a été composé au moyen de
\gls{latex}\index{\gls{latex}}.

Notre étude a porté sur l'\gls{irm} et la \gls{rmn}, définis dans la
liste des acronymes\index{acronyme}. Nous pouvons insérer d'autres
acronymes :
\begin{itemize}
\item \gls{ascii} ;
\item \gls{bios} ;
\item \gls{ctan} ;
\item \gls{dvd} ;
\item \gls{erp} ;
\item \gls{faq} ;
\item \gls{gnu} ;
\item \gls{http} ;
\item \gls{ip} ;
\item \gls{jpeg} ;
\item \gls{kdm} ;
\item \gls{lug} ;
\item \gls{mac} ;
\item \gls{nfs} ;
\item \gls{ocr} ;
\item \gls{p2p} ;
\item \gls{ram} ;
\item \gls{radar} ;
\item \gls{svg} ;
\item \gls{tft} ;
\item \gls{utf-8} ;
\item \gls{vga} ;
\item \gls{wpa} ;
\item \gls{xhtml}.
\end{itemize}
On notera que les acronymes précédents, dont ce sont les premières
occurrences dans le document, figurent sous leur forme complète,
c'est-à-dire sous leur forme développée suivie entre parenthèses de
leur forme abrégée. Ceci est assuré de façon automatique par
\gls{latex} et le package \textsf{glossaries} qui, en outre, vont
composer toutes les occurrences suivantes de ces acronymes
uniquement sous leur forme abrégée\footnote{Sauf contre-ordre.} :
\begin{itemize}
\item \gls{ascii} ;
\item \gls{bios} ;
\item \gls{ctan} ;
\item \gls{dvd} ;
\item \gls{erp} ;
\item \gls{faq} ;
\item \gls{gnu} ;
\item \gls{http} ;
\item \gls{ip} ;
\item \gls{jpeg} ;
\item \gls{kdm} ;
\item \gls{lug} ;
\item \gls{mac} ;
\item \gls{nfs} ;
\item \gls{ocr} ;
\item \gls{p2p} ;
\item \gls{ram} ;
\item \gls{radar} ;
\item \gls{svg} ;
\item \gls{tft} ;
\item \gls{utf-8} ;
\item \gls{vga} ;
\item \gls{wpa} ;
\item \gls{xhtml}.
\end{itemize}
%
\section{Cadre de travail}\label{sec-cadre}
%
Si on examine, dans le fichier\index{\gls{latex}!fichier}
\fichier{developpementI.tex} du répertoire \fichier{corps}, le code source
du tableau~\vref{tab-passionnant}\index{\gls{latex}!tableau}, on verra l'usage
de commandes permettant d'obtenir des tableaux d'allure
professionnelle\footnote{Pour obtenir de tels tableaux sous \LaTeX{}, on
  \href{http://www.tug.org/pracjourn/2007-1/mori/mori.pdf}{trouvera sur
    Internet} comment procéder.}.
%
\begin{table}[ht]
  \centering
  \begin{tabular}{ccc}
    \toprule
    \multicolumn{1}{c}{} & Word & \LaTeX{} \\
    \midrule
    Libre                & Non  & Oui      \\
    Gratuit              & Non  & Oui      \\
    Élégant              & Non  & Oui      \\
    Efficace             & Non  & Oui      \\
    Puissant             & Non  & Oui      \\
    \bottomrule
  \end{tabular}
  \caption{Un tableau passionnant}
  \label{tab-passionnant}
\end{table}

\lipsum[3-22]
%
\section{Méthode de travail}
\label{sec-methode}
%
Nous incluons la figure~\vref{fig-tigre}\index{\gls{latex}!figure} qui n'est
pas là pour faire joli, mais bien pour éclairer notre propos.
\begin{figure}
  \centering
  \capstart
  \includegraphics[width=.35\linewidth]{images/tiger}
  \caption[Un tigre]{Une figure avec une légende assez longue qui peut
    même, au besoin, s'étaler sur plusieurs lignes.}
  \label{fig-tigre}
\end{figure}
On note qu'on peut mettre, en argument optionnel de la commande
permettant de créer la légende, une légende \enquote{courte} qui sera
celle qui figurera, par exemple, dans la liste des figures.

On pourra se convaincre, à la lecture de la section~7.6 \enquote{\emph{Plots and
    Charts}} de la \href{https://ctan.org/pkg/pgf/doc}{documentation du package
  \package{TikZ}}, de ce qu'une figure\index{\gls{latex}!figure} n'est pas
toujours préférable à un tableau\index{\gls{latex}!tableau}. Plus généralement,
on pourra lire toute la section~7 \enquote{\emph{Guidelines on Graphics}}.

Avec \LaTeX{}\index{\gls{latex}|textbf}, il est extrêmement aisé de créer
un index\index{\gls{latex}!index}, comme dans les documents les mieux
composés.

\lipsum[23-42]
%
\section[Discussion]{Discussion et interprétation des résultats}
\label{sec-discussion}
%
On constate que le titre de cette section est différent de ce qui
apparaît en entête et dans la table des matières : c'est l'argument
optionnel de la commande de sectionnement qui a permis cela.

\lipsum[43-52]

% Insertion du troisième chapitre
\chapter{Conclusion du chaos du rire}
Dans ce chapitre, nous concluons l'étude du chaos du rire.

\lipsum[26-27]
\section{Une section}
\lipsum[28-29]
\subsection{Une sous-section}
\lipsum[29-31]
\subsubsection{Une sous-sous-section}
\lipsum[31-35]
\paragraph{Un paragraphe}
\lipsum[36-38]
\paragraph{Un sous-paragraphe}
\lipsum[39-41]
\paragraph{Un autre sous-paragraphe}
\lipsum[39-41]
\paragraph{Un autre paragraphe}
\lipsum[36-38]
\subsubsection{Une autre sous-sous-section}
\lipsum[31-35]
\subsection{Une autre sous-section}
\lipsum[29-31]
\section{Une autre section}
\lipsum[28-29]

%
% Deuxième partie
\part{Le rire du chaos}
% Insertion du premier chapitre
\chapter{Contexte du rire du chaos}
\epigraphhead[30]{\epigraph{La science a fait de nous des dieux avant même que
    nous méritions d'être des hommes.}{Jean Rostand}}

\lipsum[26-32]


% Insertion du deuxième chapitre
\chapter{Développement}

% Nous pouvons insérer de beaux graphiques, comme celui de la figure\vref{3D}.
% \begin{figure}[ht]
%   \centering
%   \capstart
%   \begin{tikzpicture}
%     \begin{axis}[colorbar]
%       \addplot3[surf,domain=0:360]
%       {sin(x)*sin(y)};
%     \end{axis}
%   \end{tikzpicture}
%   \caption{Représentation graphique de la fonction $f:(x,y)\mapsto
%     \sin x\times\sin y$}
%   \label{3D}
% \end{figure}

\lipsum[3-10]

%%% Local Variables: 
%%% mode: latex
%%% TeX-master: "../these"
%%% End: 

% Insertion du troisième chapitre
\chapter{Conclusion}
%
Dans ce chapitre, nous concluons notre étude.

\lipsum[6-9]

%%% Local Variables: 
%%% mode: latex
%%% TeX-master: "../these"
%%% End: 

%
% Chapitre  de conclusion générale
\include{corps/conclusion-generale}
%
% Liste des références bibliographiques (et mention automatique de celle-ci
% dans les sommaire et table des matières).
\printbibliography
%
%%%%%%%%%%%%%%%%%%%%%%%%%%%%%%%%%%%%%%%%%%%%%%%%%%%%%%%%%%%%%%%%%%%%%%
% Début de la partie annexe
%%%%%%%%%%%%%%%%%%%%%%%%%%%%%%%%%%%%%%%%%%%%%%%%%%%%%%%%%%%%%%%%%%%%%%
\appendix
%
% Chapitres annexes
\chapter{Documents juridiques}
\label{chap-juridique}

Cette partie regroupe les documents juridiques officiels.

\section{Licence sous laquelle est publié notre travail}
\label{sec-discours}

\lipsum[11-30]

\section{Transposition de la licence précédente en droit français}
\label{sec-autre-discours}

\lipsum[31-50]

\chapter{Programmes informatiques}
\label{chap-listings}

Les listings suivants sont au cœur de notre travail.

\lstinputlisting[caption={Il est l'heure}]{annexes/programmes/heure.c}
\lstinputlisting[caption={Factorielle}]{annexes/programmes/factorielle.c}

%
%%%%%%%%%%%%%%%%%%%%%%%%%%%%%%%%%%%%%%%%%%%%%%%%%%%%%%%%%%%%%%%%%%%%%%
% Début de la partie finale
%%%%%%%%%%%%%%%%%%%%%%%%%%%%%%%%%%%%%%%%%%%%%%%%%%%%%%%%%%%%%%%%%%%%%%
\backmatter
%
% Glossaire, si souhaité, et si souhaité distinct de la liste des
% acronymes. Sinon, il suffit de commenter ou de supprimer la ligne suivante.
\printglossary
%
% Index, si souhaité. Sinon, il suffit de commenter ou de supprimer la
% ligne suivante.
\printindex
%
% Table des matières
\tableofcontents
%
% Production de la 4e de couverture, si souhaitée. Sinon, il suffit de
% commenter ou de supprimer la ligne suivante.
\makebackcover
%
\end{document}
