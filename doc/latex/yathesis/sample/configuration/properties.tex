%%%%%%%%%%%%%%%%%%%%%%%%%%%%%%%%%%%%%%%%%%%%%%%%%%%%%%%%%%%%%%%%%%%%%%
% Page de couverture et page(s) de titre
%%%%%%%%%%%%%%%%%%%%%%%%%%%%%%%%%%%%%%%%%%%%%%%%%%%%%%%%%%%%%%%%%%%%%%
%
% Préparation des pages de couverture et de titre
%
% Auteur de la thèse : prénom (1er argument), nom (2e argument) et courriel (3e
% argument). Les éventuels accents devront figurer et le nom /ne/ doit /pas/
% être saisi en capitales.
\author{Alphonse}{Allais}{aa@zygo.fr}
%
% Titre de la thèse dans la langue principale (argument obligatoire) et dans la
% langue secondaire (argument optionnel).
\title[Laugh's Chaos]{Le chaos du rire}
%
% Sous-titre de la thèse dans la langue principale (argument obligatoire) et
% dans la langue secondaire (argument optionnel), si souhaité. Sinon, il suffit
% de commenter ou supprimer la ligne suivante.
\subtitle[Chaos' laugh]{Le rire du chaos}
%
% Champ disciplinaire dans la langue principale (argument obligatoire) et dans
% la langue secondaire (argument optionnel).
\academicfield[Mathematics]{Mathématiques}
%
% Spécialité dans la langue principale (argument obligatoire) et dans la langue
% secondaire (argument optionnel) si souhaitée. Sinon, il suffit de commenter
% ou supprimer la ligne suivante.
\speciality[Dynamical systems]{Systèmes dynamiques}
%
% Date de la soutenance, au format {jour}{mois}{année} donnés sous forme de
% nombres.
\date{1}{1}{2015}
%
% Sujet pour les méta-données du PDF
\subject[Chaotic Laugh]{Rire chaotique}
%
% (Tous les fichiers images stipulés ci-dessous comme valeurs des clés « logo »
% se trouvent dans le sous-dossier « images ». Il est conseillé de disposer
% d'images à un format vectoriel, par exemple PDF.)
%
% Logo du PRES, si souhaité. Sinon, il suffit de commenter ou de supprimer la
% ligne suivante.
\pres[logo=images/pres]{Université Lille Nord de France}
%
% Institut (principal en cas de cotutelle).
\institute[logo=images/ulco,url=http://www.univ-littoral.fr/]{ULCO}
%
% En cas de cotutelle (normalement, seulement dans le cas de cotutelle
% internationale), second institut. Sinon, il suffit de commenter ou de
% supprimer la ligne suivante.
\coinstitute[logo=images/paris13]{Université de Paris~13}
%
% Nom de l'école doctorale, si souhaité. Sinon, il suffit de commenter ou
% supprimer la ligne suivante.
\doctoralschool[url=http://edspi.univ-lille1.fr/]{ED Régionale SPI 72}
%
% Laboratoire (ou de l'unité) où la thèse a été préparée.
\laboratory[
logo=images/labo,
logoheight=1.25cm,
telephone=(33)(0)3 21 46 55 86,
fax=(33)(0)3 21 46 55 75,
email=secretariat@lmpa.univ-littoral.fr,
url=http://www-lmpa.univ-littoral.fr/
]{LMPA Joseph Liouville}{%
  Maison de la Recherche Blaise Pascal \\
  50, rue Ferdinand Buisson            \\
  CS 80699                             \\
  62228 Calais Cedex                   \\
  France}
%
% Les membres du jury sont saisis au moyen des commandes \supervisor,
% \cosupervisor, \comonitor, \referee, \committeepresident, \examiner, \guest,
% utilisées /autant de fois que nécessaire/. Toutes basées sur le même modèle,
% ces commandes ont 2 arguments obligatoires, successivement les prénom et nom
% de chaque personne. Si besoin est, on peut apporter certaines précisions en
% argument optionnel, au moyen des clés suivantes :
%
% -- « professor », « seniorresearcher », « mcf », « mcf* »,
% « juniorresearcher », « juniorresearcher* » (qui ne prennent pas de valeur)
% pour stipuler que la personne appartient au corps des professeurs
% d'université ;
%
% -- « affiliation » pour stipuler l'institut auquel est rattachée la personne.
%
\supervisor[professor,affiliation=ULCO]{Michel}{de Montaigne}
\cosupervisor[mcf*,affiliation=ULCO]{Charles}{Baudelaire}
\comonitor[mcf,affiliation=ULCO]{Étienne}{de la Boétie}
\referee[professor,affiliation=IHP]{René}{Descartes}
\referee[seniorresearcher,affiliation=CNRS]{Denis}{Diderot}
\committeepresident[professor,affiliation=ENS Lyon]{Victor}{Hugo}
\examiner[mcf,affiliation=Université de Paris~13]{Sophie}{Germain}
\examiner[juniorresearcher,affiliation=INRIA]{Joseph}{Fourier}
\examiner[juniorresearcher*,affiliation=CNRS]{Paul}{Verlaine}
\guest{George}{Sand}
%
% Numéro d'ordre de la thèse
\ordernumber[42]
%
%%%%%%%%%%%%%%%%%%%%%%%%%%%%%%%%%%%%%%%%%%%%%%%%%%%%%%%%%%%%%%%%%%%%%%
% Mots clés
%%%%%%%%%%%%%%%%%%%%%%%%%%%%%%%%%%%%%%%%%%%%%%%%%%%%%%%%%%%%%%%%%%%%%%
%
% Préparation des mots clés
\keywords{chaos, rire}{chaos, laugh}
