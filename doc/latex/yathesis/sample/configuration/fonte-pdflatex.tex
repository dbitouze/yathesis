%%%%%%%%%%%%%%%%%%%%%%%%%%%%%%%%%%%%%%%%%%%%%%%%%%%%%%%%%%%%%
% Pour tester une autre fonte, commenter celle décommentée et
% décommenter l'une des lignes commençant par un unique « % »
% ci-dessous
%%%%%%%%%%%%%%%%%%%%%%%%%%%%%%%%%%%%%%%%%%%%%%%%%%%%%%%%%%%%%
% % anttor
% \usepackage[math]{anttor}
% % anttor-condensed
% \usepackage[condensed,math]{anttor}
% % anttor-light
% \usepackage[light,math]{anttor}
% % anttor-condensed-light
% \usepackage[light,condensed,math]{anttor}
% % arev
% \usepackage{arev}
% % bookman
% \usepackage{bookman}
% % charter
% \usepackage{charter}
% % computer-concrete
% \usepackage{concmath}
% % charter-bt
% \usepackage[bitstream-charter]{mathdesign}
% % garamond
% \usepackage[urw-garamond]{mathdesign}
% % helvetica
% \usepackage{helvet}\usepackage{textcomp}\renewcommand{\familydefault}{\sfdefault}
% % iwona
% \usepackage[math]{iwona}
% % kerkis
% \usepackage{kmath,kerkis}
% % kpfonts
% \usepackage[frenchstyle,oldstylenums,notextcomp]{kpfonts}
% % kurier
% \usepackage[math]{kurier}
% % latin-modern
% \usepackage{lmodern}\rmfamily\DeclareFontShape{T1}{lmr}{b}{sc}{<->ssub*cmr/bx/sc}{}\DeclareFontShape{T1}{lmr}{bx}{sc}{<->ssub*cmr/bx/sc}{}
% % libertine
% \usepackage{libertine}
% % new-century-schoolbook
% \usepackage{textcomp}\usepackage{newcent}
% utopia
\usepackage{fourier}