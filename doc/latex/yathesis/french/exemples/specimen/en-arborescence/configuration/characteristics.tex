% Auteur de la thèse : prénom (1er argument obligatoire), nom (2e argument
% obligatoire) et éventuel courriel (argument optionnel). Les éventuels accents
% devront figurer et le nom /ne/ doit /pas/ être saisi en capitales
\author[aa@zygo.fr]{Alphonse}{Allais}
%
% Titre de la thèse dans la langue principale (argument obligatoire) et dans la
% langue secondaire (argument optionnel)
\title[Laugh's Chaos]{Le chaos du rire}
%
% (Facultatif) Sous-titre de la thèse dans la langue principale (argument
% obligatoire) et dans la langue secondaire (argument optionnel)
\subtitle[Chaos' laugh]{Le rire du chaos}
%
% Champ disciplinaire dans la langue principale (argument obligatoire) et dans
% la langue secondaire (argument optionnel)
\academicfield[Mathematics]{Mathématiques}
%
% (Facultatif) Spécialité dans la langue principale (argument obligatoire) et
% dans la langue secondaire (argument optionnel)
\speciality[Dynamical systems]{Systèmes dynamiques}
%
% Date de la soutenance, au format {jour}{mois}{année} donnés sous forme de
% nombres
\date{1}{1}{2015}
%
% (Facultatif) Date de la soumission, au format {jour}{mois}{année} donnés sous
% forme de nombres
\submissiondate{1}{10}{2014}
%
% (Facultatif) Sujet pour les méta-données du PDF
\subject[Chaotic Laugh]{Rire chaotique}
%
% (Facultatif) Nom (argument obligatoire) du PRES
\pres[logo=images/pres]{Université Lille Nord de France}
%
% Nom (argument obligatoire) de l'institut (principal en cas de cotutelle)
\institute[logo=images/ulco,url=http://www.univ-littoral.fr/]{ULCO}
%
% (Facultatif) En cas de cotutelle (normalement, seulement dans le cas de
% cotutelle internationale), nom (argument obligatoire) du second institut
\coinstitute[logo=images/paris13,url=http://www.univ-paris13.fr/]{Université de Paris~13}
%
% (Facultatif) Nom (argument obligatoire) de l'école doctorale
\doctoralschool[url=http://edspi.univ-lille1.fr/]{ED Régionale SPI 72}
%
% Nom (1er argument obligatoire) et adresse (2e argument obligatoire) du
% laboratoire (ou de l'unité) où la thèse a été préparée, à utiliser /autant de
% fois que nécessaire/
\laboratory[
logo=images/labo,
logoheight=1.25cm,
telephone=(33)(0)3 21 46 55 86,
fax=(33)(0)3 21 46 55 75,
email=secretariat@lmpa.univ-littoral.fr,
url=http://www-lmpa.univ-littoral.fr/
]{LMPA Joseph Liouville}{%
  Maison de la Recherche Blaise Pascal \\
  50, rue Ferdinand Buisson            \\
  CS 80699                             \\
  62228 Calais Cedex                   \\
  France}
%
% Directeur(s) de thèse et membres du jury, saisis au moyen des commandes
% \supervisor, \cosupervisor, \comonitor, \referee, \committeepresident,
% \examiner, \guest, à utiliser /autant de fois que nécessaire/ et /seulement
% si nécessaire/. Toutes basées sur le même modèle, ces commandes ont
% 2 arguments obligatoires, successivement les prénom et nom de chaque
% personne. Si besoin est, on peut apporter certaines précisions en argument
% optionnel, essentiellement au moyen des clés suivantes :
% - « professor », « seniorresearcher », « mcf », « mcf* »,
%   « juniorresearcher », « juniorresearcher* » (qui peuvent ne pas prendre de
%   valeur) pour stipuler le corps auquel appartient la personne ;
% - « affiliation » pour stipuler l'institut auquel est affiliée la personne ;
% - « female » pour stipuler que la personne est une femme pour que certains
%   mots clés soient accordés en genre.
%
\supervisor[professor,affiliation=ULCO]{Michel}{de Montaigne}
\cosupervisor[mcf*,affiliation=ULCO]{Charles}{Baudelaire}
\comonitor[mcf,affiliation=ULCO]{Étienne}{de la Boétie}
\referee[professor,affiliation=IHP]{René}{Descartes}
\referee[seniorresearcher,affiliation=CNRS]{Denis}{Diderot}
\committeepresident[professor,affiliation=ENS Lyon]{Victor}{Hugo}
\examiner[mcf,affiliation=Université de Paris~13]{Sophie}{Germain}
\examiner[juniorresearcher,affiliation=INRIA]{Joseph}{Fourier}
\examiner[juniorresearcher*,affiliation=CNRS]{Paul}{Verlaine}
\guest{George}{Sand}
%
% (Facultatif) Mention du numéro d'ordre de la thèse (s'il est connu, ce numéro
% est à spécifier en argument optionnel)
\ordernumber[42]
%
% Préparation des mots clés dans la langue principale (1er argument) et dans la
% langue secondaire (2e argument)
%%%%%%%%%%%%%%%%%%%%%%%%%%%%%%%%%%%%%%%%%%%%%%%%%%%%%%%%%%%%%%%%%%%%%%%%%%%%%%%
\keywords{chaos, rire}{chaos, laugh}

