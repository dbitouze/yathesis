\DeclareFixedFootnote{\latexmkrc}{Il s'agit du fichier de configuration du
  programme \program{latexmk} qui permet d'automatiser le processus de
  compilation complète de la thèse (cf. \vref{sec-autom-des-comp} pour plus de
  détails).}
%
\DeclareFixedFootnote{\pagededieelabo}{Produite au moyen de la commande
  facultative \protect\refCom{makelaboratory}.}
%
\DeclareFixedFootnote{\commandeacronyme}{Notamment une commande d'acronyme
  telle que \protect\docAuxCommand{gls} ou \protect\docAuxCommand{acrshort}.}
%
\DeclareFixedFootnote{\syntaxeoptions}{Le sens de la syntaxe décrivant les options est
  explicité \vref{sec-options}.}
%
\DeclareFixedFootnote{\versiontl}{L'année \enquote{\tlversion} est
  éventuellement à remplacer par celle de la version de la \texlive
  effectivement utilisée.}
%
% Fixed footnote « Selon la langue »
\newcommand{\selonlanguebase}{%
  Selon que la langue principale, ou la langue en cours, de la thèse est le
  français ou l'anglais.%
}%
\DeclareFixedFootnote{\selonlangueshort}{%
  \selonlanguebase%
}%
\DeclareFixedFootnote{\selonlangue}{%
  \selonlanguebase (cf. \vref{sec-expressions-cles} pour plus de précisions).%
}%
% \newcommand{\selonlangue}{%
%   \@ifstar{\@tempswatrue\csuse{YAD@starnostar@selonlangue}}{\@tempswafalse\csuse{YAD@starnostar@selonlangue}}%
% }%
% \newcommand{\YAD@starnostar@selonlangue}[1]{%
%   \if@tempswa%
%   \selonlangueshort%
%   \else%
%   \selonlanguelong%
%   \fi%
% }%
%
\DeclareFixedFootnote{\nofrontmatter}{Au contraire, la commande analogue
  \protect\docAuxCommand{frontmatter} pour les \protect\glspl{liminaire} ne
  doit pas être utilisée car elle l'est déjà en sous-main par la \yatCl{}.}
%
\DeclareFixedFootnote{\termesdefinisutilises}{Ne figurent dans ces listes que
  les termes, acronymes et symboles qui sont à la fois \emph{définis} et
  \emph{employés dans le texte}.}
%
\DeclareFixedFootnote{\redefexprcle}{Une autre manière de modifier cet intitulé
  est détaillé \vref{sec-expressions-standard}.}
%
\DeclareFixedFootnote{\hauteurpage}{Dans la limite de la hauteur de page.}
%
\DeclareFixedFootnote{\sepcorpaffil}{Selon l'initiale de l'institut :
  %
  \protect\lstinline[showspaces]+\ à l'+
  %
  ou
  %
  \protect\lstinline[showspaces]+\ au\ +.%
}
%
% \DeclareFixedFootnote{\noillustration}{Cette commande n'est pas illustrée car
%   elle est analogue aux commandes \protect\refCom{acknowledgements} et
%   \protect\refCom{caution}, illustrées
%   \vref{fig-acknowledgements,fig-caution}.}
%
\DeclareFixedFootnote{\nochapter}{Le contenu de ce chapitre doit donc \emph{ne
    pas} comporter d'occurrence de la commande \protect\docAuxCommand{chapter}.
  Il peut cependant contenir une ou plusieurs occurrences des autres commandes
  usuelles de structuration : \protect\docAuxCommand{section},
  \protect\docAuxCommand{subsection}, etc.}
%
\DeclareFixedFootnote{\fichierconfig}{Ceci peut être saisi directement dans le
  préambule du fichier (maître) de la thèse mais, pour optimiser l'usage de la
  \yatCl, il est conseillé de l'insérer dans un fichier nommé
  \file{\configurationfile} à placer dans un dossier nommé
  \folder{\configurationdirectory}. Le canevas de thèse livré avec la
  classe, décrit \vref{sec-canevas}, fournit ce dossier et ce fichier.}
%
\DeclareFixedFootnote{\ifscreenoutput}{Chargé seulement si le
  \Package{hyperref} l'est et si la clé \protect\refKey{output} n'a pour
  valeur ni \protect\docValue{paper}, ni \protect\docValue{paper*}.}
%
\DeclareFixedFootnote{\exceptoneside}{Sauf si l'option
  \protect\docAuxKey{oneside} est utilisée
  (cf. \vref{sec-options-usuelles-de}).}
%
\DeclareFixedFootnote{\noframe}{Sans le cadre.}
%

%%% Local Variables:
%%% mode: latex
%%% TeX-master: "../yathesis-fr"
%%% End:
