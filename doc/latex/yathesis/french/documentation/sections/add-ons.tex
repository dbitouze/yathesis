\chapter{\emph{Add-ons}}\label{cha-add-ons}

La \yatCl{} fournit des \emph{add-ons} destinés à faciliter son usage avec
différents éditeurs de texte.

\section{TeXstudio}
\label{sec-texstudio}

Le fichier de complétion \file{yathesis.cwl}, destiné à l'éditeur
\href{http://texstudio.sourceforge.net/}{TeXstudio}, se trouve dans le
répertoire \folder{\meta{racine}/% \jobdocdirectory
  /addons/completion/} où,
par défaut, \meta{racine} est, avec la distribution :
\begin{description}
\item[\TeX{}~Live :]\
  \begin{description}
  \item[sous Linux et Mac OS X :] \unixtldirectory\tldistdirectory\versiontl ;
  \item[sous Windows :] \wintldirectory\tldistdirectory\versiontl ;
  \end{description}
\item[MiK\TeX{} :] \miktexdistdirectory.
\end{description}
En attendant que ce fichier soit officiellement livré avec cet
éditeur\footnote{Ce devrait être le cas à partir de sa version
  \texttt{2.8.0}.}, ou pour être certain d'en avoir la version la plus à jour,
il suffit de le copier dans le dossier :
\begin{description}
\item[sous Linux et Mac OS X :] \urldirectory{~/.config/texstudio} ;
\item[sous Windows :] \urldirectory{C:\Documents and Settings/User/AppData/Roaming/texstudio}.
\end{description}

\section{Emacs}
\label{sec-emacs}

[TODO]
%
\iffalse
%%% Local Variables:
%%% mode: latex
%%% eval: (latex-mode)
%%% ispell-local-dictionary: "fr_FR"
%%% TeX-engine: xetex
%%% TeX-master: "../yathesis.dtx"
%%% End:
\fi
