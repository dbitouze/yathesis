\chapter{Pages de titre}\label{cha-pages-de-titre}

Ce chapitre documente la commande \refCom{maketitle} permettant de
\emph{produire}, à partir des données définies \vref{sec-proprietes-de-titre},
les pages de titre de la thèse.

\section{Production des pages de titre}

\begin{docCommand}[doc description=\mandatory]{maketitle}{\oarg{options}}%
  \index{titre!production}%
  Cette commande \emph{produit} :
  \begin{enumerate}
  \item
    \begin{enumerate}
    \item une page de 1\iere{} de couverture%
      \index{couverture}%
      \index{première de couverture}%
      \footnote{Sauf contre-ordre, cf. clé \refKey{nofrontcover}.} ;
    \item une page de titre.
    \end{enumerate}
    Ces deux pages sont composées dans la langue principale et
    sont identiques\footnote{À ceci près que le numéro d'ordre de la thèse ne figure
      que sur la page de 1\iere{} de couverture\index{couverture}.} ;
  \item \emph{automatiquement}\footnote{Sans qu'il soit nécessaire de faire
      figurer une 2\ieme{} occurrence de la commande \refCom{maketitle}.} une
    seconde page de titre \emph{si} \aside{et seulement si} l'une au moins des
    commandes \refCom{title}, \refCom{subtitle}, \refCom{academicfield} ou
    \refCom{speciality} est employée avec son argument optionnel
    (cf. \vref{rq-titre-supp}). Cette page est composée dans la langue
    secondaire.
  \end{enumerate}
\end{docCommand}

La commande \refCom{maketitle} admet un argument optionnel permettant de
personnaliser les pages de titre au moyen des clés \refKey{nofrontcover} et
\refKey{frametitle}.%
%
\changes{v0.99g}{2014-07-13}{Personnalisation des pages de titre possible en
  option de \protect\docAuxCommand{maketitle} (\emph{via}
  \protect\refKey{frametitle} et \protect\refKey{nofrontcover})}%

\begin{docKey}{nofrontcover}{=\docValue{true}\textbar\docValue{false}}{par défaut
    \docValue{true}, initialement \docValue{false}}%
  \index{couverture}%
  \index{première de couverture!suppression}%
  \index{titre!première de couverture}%
  \index{titre!première de couverture!suppression}%
  \index{suppression!première de couverture}%
  Cette clé permet de désactiver la production de la 1\iere{} de couverture.
\end{docKey}

{%
  \tcbset{before lower=\vspace*{\baselineskip}\par}
\begin{docKey}{frametitle}{=\docValue{fbox}\textbar\docValue{shadowbox}\textbar\docValue{ovalbox}\textbar\docValue{none}\textbar\marg{autre}}{pas de valeur par défaut, initialement \docValue{fbox}}
  \changes{v0.99c}{2014-06-06}{Nouvelle clé \protect\refKey{frametitle}
    permettant de personnaliser (p. ex. supprimer) le cadre autour du
    titre}%
  \index{titre!cadre entourant}%
  \index{cadre!entourant le titre!supression}%
  \index{cadre!entourant le titre!forme}%
  \index{suppression!cadre entourant le titre}%
  %
  Cette clé permet de personnaliser le cadre figurant par défaut autour du
  titre de la thèse sur les pages de titre :
  \begin{itemize}
  \item sa valeur \docValue{fbox} produit un cadre rectangulaire ;
  \item sa valeur \docValue{shadowbox} produit un cadre ombré ;
  \item sa valeur \docValue{ovalbox} produit un cadre dont les sommets sont
    arrondis ;
  \item sa valeur \docValue{none} permet de supprimer ce cadre. L'affichage des
    mentions \translateexpression{thesistitle} est alors désactivé ;
  \item toute \meta{autre} valeur lui étant passée doit être :
    \begin{enumerate}
    \item une liste de clés/valeurs propres à l'environnement
      \docAuxEnvironment{tcolorbox} du \Package{tcolorbox} (cf. la
      documentation de ce package) ;
    \item passée entre paire d'accolades :
\begin{preamblecode}
\yadsetup{frametitle={"\meta{autre}"}}
\end{preamblecode}
    \end{enumerate}
  \end{itemize}
\end{docKey}
}

\begin{dbexample}{Cadre personnalisé autour du titre de la thèse}{}
  Pour que le cadre entourant le titre de la thèse soit ombré, il suffit de
  saisir :
\begin{preamblecode}
\maketitle[frametitle=shadowbox]
\end{preamblecode}
\end{dbexample}

\begin{dbexample}{Cadre \enquote{fantaisie} autour du titre de la thèse}{}
  \index{cadre!entourant le titre!fantaisie}%
  Cet exemple, certainement déconseillé, montre comment exploiter les
  fonctionnalités du \Package{tcolorbox} pour obtenir un cadre
  \enquote{fantaisie} autour du titre de la thèse.%
  \NoAutoSpacing%
\begin{preamblecode}
\maketitle[frametitle={colback=red!50!white,beamer}]
\end{preamblecode}
\end{dbexample}

\section{Exemple complet de pages de titre}
\label{sec-exemple-complet}

Avec les données caractéristiques suivantes, la commande
\refCom{maketitle} produit :
\begin{enumerate}
\item en langue principale (ici le français),
  \begin{enumerate}
  \item \changes*{v0.99f}{2014-07-11}{Directeurs de thèse désormais dans un
      tableau indépendant de celui des membres du jury sur les pages de
      titre}%
    %
    une page de 1\iere{} de couverture\index{première de couverture} illustrée
    \vref{fig-maketitle-fr} ;
  \item une page de titre ;
  \end{enumerate}
\item en langue secondaire (ici l'anglais), une page de titre illustrée
  \vref{fig-maketitle-en}.
\end{enumerate}
%
\begin{dbremark}{Expressions clés émaillant les pages de titres}{}
  Cet exemple montre que les pages de titres (et, plus généralement, le mémoire
  de thèse) sont émaillées d'expressions, par exemple :
  \begin{itemize}
  \item \translateexpression*{doctoralschool} ;
  \item \translateexpression*{committeemembers} ;
  \end{itemize}
  que la \yatCl{} insère de façon automatique. La \vref{sec-expressions-cles}
  explique comment, au besoin, redéfinir ces expressions.
\end{dbremark}
%
\begin{dbexample}{Préparation et production des pages de titre}{}
%
  \NoAutoSpacing%
  \lstset{morecomment=[is]{\%}{\^^M}}%
  \preamblesample[configuration/characteristics]{%
    deletekeywords={author,title,subtitle,date,url},%
    deletekeywords={[2]title},%
    deletekeywords={[5]%
      academicfield,%
      speciality,%
      institute,%
      coinstitute,%
      doctoralschool,%
      supervisor,%
      cosupervisor,%
      comonitor,%
      committeepresident,%
      ordernumber%
    },%
    rangeendsuffix={\^^M},%
    linerange={%
      author-42]
    }%
  }{title=Préparation du titre (par exemple dans le \File{\characteristicsfile})}
  %
  \lstset{deletecomment=[is]{\%}{\^^M}}%
  %
  \bodysample{rangesuffix=\^^M,linerange={maketitle}}{title=Production
    du titre}
\end{dbexample}

\begin{landscape}
  \begin{figure}[htb]
    \indexex{titre}%
    \indexex{Première de couverture}%
    \centering
    \begin{subfigure}[b]{.45\linewidth}
      \centering%
      \fbox{\includegraphics[page=1,width=\linewidth-2\fboxsep-2\fboxrule]{../exemples/specimen/a-plat/these}}
      %\screenshot[1]{fr-title}
      \caption{Page de 1\iere{} de couverture en français}
      \label{fig-maketitle-fr}
    \end{subfigure}%
    \hspace{\stretch{1}}%
    \begin{subfigure}[b]{.45\linewidth}
      \centering%
      \fbox{\includegraphics[page=5,width=\linewidth-2\fboxsep-2\fboxrule]{../exemples/specimen/a-plat/these}}
      %\screenshot[1]{en-title}
      \caption{Page de titre en anglais}
      \label{fig-maketitle-en}
    \end{subfigure}%
    \caption{Pages de 1\iere{} de couverture et de titre}
    \label{fig-maketitle}
  \end{figure}
\end{landscape}

%%% Local Variables:
%%% mode: latex
%%% TeX-master: "../yathesis-fr"
%%% End:
