\chapter{Partie principale}\label{cha-corps}
\index{partie!principale}%

La partie principale de la thèse, qu'on appelle aussi son \enquote{corps},
comprend :
\begin{enumerate}
\item\index{introduction}%
 l'introduction (\enquote{générale}) ;
\item\index{chapitre!ordinaire}%
  les chapitres \enquote{ordinaires} ;
\item\index{conclusion}%
  la conclusion (\enquote{générale}) ;
\item\index{bibliographie!globale}%
  la bibliographie.
\end{enumerate}
Les introduction et conclusion peuvent éventuellement être
\enquote{générales} par exemple si la thèse comporte plusieurs
parties, chacune introduite par une introduction et conclue par
une conclusion \enquote{ordinaires}.

\begin{dbremark}{Scission du mémoire en fichiers maître et esclaves}{}
  \index{fichier!maître}%
  \index{fichier!esclave}%
  Il est vivement recommandé de scinder le mémoire de thèse,
  notamment son corps, en fichiers maître et esclaves (ces derniers
  correspondants chacun à un chapitre). La procédure
  pour ce faire, standard, est rappelée \vref{sec-repart-du-memo}.
\end{dbremark}

\begin{docCommand}[doc description=\mandatory]{mainmatter}{}
  La partie principale de la thèse doit être manuellement introduite au moyen
  de la commande usuelle \docAuxCommand{mainmatter} de la
  \Class{book}\nofrontmatter.
\end{docCommand}

\section{Chapitres et sections}

La \yatCl{} modifie les commandes usuelles \docAuxCommand{chapter} et
\docAuxCommand{section} en ce qui concerne les trois aspects suivants (examinés
aux \vref{sec-chap-non-numer,sec-intit-altern,sec-chapitres-numerotes}) :
\begin{description}
\item[titres alternatifs des chapitres et sections :] il est possible de
  différencier celui figurant en \gls{tdm} de celui figurant en
  \gls{titrecourant} ;
\item[chapitres et sections \emph{non} numérotés :] l'usage des variantes
  étoilées de ces commandes est simplifié ;
\item[têtes de chapitres numérotés :] leur mise en forme est modifiée (et
  modifiable).
\end{description}

\subsection{Titres alternatifs}
\label{sec-intit-altern}
\index{table des matières!entrée différente du titre courant}%
\index{titre courant!différent de l'entrée en table des matières}%

Par défaut, les commandes \docAuxCommand{chapter} et \docAuxCommand{section}
admettent, en plus de leur argument obligatoire permettant de stipuler le
\meta{titre} du chapitre ou de la section, un argument optionnel (\emph{unique})
permettant de remplacer le \meta{titre} par \emph{un} \meta{titre alternatif}
(\emph{identique}) :
\begin{itemize}
\item en \gls{tdm} ;
\item en \gls{titrecourant} ;
\item en signet.
\end{itemize}

Avec la \yatCl{}, ces commandes admettent un second argument optionnel
permettant de stipuler \emph{deux} \meta{titres alternatifs}
(\emph{différents}) :
\begin{itemize}
\item en \gls{tdm}\signet{} ;
\item en \gls{titrecourant}.
\end{itemize}
La nouvelle syntaxe indiquée ci-dessous, commune aux commandes \refCom{chapter}
et \refCom{section}, est précisée et synthétisée au
\vref{tab-commande-chapter-section}.

\begin{docCommand}{chapter}{\oarg{titre alt.~1}\oarg{titre alt.~2}\marg{titre}}
  \indexdef{chapitre}%
  Cette commande crée un chapitre dont le titre :
  \begin{itemize}
  \item dans le fil du texte est \meta{titre} ;
  \item alternatif en \gls{tdm} est \meta{titre alt.~1}\signet{} ;
  \item alternatif en \gls{titrecourant} est \meta{titre alt.~2}.
  \end{itemize}
\end{docCommand}
%
\begin{docCommand}{section}{\oarg{titre alt.~1}\oarg{titre alt.~2}\marg{titre}}
  \indexdef{section}%
  Cette commande crée une section dont le titre :
  \begin{itemize}
  \item dans le fil du texte est \meta{titre} ;
  \item alternatif en \gls{tdm} est \meta{titre alt.~1}\signet{} ;
  \item alternatif en \gls{titrecourant} est \meta{titre alt.~2}.
  \end{itemize}
  % Son usage précis est synthétisé au \vref{tab-commande-chapter-section}.
\end{docCommand}
%
\begin{dbremark}{Titres alternatifs aussi pour les chapitres et sections non
    numérotés}{}
  Les commandes \docAuxCommand{chapter*} et \docAuxCommand{section*}, permettant
  de créer des chapitres et sections non numérotés, partagent la syntaxe des
  commandes \docAuxCommand{chapter} et \docAuxCommand{section} synthétisée au
  \vref{tab-commande-chapter-section} : elles disposent donc elles aussi de deux
  arguments optionnels permettant différencier entrée en \gls{tdm} et
  \gls{titrecourant}.
\end{dbremark}
%
\begin{table}[htb]
  \centering
  \caption{Usage des (deux arguments optionnels des) commandes
    \protect\refCom{chapter} et \protect\refCom{section} (identique pour les
    commandes \docAuxCommand{chapter*} et \docAuxCommand{section*})}
  \label{tab-commande-chapter-section}
  \footnotesize%
\lstset{%
  deletekeywords={chapter},deletekeywords={[3]chapter},%
  deletekeywords={section},deletekeywords={[3]section},%
}
\begin{tabular}{|l|c|c|c|}
  \cline{2-4}
  \multicolumn{1}{c|}{}
                                                                                                                                                      &
    fil
    du texte                                                                                                                                          & \gls{tdm}
                                                                                                                                                      & entête                                                                                                                                                           \\\hline
  \lstinline+\chapter{+\meta{titre}\lstinline+}+                                                                                                      & \multicolumn{3}{c|}{}                                                                                                                                            \\
  \lstinline+\section{+\meta{titre}\lstinline+}+                                                                                                      & \multicolumn{3}{c|}{\multirow{-2}*{\meta{titre}}}                                                                                                                \\\hline
  \lstinline+\chapter[+\meta{alt. en {\normalfont\ttfamily\glsxtrshort*{tdm}}}\lstinline+]{+\meta{titre}\lstinline+}+                                    &                                                   & \multicolumn{2}{c|}{}                                                                                        \\
  \lstinline+\section[+\meta{alt. en {\normalfont\ttfamily\glsxtrshort*{tdm}}}\lstinline+]{+\meta{titre}\lstinline+}+                                    & \multirow{-2}*{\meta{titre}}                      & \multicolumn{2}{c|}{\multirow{-2}*{\meta{alt. en {\normalfont\ttfamily\glsxtrshort*{tdm}}}}}                    \\\hline
  \lstinline+\chapter[][+\meta{alt. en entête}\lstinline+]{+\meta{titre}\lstinline+}+                                                                 & \multicolumn{2}{c|}{}                             &                                                                                                              \\
  \lstinline+\section[][+\meta{alt. en entête}\lstinline+]{+\meta{titre}\lstinline+}+                                                                 & \multicolumn{2}{c|}{\multirow{-2}*{\meta{titre}}} & \multirow{-2}*{\meta{alt. en entête}}                                                                        \\\hline
  \lstinline+\chapter[+\meta{alt. en {\normalfont\ttfamily\glsxtrshort*{tdm}}}\lstinline+][+\meta{alt. en entête}\lstinline+]{+\meta{titre}\lstinline+}+ &                                                   &                                                                      &                                       \\
  \lstinline+\section[+\meta{alt. en {\normalfont\ttfamily\glsxtrshort*{tdm}}}\lstinline+][+\meta{alt. en entête}\lstinline+]{+\meta{titre}\lstinline+}+ & \multirow{-2}*{\meta{titre}}                      & \multirow{-2}*{\meta{alt. en {\normalfont\ttfamily\glsxtrshort*{tdm}}}} & \multirow{-2}*{\meta{alt. en entête}} \\\hline
\end{tabular}

%%% Local Variables:
%%% mode: latex
%%% TeX-master: "../yathesis-fr"
%%% End:

\end{table}

La syntaxe des commandes \docAuxCommand{subsection},
\docAuxCommand{subsubsection}, \docAuxCommand{paragraph} et
\docAuxCommand{subparagraph} n'est pas modifiée par rapport à celle de la
\Class{book} ; en effet, leurs titres ne figurant que dans le fil du texte et en
\gls{tdm}, il est inutile de pouvoir en spécifier une version spécifique aux
\glspl{titrecourant}.
%
\subsection{Chapitres non numérotés}
\label{sec-chap-non-numer}%
\index{chapitre!non numéroté}%

Si certains chapitres du corps de la thèse \aside{notamment d'introduction et de
  conclusion \enquote{générales}} doivent être \emph{non} numérotés, on recourra de
façon usuelle à la version étoilée de la commande
\docAuxCommand{chapter}. Celle-ci a toutefois été quelque peu modifiée afin
d'en simplifier l'usage.

%  : habituellement, si un chapitre non numéroté est créé
% \emph{dans la partie principale} (entre \docAuxCommand{mainmatter} et
% \docAuxCommand{backmatter}) avec la commande standard
% \docAuxCommand{chapter*} :
% \begin{enumerate}
% \item des précautions (assez techniques) doivent être prises pour que :
%   \begin{enumerate}
%   \item le titre correspondant figure dans la table des matières ;
%   \item les \glspl{titrecourant} correspondants soient corrects ;
%   \end{enumerate}
% \item toutes les (sous-(sous-))sections du chapitre, nécessairement non
%   numérotées elles aussi, doivent également être créées avec les versions
%   étoilées des commandes correspondantes : \docAuxCommand{section*},
%   \docAuxCommand{subsection*} et \docAuxCommand{subsubsection*}.
% \end{enumerate}

\begin{dbremark}{Variante étoilée de la commande \protect\docAuxCommand*{chapter} modifiée}{}
  \indexdef{chapitre!non numéroté}%
  La \yatCl{} modifie les commandes \docAuxCommand{chapter*} et
  \docAuxCommand{section*} de sorte que :
  \begin{enumerate}
  \item automatiquement, le \meta{titre} correspondant figure :
    \begin{enumerate}
    \item en table(s) des matières ;
    \item en \glspl{titrecourant} ;
    \end{enumerate}
  \item les (sous-(sous-))sections du chapitre ou de la section peuvent (et même
    \emph{doivent}) être créées avec les versions \emph{non} étoilées des
    commandes correspondantes : \docAuxCommand{section},
    \docAuxCommand{subsection} et \docAuxCommand{subsubsection}.
  \end{enumerate}
\end{dbremark}

\begin{dbexample}{Introduction}{}
  \indexex{chapitre!non numéroté}%
  Le code suivant produit la \vref{fig-introduction} illustrant une
  introduction (générale) non numérotée. On constate que, bien que seule la
  commande \docAuxCommand{chapter} figure sous sa forme étoilée, aucun élément
  de structuration de ce chapitre n'est numéroté.
  %
  \bodysample[corps/introduction]{%
    deletekeywords={[1]introduction},%
    deletekeywords={[3]section,subsection,subsubsection,paragraph,subparagraph}%
  }{}
\end{dbexample}

\begin{figure}[p]
  \centering
  \screenshot{introduction}
  \caption{Introduction (non numérotée)}
  \label{fig-introduction}
\end{figure}

\subsection{Têtes des chapitres numérotés}
\label{sec-chapitres-numerotes}%
\indexdef{chapitre!numéroté}%

Les chapitres numérotés de la thèse, introduits par la version non étoilée de la
commande \docAuxCommand{chapter}, voient leurs têtes composées par défaut avec
le style |PetersLenny| du \Package{fncychap} (cf. \vref{fig-chapitre}). La
\vref{sec-style-des-tetes} explique comment ceci peut être modifié.

\begin{figure}[ht]
  \centering
  \screenshot{chapter}
  \caption[Chapitre \enquote{ordinaire}]{(Première) Page de chapitre
    \enquote{ordinaire}}
  \label{fig-chapitre}
\end{figure}

\section{Références bibliographiques}\label{sec-refer-bibl}
\indexdef{bibliographie!globale}%

Les références bibliographiques font partie intégrante du corps de la thèse.

Tout système de gestion de bibliographie peut théoriquement être mis en œuvre
avec la \yatCl. Cependant, celle-ci a été conçue plus spécifiquement en vue
d'un usage du \Package{biblatex} et éventuellement de \package{biber},
remplaçant fortement conseillé de \hologo{BibTeX}\footnote{Dans cette section,
  leur fonctionnement est supposé connu du lecteur (sinon, cf. par exemple
  \cite{en-ligne6}).}.

\begin{docCommand}[doc description=\mandatory]{printbibliography}{\oarg{options}}
  Cette commande, fournie par \package{biblatex}, produit la liste des
  références bibliographiques saisies selon la syntaxe de ce package (cf.
  \vref{fig-printbibliography}). Mais elle a été légèrement redéfinie de sorte
  que la bibliographie figure automatiquement dans les sommaire, table des
  matières et signets du document.
\end{docCommand}

\begin{figure}[htbp]
  \indexex{bibliographie!globale}%
  \centering
  \screenshot{printbibliography}
  \caption[Bibliographie]{Bibliographie (ici composée avec le style
    bibliographique par défaut)}
  \label{fig-printbibliography}
\end{figure}

\begin{dbwarning}{Package \package{biblatex} non chargé par défaut}{}
  Le \Package{biblatex} \emph{n'étant pas} chargé par la \yatCl, on veillera
  à le charger manuellement si on souhaite l'utiliser, notamment si on souhaite
  bénéficier de l'ajout automatique de bibliographies locales en fin de
  chapitres (cf. \vref{sec-localbibs}).
\end{dbwarning}

%%% Local Variables:
%%% mode: latex
%%% TeX-master: "../yathesis-fr"
%%% End:
