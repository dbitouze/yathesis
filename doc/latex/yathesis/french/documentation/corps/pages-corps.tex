\chapter{Partie principale}\label{cha-corps}
\index{partie!principale}%

La partie principale de la thèse, qu'on appelle aussi son \enquote{corps},
comprend :
\begin{enumerate}
\item\index{introduction}%
 l'introduction (\enquote{générale}) ;
\item\index{chapitre!ordinaire}%
  les chapitres \enquote{ordinaires} ;
\item\index{conclusion}%
  la conclusion (\enquote{générale}) ;
\item\index{bibliographie!globale}%
  la bibliographie.
\end{enumerate}
Les introduction et conclusion peuvent éventuellement être
\enquote{générales} par exemple si la thèse comporte plusieurs
parties, chacune introduite par une introduction et conclue par
une conclusion \enquote{ordinaires}.

\begin{dbremark}{Scission du mémoire en fichiers maître et esclaves}{}
  \index{fichier!maître}%
  \index{fichier!esclave}%
  Il est vivement recommandé de scinder le mémoire de thèse,
  notamment son corps, en fichiers maître et esclaves (ces derniers
  correspondants chacun à un chapitre). La procédure
  pour ce faire, standard, est rappelée \vref{sec-repart-du-memo}.
\end{dbremark}

\section{Initialisation de la partie principale}

\begin{docCommand}[doc description=\mandatory]{mainmatter}{}
  La partie principale de la thèse doit être manuellement introduite au moyen
  de la commande usuelle \docAuxCommand{mainmatter}\nofrontmatter.
\end{docCommand}

\section{Commandes de structuration}

La \yatCl{} modifie les commandes usuelles de structuration
(\docAuxCommand{chapter}, \docAuxCommand{section}, \docAuxCommand{subsection},
etc.)  en ce qui concerne les trois aspects suivants (examinés aux
\vref{sec-chap-non-numer,sec-intit-altern,sec-chapitres-numerotes}) :
\begin{description}
\item[titres alternatifs des chapitres et sections :] il est possible de
  différencier celui figurant en \gls{tdm} de celui figurant en
  entête (c'est-à-dire en \gls{titrecourant}) ;
\item[unités \emph{non} numérotées :] l'usage des variantes étoilées des
  commandes de structuration est simplifié ;
\item[têtes de chapitres numérotés :] leur mise en forme est modifiée (et
  modifiable).
\end{description}

\subsection{Titres alternatifs des chapitres et sections}
\label{sec-intit-altern}
\index{table des matières!entrée différente du titre courant}%
\index{titre courant!différent de l'entrée en table des matières}%
\index{titre!d'unité!alternatif}
\index{titre!d'unité!normal}

\changes{v0.99p}{2016-12-08}{Commandes \protect\refCom{chapter} et
  \protect\refCom{section} pourvues d'un argument optionnel supplémentaire
  permettant de stipuler un titre alternatif en entête différent de celui
  en \gls{tdm}}%

Avec la \yatCl{}, les entêtes de la plupart des pages contiennent le titre du
chapitre et le titre de l'éventuelle section en cours
(cf. \vref{cha-pagination}).
%
Ce titre est par défaut celui stipulé en argument obligatoire des commandes
respectivement \docAuxCommand{chapter} et \docAuxCommand{section}, et figure
alors également dans le fil du texte et en \gls{tdm}\signet{}.

% Les commandes \docAuxCommand{chapter} et \docAuxCommand{section} admettent un
% argument obligatoire permettant de stipuler le \meta{titre} du chapitre et de la
% section dans le fil du texte, \meta{titre} qui par défaut figure également en
% \gls{tdm}\signet{} et en entête.
%
Les classes standard offrent la possibilité de faire figurer en
\gls{tdm}\signet{} et en entête \emph{un} titre alternatif, différent de celui
stipulé en argument obligatoire : il suffit pour cela de recourir
à l'\emph{unique} argument optionnel des commandes \docAuxCommand{chapter} et
\docAuxCommand{section}. Mais ce titre alternatif est alors nécessairement
\emph{identique} en \gls{tdm}\signet{} et en entête.

La \yatCl{} fournit une fonctionnalité supplémentaire : grâce aux \emph{deux}
arguments optionnels dont elle dote les commandes \docAuxCommand{chapter} et
\docAuxCommand{section}, le titre alternatif en entête peut être différencié de
celui en \gls{tdm}\signet{}.

La nouvelle syntaxe indiquée ci-dessous, commune aux commandes \refCom{chapter}
et \refCom{section},
% n'est indiquée ci-dessous que pour \refCom{chapter} mais
est précisée et synthétisée au \vref{tab-commande-chapter-section}.

\begin{docCommand}[doc new=2016-12-08]{chapter}{\oarg{alt. en {\normalfont\ttfamily\acrshort*{tdm}}}\oarg{alt. en entête}\marg{titre}}
  \indexdef{chapitre!titre alternatif}%
  % Cette commande crée un chapitre dont le titre :
  % \begin{itemize}
  % \item dans le fil du texte est \meta{titre} ;
  % \item alternatif en \gls{tdm}\signet{} est \meta{alt. en {\normalfont\ttfamily\acrshort*{tdm}}} ;
  % \item alternatif en entête est \meta{alt. en entête}.
  % \end{itemize}
\end{docCommand}
%
\begin{docCommand}[doc new=2016-12-08]{section}{\oarg{alt. en {\normalfont\ttfamily\acrshort*{tdm}}}\oarg{alt. en entête}\marg{titre}}
  \indexdef{section!titre alternatif}%
  Ces commandes créent respectivement un chapitre et une section dont le titre :
  \begin{itemize}
  \item dans le fil du texte est \meta{titre} ;
  \item alternatif en \gls{tdm}\signet{} est \meta{alt. en {\normalfont\ttfamily\acrshort*{tdm}}} ;
  \item alternatif en entête est \meta{alt. en entête}.
  \end{itemize}
  % Son usage précis est synthétisé au \vref{tab-commande-chapter-section}.
\end{docCommand}
%
\begin{table}[htb]
  \centering
  \caption{Usage des (deux arguments optionnels des) commandes
    \protect\refCom{chapter} et \protect\refCom{section}%  (identique pour les
    % commandes \docAuxCommand{chapter*} et \docAuxCommand{section*})
  }
  \label{tab-commande-chapter-section}
  \footnotesize%
\lstset{%
  deletekeywords={chapter},deletekeywords={[3]chapter},%
  deletekeywords={section},deletekeywords={[3]section},%
}
\begin{tabular}{|l|c|c|c|}
  \cline{2-4}
  \multicolumn{1}{c|}{}
                                                                                                                                                      &
    fil
    du texte                                                                                                                                          & \gls{tdm}
                                                                                                                                                      & entête                                                                                                                                                           \\\hline
  \lstinline+\chapter{+\meta{titre}\lstinline+}+                                                                                                      & \multicolumn{3}{c|}{}                                                                                                                                            \\
  \lstinline+\section{+\meta{titre}\lstinline+}+                                                                                                      & \multicolumn{3}{c|}{\multirow{-2}*{\meta{titre}}}                                                                                                                \\\hline
  \lstinline+\chapter[+\meta{alt. en {\normalfont\ttfamily\glsxtrshort*{tdm}}}\lstinline+]{+\meta{titre}\lstinline+}+                                    &                                                   & \multicolumn{2}{c|}{}                                                                                        \\
  \lstinline+\section[+\meta{alt. en {\normalfont\ttfamily\glsxtrshort*{tdm}}}\lstinline+]{+\meta{titre}\lstinline+}+                                    & \multirow{-2}*{\meta{titre}}                      & \multicolumn{2}{c|}{\multirow{-2}*{\meta{alt. en {\normalfont\ttfamily\glsxtrshort*{tdm}}}}}                    \\\hline
  \lstinline+\chapter[][+\meta{alt. en entête}\lstinline+]{+\meta{titre}\lstinline+}+                                                                 & \multicolumn{2}{c|}{}                             &                                                                                                              \\
  \lstinline+\section[][+\meta{alt. en entête}\lstinline+]{+\meta{titre}\lstinline+}+                                                                 & \multicolumn{2}{c|}{\multirow{-2}*{\meta{titre}}} & \multirow{-2}*{\meta{alt. en entête}}                                                                        \\\hline
  \lstinline+\chapter[+\meta{alt. en {\normalfont\ttfamily\glsxtrshort*{tdm}}}\lstinline+][+\meta{alt. en entête}\lstinline+]{+\meta{titre}\lstinline+}+ &                                                   &                                                                      &                                       \\
  \lstinline+\section[+\meta{alt. en {\normalfont\ttfamily\glsxtrshort*{tdm}}}\lstinline+][+\meta{alt. en entête}\lstinline+]{+\meta{titre}\lstinline+}+ & \multirow{-2}*{\meta{titre}}                      & \multirow{-2}*{\meta{alt. en {\normalfont\ttfamily\glsxtrshort*{tdm}}}} & \multirow{-2}*{\meta{alt. en entête}} \\\hline
\end{tabular}

%%% Local Variables:
%%% mode: latex
%%% TeX-master: "../yathesis-fr"
%%% End:

\end{table}
%
\begin{dbremark}{Titres alternatifs différenciables aussi pour
    \protect\docAuxCommand*{chapter*} et \protect\docAuxCommand*{section*}%  les chapitres et
    % sections non numérotés
  }{}
  Les commandes \docAuxCommand{chapter*} et \docAuxCommand{section*}, permettant
  de créer des chapitres et sections non numérotés, partagent la syntaxe des
  commandes \docAuxCommand{chapter} et \docAuxCommand{section} synthétisée au
  \vref{tab-commande-chapter-section} : elles admettent donc elles aussi deux
  arguments optionnels permettant de différencier les titres alternatifs
  en \gls{tdm}\signet{} et en entête.
\end{dbremark}
%
La syntaxe des commandes \docAuxCommand{subsection},
\docAuxCommand{subsubsection}, \docAuxCommand{paragraph} et
\docAuxCommand{subparagraph} n'est pas modifiée par rapport à celle de la
\Class{book} ; en effet, les titres correspondants ne figurant que dans le fil
du texte et (éventuellement) en \gls{tdm}\signet{}, il est inutile de pouvoir en
stipuler une version spécifique aux entêtes.
%
\subsection{Unités du mémoire non numérotées}
\label{sec-chap-non-numer}%
% \indexdef{chapitre!non numéroté}%
\indexdef{unité!du mémoire!non numérotée}%
% \indexsee{numérotation!unité}{unité!du mémoire!non numérotée}%

\changes{v0.99p}{2016-12-08}{%
  Simplification de l'usage de toutes les commandes de structuration étoilées
  (et plus seulement de \protect\docAuxCommand{chapter*})%
}

Si certaines unités du corps de la thèse \aside{par exemple des chapitres
  d'introduction et de conclusion \enquote{générales}} doivent être \emph{non}
numérotées, on recourra de façon usuelle à la version étoilée des commandes
correspondantes. Ces dernières ont toutefois été quelque peu modifiées afin d'en
simplifier l'usage.

%  : habituellement, si un chapitre non numéroté est créé
% \emph{dans la partie principale} (entre \docAuxCommand{mainmatter} et
% \docAuxCommand{backmatter}) avec la commande standard
% \docAuxCommand{chapter*} :
% \begin{enumerate}
% \item des précautions (assez techniques) doivent être prises pour que :
%   \begin{enumerate}
%   \item le titre correspondant figure dans la table des matières ;
%   \item les entêtes correspondants soient corrects ;
%   \end{enumerate}
% \item toutes les (sous-(sous-))sections du chapitre, nécessairement non
%   numérotées elles aussi, doivent également être créées avec les versions
%   étoilées des commandes correspondantes : \docAuxCommand{section*},
%   \docAuxCommand{subsection*} et \docAuxCommand{subsubsection*}.
% \end{enumerate}

\begin{dbremark}{Variantes étoilées des commandes de structuration modifiées}{}
  La \yatCl{} modifie les variantes étoilées des commandes de structuration
  (\docAuxCommand{chapter*}, \docAuxCommand{section*},
  \docAuxCommand{subsection*}, etc.) de sorte que :
  \begin{enumerate}
  \item automatiquement, le titre (alternatif le cas échéant) correspondant
    figure :
    \begin{enumerate}
    \item en \gls{tdm} (selon la profondeur choisie : cf. \refKey{depth} et
      \refKey{localtocs/depth}) ;
    \item en entête (pour les chapitres et sections seulement) ;
    \end{enumerate}
  \item si les unités correspondantes contiennent des sous-unités, ces dernières
    puissent (et même \emph{doivent}) être créées avec les versions \emph{non}
    étoilées des commandes correspondantes : elles seront néanmoins \emph{non}
    numérotées (comme l'unité les contenant).

    Ainsi, si un chapitre est non numéroté, les sections, sous-sections,
    sous-sous-sections, etc. qu'il contient doivent aussi être non
    numérotées. Et, avec la \yatCl{}, elles seront cependant introduites par les
    commandes \emph{non} étoilées correspondantes : \docAuxCommand{section},
    \docAuxCommand{subsection}, \docAuxCommand{subsubsection}, etc.
  \end{enumerate}
\end{dbremark}

\begin{dbexample}{Introduction}{}
  \indexex{chapitre!non numéroté}%
  \indexex{unité!du mémoire!non numérotée}%
  Le code suivant produit la \vref{fig-introduction} illustrant une
  introduction (générale) non numérotée. On constate que, bien que seule la
  commande \docAuxCommand{chapter} figure sous sa forme étoilée, aucun élément
  de structuration de ce chapitre n'est numéroté.
  %
  \bodysample{introduction}{%
    deletekeywords={[1]introduction},%
    deletekeywords={[3]section,subsection,subsubsection,paragraph,subparagraph}%
  }{}
\end{dbexample}

\begin{figure}[p]
  \centering
  \screenshot{introduction}
  \caption{Introduction (non numérotée)}
  \label{fig-introduction}
\end{figure}

\subsection{Têtes des chapitres numérotés}
\label{sec-chapitres-numerotes}%
\indexdef{chapitre!numéroté}%

Les chapitres numérotés de la thèse, introduits par la version non étoilée de la
commande \docAuxCommand{chapter}, voient leurs têtes composées par défaut avec
le style |PetersLenny| du \Package{fncychap} (cf. \vref{fig-chapitre}). La
\vref{sec-style-des-tetes} explique comment ceci peut être modifié.

\begin{figure}[ht]
  \centering
  \screenshot{chapter}
  \caption[Chapitre \enquote{ordinaire}]{(Première) Page de chapitre
    \enquote{ordinaire}}
  \label{fig-chapitre}
\end{figure}

\section{Références bibliographiques}\label{sec-refer-bibl}
\indexdef{bibliographie!globale}%

Les références bibliographiques font partie intégrante du corps de la thèse.

Tout système de gestion de bibliographie peut théoriquement être mis en œuvre
avec la \yatCl. Cependant, celle-ci a été conçue plus spécifiquement en vue
d'un usage du \Package{biblatex} et éventuellement de \package{biber},
remplaçant fortement conseillé de \hologo{BibTeX}\footnote{Dans cette section,
  leur fonctionnement est supposé connu du lecteur (sinon, cf. par exemple
  \cite{en-ligne6}).}.

\begin{docCommand}[doc description=\mandatory]{printbibliography}{\oarg{options}}
  Cette commande, fournie par \package{biblatex}, produit la liste des
  références bibliographiques saisies selon la syntaxe de ce package (cf.
  \vref{fig-printbibliography}). Mais elle a été légèrement redéfinie de sorte
  que la bibliographie figure automatiquement dans les sommaire, table des
  matières et signets du document.
\end{docCommand}

\begin{figure}[htbp]
  \indexex{bibliographie!globale}%
  \centering
  \screenshot{printbibliography}
  \caption[Bibliographie]{Bibliographie (ici composée avec le style
    bibliographique par défaut)}
  \label{fig-printbibliography}
\end{figure}

\begin{dbwarning}{Package \package{biblatex} non chargé par défaut}{}
  Le \Package{biblatex} \emph{n'étant pas} chargé par la \yatCl, on veillera
  à le charger manuellement si on souhaite l'utiliser, notamment si on souhaite
  bénéficier de l'ajout automatique de bibliographies locales en fin de
  chapitres (cf. \vref{sec-localbibs}).
\end{dbwarning}

%%% Local Variables:
%%% mode: latex
%%% TeX-master: "../yathesis-fr"
%%% End:
