\chapter{Caractéristiques de la thèse}
\label{cha-caract-du-docum}%
\indexdef{caractéristiques de la thèse}%

Ce chapitre liste les commandes et options permettant de spécifier les données
caractéristiques de la thèse. La plupart d'entre elles sont ensuite affichées
en divers emplacements du mémoire :
\begin{itemize}
\item%
  \index{titre}%
  \index{couverture}%
  \index{première de couverture}%
  sur les pages de 1\iere{} de couverture et de titre(s), produites par la
  commande \refCom{maketitle} ;
\item\index{laboratoire}%
  sur l'éventuelle page dédiée au(x) laboratoire(s) où la thèse a été préparée,
  produite par la commande \refCom{makelaboratory} ;
\item\index{mot clé}%
  sur l'éventuelle page dédiée aux mots clés, produite par la commande
  \refCom{makekeywords} ;
\item\index{résumé}%
  sur la page dédiée aux résumés, produite par la commande \refCom{makeabstract} ;
\item\index{quatrième de couverture}%
  sur l'éventuelle 4\ieme{} de couverture, produite par la commande
  \refCom{makebackcover}.
\end{itemize}
Certaines de ces caractéristiques figurent également comme métadonnées du fichier
\pdf{} produit.

\section{Où spécifier les caractéristiques de la thèse ?}
\label{sec-lieu-de-saisie}%
\index{caractéristiques de la thèse!lieu de spécification}%

Les commandes permettant de définir les caractéristiques de la thèse peuvent être
saisies, au choix :
\begin{description}
\item[dans le fichier (maître) de la thèse :]\
  \begin{enumerate}
  \item soit dans son préambule ;
  \item soit dans son corps ;
    \begin{dbwarning}{Caractéristiques de la thèse à saisir \emph{avant}
        \protect\refCom*{maketitle}}{avant-maketitle}
    Si les caractéristiques de la thèse sont saisies dans le corps du fichier
    (maître) de la thèse, elles doivent nécessairement l'être \emph{avant} la
    commande \refCom{maketitle}.
  \end{dbwarning}
  \end{enumerate}
\item[dans un fichier dédié]
  \index{fichier!des caractéristiques de la thèse}%
  \index{dossier!de configuration}%
  % \item\label{characteristicsfile} dans un fichier dédié,
  à nommer \file{\characteristicsfile} et à placer dans un sous-dossier à nommer
  \folder{\configurationdirectory}. Ces fichier et sous-dossier \aside{tous
    deux prévus à cet effet} sont à créer au besoin mais ils sont fournis par le
  canevas de thèse \enquote{en arborescence} livré avec la classe, décrit
  \vref{sec-canevas-arborescence}.
  \begin{dbwarning}{Fichier de caractéristiques à ne pas importer manuellement}{}
    Le \File{\characteristicsfile} est \emph{automatiquement} importé par la
    \yatCl{} et il doit donc \emph{ne pas} être explicitement importé : on
    \emph{ne} recourra donc \emph{pas} à la commande
    |\input{|\file{\characteristicsfile}|}| (ou autre commande d'importation
    similaire à \docAuxCommand{input}).
  \end{dbwarning}
\end{description}

\section{Caractéristiques de titre}
\label{sec-proprietes-de-titre}

Cette section liste les commandes et options permettant de \emph{préparer} les
pages de 1\iere{} de couverture et de titre de la thèse\footnote{Sauf cas
  particulier, ces pages seront dans la suite appelées simplement
  \enquote{pages de titre}.}.

\subsection{Auteur, (sous-)titre, spécialité, sujet,
  date}\label{sec-caracteristiques}

Les commandes suivantes permettent de stipuler les auteur, titre et éventuel
sous-titre, champ disciplinaire, spécialité, date et sujet de la thèse. Toutes
ces données, sauf le sujet, figureront automatiquement sur les pages de
titre\footnote{En outre, les titres et éventuels sous-titres figureront sur les
  pages de résumé (cf. \vref{sec-abstract}) et de 4\ieme{} de couverture (cf.
  \vref{sec-quatr-de-couv}).}.
%
\begin{docCommand}[doc description=\mandatory]{author}{\oarg{adresse
      courriel}\marg{prénom}\marg{nom}}%
  \index{caractéristiques de la thèse!liste!auteur}%
  \indexdef{auteur}%
  \index{auteur!courriel}%
  \index{courriel}%
  \indexdef{courriel!auteur}%
  \index{mail|see{courriel}}%
  \index{email|see{courriel}}%
  Cette commande définit l'auteur de la thèse. Ses \meta{prénom} et
  \meta{nom} :
  \begin{itemize}
  \item figureront sur la ou les pages de titre ;
  \item%
    \index{lien hypertexte!courriel}%
    seront un lien hypertexte vers l'\meta{adresse courriel} si celle-ci est
    renseignée en argument optionnel ;
  \item apparaîtront aussi comme métadonnée \enquote{Auteur} du
    fichier \pdf{} de la thèse.
  \end{itemize}
  \begin{dbwarning}{Format des prénom et nom de l'auteur}{}
    \index{auteur!format}%
    \index{nom!format}%
    \index{prénom!format}%
    \index{format!nom}%
    \index{format!prénom}%
    On veillera à ce que :
    \begin{enumerate}
    \item les éventuels accents figurent dans les \meta{prénom} et
      \meta{nom};
    \item \index{caractéristiques de la thèse!liste!auteur}%
      \index{capitales}%
      \index{majuscules|see{capitales}}%
      le \meta{nom} \emph{ne} soit \emph{pas} saisi en capitales (sauf pour la
      ou les majuscules) car il sera automatiquement composé en petites
      capitales.
    \end{enumerate}
  \end{dbwarning}
\end{docCommand}
%
\begin{docCommand}[doc description=\mandatory]{title}{\oarg{titre dans la langue
      secondaire}\marg{titre}}%
  \index{caractéristiques de la thèse!liste!titre}%
  \indexdef{titre}%
  Cette commande définit le \meta{titre} de la thèse. Celui-ci apparaît alors
  aussi comme métadonnée \enquote{Titre} du fichier \pdf{} de la thèse.
\end{docCommand}
%
\begin{docCommand}{subtitle}{\oarg{sous-titre dans la langue
      secondaire}\marg{sous-titre}}%
  \index{caractéristiques de la thèse!liste!sous-titre}%
  \indexdef{sous-titre}%
  \index{titre!sous-titre}%
  Cette commande définit l'éventuel \meta{sous-titre} de
  la thèse.
\end{docCommand}
%
\begin{docCommand}[doc description=\mandatory]{academicfield}{\oarg{discipline dans la langue
      secondaire}\marg{discipline}}%
  \index{caractéristiques de la thèse!liste!discipline}%
  \indexdef{discipline}%
  Cette commande définit la \meta{discipline} \aside{ou champ disciplinaire}
  de la thèse. Celui-ci apparaît alors aussi comme métadonnée \enquote{Sujet} du
  fichier \pdf{} de la thèse, sauf si la commande \refCom{subject} est utilisée.
\end{docCommand}
%
\begin{docCommand}{speciality}{\oarg{spécialité dans la langue
      secondaire}\marg{spécialité}}%
  \index{caractéristiques de la thèse!liste!spécialité}%
  \indexdef{spécialité}%
  Cette commande définit la \meta{spécialité} (du champ
  disciplinaire) de la thèse.
\end{docCommand}
%
\begin{dbremark}{Titre, sous-titre, champ disciplinaire et spécialité dans la
    langue secondaire}{titre-supp}%
  \index{caractéristiques de la thèse!liste!langue}%
  \emph{Via} leur argument obligatoire, les commandes \refCom{title},
  \refCom{subtitle}, \refCom{academicfield} et \refCom{speciality} définissent
  les titre, sous-titre, champ disciplinaire et spécialité, \emph{dans la
    langue principale} de la thèse \aside*{par défaut le français}. Chacune de
  ces commandes admet un argument optionnel permettant de stipuler la donnée
  correspondante \emph{dans la langue secondaire} de la thèse \aside*{par
    défaut l'anglais\footnote{Les langues principale et secondaire de la thèse
      sont détaillées \vref{sec-langues}.}}.

  Dès lors qu'une au moins des ces commandes est employée avec son argument
  optionnel, la commande \refCom{maketitle}, qui produit les pages de titre
  composées dans la langue principale, génère \emph{automatiquement} une page
  de titre \emph{supplémentaire} composée dans la langue secondaire.
\end{dbremark}
%
\begin{docCommand}[doc description=\mandatory]{date}{\marg{jour}\marg{mois}\marg{année}}
  \index{caractéristiques de la thèse!liste!date de soutenance}%
  \index{date!de soutenance}%
  Cette commande définit la date de la soutenance.
\end{docCommand}
%
\begin{docCommand}{submissiondate}{\marg{jour}\marg{mois}\marg{année}}
  \index{caractéristiques de la thèse!liste!date de soumission}%
  \index{date!de soumission}%
  %
  \changes{v0.99k}{2014-10-01}{Nouvelle commande \protect\refCom{submissiondate}
    permettant de stipuler une date de soumission du mémoire aux rapporteurs}%
  %
  Cette commande définit la date de la soumission du mémoire (qui ne figure
  qu'en version \enquote{à soumettre}, cf. option \docValue{submitted*}
  \vref{sec-versions}). Ses arguments sont soumis aux mêmes contraintes que ceux
  de la commande \refCom{date} (cf. \vref{wa-format-date}).
\end{docCommand}
%
\begin{dbwarning}{Format des jour, mois et année des dates de
    soutenance et de soumission}{format-date}
  \index{date!format}%
  \index{format!date}%
  Les \meta{jour}, \meta{mois} et \meta{année} doivent être des nombres
  (entiers), \meta{jour} et \meta{mois} étant compris respectivement :
  \begin{itemize}
  \item entre |1| et |31| ;
  \item entre |1| et |12|.
  \end{itemize}
\end{dbwarning}
%
\begin{docCommand}{subject}{\oarg{sujet dans la langue
      secondaire}\marg{sujet de la thèse}}
   \index{caractéristiques de la thèse!liste!sujet}%
   \indexdef{sujet}%
  Cette commande définit le \meta{sujet de la thèse}.  Celui-ci ne figure nulle
  part dans la version papier du mémoire : il n'apparaît que comme métadonnée
  \enquote{Sujet} du fichier \pdf{} de la thèse. Si cette commande n'est pas
  employée, c'est le champ disciplinaire (commande \refCom{academicfield}) qui
  apparaît comme métadonnée \enquote{Sujet}.
\end{docCommand}

\begin{dbexample}{Auteur, (sous-)titre, spécialité, sujet, date}{}
   \indexex{caractéristiques de la thèse}%
  Les données principales d'une thèse peuvent être les suivantes.
  % \tcbset{listing options={deletekeywords={[2]title}}}
  \NoAutoSpacing%
\begin{preamblecode}[title=Par exemple dans le \File{\characteristicsfile},listing options={deletekeywords={author,title,subtitle,date},deletekeywords={[2]title},deletekeywords={[5]academicfield,speciality}}]
\author[aa@zygo.fr]{Alphonse}{Allais}
\title[Laugh's Chaos]{Le chaos du rire}
\subtitle[Chaos' laugh]{Le rire du chaos}
\academicfield[Mathematics]{Mathématiques}
\speciality[Dynamical systems]{Systèmes dynamiques}
\date{1}{1}{2015}
\subject{Rire chaotique}
\end{preamblecode}
\end{dbexample}

\subsection{Instituts et entités}\label{sec-entites}

Les instituts et entités dans lesquels \aside{ou grâce auxquels} la thèse
a été préparée sont définis et précisés au moyen des commandes et options
listés dans cette section. Ils figureront automatiquement sur la ou les pages
de titre\footnote{Le ou les laboratoires apparaissent en outre sur les pages
  dédiée aux laboratoires, de résumés et de 4\ieme{} de couverture.}.

\subsubsection{Définition}
%
\begin{docCommand}{comue}{\oarg{précision(s)}\marg{nom de la
      COMUE}}
   \index{caractéristiques de la thèse!liste!\acrshort{comue}}%
   \indexdef{\acrshort{comue}}%
  Cette commande définit la \gls{comue}. Celle-ci ne figure que par
  l'intermédiaire de ses logo et \acrshort{url} spécifiés au moyen des clés
  \refKey{logo} et \refKey{url}.
\end{docCommand}
%
\begin{docCommand}[doc description=\mandatory]{institute}{\oarg{précision(s)}\marg{nom de
      l'institut}}
  \index{caractéristiques de la thèse!liste!institut principal}%
  \indexdef{institut}%
  \indexdef{institut!principal}%
  \indexsee{université}{institut}%
  \indexsee{école}{institut}%
  Cette commande définit l'institut (ou l'université, l'école, etc.), principal
  en cas de cotutelle.
\end{docCommand}
%
\begin{docCommand}{coinstitute}{\oarg{précision(s)}\marg{nom de
      l'institut}}
  \index{caractéristiques de la thèse!liste!institut de cotutelle}%
  \index{institut!de cotutelle}%
  \indexdef{cotutelle!institut}%
  Cette commande définit l'institut de cotutelle. Celle-ci ne
  devrait être employée qu'en cas de thèse cotutelle de nature
  \emph{internationale}.
\end{docCommand}
%
\begin{docCommand}{company}{\oarg{précision(s)}\marg{nom de l'entreprise}}
  \index{caractéristiques de la thèse!liste!entreprise}%
  \indexdef{entreprise}%
  \indexsee{thèse industrielle}{entreprise}%
  \indexsee{\acrshort{cifre}}{entreprise}%
  Cette commande définit l'entreprise ayant (co)financé la thèse.
  Celle-ci ne devrait être employée qu'en cas de thèse industrielle (par
  exemple dans le cadre d'un dispositif \acrshort{cifre}).
\end{docCommand}
%
\begin{docCommand}[doc description=\mandatory]{doctoralschool}{\oarg{précision(s)}\marg{nom de l'école
      doctorale}}
  \index{caractéristiques de la thèse!liste!école doctorale}%
  \indexdef{école doctorale}%
  Cette commande définit l'école doctorale.
\end{docCommand}
%
\begin{docCommand}[doc description=\mandatory]{laboratory}{\oarg{précision(s)}\marg{nom}\marg{adresse}}
  \index{caractéristiques de la thèse!liste!laboratoire}%
  \indexdef{laboratoire}%
  \index{laboratoire!nom}%
  \index{laboratoire!adresse}%
  Cette commande définit le nom et l'adresse du laboratoire.
\end{docCommand}
%
\begin{dbremark}{Changements de ligne dans l'adresse du laboratoire}{}
  Il est possible de composer l'\meta{adresse} du laboratoire sur plusieurs
  lignes au moyen de la commande |\\|.
\end{dbremark}
%
\begin{dbexample}{Instituts et entités}{}
  \indexex{laboratoire}%
  Si la thèse a été préparée au \gls{lmpa} de l'\gls{ulco}, on
  pourra recourir à :
  \NoAutoSpacing%
\begin{preamblecode}[listing options={deletekeywords={[5]institute,doctoralschool}}]
\comue{Université Lille Nord de France}
\institute{ULCO}
\doctoralschool{ED Régionale SPI 72}
\laboratory{LMPA}{%
  Maison de la Recherche Blaise Pascal \\
  50, rue Ferdinand Buisson            \\
  CS 80699                             \\
  62228 Calais Cedex                   \\
  France%
}
\end{preamblecode}
\end{dbexample}
%
\begin{dbremark}{Laboratoires multiples}{}
  \index{laboratoire!multiple}%
  Si la thèse a été préparée dans plusieurs laboratoires, il est possible de
  tous les spécifier en utilisant la commande \refCom{laboratory} autant de
  fois que nécessaire. Par convention, le laboratoire stipulé à la première
  \aside{et éventuellement seule} occurrence de la commande
  \refCom{laboratory} est le laboratoire \emph{principal}.

  En cas de laboratoires multiples, tous ne figurent pas systématiquement :
  \begin{itemize}
  \item sur les pages de titre, le seul laboratoire affiché est le
    laboratoire principal ;
  \item sur les pages de résumés et de 4\ieme{} de couverture
    (cf. \vref{sec-abstract,sec-quatr-de-couv}), par défaut seul le laboratoire
    principal est affiché (mais un nombre arbitraire de laboratoires peut être
    affiché grâce à la clé \refKey{numlaboratories}) ;
  \item sur la page \aside{facultative} qui leur est dédiée
    (cf. \vref{sec-laboratoires}), tous les laboratoires stipulés sont
    affichés.
  \end{itemize}
\end{dbremark}

\subsubsection{Précisions}
\index{caractéristiques de la thèse!précision}%

Toutes les commandes précédentes admettent un argument optionnel permettant
d'apporter sur les instituts ou entités des \meta{précisions} --- sous la forme
d'une liste \meta{clé}|=|\meta{valeur}.
%
\paragraph{Pour tout institut ou entité}

Les clés suivantes\syntaxeoptions{} sont valables pour tout institut ou entité.

\begin{docKey}{logo}{=\meta{fichier image}}{pas de valeur
    par défaut, initialement vide}
  \indexdef{logo}%
  \index{laboratoire!logo}%
  \index{institut!logo}%
  Cette option définit le logo d'un institut, spécifié sous la forme de (du
  chemin menant à) son \meta{fichier image}.
  \begin{dbexample}{Logo d'institut}{logoinst}
    Supposons que la thèse ait été préparée à l'\gls{ulco} et qu'on dispose du
    logo de cette université sous la forme d'un fichier nommé
    \texttt{ulco.pdf}, situé dans le sous-dossier \folder{images}. On
    saisira alors :
\begin{preamblecode}[listing options={deletekeywords={[5]institute}}]
\institute[logo=images/ulco]{ULCO}
\end{preamblecode}
\end{dbexample}
Tous les logos apparaissent automatiquement en haut de la ou des
pages de titre, sauf :
\begin{itemize}
\item ceux des laboratoires qui ne figurent que sur l'éventuelle page qui leur
  est dédiée ;
\item celui de l'école doctorale qui ne figure nulle part et qu'il est donc
  inutile de spécifier.
\end{itemize}
\end{docKey}
%
\begin{docKey}{logoheight}{=\meta{dimension}}{pas de valeur par
    défaut, initialement \docValue*{1.5cm}}
  \indexdef{logo!taille}%
  \indexdef{taille!logo}%
  Par défaut, tous les logos ont une même hauteur de \SI{1.5}{\cm}
  mais la clé \refKey{logoheight} permet de spécifier une hauteur
  différente.
  \begin{dbexample}{Hauteur du logo d'institut}{}
    La commande de l'\vref{ex-logoinst} aurait ainsi pu contenir :
\begin{preamblecode}[listing options={deletekeywords={[5]institute}}]
\institute[logoheight=1cm,logo=images/ulco]{ULCO}
\end{preamblecode}
\end{dbexample}
\end{docKey}
%
% \DescribeOption{nologo}
% L'option |nologo| (qui ne prend pas de valeur) pour que le logo d'un
% institut ne figure pas, même s'il a été précisé.
%
\begin{docKey}{url}{=\meta{\acrshort*{url} de l'institut}}{pas de valeur par
    défaut, initialement vide}
  \indexdef{\acrshort{url}}%
  \index{laboratoire!\acrshort{url}}%
  \index{institut!\acrshort{url}}%
  \index{lien hypertexte}%
  Cette option définit l'\acrshort{url} d'un institut. Les noms et éventuels
  logos des instituts sont alors des liens hypertextes pointant vers cette
  \acrshort{url}.
  \begin{dbexample}{\acrshort*{url} d'institut}{}
    Si la thèse a été préparée à l'\gls{ulco}, on pourra recourir à :%
    \NoAutoSpacing%
\begin{preamblecode}[listing options={deletekeywords={url},deletekeywords={[5]institute}}]
\institute[url=http://www.univ-littoral.fr/]{ULCO}
\end{preamblecode}
\end{dbexample}
\begin{dbwarning}{Caractère \protect\lstinline+\#+ à protéger dans les
    \acrshortpl*{url} d'instituts et entités}{}
  \index{\acrshort{url}!format}%
  \index{format!\acrshort{url}}%
  Au cas (peu probable) où le caractère |#| doive figurer dans ces
  \acrshortpl{url}, il doit être \enquote{protégé} au moyen d'une
  contre-oblique le précédant : |\#|.
\end{dbwarning}
\end{docKey}

\paragraph{Pour le laboratoire seulement}
  \index{caractéristiques de la thèse!liste!laboratoire}%

Les options supplémentaires suivantes \emph{ne} sont prévues
\emph{que} pour l'entité \enquote{laboratoire} qui, contrairement
aux autres, peut disposer d'une page dédiée\pagededieelabo.
%
\begin{docKey}{telephone}{=\meta{numéro}}{pas de valeur par défaut,
    initialement vide}
  \indexdef{téléphone}%
  \index{laboratoire!téléphone}%
  Cette option définit le numéro de téléphone du laboratoire.
\end{docKey}
%
\begin{docKey}{fax}{=\meta{numéro}}{pas de valeur par défaut,
    initialement vide}
  \indexdef{fax}%
  \index{laboratoire!fax}%
  Cette option définit le numéro de fax du laboratoire.
\end{docKey}
%
\begin{docKey}{email}{=\meta{adresse courriel}}{pas de valeur par
    défaut, initialement vide}
  \index{courriel}%
  \indexdef{courriel!laboratoire}%
  \index{laboratoire!courriel}%
  Cette option définit l'adresse courriel du laboratoire.
\end{docKey}
%
\begin{docKey}{nonamelink}{=\docValue{true}\textbar\docValue{false}}{par défaut
    \docValue{true}, initialement \docValue{false}}
  \changes{v0.99i}{2014-07-17}{Nouvelle option \protect\refKey{nonamelink}
    agissant sur les hyperliens des laboratoires}%
  \index{lien hypertexte!suppression}%
  \indexsee{hyperlien}{lien hypertexte}%
%
  Cette option a pour effet que, si l'\acrshort{url} du laboratoire a été
  définie au moyen de l'option \refKey{url}, le nom de celui-ci n'est pas un
  lien hypertexte : seuls l'\acrshort{url} en regard de la mention du site Web
  et l'éventuel logo figurant la page dédiée aux laboratoires\pagededieelabo{}
  sont des liens hypertextes pointant vers cette \acrshort{url}.
\end{docKey}
%
\begin{dbexample}{Laboratoire}{}
  \indexex{laboratoire}%
  Si la thèse a été préparée au \gls{lmpa}, on peut recourir à :
  \NoAutoSpacing%
\begin{preamblecode}[listing options={deletekeywords={url}}]
\laboratory[
telephone=(33) 03 21 46 55 86,
fax=(33) 03 21 46 55 75,
email=secretariat@lmpa.univ-littoral.fr,
url=http://www-lmpa.univ-littoral.fr/
]{LMPA}{%
  Maison de la Recherche Blaise Pascal \\
  50, rue Ferdinand Buisson            \\
  CS 80699                             \\
  62228 Calais Cedex                   \\
  France%
}
\end{preamblecode}
\end{dbexample}
%
\begin{dbremark}{Téléphone, fax et courriel : pour le
    laboratoire seulement}{}
  Spécifier les options \refKey{telephone}, \refKey{fax}, \refKey{email} et
  \refKey{nonamelink} pour un autre institut que le laboratoire est inutile :
  ces précisions complémentaires n'auront aucun effet.
\end{dbremark}
%
\begin{dbremark}{Instituts sous forme d'acronymes}{acronymes}
  \index{acronyme}%
  \indexsee{sigle}{acronyme}%
  Si l'institut ou l'entité doit figurer sous la forme d'un acronyme, on aura
  intérêt à ne pas les saisir tels quel comme on l'a fait jusqu'ici
  (\lstinline[deletekeywords={[5]institute}]|\institute{ULCO}| ou
  |\laboratory{LMPA}|) mais à recourir aux fonctionnalités du
  \Package{glossaries}. L'\vref{acronymes} donne un aperçu de la procédure.
\end{dbremark}

\subsection{Directeur(s) de thèse et membres du jury}\label{sec-jury}

Les directeur(s) et membres du jury de la thèse sont définis et précisés au
moyen des commandes et options listés dans cette section. Ils figurent
automatiquement sur la ou les pages de titre\footnote{En versions \enquote{à
    soumettre} aux rapporteurs (cf. valeurs \protect\docValue{submitted} et
  \protect\docValue{submitted*} de la clé \protect\refKey{version}), les
  membres du jury ne figurent pas car le doctorant ne peut alors préjuger d'un
  jury, ne sachant pas encore s'il va être autorisé à soutenir.}.

\subsubsection{Définition}\label{sec-definition-directeurs-jury}

\paragraph{Directeurs}\label{sec-definition-directeurs}
\index{rôle!prédéfini}%

Parmi la ou les personnes assurant l'encadrement de la thèse,
celles ayant les rôles :
\begin{itemize}
\item de directeur  ;
\item de co-directeur ;
\item de co-encadrant ;
\end{itemize}
sont distinguées au moyen des commandes respectives \refCom{supervisor},
\refCom{cosupervisor} et \refCom{comonitor}, en versions éventuellement étoilées
pour désigner celles qui ne sont pas membres du jury.%
%
\changes{v0.99f}{2014-07-11}{Nouvelles commandes \protect\refCom{supervisor*},
  \protect\refCom{cosupervisor*} et \protect\refCom{comonitor*} permettant de
  spécifier des directeurs de thèses non membres du jury}%
%
\begin{docCommand}[doc
  description=\mandatory]{supervisor}{\oarg{précision(s)}\marg{prénom}\marg{nom}}
  \index{caractéristiques de la thèse!liste!directeur de thèse}%
  \indexdef{directeur de thèse}%
  \index{membre du jury!directeur de thèse}%
  Cette commande définit un directeur de la thèse (également membre du jury).
\end{docCommand}

\begin{docCommand}[doc description=\mandatory]{supervisor*}{\oarg{précision(s)}\marg{prénom}\marg{nom}}
  \index{caractéristiques de la thèse!liste!directeur de thèse}%
  \indexdef{directeur de thèse}%
  Cette commande définit un directeur de la thèse (non membre du jury).
\end{docCommand}

\begin{docCommand}{cosupervisor}{\oarg{précision(s)}\marg{prénom}\marg{nom}}
  \index{caractéristiques de la thèse!liste!co-directeur de thèse}%
  \indexdef{co-directeur de thèse}%
  \index{membre du jury!co-directeur de thèse}%
  Cette commande définit un éventuel co-directeur de la thèse (également membre du jury).
\end{docCommand}

\begin{docCommand}{cosupervisor*}{\oarg{précision(s)}\marg{prénom}\marg{nom}}
  \index{caractéristiques de la thèse!liste!co-directeur de thèse}%
  \indexdef{co-directeur de thèse}%
  Cette commande définit un éventuel co-directeur de la thèse (non membre du jury).
\end{docCommand}

\begin{docCommand}{comonitor}{\oarg{précision(s)}\marg{prénom}\marg{nom}}
  \index{caractéristiques de la thèse!liste!co-encadrant de thèse}%
  \indexdef{co-encadrant de thèse}%
  \index{membre du jury!co-encadrant de thèse}%
  Cette commande définit un éventuel co-encadrant de la thèse (également membre du jury).
\end{docCommand}

\begin{docCommand}{comonitor*}{\oarg{précision(s)}\marg{prénom}\marg{nom}}
  \index{caractéristiques de la thèse!liste!co-encadrant de thèse}%
  \indexdef{co-encadrant de thèse}%
  Cette commande définit un éventuel co-encadrant de la thèse (non membre du jury).
\end{docCommand}

\paragraph{Membres du jury}\label{sec-definition-jury}
\index{membre du jury}%
\indexsee{jury}{membre du jury}%

\begin{docCommand}{referee}{\oarg{précision(s)}\marg{prénom}\marg{nom}}
  \index{caractéristiques de la thèse!liste!rapporteur}%
  \indexdef{rapporteur de la thèse}%
  \index{membre du jury!rapporteur}%
  Cette commande définit un rapporteur de la thèse.
\end{docCommand}

\begin{docCommand}{committeepresident}{\oarg{précision(s)}\marg{prénom}\marg{nom}}
  \index{caractéristiques de la thèse!liste!président du jury}%
  \indexdef{président du jury}%
  \index{membre du jury!président}%
  \index{rôle!prédéfini}%
  Cette commande définit le président du jury de la thèse (dont le rôle figure
  sur la ou les pages de titre).
\end{docCommand}

\begin{docCommand}{examiner}{\oarg{précision(s)}\marg{prénom}\marg{nom}}
  \index{caractéristiques de la thèse!liste!examinateur}%
  \indexdef{examinateur}%
  \index{membre du jury!examinateur}%
  Cette commande définit un examinateur ordinaire de la thèse.
\end{docCommand}

\begin{docCommand}{guest}{\oarg{précision(s)}\marg{prénom}\marg{nom}}
  \index{caractéristiques de la thèse!liste!invité}%
  \indexdef{invité}%
  \index{membre du jury!invité}%
  Cette commande définit une éventuelle personne invitée au jury de la thèse.
\end{docCommand}
%
\begin{dbwarning}{Usage multiple et facultatif des commandes du
    jury}{}
  Toutes ces commandes sont à utiliser :
  \begin{description}
  \item[autant de fois que nécessaire :]
    \refCom{referee} et \refCom{examiner} (par exemple) seront
    certainement employées à plusieurs reprises ;
  \item[seulement si nécessaire :]
    \refCom{cosupervisor}, \refCom{comonitor} et \refCom{guest} (par
    exemple) peuvent ne pas être employées.
  \end{description}
  La commande \refCom{supervisor} (ou sa variante étoilée), utilisable elle
  aussi plusieurs fois, doit être employée au moins une fois.
\end{dbwarning}

\begin{dbexample}{Jury}{}
  \indexex{membre du jury}%
\begin{preamblecode}[listing options={deletekeywords={[5]supervisor,cosupervisor,committeepresident}}]
\supervisor{Michel}{de Montaigne}
\cosupervisor{Étienne}{de la Boétie}
%
\referee{René}{Descartes}
\referee{Denis}{Diderot}
%
\committeepresident{Victor}{Hugo}
\examiner{Charles}{Baudelaire}
\examiner{Émile}{Zola}
\examiner{Paul}{Verlaine}
%
\guest{George}{Sand}
\end{preamblecode}
\end{dbexample}

\begin{dbwarning}{Format des prénoms et noms des directeurs de thèse et membres du jury}{}
  \index{directeur de thèse!format}%
  \index{membre du jury!format}%
  \index{nom!format}%
  \index{prénom!format}%
  \index{format!nom}%
  \index{format!prénom}%
  Comme pour les prénom et nom de l'auteur de la thèse, on veillera à ce que :
  \begin{enumerate}
  \item les éventuels accents figurent dans les \meta{prénom} et \meta{nom};
  \item les \meta{nom} \emph{ne} soient \emph{pas} saisis en capitales (sauf
    pour la ou les majuscules) car ils seront automatiquement composés en
    petites capitales.
  \end{enumerate}
\end{dbwarning}

\subsubsection{Précisions}\label{sec-options-staff}

Toutes les commandes précédentes admettent un argument optionnel permettant
d'apporter sur les directeurs de thèse et membres du jury\footnote{Pour les
  directeurs de thèse \emph{non} membres du jury, ces \meta{précisions} sont
  inutiles car elles ne figureront nulle part.}
des \meta{précisions} :
\begin{enumerate}
\item corporation ;
\item affiliation ;
\item homme/femme.
\end{enumerate}
% --- sous la forme d'une liste \meta{clé}|=|\meta{valeur}.

\paragraph{Corporation}
\label{sec-corps}%
\indexsee{corps de métier}{corporation}%
\index{caractéristiques de la thèse!liste!corporation}%
\indexsee{fonction}{corporation}%
\indexsee{grade}{corporation}%
\index{corporation}%
\index{corporation!prédéfinie}%
\index{directeur de thèse!corporation}%
\index{membre du jury!corporation}%

Les clés suivantes\syntaxeoptions{} permettent de spécifier les corporations (ou
corps de métier) des membres du jury parmi celles prédéfinies par la \yatCl{}.

\begin{docKey}{professor}{=\docValue{true}\textbar\docValue{false}}{par défaut
    \docValue{true}, initialement \docValue{false}}
  \index{corporation!prédéfinie!professeur}%
  \indexsee{professeur}{corporation}%
  Cette clé permet de spécifier qu'une personne appartient à la corporation des
  professeurs d'université.
\end{docKey}
%
\begin{docKey}{seniorresearcher}{=\docValue{true}\textbar\docValue{false}}{par
    défaut \docValue{true}, initialement \docValue{false}}
  \index{corporation!prédéfinie!directeur de recherche}%
  \indexsee{directeur de recherche}{corporation}%
  Cette clé permet de spécifier qu'une personne appartient à la corporation des
  directeurs de recherche du \gls{cnrs}.
\end{docKey}
%
\begin{docKey}[][doc updated=2016-10-24]{associateprofessor}{=\docValue{true}\textbar\docValue{false}}{par défaut
    \docValue{true}, initialement \docValue{false}}
  \index{corporation!prédéfinie!\acrshort{mcf} (non) \acrshort{hdrpeople}}%
  \indexsee{\acrshort{mcf} (non) \acrshort{hdrpeople}}{corporation}%
  \changes{v0.99o}{2016-10-24}{Clés \protect\refAux{mcf} et
    \protect\refAux{mcf*} remplacées par les (et alias des) clés
    \protect\refKey{associateprofessor} et
    \protect\refKey{associateprofessor*}}%
  \changes*{v0.99o}{2016-10-24}{\acrshort{mcf} désormais traduit en anglais par
    \enquote{\foreignlanguage{english}{\translate{associateprofessor}}}}%
  Cette clé permet de spécifier qu'une personne appartient à la corporation des
  \glspl{mcf}\footnote{Par souci de compatibilité ascendante, la clé désormais
    obsolète \refAux{mcf} est un alias de la clé \refKey{associateprofessor}.}.
\end{docKey}
%
\begin{docKey}[][doc updated=2016-10-24]{associateprofessor*}{=\docValue{true}\textbar\docValue{false}}{par défaut
    \docValue{true}, initialement \docValue{false}}
  \index{corporation!prédéfinie!\acrshort{mcf} (non) \acrshort{hdrpeople}}%
  \indexsee{\acrshort{mcf} (non) \acrshort{hdrpeople}}{corporation}%
  \indexsee{\acrshort{hdrpeople} (habilité)}{corporation}%
  Cette clé permet de spécifier qu'une personne appartient à la corporation des
  \glspl{mcf} \acrshort{hdrpeople}\footnote{Par souci de compatibilité
    ascendante, la clé désormais obsolète \refAux{mcf*} est un alias de la clé
    \refKey{associateprofessor*}.}.
\end{docKey}
%
\begin{docKey}{juniorresearcher}{=\docValue{true}\textbar\docValue{false}}{par
    défaut \docValue{true}, initialement \docValue{false}}
  \index{corporation!prédéfinie!chargé de recherche (non) \acrshort{hdrpeople}}%
  \indexsee{chargé de recherche (non) \acrshort{hdrpeople}}{corporation}%
  Cette clé permet de spécifier qu'une personne appartient à la corporation des
  \glspl{cr} du \gls{cnrs}.
\end{docKey}
%
\begin{docKey}{juniorresearcher*}{=\docValue{true}\textbar\docValue{false}}{par
    défaut \docValue{true}, initialement \docValue{false}}
  \index{corporation!prédéfinie!chargé de recherche (non) \acrshort{hdrpeople}}%
  \indexsee{chargé de recherche (non) \acrshort{hdrpeople}}{corporation}%
  \indexsee{\acrshort{hdrpeople} (habilité)}{corporation}%
  Cette clé permet de spécifier qu'une personne appartient à la corporation des
  \glspl{cr} \acrshort{hdrpeople} du \gls{cnrs}.
\end{docKey}
%
\begin{dbexample}{Corporation (prédéfinies)}{}
  \indexex{corporation!prédéfinie}%
\begin{preamblecode}[listing options={deletekeywords={[5]supervisor,cosupervisor,committeepresident}}]
\supervisor[professor]{Michel}{de Montaigne}
\cosupervisor[juniorresearcher*]{Étienne}{de la Boétie}
%
\referee{René}{Descartes}
\referee[seniorresearcher]{Denis}{Diderot}
%
\committeepresident[professor]{Victor}{Hugo}
\examiner[associateprofessor*]{Charles}{Baudelaire}
\examiner[professor]{Émile}{Zola}
\examiner{Paul}{Verlaine}
\end{preamblecode}
\end{dbexample}
%
\begin{dbremark}{Corporation non prédéfinies}{}
  \index{corporation!non prédéfinie}%
  Il est possible de spécifier d'autres corporations que celles prédéfinies
  ci-dessus. La \vref{sec-nouveaux-corps} explique comment procéder.
\end{dbremark}
%
% \begin{docKey}{distinction}{=\meta{distinction}}{pas de valeur par
%     défaut, initialement vide}
%   Cette clé définit une distinction, par exemple un prix, à faire
%   apparaître sur la page de titre en français.
% \end{docKey}
% %
% \begin{docKey}{award}{=\meta{distinction}}{pas de valeur par défaut,
%     initialement vide}
%   Cette clé définit une distinction, par exemple un prix, à faire
%   apparaître sur la page de titre en anglais.
% \end{docKey}
%
% \begin{dbexample}{Distinctions}{}
%   \begin{preamblecode}[title=Préparation du titre (p. ex. dans le \File{\characteristicsfile})]
% \cosupervisor[distinction=prix Nobel,award=Nobel Price]{Étienne}{de la Boétie}
% \referee[distinction=médaille Fields,award=Fields Medal]{René}{Descartes}
% \end{preamblecode}
% \end{dbexample}

\paragraph{Affiliation}
\label{sec-inst-de-prov}%
\index{affiliation}%
\index{caractéristiques de la thèse!liste!affiliation}%

\begin{docKey}{affiliation}{=\meta{institut}}{pas de valeur par défaut,
    initialement vide}
  \indexdef{affiliation}%
  \index{directeur de thèse!affiliation}%
  \index{membre du jury!affiliation}%
  Cette clé définit l'\meta{institut}%
  \footnote{La \vref{rq-acronymes} s'applique également ici : plutôt que
    spécifié tel quel, l'acronyme d'un \meta{institut} peut être géré par le
    \Package{glossaries}.}  auquel est affilié un membre du jury.
\end{docKey}
\begin{dbexample}{Institut d'affiliation}{}
\begin{preamblecode}[listing options={deletekeywords={[5]supervisor}}]
\supervisor[affiliation=ULCO]{Michel}{de Montaigne}
\end{preamblecode}
\end{dbexample}
%
\begin{dbwarning}{Virgule(s) dans les valeurs des clés}{virgule}
  Dans toute option de la forme \meta{clé}|=|\meta{valeur}, si \meta{valeur}
  contient une ou plusieurs virgules, il faut \emph{impérativement} la placer
  entre paire d'accolades ainsi : \meta{clé}|={|\meta{valeur}|}|. Cela peut
  notamment être le cas de la \meta{valeur} de la clé \refKey{affiliation}.
\end{dbwarning}
%
\begin{dbexample}{Multiples instituts d'affiliation}{}
  \index{affiliation!multiple}%
  Si en plus d'être affilié à l'\gls{ulco}, René Descartes était membre du
  \gls{cnrs}, on pourait procéder comme suit :
\begin{preamblecode}
\referee[affiliation={ULCO, CNRS}]{René}{Descartes}
\end{preamblecode}
On notera la paire d'accolades, nécessaire conformément
à l'\vref{wa-virgule}. De façon générale, il n'est pas indispensable de faire
figurer tant de précisions et, ne serait-ce que pour des raisons de place, on
veillera à ne pas multiplier celles-ci.
\end{dbexample}

\paragraph{Homme/femme}
\label{sec-hommefemme}%
\index{caractéristiques de la thèse!liste!homme}%
\index{caractéristiques de la thèse!liste!femme}%

\changes{v0.99f}{2014-07-11}{Nouvelles clés \protect\refKey{male} et
  \protect\refKey{female} permettant de spécifier si une personne est un homme
  ou une femme}%
%
Par défaut, les directeurs de thèse et membres du jury sont supposés être des
hommes\footnote{Je promets d'envisager mon adhésion au \acrshort{mlf} pour une
  supposition aussi sexiste !}, si bien qu'un certain nombre de mots clés de la
\yatCl{} sont de genre masculin (\enquote{directeur}, \enquote{chargé de
  recherche}, etc.). Il est possible de spécifier qu'un directeur de thèse ou
un membre du jury est un homme ou une femme au moyen des clés \refKey{male} et
\refKey{female} suivantes.

\begin{docKey}{male}{=\docValue{true}\textbar\docValue{false}}{par défaut
    \docValue{true}, initialement \docValue{true}}
  \index{homme}%
  Cette clé permet de spécifier qu'une personne est ou pas de sexe masculin.
\end{docKey}

\begin{docKey}{female}{=\docValue{true}\textbar\docValue{false}}{par défaut
    \docValue{true}, initialement \docValue{false}}
  \index{femme}%
  Cette clé permet de spécifier qu'une personne est ou pas de sexe féminin.
\end{docKey}

L'option |female|\footnote{Ou, de façons équivalentes,
  \protect\lstinline+male=false+ ou \protect\lstinline+female=true+.} n'a pour
effet que d'accorder en genre féminin un certain nombre de mots clés de la
\yatCl{} (\enquote{directrice} au lieu \enquote{directeur}, \enquote{chargée de
  recherche} au lieu \enquote{chargé de recherche}, etc.).

\begin{dbexample}{Directrice de thèse}{}
  \indexex{femme}%
  S'il est demandé que le rôle de Sophie \textsc{Germain}, directrice de thèse,
  soit accordé en genre (\enquote{directrice} et non \enquote{directeur}), il
  suffit de saisir :
\begin{preamblecode}[listing options={deletekeywords={[5]supervisor}}]
\supervisor[female]{Sophie}{Germain}
\end{preamblecode}
\end{dbexample}

\subsection{Numéro d'ordre}
\label{sec-numero-dordre}

Certains instituts exigent que le numéro d'ordre de la thèse figure sur la page
de 1\iere{} de couverture.

\begin{docCommand}{ordernumber}{\oarg{numéro d'ordre}}
  \index{caractéristiques de la thèse!liste!numéro d'ordre}%
  \indexdef{numéro d'ordre}%
  Cette commande définit le \meta{numéro d'ordre} de la thèse et s'utilise sans
  son argument optionnel si on ne connaît pas \aside{encore} le \meta{numéro
    d'ordre} : ce dernier est alors remplacé par une espace horizontale vide
  permettant de l'inscrire à la main \emph{a posteriori}. Vide ou pas, le
  \meta{numéro d'ordre} figure sur \aside{et seulement sur} la 1\iere{} page du
  mémoire\footnote{Première de couverture s'il y a, page de titre en langue
    principale sinon.}, précédé de l'expression
  \translateexpression{ordernumber}.
\end{docCommand}

\section{Caractéristiques de mots clés}
\label{sec-proprietes-de-mots}

Les mots clés de la thèse sont stipulés au moyen de la commande
\refCom{keywords} suivante.
%
\begin{docCommand}[doc description=\mandatory]{keywords}{\marg{mots clés}\marg{mots clés dans la langue
      secondaire}}
  \index{caractéristiques de la thèse!liste!mot clé}%
  \indexdef{mot clé}%
  Cette commande définit les \meta{mots clés} de la thèse dans
  les langues principale et secondaire. Ceux-ci :
  \begin{itemize}
  \item apparaissent comme métadonnée \enquote{Mots-clés} du fichier \pdf{} ;
  \item figurent, dans les deux langues principale et secondaire, précédés des
    expressions \translateexpression{keywords} :
    \begin{itemize}
    \item sur la page qui leur est dédiée (si la commande \refCom{makekeywords}
      est employée) ;
    \item sur la page dédiée au(x) résumé(s) de la thèse générée par la
      commande \refCom{makeabstract} ;
    \item sur la 4\ieme{} de couverture (si la commande \refCom{makebackcover}
      est employée).
    \end{itemize}
  \end{itemize}
\end{docCommand}

%%% Local Variables:
%%% mode: latex
%%% TeX-master: "../yathesis-fr"
%%% End:
