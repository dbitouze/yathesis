\chapter{Partie liminaire}\label{cha-liminaires}
\index{liminaire}%
\indexsee{préliminaire}{liminaire}%
\index{partie!liminaire}%

La \gls{liminaire} de la thèse comprend :
\begin{enumerate}
\item la page (éventuelle) de clause de non-responsabilité ;
\item la page (éventuelle) des mots clés de la thèse ;
\item la page (éventuelle) du ou des laboratoires où a été préparée la thèse ;
\item la page (éventuelle) des dédicaces ;
\item la page (éventuelle) des épigraphes ;
\item la page de résumés dans les langues principale et secondaire ;
\item les (éventuels) avertissement, remerciements, résumé substantiel en
  français, avant-propos, etc.
\item la ou les listes (éventuelles), commune ou distinctes :
  \begin{itemize}
  \item des sigles et acronymes\footnote{Par commodité, nous ne parlerons plus
      dans la suite que d'acronymes mais ce qui les concernera s'appliquera de
      façon identique aux sigles.} ;
  \item des symboles ;
  \item des termes du glossaire ;
  \end{itemize}
\item le sommaire ou la table des matières ;
\item la liste (éventuelle) des tableaux ;
\item la liste (éventuelle) des figures ;
\item la liste (éventuelle) des listings informatiques.
\end{enumerate}

\begin{dbremark}{Commande \protect\docAuxCommand*{frontmatter} non nécessaire}{nofrontmatter}
  La commande \docAuxCommand{frontmatter} usuelle de la \Class{book}, employée
  habituellement pour entamer la partie liminaire du document, n'est pas
  nécessaire car la \yatCl{} la charge déjà en sous-main. On verra plus loin
  que, au contraire, la commande analogue \refCom{mainmatter} doit être
  explicitement employée pour entamer la partie principale du document (il en
  est de même des commandes \refCom{appendix} et \refCom{backmatter} pour les
  éventuelles parties annexe et finale).
\end{dbremark}

\section{Clause de non-responsabilité}
\label{sec-clause-de-non}%
\index{clause de non-responsabilité}%

\changes{v0.99d}{2014-06-08}{Élision \enquote{automatique} des articles définis
  précédant \meta{institut} et \meta{co-institut} dans la clause de
  non-responsabilité}%
%
La \yatCl{} permet de faire figurer une clause de non-responsabilité, telle
qu'exigée par certains instituts. Celle-ci apparaît sur une page dédiée et
a pour contenu par défaut une phrase semblable à\selonlangue{} :
  \begin{itemize}
  \item \enquote{L'\meta{institut} n'entend donner aucune
      approbation ni improbation aux opinions \'emises dans les th\`eses : ces
      opinions devront \^etre consid\'er\'ees comme propres \`a leurs auteurs.}
  \item \foreignquote{english}{The \meta{institut} neither endorse
      nor censure authors' opinions expressed in the theses: these opinions
      must be considered to be those of their authors.}
  \end{itemize}
  où l'\meta{institut} est celui défini par la commande \refCom{institute}
  \aside*{auquel est adjoint l'éventuel institut de cotutelle}.

La page dédiée à la clause de non-responsabilité est produite par la commande
\refCom{makedisclaimer}.

\begin{docCommand}{makedisclaimer}{}
  Cette commande produit une page où figure, seule et centrée
  verticalement, la clause de non-responsabilité.
\end{docCommand}

\begin{docCommand}{makedisclaimer*}{}
  Cette commande a le même effet que la commande
  \refCom{makedisclaimer} sauf que la clause de non-responsabilité est alignée
  sur le haut de la page et non centrée verticalement.
\end{docCommand}

\begin{dbexample}{Production de la page dédiée à la clause de
    non-responsabilité}{}
  \indexex{clause de non-responsabilité}%
  \NoAutoSpacing%
%
\bodysample{rangesuffix=\^^M,linerange={makedisclaimer}}{}%
  Le résultat de ce code est illustré \vref{fig-disclaimerpage}.
\end{dbexample}

\begin{figure}[htbp]
  \centering
  \screenshot{disclaimer}%
  \caption{Page de clause de non-responsabilité}
  \label{fig-disclaimerpage}
\end{figure}

\begin{dbwarning}{Élision automatique non robuste}{elision-disclaimer}
  Dans la clause de non-responsabilité, l'article défini précédant
  \meta{institut} est automatiquement élidé selon l'initiale (voyelle ou
  consonne) du mot suivant. Cette élision automatique n'est donc pas robuste :
  elle peut ne pas donner le résultat escompté si \meta{institut} a pour
  initiale :
  \begin{itemize}
  \item une consonne, mais est de genre féminin ;
  \item une voyelle, mais par le truchement d'une commande\commandeacronyme, et
    non pas \enquote{directement}.
  \end{itemize}
\end{dbwarning}

Pour pallier cet inconvénient, et aussi pour permettre de redéfinir la phrase
par défaut si elle ne convient pas, on pourra recourir à la commande
\refCom{disclaimer}.

\begin{docCommand}{disclaimer}{\marg{clause}}
  \index{clause de non-responsabilité!modification}%
  Cette commande, à placer avant \refCom{makedisclaimer}, permet de redéfinir
  le contenu par défaut de la \meta{clause} de non-responsabilité.
\end{docCommand}

\section{Mots clés}\label{sec-mots-cles}

\begin{docCommand}{makekeywords}{}
  \indexdef{mot clé}%
  Cette commande produit une page où figurent, seuls et centrés
  verticalement, les mots clés de la thèse stipulés au moyen de la commande
  \refCom{keywords}.
\end{docCommand}
%
\begin{docCommand}{makekeywords*}{}
  \indexdef{mot clé}%
  Cette commande a le même effet que la commande
  \refCom{makekeywords} sauf que les mots clés sont alignés sur le haut de la
  page et non centrés verticalement.
\end{docCommand}

\begin{dbexample}{Préparation et production de la page dédiée aux mots clés}{}
  \indexex{mot clé}%
  Les codes suivants produisent la page illustrée \vref{fig-makekeywords}.
  \preamblesample[configuration/characteristics]{%
    linerange={keywords-laugh},%
    deletekeywords={[1]{keywords}},
    deletekeywords={[5]{keywords}}%
  }{title=Préparation}
%
  \bodysample{rangesuffix=\^^M,linerange={makekeywords}}{title=Production}
\end{dbexample}

\begin{figure}[htbp]
  \centering
  \screenshot{keywords}%
  \caption{Page dédiée aux mots clés}
  \label{fig-makekeywords}
\end{figure}

\section{Laboratoire(s)}
\label{sec-laboratoires}

\begin{docCommand}{makelaboratory}{}
  \indexdef{laboratoire}%
  Cette commande produit une page où figure, seul(s) et centré(s)
  verticalement, le ou les laboratoires où a été préparée la thèse, stipulés au
  moyen de la commande \refCom{laboratory} et éventuellement précisés au moyen
  des clés \refKey{logo}, \refKey{logoheight}, \refKey{telephone},
  \refKey{fax}, \refKey{email} et \refKey{nonamelink}.
\end{docCommand}
%
\begin{docCommand}{makelaboratory*}{}
  \index{laboratoire}%
  Cette commande a le même effet que la commande \refCom{makelaboratory} sauf
  que le ou les laboratoires sont alignés sur le haut de la page et non centrés
  verticalement.
\end{docCommand}

\begin{dbexample}{Préparation et production de la page dédiée au(x) laboratoire(s)}{}
  \indexex{laboratoire}%
  Les codes suivants produisent la page illustrée \vref{fig-makelaboratory}.
  \NoAutoSpacing%
  \preamblesample[configuration/characteristics]{%
    deletekeywords={url},%
    morekeywords={[2]url},%
    linerange={laboratory-Liouville}%
  }{title=Préparation}
%
  \bodysample{rangesuffix=\^^M,linerange={makelaboratory}}{title=Production}
\end{dbexample}

\begin{figure}[htbp]
  \centering
  \screenshot{laboratory}
  \caption{Page dédiée au(x) laboratoire(s)}
  \label{fig-makelaboratory}
\end{figure}

\section{Dédicaces}

\begin{docCommand}{dedication}{\marg{dédicace}}
  \indexdef{dédicace}%
  Cette commande, à employer autant de fois que
  souhaité\hauteurpage{}, permet de préparer une dédicace.
\end{docCommand}

\begin{docCommand}{makededications}{}
  \index{dédicace}%
  Cette commande produit une page où figurent, seules, alignées à droite et
  centrées verticalement, la ou les dédicaces stipulées au moyen de la commande
  \refCom{dedication}.
\end{docCommand}
%
\begin{docCommand}{makededications*}{}
  \index{dédicace}%
  Cette commande a le même effet que la commande \refCom{makededications} sauf
  que la ou les dédicaces sont alignées sur le haut de la page et non centrées
  verticalement.
\end{docCommand}

\begin{dbexample}{Préparation et production de la page dédiée aux dédicaces}{}
  \indexex{dédicace}%
  \NoAutoSpacing%
  \preamblesample[liminaires/dedicaces]{linerange=dedication-ritent}{title=Préparation}
%
  \bodysample[liminaires/dedicaces]{rangesuffix=\^^M,linerange={makededications}}{title=Production}
  Le résultat de ce code est illustré \vref{fig-dedicationspage}.
\end{dbexample}

\begin{figure}[htbp]
  \centering
  \screenshot{dedications}%
  \caption{Page de dédicaces}
  \label{fig-dedicationspage}
\end{figure}

\section{Épigraphes liminaires}

\begin{docCommand}{frontepigraph}{\oarg{langue}\marg{épigraphe}\marg{auteur}}
  \indexdef{épigraphe}%
  Cette commande, à employer autant de fois que souhaité\hauteurpage{}, permet
  de préparer une épigraphe destinée à apparaître sur une page dédiée de la
  \gls{liminaire}.

  Si l'épigraphe est exprimée dans une \meta{langue} \aside{connue du
    \Package{babel}} autre que la langue principale du document, on peut le
  spécifier en argument optionnel%
  \footnote{Si cette \meta{langue} est autre que le français ou l'anglais, elle
    doit être explicitement chargée en option de la commande
    \docAuxCommand{documentclass} (cf.  \vref{rq-languessupplementaires}).}.
\end{docCommand}

\begin{docCommand}{makefrontepigraphs}{}
  \index{épigraphe}%
  Cette commande produit une page où figurent, seules, alignées à droite et
  centrées verticalement, la ou les épigraphes stipulées au moyen de la
  commande \refCom{frontepigraph}.
\end{docCommand}
%
\begin{docCommand}{makefrontepigraphs*}{}
  \index{épigraphe}%
  Cette commande a le même effet que la commande \refCom{makefrontepigraphs}
  sauf que la ou les épigraphes sont alignées sur le haut de la page et non
  centrées verticalement.
\end{docCommand}

\begin{dbexample}{Préparation et production de la page dédiée aux épigraphes
    liminaires}{}
  \indexex{épigraphe}%
  \NoAutoSpacing%
  Les codes suivants produisent la page illustrée \vref{fig-epigraphspage}.
  \preamblesample[liminaires/epigraphes]{linerange={frontepigraph-Einstein}}{title=Préparation}
  %
  \bodysample[liminaires/epigraphes]{rangesuffix=\^^M,linerange={makefrontepigraphs}}{title=Production}
\end{dbexample}

\begin{figure}[htbp]
  \centering
  \screenshot{frontepigraphs}
  \caption{Page d'épigraphes liminaires}
  \label{fig-epigraphspage}
\end{figure}

\begin{dbremark}{Épigraphes ailleurs dans le document}{}
  Pour gérer les épigraphes liminaires, la \yatCl{} exploite le
  \Package*{epigraph} \aside*{qui est automatiquement chargé}. Il est bien sûr
  possible de recourir aux commandes de ce package pour faire figurer, ailleurs
  dans le mémoire, d'autres épigraphes.
\end{dbremark}

\section{Avertissement, remerciements, résumé substantiel, avant-propos, etc.}
\index{avertissement}%
\index{remerciements}%
\index{résumé}%
\index{avant-propos}%

La \gls{liminaire} d'un mémoire de thèse peut contenir un avertissement, des
remerciements, un résumé substantiel en français (cf. \vref{wa-frenchabstract}),
un avant-propos, etc.  à considérer et à composer comme des chapitres
\enquote{ordinaires}.

\begin{dbwarning}{Chapitres \enquote{ordinaires} de la \gls{liminaire}
    automatiquement \emph{non} numérotés}{}
  \index{chapitre!non numéroté}%
  Les chapitres \enquote{ordinaires} de la \gls{liminaire} doivent être
  introduits au moyen de la commande usuelle \docAuxCommand{chapter}, sous sa
  forme \emph{non} étoilée : puisqu'ils seront situés dans la partie liminaire
  du mémoire, ces chapitres seront automatiquement \emph{non} numérotés.
\end{dbwarning}

%\begin{dbremark}{\protect\Glspl{titrecourant} des chapitres de la
%  \protect\gls{liminaire}}{titrecourant}%
%  Les chapitres \enquote{ordinaires} sont pourvus de \glspl{titrecourant}
%  si (et seulement si) ils figurent après la page dédiée aux résumés
%  (cf. \vref{sec-abstract}).
%\end{dbremark}

\section{Résumés succincts en français et en anglais}\label{sec-abstract}

Une page contenant de courts résumés en français et en anglais est requise.
L'environnement \refEnv{abstract} suivant permet de préparer une telle
page.
%
\begin{docEnvironment}[doclang/environment content=résumé,doc description=\mandatory]{abstract}{\oarg{titre alternatif}}
  \indexdef{résumé}%
  \index{résumé!en français}%
  \index{résumé!en anglais}%
  Cet environnement, destiné à recevoir le ou les résumés de la thèse, est
  conçu pour être employé une ou deux fois :
  \begin{enumerate}
  \item sa 1\iere{} occurrence doit contenir le résumé dans la langue
    principale ;
  \item sa 2\ieme{} occurrence, si présente, doit contenir le résumé dans la
    langue secondaire.
  \end{enumerate}
  Ces résumés figurent, dans les langues principale et secondaire :
  \begin{itemize}
  \item sur la page dédiée au(x) résumé(s) de la thèse produite par la commande
    \refCom{makeabstract} ;
  \item sur la 4\ieme{} de couverture si la commande \refCom{makebackcover} est
    employée.
  \end{itemize}
  Ils \index{nom!résumé}%
  sont respectivement intitulés \enquote{\abstractname} ou
  \enquote{\selectlanguage{english}\abstractname}\selonlangueshort{} mais
  l'argument optionnel permet de spécifier un \meta{titre} (ou \meta{nom}
    \meta{alternatif}\redefexprcle.
\end{docEnvironment}

\begin{docCommand}[doc description=\mandatory]{makeabstract}{}
  \index{résumé}%
  Cette commande produit une page dédiée aux résumés en y faisant
  apparaître automatiquement :
  \begin{enumerate}
  \item dans les langues principale et secondaire :
    \begin{itemize}
    \item les titre, éventuel sous-titre et mots clés de la thèse, stipulés au
      moyen des commandes respectives \refCom{title}, \refCom{subtitle} et
      \refCom{keywords} ;
    \item les résumés saisis au moyen de l'environnement \refEnv{abstract} ;
    \end{itemize}
  \item le nom et l'adresse du laboratoire (principal)\footnote{Il est possible
      de faire figurer sur les pages de résumés et de 4\ieme{} de couverture un
      nombre arbitraire de laboratoires au moyen de la clé
      \refKey{numlaboratories}.} dans lequel la thèse a été préparée, stipulés
    au moyen de la commande \refCom{laboratory}.
  \end{enumerate}
\end{docCommand}

\begin{dbexample}{Préparation et production de la page dédiée aux résumés}{}
  \indexex{résumé}%
  Les codes suivants produisent la page illustrée \vref{fig-resumes-succincts}.
% \preamblesample{%
%   includerangemarker=false,%
%   rangebeginprefix={»).\^^M},%
%   rangeendsuffix={\^^M\%\ Page},%
%   linerange={\\begin\{abstract-\\end\{abstract\}}%
% }{title=Préparation des résumés}
\begin{bodycode}
\begin{abstract}
  \lipsum[1-2]
\end{abstract}
\begin{abstract}
  \lipsum[3-4]
\end{abstract}
\end{bodycode}
  %
  \bodysample[liminaires/resumes]{rangesuffix=\^^M,linerange={makeabstract}}{title=Production
    des résumés}
\end{dbexample}

\begin{figure}[htbp]
  \centering
  \screenshot{abstract}%
  \caption{Page de résumés succincts en français et en anglais}
  \label{fig-resumes-succincts}
\end{figure}

\begin{dbwarning}{Résumés nécessairement courts dans l'environnement
    \protect\lstinline+abstract+}{}
  L'environnement \refEnv{abstract} est prévu pour des résumés courts, leurs
  versions dans les langues principale et secondaire devant tenir l'une sous
  l'autre sur une seule et même page. Cette limitation est en phase avec les
  recommandations du ministère stipulant que ces résumés doivent chacun
  contenir au maximum 1700~caractères, espaces compris\footnote{En cas de
    débordement sur plus d'une page, on pourra toujours recourir à un
    changement local de taille des caractères.}.
\end{dbwarning}

\begin{dbwarning}{Résumé en français nécessaire en cas de mémoire en langue
    étrangère}{frenchabstract}
  Un mémoire composé principalement en langue étrangère \aside{notamment dans
    le cadre d'une cotutelle internationale} requiert, en sus de la page de
  résumé(s) ci-dessus, un résumé \emph{en français} de la thèse. Celui-ci doit
  être \emph{substantiel}, d'une dizaine de pages environ.
\end{dbwarning}

\section{Liste d'acronymes, liste de symboles,
  glossaire}\label{sec-sigl-gloss-nomencl}
\index{acronyme!liste d'---s}%
\index{symbole!liste de ---s}%
\index{glossaire}%

\begin{dbremark*}{Section à passer en 1\iere{} lecture}
  Cette section est à passer en 1\iere{} lecture si on ne compte faire figurer
  ni listes d'acronymes, ni listes de symboles, ni glossaire.
\end{dbremark*}

Tout système de gestion de glossaire peut théoriquement être mis en œuvre avec
la \yatCl. Cependant, celle-ci fournit des fonctionnalités propres au
\Package{glossaries}\footnote{Dans ses versions à partir de la \texttt{4.0} en
  date du \formatdate{14}{11}{2013}. Dans cette section, le fonctionnement de
  ce package est supposé connu du lecteur (sinon, cf. par exemple
  \cite{en-ligne7}).} :
\begin{itemize}
\item une commande \refCom{newglssymbol}, destinée à faciliter la définition de
  symboles dans la base terminologique ;
\item un style de glossaire \docValue{yadsymbolstyle}, destiné à composer la
  liste des symboles sous forme de \enquote{nomenclature} (dans l'esprit du
  \Package*{nomencl}).
\end{itemize}

\begin{dbwarning}{Package \package{glossaries} non chargé par défaut}{}
  Le \Package{glossaries} \emph{n'étant pas} chargé par la \yatCl, on veillera
  à le charger manuellement si on souhaite l'utiliser.
\end{dbwarning}

% \begin{enumerate}
% \item les commandes de ce package produisant les liste des termes du ou des
%   glossaires sont légèrement modifiées (un style de pages propre leur étant
%   appliqué) :
%   \begin{itemize}
%   \item \docAuxCommand{printglossary} et \docAuxCommand{printglossaries} qui
%     produisent la liste des termes du ou des glossaires\termesdefinisutilises{}
%     (cf. \vref{fig-printglossary}) ;
%   \item \docAuxCommand{printacronyms}\footnote{L'usage de la commande
%       \docAuxCommand{printacronyms} nécessite que l'option \docAuxKey{acronyms}
%       soit passée au \Package{glossaries}.} qui produit la liste des
%     acronymes\termesdefinisutilises{} (cf. \vref{fig-printacronyms}) ;
%   \end{itemize}
% \item les commande et style propres \refCom{newglssymbol}, et
%   \docValue{yadsymbolstyle}, précisés ci-dessous, sont définis.
% \end{enumerate}

\begin{docCommand}{newglssymbol}{\oarg{classement}\marg{label}\marg{symbole}\marg{nom}\marg{description}}
  \indexdef{symbole}%
  Cette commande définit un symbole au moyen :
  \begin{itemize}
  \item de son \meta{label}\footnote{Ce \meta{label}, qui identifie le symbole de
      manière unique dans la base terminologique, est notamment utilisé dans
      les commandes qui produisent celui-ci dans le texte \aside*{par exemple
      \docAuxCommand{gls}\marg{label}}.} ;
\item du \meta{symbole} proprement dit\footnote{Ce symbole peut notamment être
    composé au moyen de la commande \docAuxCommand{ensuremath}\marg{symbole
      mathématique} ou de la commande \docAuxCommand{si}\marg{commande d'unité}
    du \Package*{siunitx} (à charger).} ;
  \item de son \meta{nom} ;
  \item de sa \meta{description}.
  \end{itemize}
  Dans la liste des symboles produite par la commande \refCom{printsymbols}, un
  symbole est par défaut classé selon l'ordre alphabétique de son \meta{label}
  mais peut optionnellement l'être selon celui d'une autre chaîne de
  \meta{classement}.
\end{docCommand}

\begin{dbwarning}{Option \texttt{symbols} nécessitée par la commande
    \protect\refCom*{newglssymbol}}{}
  L'usage de la commande \refCom{newglssymbol} nécessite que l'option
  \docAuxKey{symbols} soit passée au \Package{glossaries}.
\end{dbwarning}

\begin{docCommand}{printsymbols}{\oarg{options}}
  \index{symbole!liste de ---s}%
  Cette commande, fournie par le \Package{glossaries}, produit la liste des
  symboles saisies (par exemple) au moyen de la \refCom{newglssymbol}. Mais
  elle a été légèrement redéfinie, sa clé \refKey{style} ayant pour valeur par
  défaut \docValue{yadsymbolstyle} (et non \docValue{list}) :
  \begin{docKey}{style}{=\docValue{yadsymbolstyle}\textbar\meta{style}}{pas de valeur
      par défaut, initialement \docValue{yadsymbolstyle}}
    Cette clé permet de spécifier le style appliqué à la liste des
    symboles. Tout \meta{style} spécifié, autre que \docValue{yadsymbolstyle},
    doit être l'un de ceux acceptés par la clé \refKey{style} du
    \Package{glossaries}.
  \end{docKey}
\end{docCommand}

\begin{dbexample}{Définitions et liste des symboles}{}
  \indexex{symbole}%
  Le code suivant définit certains symboles.
  \preamblesample[auxiliaires/symboles.tex]{}{}%
  Le code suivant produit la liste de ces symboles \aside*{composée avec le
    style \docValue{yadsymbolstyle}}.
  \bodysample{rangesuffix=\^^M,linerange={printsymbols}}{} Le résultat de ce
  code est illustré \vref{fig-printsymbols}.
\end{dbexample}

% \afterpage{%
\begin{landscape}
  \begin{figure}[p]
    \centering
    \begin{subfigure}[b]{.45\linewidth}
      \centering
      \screenshot[1]{printacronyms}
      \caption{Acronymes}
      \label{fig-printacronyms}
    \end{subfigure}%
    \hspace{\stretch{1}}%
    \begin{subfigure}[b]{.45\linewidth}
      \centering
      \screenshot[1]{printsymbols}
      \caption{Symboles}
      \label{fig-printsymbols}
    \end{subfigure}%
    \caption{Listes des acronymes et des symboles}
    \label{fig-printacronyms-printsymbols}
  \end{figure}
\end{landscape}
% }

% Si on souhaite faire figurer :
% \begin{enumerate}
% \item une liste \emph{commune} des acronymes et des termes du glossaire, on
%   chargera \package{glossaries} \emph{sans} l'option |acronym| et on recourra à
%   la commande \docAuxCommand{printglossary} ;
% \item deux listes \emph{distinctes}, on chargera \package{glossaries}
%   \emph{avec} l'option |acronym|. et on recourra à la commande
%   \begin{enumerate}
%   \item \refCom{printacronyms} pour celle des acronymes (cf.
%     \vref{fig-acronymes}) ;
%   \item\label{item:1} \docAuxCommand{printglossary} pour celle des termes du
%     glossaire (cf. \vref{fig-printglossary}).
%   \end{enumerate}
% \end{enumerate}

Dans un mémoire de thèse, les emplacements des listes des termes du glossaire,
des acronymes\footnote{Les commandes \docAuxCommand{printglossary} et
  \docAuxCommand{printacronyms} du \Package{glossaries}, produisant les listes
  des termes du glossaire et des acronymes, sont illustrées
  \vref{fig-printglossary,fig-printacronyms}.} et des symboles sont \emph{a
  priori} arbitraires. Il est cependant parfois conseillé de placer :
\begin{itemize}
\item si elles sont \emph{communes}, \emph{la} liste résultante en partie finale ;
\item si elles sont \emph{distinctes} :
  \begin{enumerate}
  \item les listes des acronymes et des symboles avant qu'ils soient utilisés
    pour la première fois donc, \emph{a priori}, avant le ou les résumés ;
  \item la liste des termes du glossaire en partie finale.
  \end{enumerate}
\end{itemize}

\section{Sommaire et/ou table des matières}\label{sec-table-des-matieres}

La \yatCl{} redéfinit la commande \refCom{tableofcontents} habituelle de
création des tables des matières \enquote{globales}\footnote{Par opposition aux
  tables des matières locales\index{table des matières!locale},
  cf. \vref{sec-localtoc}.} pour permettre de facilement en spécifier la
profondeur et en modifier le nom.

\begin{docCommand}[doc description=\mandatory]{tableofcontents}{\oarg{options}}
  \indexdef{table des matières}%
  Cette commande produit une table des matières dont le \enquote{niveau de
    profondeur} par défaut est celui des sous-sections : les intitulés des
  commandes de structuration qui y figurent sont (seulement) ceux des parties
  (éventuelles), des chapitres, des sections et des sous-sections.
\end{docCommand}

L'argument optionnel de la commande \refCom{tableofcontents} permet de stipuler
des \meta{options} sous la forme d'une liste \meta{clé}|=|\meta{valeur} dont
les clés disponibles sont les deux suivantes.
  %
{%
  \tcbset{before lower=\vspace*{\baselineskip}\par}
    %
  \begin{docKey}{depth}{=\docValue{part}\textbar\docValue{chapter}\textbar\docValue{section}\textbar\docValue{subsection}\textbar\docValue{subsubsection}\textbar\docValue{paragraph}\textbar\docValue{subparagraph}}{pas
      de valeur par défaut, initialement \docValue{subsection}}
    \index{table des matières!globale!profondeur}%
    \index{profondeur!table des matières!globale}%
    Cette clé permet de modifier le \enquote{niveau de profondeur} de la table
    des matières, respectivement jusqu'aux : parties, chapitres, sections,
    sous-sections, sous-sous-sections, paragraphes, sous-paragraphes.
  \end{docKey}
}
%
\begin{docKey}{name}{=\meta{nom alternatif}}{pas de valeur par défaut,
    initialement \docAuxCommand{contentsname}}
  \index{table des matières!globale!nom}%
  \index{table des matières!globale!titre}%
  \index{nom!table des matières}%
  Par défaut, le nom de la table des matières est \docAuxCommand{contentsname},
  c'est-à-dire \enquote{\contentsname} ou
  \enquote{\selectlanguage{english}\contentsname}\selonlangueshort{}. Cette clé
  permet de spécifier un \meta{nom alternatif}\redefexprcle.
\end{docKey}

\begin{dbremark}{Tables des matières multiples}{}
  \index{table des matières!globale!multiple}%
  Si la table des matières est longue, il est conseillé de la placer en fin de
  document mais de faire alors figurer, en \gls{liminaire}, un sommaire
  c'est-à-dire par une table des matières allégée.

  À cet effet, la \yatCl{} permet de faire figurer, dans un même document,
  plusieurs tables des matières au moyen d'occurrences multiples de la commande
  \refCom{tableofcontents}, chacune d'elles étant sujette aux options
  précédentes.
\end{dbremark}

\begin{dbexample}{Sommaire et table des matières}{sommaire-table-des-matieres}
  \indexex{table des matières}%
  Pour faire figurer, dans un même document :
  \begin{enumerate}
  \item un sommaire :
    \begin{itemize}
    \item ne faisant apparaître que les chapitres (et éventuelles parties) ;
    \item nommé \enquote{Sommaire} ;
    \end{itemize}
  \item la table des matières ;
  \end{enumerate}
  on insérera respectivement :
  %
  \bodysample{%
    rangeendsuffix=\],%
    deletekeywords={chapter},%
    linerange={tableofcontents-Sommaire},
  }{}
  %
  et :
  %
  \bodysample{rangesuffix=\^^M,linerange={tableofcontents}}{}
  %
  La \vref{fig-tableofcontentsto-tableofcontents} illustre ce code.
\end{dbexample}

\afterpage{%
  \begin{landscape}
    \begin{figure}[p]
      \centering
      \begin{subfigure}[b]{.45\linewidth}
        \centering%
        \screenshot[1]{tableofcontents-withargument}
        \caption{Sommaire allant jusqu'aux chapitres}
        \label{fig-tableofcontentsto}
      \end{subfigure}%
      \hspace{\stretch{1}}%
      \begin{subfigure}[b]{.45\linewidth}
        \centering%
        \screenshot[1]{tableofcontents-withoutargument}
        \caption{Table des matières allant jusqu'aux sous-sections}
        \label{fig-tableofcontents}
      \end{subfigure}%
      \caption[Sommaire et table des matières]{Sommaire et table des matières
        de profondeurs différentes dans un même document}
      \label{fig-tableofcontentsto-tableofcontents}
    \end{figure}
  \end{landscape}
}

\section{Tables et listes usuelles}
\index{figure!table des ---s}%
\index{table des figures}%
\index{liste des tableaux}%
\indexsee{table des tableaux}{liste des tableaux}%
\index{tableau!liste des ---x}%
\index{table des listings}%
\index{listing informatique!table des ---s}%

Les commandes usuelles |\listoftables| et |\listoffigures| produisent les
listes respectivement des tableaux et des figures.
%
On peut faire figurer d'autres listes, par exemple celle des listings
informatiques au moyen de la commande |\lstlistoflistings| du
\Package*{listings}.
%
Nous n'illustrons pas ces commandes, classiques.

%%% Local Variables:
%%% mode: latex
%%% TeX-master: "../yathesis-fr"
%%% End:
