\chapter{\emph{Add-ons}}\label{cha-add-ons}

La \yatCl{} fournit des \emph{add-ons} destinés à faciliter son usage avec
différents éditeurs de texte.

\section{\texorpdfstring{\texstudio}{TeXstudio}}
\label{sec-texstudio}
\index{éditeur de texte!TeXstudio@\texstudio}%
\index{TeXstudio@\texstudio}%

L'éditeur \href{http://texstudio.sourceforge.net/}{\texstudio} est livré avec un
système de complétion\index{complétion} et de vérification de l'orthographe des
commandes, environnemens et clés pour un grand nombre de classes et
packages. C'est notamment le cas pour la \yatCl{} (par le biais du fichier
\file{yathesis.cwl} dont la version la plus récente est livrée avec les
distributions\index{distribution \TeX} \texlive et \miktex{}).

% , se trouve dans
% le répertoire \folder{\meta{racine}/% \jobdocdirectory
%   /addons/completion/} où, par défaut, \meta{racine} est, avec la distribution :
% \begin{description}
% \item[\texlive :]\
%   \begin{description}
%   \item[sous \linux et \macos :] \unixtldirectory\tldistdirectory\versiontl ;
%   \item[sous \windows :] \wintldirectory\tldistdirectory\versiontl ;
%   \end{description}
% \item[\miktex :] \miktexdistdirectory.
% \end{description}
% % En attendant que ce fichier soit officiellement livré avec cet
% % éditeur\footnote{Ce devrait être le cas à partir de sa version
% %   \texttt{2.8.0}.}, ou pour être certain d'en avoir la version la plus à jour,
% % il suffit de le copier dans le dossier :
% % \begin{description}
% % \item[sous \linux et \macos :] \urldirectory{~/.config/texstudio} ;
% % \item[sous \windows{} :] \urldirectory{C:\Documents and Settings/User/AppData/Roaming/texstudio}.
% % \end{description}

\section{\texorpdfstring{\emacs}{Emacs}}
\label{sec-emacs}
\index{éditeur de texte!Emacs@\emacs}%
\index{Emacs@\emacs}%

[TODO]

%%% Local Variables:
%%% mode: latex
%%% TeX-master: "../yathesis-fr"
%%% End:
