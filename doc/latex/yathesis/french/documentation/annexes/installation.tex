\chapter{Installation}\label{cha-installation}
\index{installation}%

\changes{v0.99}{2014-05-18}{Procédure d'installation précisée}%
La procédure d'installation de la \yatCl{} dépend de la version souhaitée :
stable ou de développement.

\section{Version stable}
\label{sec-version-stable}
\index{distribution \TeX}%
\indexsee{TeX Live@\texlive}{distribution \protect\TeX}%
\indexsee{MiKTeX@\miktex}{distribution \protect\TeX}%

La version stable de la classe est normalement fournie par les distributions de
\TeX{}, notamment \texlive\footnote{Par mise à jour de sa version \texttt{2014},
  et d'emblée pour les versions suivantes.} et \miktex\footnote{Par mise à jour
  de sa version \texttt{2.9}, et d'emblée pour les versions suivantes.}. Pour
s'assurer que cette version stable est la plus récente, il est de toute façon
conseillé de mettre à jour sa distribution \TeX{}.

\section{Version de développement}
\label{sec-vers-de-devel}

Si on souhaite utiliser (à ses risques et périls !) la version de développement
de la \yatCl{}, on clonera son dépôt \program{Git} à la page
\url{https://github.com/dbitouze/yathesis}. La procédure pour ce faire, hors
sujet ici, n'est pas détaillée.

%%% Local Variables:
%%% mode: latex
%%% TeX-master: "../yathesis-fr"
%%% End:
