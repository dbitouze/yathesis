\chapter{Usage avancé}\label{cha-usage-avance}

% \section{(Dés)Activation des erreurs ciblées propres aux éléments
%   \enquote{obligatoires}}\label{sec-desact-des-erre}

On a vu \vref{sec-comm-oblig} que la \yatCl{} considère comme
\enquote{obligatoires} certains éléments (commandes et environnements) et que,
parmi eux, certains ne le sont que modérément. La liste complète de ces
derniers figure à la 1\iere{} colonne du \vref{tab-no-warnings}.
\begin{table}[ht]
  \centering
  \caption[Éléments modérément \enquote{obligatoires} et désactivation des
  erreurs associées]{Éléments modérément \enquote{obligatoires} et options de
    désactivation des erreurs ciblées associées}
  \label{tab-no-warnings}
  \begin{tabular}{ll}
  Élément                    & Clé(s) de désactivation individuelle de l'erreur   \\\toprule
  \refCom{author}            & \refKey{noauthor}                                  \\
  \refCom{title}             & \refKey{notitle}                                   \\
  \refCom{academicfield}     & \refKey{noacademicfield}                           \\
  \refCom{date}              & \refKey{nodate}                                    \\
  \refCom{institute}         & \refKey{noinstitute}                               \\
  \refCom{doctoralschool}    & \refKey{nodoctoralschool}                          \\
  \refCom{laboratory}        & \refKey{nolaboratory}, \refKey{nolaboratoryadress} \\
  \refCom{supervisor}        & \refKey{nosupervisor}                              \\
  \refCom{maketitle}         & \refKey{nomaketitle}                               \\
  \refCom{keywords}          & \refKey{nokeywords}                                \\
  \refEnv{abstract}          & \refKey{noabstract}                                \\
  \refCom{makeabstract}      & \refKey{nomakeabstract}                            \\
  \refCom{tableofcontents}   & \refKey{notableofcontents}                         \\
  \refCom{printbibliography} & \refKey{noprintbibliography}                       \\\bottomrule
\end{tabular}

\end{table}

Cependant, au cours de la rédaction, certains éléments \enquote{obligatoires}
de la thèse ne sont pas encore connus de l'utilisateur, par exemple la date de
soutenance (commande \refCom{date}).
%
De ce fait, le signalement des éléments \enquote{obligatoires} manquants est :
\begin{description}
\item[désactivé] par défaut\footnote{C'est-à-dire en version intermédiaire du
    document (cf. valeur initiale \docValue{inprogress} de la clé
    \refKey{version}). Le signalement est également désactivé en versions
    intermédiaire alternative et brouillon (cf. valeurs \docValue{inprogress*}
    et \docValue{draft} de la clé \refKey{version}).} ;
\item[activé] (seulement) en versions \enquote{à
    soumettre}\footnote{Cf. valeurs \docValue{submitted} et
    \docValue{submitted*} de la clé \refKey{version}.} et
  \emph{finale}\footnote{Cf. valeur \docValue{final} de la clé
    \refKey{version}.} du document, par le biais d'une erreur ciblée, émise
  à chaque compilation.
\end{description}

Or, l'utilisateur \emph{avancé} peut vouloir passer outre le caractère
\enquote{obligatoire} de tel ou tel élément\footnote{Par exemple, il peut juger
  inutile de faire figurer des mots clés dans sa thèse et donc vouloir passer
  outre la commande \enquote{obligatoire} \protect\refCom{keywords}.},
y compris en versions à soumettre ou finale du document, et pour éviter que
chaque compilation ne soit alors perturbée par les erreurs ciblées
correspondantes, la \yatCl{} propose de désactiver ces erreurs de façon :
\begin{description}
\item[soit globale] au moyen de la clé \refKey{noerror} suivante.
  \begin{docKey}{noerror}{=\docValue{true}\textbar\docValue{false}}{par défaut
      \docValue{true}, initialement \docValue{true}}
    Cette option est en sous-main initialement égale à \docValue{true}, et
    devient automatiquement égale à \docValue{false} si la clé \refKey{version}
    a pour valeur \docValue{submitted}, \docValue{submitted*} ou
    \docValue{final}. L'employer la rend égale à \docValue{true} et désactive
    donc, dans tous les cas, toutes les erreurs propres à la \yatCl{}.
  \end{docKey}
\item[soit individuelle] au moyen des clés listées à la 2\ieme{} colonne du
  \vref{tab-no-warnings} et dont le rôle est précisé ci-après.
  \begin{docKey}{noauthor}{=\docValue{true}\textbar\docValue{false}}{par défaut
      \docValue{true}, initialement \docValue{false}}
    Cette option désactive l'erreur émise si la commande \refCom{author} est
    omise (ou à arguments obligatoires vides).
  \end{docKey}
  \begin{docKey}{notitle}{=\docValue{true}\textbar\docValue{false}}{par défaut
      \docValue{true}, initialement \docValue{false}}
    Cette option désactive l'erreur émise si la commande \refCom{title} est
    omise (ou à argument obligatoire vide).
  \end{docKey}
  \begin{docKey}{noacademicfield}{=\docValue{true}\textbar\docValue{false}}{par
      défaut \docValue{true}, initialement \docValue{false}}
    Cette option désactive l'erreur émise si la commande \refCom{academicfield}
    est omise (ou à argument obligatoire vide).
  \end{docKey}
  \begin{docKey}{nodate}{=\docValue{true}\textbar\docValue{false}}{par défaut
      \docValue{true}, initialement \docValue{false}}
    Cette option désactive l'erreur émise si la commande \refCom{date} est
    omise (ou à arguments obligatoires vides ou incorrects)\footnote{La
      commande \refCom{date} n'est en fait pas \enquote{obligatoire} en
      versions \enquote{à soumettre} (valeurs \docValue{submitted} et
      \docValue{submitted*} de la clé \refKey{version}).}.
  \end{docKey}
  \begin{docKey}{noinstitute}{=\docValue{true}\textbar\docValue{false}}{par
      défaut \docValue{true}, initialement \docValue{false}}
    Cette option désactive l'erreur émise si la commande \refCom{institute} est
    omise (ou à argument obligatoire vide).
  \end{docKey}
  \begin{docKey}{nodoctoralschool}{=\docValue{true}\textbar\docValue{false}}{par
      défaut \docValue{true}, initialement \docValue{false}}
    Cette option désactive l'erreur émise si la commande
    \refCom{doctoralschool} est omise (ou à argument obligatoire vide).
  \end{docKey}
  \begin{docKey}{nolaboratory}{=\docValue{true}\textbar\docValue{false}}{par
      défaut \docValue{true}, initialement \docValue{false}}
    Cette option désactive l'erreur émise si la commande \refCom{laboratory}
    est omise (ou à 1\ier{} argument obligatoire vide).
  \end{docKey}
  \begin{docKey}{nolaboratoryadress}{=\docValue{true}\textbar\docValue{false}}{par
      défaut \docValue{true}, initialement \docValue{false}}
    Cette option désactive l'erreur émise si la commande \refCom{laboratory}
    est omise (ou à 2\ieme{} argument obligatoire vide).
  \end{docKey}
  \begin{docKey}{nosupervisor}{=\docValue{true}\textbar\docValue{false}}{par
      défaut \docValue{true}, initialement \docValue{false}}
    Cette option désactive l'erreur émise si la commande \refCom{supervisor}
    est omise (ou à argument obligatoire vide).
  \end{docKey}
  \begin{docKey}{nomaketitle}{=\docValue{true}\textbar\docValue{false}}{par
      défaut \docValue{true}, initialement \docValue{false}}
    Cette option désactive l'erreur émise si la commande \refCom{maketitle} est
    omise.
  \end{docKey}
  \begin{docKey}{nokeywords}{=\docValue{true}\textbar\docValue{false}}{par
      défaut \docValue{true}, initialement \docValue{false}}
    Cette option désactive l'erreur émise si la commande \refCom{keywords} est
    omise (ou à arguments obligatoires vides).
  \end{docKey}
  \begin{docKey}{noabstract}{=\docValue{true}\textbar\docValue{false}}{par
      défaut \docValue{true}, initialement \docValue{false}}
    Cette option désactive l'erreur émise si l'environnement \refEnv{abstract}
    est omis (ou à contenu vide).
  \end{docKey}
  \begin{docKey}{nomakeabstract}{=\docValue{true}\textbar\docValue{false}}{par
      défaut \docValue{true}, initialement \docValue{false}}
    Cette option désactive l'erreur émise si la commande \refCom{makeabstract}
    est omise.
  \end{docKey}
  \begin{docKey}{notableofcontents}{=\docValue{true}\textbar\docValue{false}}{par
      défaut \docValue{true}, initialement \docValue{false}}
    Cette option désactive l'erreur émise si la commande
    \refCom{tableofcontents} est omise.
  \end{docKey}
  \begin{docKey}{noprintbibliography}{=\docValue{true}\textbar\docValue{false}}{par
      défaut \docValue{true}, initialement \docValue{false}}
    Cette option désactive l'erreur émise si la commande
    \refCom{printbibliography} est omise.
  \end{docKey}
\end{description}

\begin{dbwarning}{Éléments \enquote{obligatoires} : modérément obligatoires
    mais fortement conseillés}{}
  En versions à soumettre et finale du document, il est conseillé d'employer
  tous les éléments \enquote{obligatoires} et, sinon, d'utiliser à bon escient
  les clés de désactivation ci-dessus, sans quoi :
  \begin{itemize}
  \item des textes génériques peuvent figurer dans le document en remplacement
    de certains éléments manquants ;
  \item le bon fonctionnement de la \yatCl{} peut être sérieusement altéré.
  \end{itemize}
  Donc, pour ces versions, on ne passera outre les éléments
  \enquote{obligatoires} de la \yatCl{} que dans le cadre d'un usage avancé, si
  l'on est sûr de ce que l'on fait et qu'on pourra en gérer \emph{seul} les
  conséquences.  On évitera notamment de recourir de façon trop légère à la clé
  \refKey{noerror}.
\end{dbwarning}

%%% Local Variables:
%%% mode: latex
%%% TeX-master: "../yathesis-fr"
%%% End:
