\chapter{Canevas et spécimens de thèse}\label{cha-specimen-canevas}%
\index{canevas}%
\index{spécimen}%

Un canevas et un spécimen de mémoires de thèse créés avec la \yatCl sont
fournis, chacun en deux versions illustrant chacune une façon d'organiser le
source \file{.tex} du mémoire :
\begin{description}
\item[\enquote{à plat} :] le source est tout entier dans un unique fichier,
  situé dans le même dossier que les fichiers annexes (bibliographie et
  images) ;
\item[\enquote{en arborescence} :]%
  \index{fichier!maître}%
  \index{fichier!esclave}%
  le source est scindé en fichiers maître et esclaves\footnote{Comme cela est en
    général recommandé, cf. \vref{sec-repart-du-memo}.}, situés (ainsi que
  l'ensemble des fichiers annexes) dans différents (sous-)dossiers.
\end{description}
Les deux canevas et deux spécimens ainsi proposés ont pour but :
\begin{itemize}
\item d'aider à la mise en œuvre de la classe en fournissant une base de départ
  que chacun peut progressivement adapter à ses propres
  besoins ;
\item d'illustrer les fonctionnalités de la classe.
\end{itemize}

La version électronique (\pdf{}) de la présente
documentation\footnote{Disponible à l'adresse
  \url{http://ctan.org/pkg/yathesis}, si besoin est.} intègre ces canevas et
spécimens par le biais d'une archive \gls{zip}, normalement accessible par
simple clic sur le lien suivant :
\textattachfile{../exemples/canevas-specimen.zip}{\file{canevas-specimen.zip}}\footnote{En
  tous cas avec les afficheurs \pdf:{} \program{Evince} sous \linux et
  \href{http://www.sumatrapdfreader.org/free-pdf-reader-fr.html}{\program{SumatraPDF}}
  sous \windows.}. L'extraction de cette archive fournit un dossier nommé
\folder{exemples} dont l'arborescence est la suivante :

\setlength{\DTbaselineskip}{15pt}
\begin{tcolorbox}
  \dirtree{%
  .1 \folder{exemples}/.
  .2 \folder{canevas}/.
  .3 \folder{a-plat}/.
  .3 \folder{en-arborescence}/.
  .2 \folder{specimen}/.
  .3 \folder{a-plat}/.
  .3 \folder{en-arborescence}/.
}%
\end{tcolorbox}

\changes{v0.99n}{2016-06-11}{Réorganisation des spécimens et canevas}%
\changes{v0.99m}{2016-05-22}{Réorganisation et changement de noms des spécimens
  et canevas}%
\changes{v0.99m}{2016-05-22}{Spécimens et canevas intégrés au \acrshort{pdf} de
  la documentation sous la forme d'archives \gls{zip}}%
\changes{v0.99l}{2014-10-23}{Réorganisation et changement de noms des spécimens
  et canevas}%
\changes{v0.99c}{2014-06-06}{Spécimens et canevas fournis sous forme d'archives
  \file{.zip}}%
\changes{v0.99b}{2014-06-02}{Réorganisation des spécimens et canevas}%
\changes{v0.99a}{2014-06-02}{Spécimens et canevas enrichis}%
\begin{comment}
  \begin{itemize}
  \item pour la distribution \texlive\versiontl, sur les systèmes :
    \begin{itemize}
    \item \linux et \macos{} :
      \href{./.}{\folder{\unixtldirectory\tldistdirectory\jobdocdirectory/}} ;
    \item \windows{} :
      \href{./.}{\folder{\wintldirectory\tldistdirectory\jobdocdirectory/}} ;
    \end{itemize}
  \item pour la distribution \miktex : \folder{\miktexdistdirectory}.
  \end{itemize}
\end{comment}

\begin{dbwarning}{Archive à extraire avant toute chose !}{}
  Pour pouvoir consulter et surtout tester sans problème les canevas et
  spécimens de l'archive \file{canevas-specimen.zip}, celle-ci \emph{doit} être
  extraite avant toute chose !
\end{dbwarning}

Il est également possible de tester directement au moyen des éditeurs (et
compilateurs) \LaTeX{} en ligne%
\index{éditeur de texte!en ligne}%
\index{compilation!en ligne}%
\index{en ligne!éditeur de texte}%
\index{en ligne!compilation}
\begin{itemize}
\item \href{https://fr.sharelatex.com/}{ShareLaTeX} : le
  \href{https://fr.sharelatex.com/templates/thesis/yathesis-template}{canevas}
  et le
  \href{https://fr.sharelatex.com/templates/thesis/yathesis-specimen}{spécimen}\enarborescence ;
\item \href{https://www.overleaf.com/}{Overleaf} : le
  \href{https://www.overleaf.com/latex/templates/template-of-a-thesis-written-with-yathesis-class/nhtmtthnqwtd}{canevas}
  et le
  \href{https://www.overleaf.com/latex/examples/sample-of-a-thesis-written-with-yathesis-class/nbcfvfqgnjfq}{spécimen}\enarborescence.
\end{itemize}

Les \vref{sec-canevas,sec-specimens} détaillent les fichiers qui constituent
chacun de ces canevas et spécimens.

% Parmi eux, \file{latexmkrc}, fichier de configuration du programme
% \program{latexmk} qui permet d'automatiser le processus de compilation
% complète de la thèse (cf. \vref{sec-autom-des-comp} pour plus de détails).

% Les \vref{sec-specimens,sec-canevas} détaillent davantage ces spécimens et
% canevas, en indiquant notamment tous les fichiers qui les constituent. Chacun
% des spécimens et canevas fournit un fichier, nommé \file{latexmkrc}, de
% configuration du programme \program{latexmk} qui permet d'automatiser le
% processus de compilation complète de la thèse (cf. \vref{sec-autom-des-comp}
% pour plus de détails).

% La commande à utiliser pour lister le contenu du répertoire est :%
% tree --charset=ascii -F -I \ %
% "*aux|*idx|*ind|*lof|*lot|*out|*toc|*acn|*acr|*alg|*bcf|*glg|*glo|*gls|*glg2|*gls2|*glo2|*ist|*run.xml|*xdy|*lol|*fls|*slg|*slo|*sls|*unq|*synctex.gz|*mw|*bbl|*blg|*fdb_latexmk|*log|*auto"

\section{Canevas}
\label{sec-canevas}

Les \emph{canevas} fournis (regroupés dans le dossier \folder{canevas}) ne sont
rien d'autre que les (quasi-)répliques des \emph{spécimens} correspondants dont
les données ont été vidées : pour les exploiter, il suffit donc de remplir les
\enquote{cases} vides.

\subsection{Canevas \enquote{à plat}}
\label{sec-canevas-a-plat}
\index{canevas!à plat}%

Le dossier (\folder{canevas/a-plat}) de ce canevas ne contient que trois fichiers :
\begin{enumerate}
\item \file{these.tex}, source \file{.tex} (unique) de la thèse  ;
\item \file{these.pdf} produit par compilation du \File{these.tex} ;
\item \file{latexmkrc}.
\end{enumerate}

[TODO]

\subsection{Canevas \enquote{en arborescence}}
\label{sec-canevas-arborescence}
\index{canevas!en arborescence}%
\index{fichier!maître}%
\index{fichier!esclave}%

Le dossier (\folder{canevas/en-arborescence}) de ce canevas contient les fichiers :
\begin{enumerate}
\item ...
\item \file{latexmkrc}.
\end{enumerate}

[TODO]

\section{Spécimens}
\label{sec-specimens}

Sur la base de données plus ou moins fictives, de textes arbitraires et de
\gls{fauxtexte}, les spécimens (regroupés dans le dossier \folder{specimen})
mettent en évidence l'ensemble des possibilités offertes par la \yatCl{}.

\subsection{Spécimen \enquote{à plat}}
\label{sec-specimen-a-plat}
\index{spécimen!à plat}%

Le dossier (\folder{specimen/a-plat}) de ce spécimen contient les fichiers :
\begin{enumerate}
\item \file{these.tex} qui est le source \file{.tex} (unique) de la thèse ;
\item \file{bibliographie.bib}, contenant les références bibliographiques de
  la thèse ;
\item \file{these.pdf} produit par compilation du \File{these.tex} ;
\item \file{labo.pdf}, \file{paris13.pdf}, \file{pres.pdf}, \file{tiger.pdf},
  \file{ulco.pdf} (images : logos, etc.) ;
\item \file{latexmkrc}.
\end{enumerate}

[TODO]

\subsection{Spécimen \enquote{en arborescence}}
\label{sec-specimen-arborescence}
\index{spécimen!en arborescence}%
\index{fichier!maître}%
\index{fichier!esclave}%

Le dossier (\folder{specimen/en-arborescence}) de ce spécimen contient les
fichiers :
\begin{enumerate}
\item ...
\item \file{latexmkrc}.
\end{enumerate}

[TODO]

%%% Local Variables:
%%% mode: latex
%%% TeX-master: "../yathesis-fr"
%%% End:
