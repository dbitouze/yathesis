\chapter{Fichiers automatiquement importés par la \yatCl{}}
\label{cha-fichiers-importes-par}

Pour faciliter son utilisation, la \yatCl{} importe automatiquement deux
fichiers :
\begin{enumerate}
\item%
  \index{fichier!des caractéristiques de la thèse}%
  un fichier nommé \file{\characteristicsfile} dédié aux données
  caractéristiques du document amenées à figurer en divers emplacements ou comme
  métadonnées du fichier \pdf{} produit (cf. \vref{sec-lieu-de-saisie}) ;
\item%
  \index{fichier!de configuration de \yatcl}%
  un fichier nommé \file{\configurationfile} dédié à la configuration du
  document, où stocker notamment les réglages :
  \begin{itemize}
  \item de la \yatCl (cf. \vref{cha-configuration}) ;
  \item des différents packages chargés soit par la classe, soit manuellement
    (cf. \vref{cha-packages-charges}).
  \end{itemize}
  % \item un fichier nommé \file{\macrosfile} dédié aux macros personnelles
  % créées pour le document.
\end{enumerate}
\begin{dbwarning}{Fichiers de données et de configuration automatiquement importés}{import-sous-cond}
  Pour que ces fichiers
  % \file{\characteristicsfile} et \file{\configurationfile}
  soient automatiquement importés, il est nécessaire :
  \begin{enumerate}
  \item qu'ils existent\footnote{Ces fichiers et sous-répertoire sont donc
      à créer au besoin mais le canevas de thèse \enquote{en arborescence} livré avec
      la classe, décrit \vref{sec-canevas-arborescence}, les fournit d'emblée.} ;
  \item%
    \index{dossier!de configuration}%
    qu'ils soient situés dans le répertoire \emph{ad hoc}, à savoir un
    sous-répertoire nommé \folder{\configurationdirectory} du répertoire où se
    trouve le fichier (maître) du document.
  \end{enumerate}
\end{dbwarning}
%
\begin{dbwarning}{Fichiers de données et de configuration à ne pas importer manuellement}{}
  \index{fichier!des caractéristiques de la thèse}%
  \index{fichier!de configuration de \yatcl}%
  Si ces fichiers
  % \file{\characteristicsfile} et \file{\configurationfile}
  vérifient les conditions de l'avertissement précédent, la \yatCl{} les
  importe \emph{automatiquement} : ils doivent donc \emph{ne pas} être
  explicitement importés \aside*{au moyen d'une commande \docAuxCommand{input}
    ou assimilée}.
\end{dbwarning}
