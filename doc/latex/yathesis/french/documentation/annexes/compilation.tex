\chapter{Compilation de la présente documentation}\label{cha-comp-de-la-pres}

\index{compilation!du présent document}%

Pour compiler soi-même le présent document, il est :
\begin{enumerate}
\item nécessaire de :
  \begin{enumerate}
  \item copier le \Folder{.../yathesis/doc/latex/yathesis/french} dans un
    dossier personnel accessible en écriture ;
  \item se rendre dans le \Folder{french/exemples/specimen/a-plat} et compiler
    le \File{these.tex} au moyen de la commande suivante\uneseuleligne :
  %
\begin{listingshell}[before=\smallskip]
latexmk -g -norc -r ./latexmkrc -jobname=these \
-pdflatex="pdflatex %O '\RequirePackage{etoolbox} \
\AtEndPreamble{\RequirePackage{yathesis-demo}} \input{%S}'" these
\end{listingshell}
%
\item accéder au \File{canevas-specimen.zip} attaché à la version électronique
  originale de la présente documentation\footnote{Disponible à l'adresse
    \url{http://ctan.org/pkg/yathesis}, si besoin est.} et le placer dans le
  \Folder{../exemples} ;
\end{enumerate}
\item suffisant d'ensuite :
  \begin{enumerate}
  \item se rendre dans le dossier \Folder{french/documentation} ;
  \item lancer la commande :
%
\begin{listingshell}[before=\medskip]
latexmk -norc -r ./latexmkrc yathesis-fr
\end{listingshell}
%
  \end{enumerate}
\end{enumerate}

%%% Local Variables:
%%% mode: latex
%%% TeX-master: "../yathesis-fr"
%%% End:
