\chapter{Questions fréquemment posées}\label{cha-faq}

Ce chapitre est une \gls{faq} \aside{autrement dit une liste des questions
  fréquemment posées} sur la \yatCl{}.

\section{Problèmes d'utilisation}

\begin{dbfaq}{Comment faire en cas de problème d'utilisation de la \yatCl{} ?}{probleme-utilisation}
  \index{problème d'utilisation}%
  La \yatCl{} est vraiment formidable, mais je rencontre un problème en
  l'utilisant.  Comment faire ?
\end{dbfaq}

En cas de problème d'utilisation\footnote{À ne pas confondre avec un bogue ou
  une fonctionnalité manquante, cf. \vref{faq-bogues}.} :
\begin{enumerate}
\item commencer par chercher s'il n'a pas déjà été signalé (et surtout
  solutionné) en consultant par exemple la liste des questions concernant la
  \yatCl{} sur les sites de questions \& réponses dédiés à \LaTeX{} :
  \begin{itemize}
  \item \url{https://texnique.fr/osqa/tags/yathesis/}\footnote{Site
      francophone.} ;
  \item
    \url{https://tex.stackexchange.com/questions/tagged/yathesis}\footnote{Site
      anglophone.} ;
  \end{itemize}
\item s'il semble inédit (ou n'est pas \aside{ou mal} solutionné), poser
  soi-même une question sur un des lieux d'entraide dédiés à \LaTeX{}, par
  exemple sur l'un des sites ci-dessus\ecm{}.
\end{enumerate}

\section{Communication}
\label{sec-communication}

\begin{dbfaq}{Comment communiquer avec l'auteur de la \yatCl{} ?}{bogues}
  \index{bogue}%
  \index{bogue!rapport}%
  \indexsee{bug}{bogue}%
  \index{fonctionnalité!demande}%
  La \yatCl{} est vraiment formidable, mais :
  \begin{enumerate}
  \item je souhaite signaler un dysfonctionnement (un bogue) ou suggérer une
    amélioration (par exemple en demandant une nouvelle fonctionnalité) ;
  \item je souhaite communiquer avec son auteur.
  \end{enumerate}
  Comment faire ?
\end{dbfaq}

\begin{enumerate}
\item Pour un dysfonctionnement\footnote{À ne pas confondre avec un
    \enquote{simple} problème d'utilisation,
    cf. \vref{faq-probleme-utilisation}.}  ou une amélioration :
  \begin{enumerate}
  \item avant de le signaler ou de la suggérer, s'assurer qu'ils n'ont pas déjà
    été répertoriés :
    \begin{enumerate}
    \item en consultant la liste de ceux qui le sont déjà\issues ;
    \item en lisant la suite du présent chapitre ;
    \item en lisant l'\vref{cha-incomp-conn} ;
      % des dysfonctionnements déjà répertoriés à l'adresse
      % \url{https://github.com/dbitouze/yathesis/issues?q=is%3Aopen+is%3Aissue+label%3Abug} ;
    \end{enumerate}
  \item s'ils n'ont pas déjà été répertoriés, signaler ce
    dysfonctionnement\ecm{} ou suggérer cette amélioration\newissues{}.
  \end{enumerate}
  % \item Pour une nouvelle fonctionnalité (ou suggestion d'amélioration) :
  %   \begin{enumerate}
  %   \item avant de la demander, s'assurer qu'elle n'a pas déjà été
  %     répertoriée :
  %     \begin{enumerate}
  %     \item en consultant la liste de celles qui le sont déjà\issues ;
  %       %       \url{https://github.com/dbitouze/yathesis/issues?q=is%3Aopen+is%3Aissue+label%3Aenhancement} ;
  %     \item en lisant la suite du présent chapitre ;
  %     \end{enumerate}
  %   \item si elle n'a pas déjà été répertoriée, demander cette fonctionnalité
  %     (ou suggérer une amélioration)\newissues.
  %   \end{enumerate}
\item Pour communiquer avec l'auteur de la classe, il est possible d'utiliser
  l'adresse indiquée à la page \url{https://github.com/dbitouze/yathesis/}.
\end{enumerate}

\section{Avertissements}
\label{sec-avertissements}

\begin{dbfaq}{Puis-je ignorer un avertissement signalant une version trop
    ancienne d'un package ?}{}
  \index{avertissement de compilation}%
  \index{compilation!avertissement}%
  \index{package!ancien}%
  Je suis confronté à un avertissement de la forme \enquote{You have requested,
    on input line \meta{numéro}, version `\meta{date plus récente}' of package
    \meta{nom d'un package}, but only version `\meta{date moins récente} ...'
    is available.}. Est-ce grave, docteur ?
\end{dbfaq}

Ça peut être grave. Cf. \vref{rq-packages-anciens} pour plus de précisions.

\section{Erreurs}
\label{sec-erreurs}%
\index{erreur de compilation}%
\index{compilation!erreur}%

\begin{dbfaq}{Comment éviter l'erreur \enquote{Option clash for package
      \meta{package}} ?}{option-clash}
  Je suis confronté à l'erreur \enquote{Option clash for package \meta{package}}
  (notamment avec \meta{package}|=|\package{babel}). Comment l'éviter ?
\end{dbfaq}

Cette erreur est probablement due au fait que le \meta{package} a été
manuellement chargé au moyen de la commande
|\usepackage[...]{|\meta{package}|}|, alors que la \yatCl{} le charge déjà
automatiquement (cf. l'\vref{sec-packages-charges-par} pour la liste des
packages automatiquement chargés). Supprimer cette commande devrait résoudre le
problème (cf. également l'\vref{wa-packages-a-ne-pas-charger}).

\begin{dbfaq}{Comment éviter l'erreur \enquote{Command
      \protect\docAuxCommand*{nobreakspace} unavailable in encoding T1} ?}{}
  Lorsque je compile ma thèse avec \hologo{XeLaTeX} ou \hologo{LuaLaTeX}, je
  suis confronté à l'erreur \enquote{Command \docAuxCommand*{nobreakspace}
    unavailable in encoding T1}. Comment l'éviter ?
\end{dbfaq}

(Cette question ne concerne pas directement la \yatCl{}.) Il suffit
d'insérer, en préambule du fichier (maître) de la thèse, la ligne :
\begin{preamblecode}[title=Par exemple dans le \File{\configurationfile}]
\DeclareTextCommand{\nobreakspace}{T1}{\leavevmode\nobreak\ }
\end{preamblecode}

% \begin{dbfaq}{Comment éviter l'erreur \enquote{No room for a new%
%   \protect\docAuxCommand*{write}} ?}{}%
%   Je suis confronté à l'erreur \enquote{no room for a new%
%   \docAuxCommand{write}}. Comment l'éviter ?%
% \end{dbfaq}

% Il devrait suffire de charger le \Package{morewrites} (plutôt parmi%
% les premiers packages).%

\section{Mise en page}
\label{sec-mise-en-page}

\subsection{Pages de titre}
\label{sec-pages-de-titre}
\index{page de titre!mise en page}%
\index{page de titre!apparence}%

\begin{dbfaq}{Comment modifier l'apparence de la page de titre ?}{}
  L'apparence par défaut de la page de titre ne me convient pas et je voudrais
  la modifier. Comment faire ?
\end{dbfaq}

Il est prévu de permettre de modifier certains aspects de la mise en page de la
page de titre, et même de fournir une documentation permettant d'obtenir une
apparence complètement personnalisée, mais ce n'est pas encore implémenté.  En
attendant que ça le soit, il faut composer cette page soi-même :
\begin{itemize}
\item soit en y resaisissant manuellement toutes les caractéristiques
  nécessaires définies au \vref{cha-caract-du-docum} ;
\item soit, mieux, en se rendant sur le site
  \href{https://texnique.fr/osqa/tags/yathesis/}{\TeX{}nique} pour :
  \begin{itemize}
  \item y examiner les réponses apportées aux questions similaires ;
  \item le cas échéant, y poser soi-même une question.
  \end{itemize}
\end{itemize}

\subsection{Table des matières}
\label{sec-table-des-matieres-faq}

\begin{dbfaq}{Pourquoi les glossaire, listes d'acronymes et de symboles
    apparaissent en double dans la table des matières et dans les signets ?}{}
  \index{table des matières!globale!entrée en double}%
  \index{signets!entrée en double}%
  Les glossaire, listes d'acronymes et de symboles apparaissent en double dans
  la table des matières et dans les signets. Comment éviter cela ?%
\end{dbfaq}

La \yatCl{} fait d'elle-même figurer les glossaire, listes d'acronymes et de
symboles à la fois dans la table des matières et dans les signets. Pour régler
le problème, il devrait donc suffire de \emph{ne pas} explicitement demander que
ce soit le cas, en \emph{ne} recourant \emph{ni} à l'option \docAuxKey*{toc},
\emph{ni} à la commande \docAuxCommand*{glstoctrue} du \Package{glossaries}.

\begin{dbfaq}{Comment faire en sorte que, dans la table des matières, seuls
    les numéros de page soient des liens hypertextes ?}{}
  \index{table des matières!hyperlien}%
  J'ai chargé le \Package{hyperref} et, par défaut, les entrées de la table des
  matières sont toutes entières des liens hypertextes, ce qui est trop
  envahissant. Comment faire en sorte que seuls les numéros de page soient des
  liens hypertextes ?
\end{dbfaq}

(Cette question ne concerne pas directement la \yatCl{}.) Il suffit de
passer l'option \lstinline|linktoc=false| au \Package{hyperref}.

\begin{dbfaq}{Comment supprimer la bibliographie des sommaire, table des
    matières et signets ?}{}
  \index{table des matières!globale!bibliographie}%
  \index{signets!bibliographie}%
  Par défaut, la bibliographie figure dans les sommaire, table des matières et
  signets du document. Comment éviter cela ?
\end{dbfaq}

(Cette question ne concerne pas directement la \yatCl{}.) Il suffit de
passer à la commande \docAuxCommand{printbibliography} l'option
|heading=|\meta{entête}, où \meta{entête} vaut par exemple
\docValue*{bibliography} (cf. la documentation du \Package{biblatex} pour plus
de détails).

\begin{dbfaq}{Comment affecter des profondeurs différentes aux signets et à la
    table des matières ?}{}
  \index{table des matières!globale!signet}%
  \index{table des matières!globale!profondeur}%
  \index{signets!profondeur}%
  \index{profondeur!signets}%
  Grâce au chargement du \Package{hyperref}, mon fichier \glsxtrshort{pdf}
  dispose de signets mais, par défaut, ceux-ci ont même niveau de profondeur que
  la table des matières. Comment leur affecter une profondeur différente ?
\end{dbfaq}

(Cette question ne concerne pas directement la \yatCl{}.) L'option
\docAuxKey{depth} du \Package*{bookmark} permet d'affecter aux signets un autre
niveau que celui par défaut :
\begin{preamblecode}[title=Par exemple dans le \File{\configurationfile}]
\bookmarksetup{depth="\meta{autre niveau}"}
\end{preamblecode}
où \meta{autre niveau} est l'une des valeurs possibles de la clé \refKey{depth}.

\begin{dbfaq}{Comment éviter que, dans la table des matières, certains numéros
    de pages débordent dans la marge de droite ?}{}
  \index{table des matières!globale!débordement dans la marge}%
  Dans la table des matières, certains numéros de pages (en chiffres romains
  notamment) débordent dans la marge de droite. Comment l'éviter ?
\end{dbfaq}

Il suffit d'insérer, en préambule du fichier (maître) de la thèse, les lignes :
\begin{preamblecode}[title=Par exemple dans le \File{\configurationfile}]
\makeatletter
\renewcommand*\@pnumwidth{"\meta{distance}"}
\makeatother
\end{preamblecode}
où \meta{distance}, à exprimer par exemple en points (par exemple |27pt|), est
à déterminer par \enquote{essais/erreurs} de sorte que \meta{distance} soit :
\begin{enumerate}
\item suffisamment grande, pour empêcher les débordements de numéros de pages ;
\item aussi petite que possible, pour éviter les lignes de pointillés trop
  courtes.
\end{enumerate}

\subsection[Titres
courants]{\texorpdfstring{\Glsentryplural{titrecourant}}{Titres courants}}
\label{sec-titres-courants}

\glsreset{tdm}
\begin{dbfaq}{Est-il possible d'obtenir des \glsplural{titrecourant} distincts
    des titres figurant en table(s) des matières ?}{}
  \index{table des matières!entrée différente du titre courant}%
  \index{titre courant!différent de l'entrée en table des matières}%
  %
  Les titres des chapitres et des sections sont reproduits en
  \glsplural{titrecourant} et, si certains d'entre eux sont longs au point de
  déborder de l'entête, je peux recourir à l'argument optionnel des commandes
  \docAuxCommand{chapter} et \docAuxCommand{section} pour qu'ils y soient
  remplacés par des titres alternatifs \enquote{courts}.

  Mais ce remplacement a alors lieu \emph{aussi} en \gls{tdm}, et cela me pose
  problème dans les cas suivants.
  % Mais alors les titres \enquote{normaux} de ces chapitres sont remplacés par
  % leurs versions alternatives \enquote{courtes} \emph{également} en table(s)
  % des
  % matières, et cela me pose problème dans les cas suivants.
  \begin{description}
  \item[Cas 1.] Je souhaite que, en \gls{tdm}, figurent systématiquement les
    titres \enquote{normaux}, et pas d'éventuels titres alternatifs
    \enquote{courts}.
  \item[Cas 2.] Je souhaite que les titres alternatifs \enquote{courts} des
    titres des chapitres puissent être différents en \gls{tdm} et en
    \glsplural{titrecourant}.
  \end{description}
  Est-il donc possible d'obtenir des \glsplural{titrecourant} distincts des
  titres figurant en \gls{tdm} ?%
\end{dbfaq}

Il suffit de recourir à l'argument optionnel supplémentaire des commandes
\refCom{chapter} et \refCom{section} fourni par la \yatCl{}
(cf. \vref{sec-intit-altern}).

\subsection{Divers}
\label{sec-divers}

\begin{dbfaq}{Pourquoi mes signes de ponctuation haute ne sont pas précédés des
    espaces adéquates ?}{}
  \index{espace!avant \enquote{?;:"!}}%
  Certains éléments que j'ai saisis en préambule contiennent des signes de
  ponctuation haute ({\NoAutoSpacing?;:!}) mais, dans le \pdf{} produit, ces
  derniers ne sont pas précédés des espaces adéquates. Comment régler ce
  problème ?
\end{dbfaq}

(Cette question ne concerne pas directement la \yatCl{}.) Le problème
est dû aux caractères actifs du module \package*+{babel-french} du
\Package{babel}. Si ces éléments concernent :
\begin{enumerate}
\item les caractéristiques du document (cf. \vref{cha-caract-du-docum}), il
  suffit de les saisir\footnote{Cf. \vref{sec-lieu-de-saisie}.} :
  \begin{itemize}
  \item soit dans le \emph{corps} du fichier (maître) de la thèse\footnote{Mais
      cf. alors \vref{wa-avant-maketitle}.} (et donc \emph{pas} dans son
    \emph{préambule}) ;
  \item soit dans le \File{\characteristicsfile} prévu à cet effet ;
  \item soit entre |\shorthandon{;:!?}| et |\shorthandoff{;:!?}| si on tient
    absolument à ce qu'ils soient saisis en préambule.
  \end{itemize}
\item les termes du glossaire, des acronymes ou des symboles, il suffit de
  définir les entrées correspondantes ou d'utiliser la ou les commandes
  \docAuxCommand{loadglsentries} :
  \begin{itemize}
  \item soit dans le \File{\configurationfile}
    (cf. \vref{rq-configurationfile}) ;
  \item soit entre |\shorthandon{;:!?}| et |\shorthandoff{;:!?}|. Cette solution
    peut être préférée à la précédente pour ne pas perdre les fonctionnalités de
    complétion pour les labels des termes de glossaire fournies par certains
    éditeurs de texte orientés \LaTeX{}.
  \end{itemize}
\end{enumerate}

\begin{dbfaq}{Pourquoi \protect\docAuxCommand*{setcounter} n'a-t-elle pas
    d'effet sur \protect\docAuxKey*{secnumdepth} ?}{}
  \index{profondeur!numérotation des unités}%
  \index{numérotation!des unités!profondeur}%
  J'essaie de modifier la profondeur de numérotation de mon document en
  spécifiant la valeur du compteur \docAuxKey*{secnumdepth} au moyen de la
  commande :
\begin{preamblecode}
\setcounter{secnumdepth}{"\meta{nombre}"}
\end{preamblecode}
  mais cela n'a aucun effet. Pourquoi ?
\end{dbfaq}

La profondeur de numérotation d'un document composé avec la \yatCl{} est
à spécifier au moyen de l'option de classe
\refKey{secnumdepth}. Cf. \vref{sec-profondeur-de-la} pour plus de précisions.

\section{Validation}
\label{sec-validation}

\begin{dbfaq}{Le \glsxtrshort{pdf} de mon mémoire n'est pas valide au yeux du
    \glsxtrshort{cines}. Comment y remédier ?}{}
  \index{pdf@\glsxtrshort{pdf}!valide} \index{validité!pdf@\glsxtrshort{pdf}}
  Conformément aux dispositions propres au dépôt sur support électronique
  \autocite{guidoct-abes}, j'ai testé sur le site \url{https://facile.cines.fr/}
  la validité du fichier \glsxtrshort{pdf} de mon mémoire de thèse créé avec la
  \yatCl{}, et il s'avère que celui-ci n'est pas valide. Comment y remédier ?
\end{dbfaq}

(Cette question ne concerne pas directement la \yatCl{}.) Le problème
vient de ce que le site \url{https://facile.cines.fr/} reconnaît mal les
méta-données des fichiers \glsxtrshort{pdf} produits par \hologo{pdfLaTeX},
\hologo{XeLaTeX} ou \hologo{LuaLaTeX}.
  %
Pour pallier cela, il devrait suffire\footnote{Plus de précisions à l'adresse
  \url{https://facile.cines.fr/\#latex}.} d'insérer en introduction du fichier
(maître) \file{.tex}, avant même la déclaration \docAuxCommand{documentclass} :
\begin{preamblecode}
\pdfobjcompresslevel 0
\end{preamblecode}

%%% Local Variables:
%%% mode: latex
%%% TeX-master: "../yathesis-fr"
%%% End:
