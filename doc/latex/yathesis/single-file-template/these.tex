% Document de classe yathesis
\documentclass{yathesis}
%
% Chargement manuel de packages (pas déjà chargés par la classe yathesis)
\usepackage[T1]{fontenc}
\usepackage[utf8]{inputenc}
\usepackage{kpfonts}
\usepackage{booktabs}
\usepackage{siunitx}
\usepackage{pgfplots}
\usepackage{floatrow}
\usepackage{caption}
\usepackage{microtype}
\usepackage{varioref}
%\usepackage[xindy,quiet]{imakeidx}
%\usepackage[autostyle]{csquotes}
%\usepackage[backend=biber,safeinputenc]{biblatex}
\usepackage{hyperref}
%\usepackage[xindy,acronyms,symbols]{glossaries}
%
% (Facultatif) Génération de l'index (obligatoire si un package d'index, par
% exemple « imakeidx », est chargé):
% \makeindex
%
% (Facultatif) Spécification de la ou des ressources bibliographiques
% (obligatoire si le package « biblatex » est chargé) :
% \addbibresource{}
% \addbibresource{}
%
% (Facultatif) Génération du glossaire (obligatoire si le package « glossaries »
% est chargé) :
% \makeglossaries
%
% (Facultatif) Spécification de la ou des ressources terminologiques :
% \loadglsentries{}
% \loadglsentries{}
% \loadglsentries{}
%
% (Facultatif) Configuration des styles du glossaire et de la liste
% d'acronymes :
% \setglossarystyle{}
% \setacronymstyle{}
%
%%%%%%%%%%%%%%%%%%%%%%%%%%%%%%%%%%%%%%%%%%%%%%%%%%%%%%%%%%%%%%%%%%%%%%%%%%%%%%%
%%%%%%%%%%%%%%%%%%%%%%%%%%%%%%%%%%%%%%%%%%%%%%%%%%%%%%%%%%%%%%%%%%%%%%%%%%%%%%%
% Début du document
%%%%%%%%%%%%%%%%%%%%%%%%%%%%%%%%%%%%%%%%%%%%%%%%%%%%%%%%%%%%%%%%%%%%%%%%%%%%%%%
%%%%%%%%%%%%%%%%%%%%%%%%%%%%%%%%%%%%%%%%%%%%%%%%%%%%%%%%%%%%%%%%%%%%%%%%%%%%%%%
\begin{document}
%
%%%%%%%%%%%%%%%%%%%%%%%%%%%%%%%%%%%%%%%%%%%%%%%%%%%%%%%%%%%%%%%%%%%%%%%%%%%%%%%
% Caractéristiques du document
%%%%%%%%%%%%%%%%%%%%%%%%%%%%%%%%%%%%%%%%%%%%%%%%%%%%%%%%%%%%%%%%%%%%%%%%%%%%%%%
%
% Préparation des pages de couverture et de titre
%%%%%%%%%%%%%%%%%%%%%%%%%%%%%%%%%%%%%%%%%%%%%%%%%%%%%%%%%%%%%%%%%%%%%%%%%%%%%%%
% Auteur de la thèse : prénom (1er argument obligatoire), nom (2e argument
% obligatoire) et éventuel courriel (argument optionnel). Les éventuels accents
% devront figurer et le nom /ne/ doit /pas/ être saisi en capitales :
\author[]{}{}
%
% Titre de la thèse dans la langue principale (argument obligatoire) et dans la
% langue secondaire (argument optionnel) :
\title[]{}
%
% (Facultatif) Sous-titre de la thèse dans la langue principale (argument
% obligatoire) et dans la langue secondaire (argument optionnel) :
% \subtitle[]{}
%
% Champ disciplinaire dans la langue principale (argument obligatoire) et dans
% la langue secondaire (argument optionnel) :
\academicfield[]{}
%
% (Facultatif) Spécialité dans la langue principale (argument obligatoire) et
% dans la langue secondaire (argument optionnel) :
\speciality[]{}
%
% Date de la soutenance, au format {jour}{mois}{année} donnés sous forme de
% nombres :
\date{}{}{}
%
% (Facultatif) Date de la soumission, au format {jour}{mois}{année} donnés sous
% forme de nombres :
%\submissiondate{}{}{}
%
% (Facultatif) Sujet pour les méta-données du PDF :
\subject[]{}
%
% (Facultatif) Nom (argument obligatoire) du PRES :
\pres[logo=,url=]{}
%
% Nom (argument obligatoire) de l'institut (principal en cas de cotutelle) :
\institute[logo=,url=]{}
%
% (Facultatif) En cas de cotutelle (normalement, seulement dans le cas de
% cotutelle internationale), nom (argument obligatoire) du second institut :
% \coinstitute[logo=]{}
%
% (Facultatif) Nom (argument obligatoire) de l'école doctorale :
\doctoralschool[url=]{}
%
% Nom (1er argument obligatoire) et adresse (2e argument obligatoire) du
% laboratoire (ou de l'unité) où la thèse a été préparée, à utiliser /autant de
% fois que nécessaire/ :
\laboratory[
logo=,
telephone=,
fax=,
email=,
url=
]{}{%
  \\
  \\
  \\
  \\
  \\
  }
%
% Directeur(s) de thèse et membres du jury, saisis au moyen des commandes
% \supervisor, \cosupervisor, \comonitor, \referee, \committeepresident,
% \examiner, \guest, à utiliser /autant de fois que nécessaire/ et /seulement
% si nécessaire/. Toutes basées sur le même modèle, ces commandes ont
% 2 arguments obligatoires, successivement les prénom et nom de chaque
% personne. Si besoin est, on peut apporter certaines précisions en argument
% optionnel, essentiellement au moyen des clés suivantes :
% - « professor », « seniorresearcher », « mcf », « mcf* »,
%   « juniorresearcher », « juniorresearcher* » (qui peuvent ne pas prendre de
%   valeur) pour stipuler le corps auquel appartient la personne ;
% - « affiliation » pour stipuler l'institut auquel est affiliée la personne ;
% - « female » pour stipuler que la personne est une femme pour que certains
%   mots clés soient accordés en genre.
%
\supervisor[,affiliation=]{}{}
% \cosupervisor[,affiliation=]{}{}
% \comonitor[,affiliation=]{}{}
\referee[,affiliation=]{}{}
\referee[,affiliation=]{}{}
\committeepresident[,affiliation=]{}{}
\examiner[,affiliation=]{}{}
\examiner[,affiliation=]{}{}
\examiner[,affiliation=]{}{}
% \guest{}{}
%
% (Facultatif) Mention du numéro d'ordre de la thèse (s'il est connu, ce numéro
% est à spécifier en argument optionnel) :
% \ordernumber[]
%
% Préparation des mots clés dans la langue principale (1er argument) et dans la
% langue secondaire (2e argument)
%%%%%%%%%%%%%%%%%%%%%%%%%%%%%%%%%%%%%%%%%%%%%%%%%%%%%%%%%%%%%%%%%%%%%%%%%%%%%%%
\keywords{}{}
%
% Production des pages de couverture et de titre
%%%%%%%%%%%%%%%%%%%%%%%%%%%%%%%%%%%%%%%%%%%%%%%%%%%%%%%%%%%%%%%%%%%%%%%%%%%%%%%
\maketitle
%
%%%%%%%%%%%%%%%%%%%%%%%%%%%%%%%%%%%%%%%%%%%%%%%%%%%%%%%%%%%%%%%%%%%%%%%%%%%%%%%
% Début de la partie liminaire de la thèse
%%%%%%%%%%%%%%%%%%%%%%%%%%%%%%%%%%%%%%%%%%%%%%%%%%%%%%%%%%%%%%%%%%%%%%%%%%%%%%%
%
% (Facultatif) Production de la page de clause de non-responsabilité :
\makedisclaimer
%
% (Facultatif) Production de la page de mots clés :
\makekeywords
%
% (Facultatif) Production de la page affichant les logo, nom et coordonnées du
% ou des laboratoires (ou unités de recherche) où la thèse a été préparée :
\makelaboratory
%
% (Facultatif) Dédicace(s) :
\dedication{}
\dedication{}
% (Facultatif) Production de la page de dédicace(s) :
\makededications
%
% (Facultatif) Épigraphe(s) :
\frontepigraph{}{}
\frontepigraph{}{}
% (Facultatif) Production de la page de d'épigraphe(s) :
\makefrontepigraphs
%
% Résumés (de 1700 caractères maximum, espaces compris) dans la
% langue principale (1re occurrence de l'environnement « abstract »)
% et, facultativement, dans la langue secondaire (2e occurrence de
% l'environnement « abstract ») :
\begin{abstract}
% ...
\end{abstract}
\begin{abstract}
% ...
\end{abstract}
%
% Production de la page de résumés :
\makeabstract
%
% (Facultatif) Chapitre de remerciements :
\chapter{Remerciements}
% ...
%
% (Facultatif) Chapitre d'avertissement :
% \chapter{Avertissement}
% ...
%
% (Facultatif) Liste des acronymes :
% \printacronyms
%
% (Facultatif) Liste des symboles :
% \printsymbols
%
% (Facultatif) Chapitre d'avant-propos :
% \chapter{Avant-propos}
% ...
%
% Sommaire
\tableofcontents[depth=chapter,name=Sommaire]
%
% (Facultatif) Liste des tableaux :
\listoftables
%
% (Facultatif) Table des figures :
\listoffigures
%
% (Facultatif) Table des listings (nécessite que le package « listings » soit
% chargé) :
% \lstlistoflistings
%
%%%%%%%%%%%%%%%%%%%%%%%%%%%%%%%%%%%%%%%%%%%%%%%%%%%%%%%%%%%%%%%%%%%%%%%%%%%%%%%
% Début de la partie principale (du « corps ») de la thèse
%%%%%%%%%%%%%%%%%%%%%%%%%%%%%%%%%%%%%%%%%%%%%%%%%%%%%%%%%%%%%%%%%%%%%%%%%%%%%%%
\mainmatter
%
% Chapitre d'introduction (générale)
%%%%%%%%%%%%%%%%%%%%%%%%%%%%%%%%%%%%%%%%%%%%%%%%%%%%%%%%%%%%%%%%%%%%%%%%%%%%%%%
\chapter*{Introduction}
% ...
%
% Chapitres ordinaires (avec parties éventuelles)
%%%%%%%%%%%%%%%%%%%%%%%%%%%%%%%%%%%%%%%%%%%%%%%%%%%%%%%%%%%%%%%%%%%%%%%%%%%%%%%
%
% Première partie éventuelle
% \part{...}
%
% Premier chapitre
% \chapter{...}
% ...
% Deuxième chapitre
% \chapter{...}
% ...
% Troisième chapitre
% \chapter{...}
% ...
%
%
% Deuxième partie éventuelle
% \part{...}
%
% Quatrième chapitre
% \chapter{...}
% ...
% Cinquième chapitre
% \chapter{...}
% ...
% Sixième chapitre
% \chapter{...}
% ...
%
% Chapitre  de conclusion (générale)
%%%%%%%%%%%%%%%%%%%%%%%%%%%%%%%%%%%%%%%%%%%%%%%%%%%%%%%%%%%%%%%%%%%%%%%%%%%%%%%
\chapter*{Conclusion}
% ...
%
% Liste des références bibliographiques
%\printbibliography
%
%%%%%%%%%%%%%%%%%%%%%%%%%%%%%%%%%%%%%%%%%%%%%%%%%%%%%%%%%%%%%%%%%%%%%%%%%%%%%%%
% Début de la partie annexe éventuelle
%%%%%%%%%%%%%%%%%%%%%%%%%%%%%%%%%%%%%%%%%%%%%%%%%%%%%%%%%%%%%%%%%%%%%%%%%%%%%%%
% \appendix
%
% Premier chapitre annexe (éventuel)
% \chapter{...}
% ...
% Deuxième chapitre annexe (éventuel)
% \chapter{...}
% ...
%
%%%%%%%%%%%%%%%%%%%%%%%%%%%%%%%%%%%%%%%%%%%%%%%%%%%%%%%%%%%%%%%%%%%%%%%%%%%%%%%
% Début de la partie finale
%%%%%%%%%%%%%%%%%%%%%%%%%%%%%%%%%%%%%%%%%%%%%%%%%%%%%%%%%%%%%%%%%%%%%%%%%%%%%%%
\backmatter
%
% (Facultatif) Glossaire (si souhaité distinct de la liste des acronymes) :
% \printglossary
%
% (Facultatif) Index :
% \printindex
%
% Table des matières
\tableofcontents
%
% (Facultatif) Production de la 4e de couverture :
\makebackcover
%
\end{document}
