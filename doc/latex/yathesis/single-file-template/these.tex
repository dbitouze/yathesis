% Document de classe yathesis
\documentclass{yathesis}
%
% Chargement manuel de packages (pas déjà chargés par la classe yathesis)
\usepackage[utf8]{inputenc}
\usepackage[T1]{fontenc}
\usepackage{kpfonts}
\usepackage{booktabs}
\usepackage{siunitx}
\usepackage{pgfplots}
\usepackage{floatrow}
\usepackage{caption}
\usepackage{microtype}
\usepackage[autostyle]{csquotes}
\usepackage[backend=biber,safeinputenc]{biblatex}
% Pour un éventuel glossaire, une liste d'acronymes et/ou une liste de
% symboles
% \usepackage[xindy,acronyms,symbols,toc]{glossaries}
%
% Spécification de la ou des ressources bibliographiques
\addbibresource{}
% \addbibresource{}
%
% Configuration des styles du glossaire et de la liste d'acronymes, si
% souhaitée. Sinon, il suffit de commenter ou de supprimer les lignes
% suivantes.
% \setglossarystyle{}
% \setacronymstyle{}
%
% Génération du glossaire (obligatoire si le package « glossaries » est
% chargé), si souhaitée. Sinon, il suffit de commenter ou de supprimer la
% ligne suivante.
% \makeglossaries
%
% Spécification de la ou des ressources terminologiques, si souhaitée. Sinon,
% il suffit de commenter ou de supprimer les lignes suivantes.
% \loadglsentries{}
% \loadglsentries{}
% \loadglsentries{}
%
% Génération de l'index (obligatoire si un index est souhaité), si
% souhaitée. Sinon, il suffit de commenter ou de supprimer la ligne suivante.
% \makeindex
%
%%%%%%%%%%%%%%%%%%%%%%%%%%%%%%%%%%%%%%%%%%%%%%%%%%%%%%%%%%%%%%%%%%%%%%%%%%%%%%%
%%%%%%%%%%%%%%%%%%%%%%%%%%%%%%%%%%%%%%%%%%%%%%%%%%%%%%%%%%%%%%%%%%%%%%%%%%%%%%%
% Début du document
%%%%%%%%%%%%%%%%%%%%%%%%%%%%%%%%%%%%%%%%%%%%%%%%%%%%%%%%%%%%%%%%%%%%%%%%%%%%%%%
%%%%%%%%%%%%%%%%%%%%%%%%%%%%%%%%%%%%%%%%%%%%%%%%%%%%%%%%%%%%%%%%%%%%%%%%%%%%%%%
\begin{document}
%
%%%%%%%%%%%%%%%%%%%%%%%%%%%%%%%%%%%%%%%%%%%%%%%%%%%%%%%%%%%%%%%%%%%%%%%%%%%%%%%
% Caractéristiques du document
%%%%%%%%%%%%%%%%%%%%%%%%%%%%%%%%%%%%%%%%%%%%%%%%%%%%%%%%%%%%%%%%%%%%%%%%%%%%%%%
%
% Préparation des pages de couverture et de titre
%%%%%%%%%%%%%%%%%%%%%%%%%%%%%%%%%%%%%%%%%%%%%%%%%%%%%%%%%%%%%%%%%%%%%%%%%%%%%%%
% Auteur de la thèse : prénom (1er argument obligatoire), nom (2e argument
% obligatoire) et éventuel courriel (argument optionnel). Les éventuels accents
% devront figurer et le nom /ne/ doit /pas/ être saisi en capitales.
\author[]{}{}
%
% Titre de la thèse dans la langue principale (argument obligatoire) et dans la
% langue secondaire (argument optionnel).
\title[]{}
%
% Sous-titre de la thèse dans la langue principale (argument obligatoire) et
% dans la langue secondaire (argument optionnel). À décommenter si souhaité,
% à commenter ou à supprimer sinon.
% \subtitle[]{}
%
% Champ disciplinaire dans la langue principale (argument obligatoire) et dans
% la langue secondaire (argument optionnel).
\academicfield[]{}
%
% Spécialité dans la langue principale (argument obligatoire) et dans la langue
% secondaire (argument optionnel). À décommenter si souhaitée, à commenter ou
% à supprimer sinon.
\speciality[]{}
%
% Date de la soutenance, au format {jour}{mois}{année} donnés sous forme de
% nombres.
\date{}{}{}
%
% Sujet pour les méta-données du PDF. À décommenter si souhaité, à commenter ou
% à supprimer sinon.
\subject[]{}
%
% Nom (argument obligatoire) du PRES. À décommenter si souhaité, à commenter ou
% à supprimer sinon.
\pres[logo=,url=]{}
%
% Nom (argument obligatoire) de l'institut (principal en cas de cotutelle).
\institute[logo=,url=]{}
%
% En cas de cotutelle (normalement, seulement dans le cas de cotutelle
% internationale), nom (argument obligatoire) du second
% institut. À décommenter si souhaité, à commenter ou à supprimer sinon.
% \coinstitute[logo=]{}
%
% Nom (argument obligatoire) de l'école doctorale. À décommenter si souhaitée,
% à commenter ou à supprimer sinon.
\doctoralschool[url=]{}
%
% Nom (1er argument obligatoire) et adresse (2e argument obligatoire) du
% laboratoire (ou de l'unité) où la thèse a été préparée, à utiliser /autant de
% fois que nécessaire/.
\laboratory[
logo=,
telephone=,
fax=,
email=,
url=
]{}{%
  \\
  \\
  \\
  \\
  \\
  }
%
% Membres du jury, saisis au moyen des commandes \supervisor, \cosupervisor,
% \comonitor, \referee, \committeepresident, \examiner, \guest, à utiliser
% /autant de fois que nécessaire/ et /seulement si nécessaire/. Toutes basées sur
% le même modèle, ces commandes ont 2 arguments obligatoires, successivement les
% prénom et nom de chaque personne. Si besoin est, on peut apporter certaines
% précisions en argument optionnel, au moyen des clés suivantes :
% - « professor », « seniorresearcher », « mcf », « mcf* »,
%   « juniorresearcher », « juniorresearcher* » (qui peuvent ne pas prendre de
%   valeur) pour stipuler le corps auquel appartient la personne ;
% - « affiliation » pour stipuler l'institut auquel est affiliée la personne.
%
\supervisor[professor,affiliation=]{}{}
% \cosupervisor[mcf*,affiliation=]{}{}
% \comonitor[mcf,affiliation=]{}{}
\referee[professor,affiliation=]{}{}
\referee[professor,affiliation=]{}{}
\committeepresident[professor,affiliation=]{}{}
\examiner[affiliation=]{}{}
\examiner[affiliation=]{}{}
\examiner[affiliation=]{}{}
% \guest{}{}
%
% Mention du numéro d'ordre de la thèse (s'il est connu, ce numéro est
% à spécifier en argument optionnel). À décommenter si souhaité, à commenter
% ou à supprimer sinon.
% \ordernumber[]
%
% Préparation des mots clés dans la langue principale (1er argument) et dans la
% langue secondaire (2e argument)
%%%%%%%%%%%%%%%%%%%%%%%%%%%%%%%%%%%%%%%%%%%%%%%%%%%%%%%%%%%%%%%%%%%%%%%%%%%%%%%
\keywords{}{}
%
% Production des pages de couverture et de titre
%%%%%%%%%%%%%%%%%%%%%%%%%%%%%%%%%%%%%%%%%%%%%%%%%%%%%%%%%%%%%%%%%%%%%%%%%%%%%%%
\maketitle
%
%%%%%%%%%%%%%%%%%%%%%%%%%%%%%%%%%%%%%%%%%%%%%%%%%%%%%%%%%%%%%%%%%%%%%%%%%%%%%%%
% Début de la partie liminaire de la thèse
%%%%%%%%%%%%%%%%%%%%%%%%%%%%%%%%%%%%%%%%%%%%%%%%%%%%%%%%%%%%%%%%%%%%%%%%%%%%%%%
%
% Production de la page de clause de non-responsabilité. À décommenter si
% souhaitée, à commenter ou à supprimer sinon.
\makedisclaimer
%
% Production de la page de mots clés. À décommenter si souhaitée, à commenter
% ou à supprimer sinon.
\makekeywords
%
% Production de la page affichant les logo, nom et coordonnées du ou des
% laboratoires (ou unités de recherche) où la thèse a été
% préparée. À décommenter si souhaitée, à commenter ou à supprimer sinon.
\makelaboratory
%
% Dédicace(s). À décommenter si souhaitée(s), à commenter ou à supprimer
% sinon.
\dedication{}
\dedication{}
% Production de la page de dédicace(s). À décommenter si souhaitée, à commenter
% ou à supprimer sinon.
\makededications
%
% Épigraphes(s). À décommenter si souhaitée(s), à commenter ou à supprimer
% sinon.
\frontepigraph{}{}
\frontepigraph{}{}
% Production de la page d'épigraphe(s). À décommenter si souhaitée, à commenter
% ou à supprimer sinon.
\makefrontepigraphs
%
% Chapitre de remerciements. À décommenter si souhaité, à commenter ou
% à supprimer sinon.
\chapter{Remerciements}
% ...
%
% Chapitre d'avertissement. À décommenter si souhaité, à commenter ou
% à supprimer sinon.
% \chapter{Avertissement}
% ...
%
% Résumés (de 1700 caractères maximum, espaces compris) dans la
% langue principale (1re occurrence de l'environnement « abstract »)
% et, facultativement, dans la langue secondaire (2e occurrence de
% l'environnement « abstract »).
\begin{abstract}
% ...
\end{abstract}
\begin{abstract}
% ...
\end{abstract}
%
% Production de la page de résumés.
\makeabstract
%
% Liste des acronymes. À décommenter si souhaitée, à commenter ou à supprimer
% sinon.
% \printacronyms
%
% Liste des symboles. À décommenter si souhaitée, à commenter ou à supprimer
% sinon.
% \printsymbols
%
% Chapitre d'avant-propos. À décommenter si souhaité, à commenter ou
% à supprimer sinon.
% \chapter{Avant-propos}
% ...
%
% Sommaire
\tableofcontents[depth=chapter,name=Sommaire]
%
% Liste des tableaux. À décommenter si souhaitée, à commenter ou à supprimer
% sinon.
\listoftables
%
% Table des figures. À décommenter si souhaitée, à commenter ou à supprimer
% sinon.
\listoffigures
%
% Table des listings (nécessite que le package « listings » soit
% chargé). À décommenter si souhaitée, à commenter ou à supprimer sinon.
% \lstlistoflistings
%
%%%%%%%%%%%%%%%%%%%%%%%%%%%%%%%%%%%%%%%%%%%%%%%%%%%%%%%%%%%%%%%%%%%%%%%%%%%%%%%
% Début de la partie principale (du « corps ») de la thèse
%%%%%%%%%%%%%%%%%%%%%%%%%%%%%%%%%%%%%%%%%%%%%%%%%%%%%%%%%%%%%%%%%%%%%%%%%%%%%%%
\mainmatter
%
% Chapitre d'introduction (générale)
%%%%%%%%%%%%%%%%%%%%%%%%%%%%%%%%%%%%%%%%%%%%%%%%%%%%%%%%%%%%%%%%%%%%%%%%%%%%%%%
\chapter*{Introduction}
% ...
%
% Chapitres ordinaires (avec parties éventuelles)
%%%%%%%%%%%%%%%%%%%%%%%%%%%%%%%%%%%%%%%%%%%%%%%%%%%%%%%%%%%%%%%%%%%%%%%%%%%%%%%
%
% Première partie éventuelle
% \part{...}
%
% Premier chapitre
% \chapter{...}
% ...
% Deuxième chapitre
% \chapter{...}
% ...
% Troisième chapitre
% \chapter{...}
% ...
%
%
% Deuxième partie éventuelle
% \part{...}
%
% Quatrième chapitre
% \chapter{...}
% ...
% Cinquième chapitre
% \chapter{...}
% ...
% Sixième chapitre
% \chapter{...}
% ...
%
% Chapitre  de conclusion (générale)
%%%%%%%%%%%%%%%%%%%%%%%%%%%%%%%%%%%%%%%%%%%%%%%%%%%%%%%%%%%%%%%%%%%%%%%%%%%%%%%
\chapter*{Conclusion}
% ...
%
% Liste des références bibliographiques
\printbibliography
%
%%%%%%%%%%%%%%%%%%%%%%%%%%%%%%%%%%%%%%%%%%%%%%%%%%%%%%%%%%%%%%%%%%%%%%%%%%%%%%%
% Début de la partie annexe éventuelle
%%%%%%%%%%%%%%%%%%%%%%%%%%%%%%%%%%%%%%%%%%%%%%%%%%%%%%%%%%%%%%%%%%%%%%%%%%%%%%%
% \appendix
%
% Premier chapitre annexe (éventuel)
% \chapter{...}
% ...
% Deuxième chapitre annexe (éventuel)
% \chapter{...}
% ...
%
%%%%%%%%%%%%%%%%%%%%%%%%%%%%%%%%%%%%%%%%%%%%%%%%%%%%%%%%%%%%%%%%%%%%%%%%%%%%%%%
% Début de la partie finale
%%%%%%%%%%%%%%%%%%%%%%%%%%%%%%%%%%%%%%%%%%%%%%%%%%%%%%%%%%%%%%%%%%%%%%%%%%%%%%%
\backmatter
%
% Glossaire (si souhaité distinct de la liste des acronymes). À décommenter si
% souhaité, à commenter ou à supprimer sinon.
% \printglossary
%
% Index. À décommenter si souhaité, à commenter ou à supprimer sinon.
% \printindex
%
% Table des matières
\tableofcontents
%
% Production de la 4e de couverture. À décommenter si souhaitée, à commenter ou
% à supprimer sinon.
\makebackcover
%
\end{document}
