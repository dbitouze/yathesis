% Auteur de la thèse : prénom (1er argument obligatoire), nom (2e argument
% obligatoire) et éventuel courriel (argument optionnel). Les éventuels accents
% devront figurer et le nom /ne/ doit /pas/ être saisi en capitales :
\author[]{}{}
%
% Titre de la thèse dans la langue principale (argument obligatoire) et dans la
% langue secondaire (argument optionnel) :
\title[]{}
%
% (Facultatif) Sous-titre de la thèse dans la langue principale (argument
% obligatoire) et dans la langue secondaire (argument optionnel) :
% \subtitle[]{}
%
% Champ disciplinaire dans la langue principale (argument obligatoire) et dans
% la langue secondaire (argument optionnel) :
\academicfield[]{}
%
% (Facultatif) Spécialité dans la langue principale (argument obligatoire) et
% dans la langue secondaire (argument optionnel) :
\speciality[]{}
%
% Date de la soutenance, au format {jour}{mois}{année} donnés sous forme de
% nombres :
\date{}{}{}
%
% (Facultatif) Date de la soumission, au format {jour}{mois}{année} donnés sous
% forme de nombres :
%\submissiondate{}{}{}
%
% (Facultatif) Sujet pour les méta-données du PDF :
\subject[]{}
%
% (Facultatif) Nom (argument obligatoire) du PRES :
\pres[logo=,url=]{}
%
% Nom (argument obligatoire) de l'institut (principal en cas de cotutelle) :
\institute[logo=,url=]{}
%
% (Facultatif) En cas de cotutelle (normalement, seulement dans le cas de
% cotutelle internationale), nom (argument obligatoire) du second institut :
% \coinstitute[logo=]{}
%
% (Facultatif) Nom (argument obligatoire) de l'école doctorale :
\doctoralschool[url=]{}
%
% Nom (1er argument obligatoire) et adresse (2e argument obligatoire) du
% laboratoire (ou de l'unité) où la thèse a été préparée, à utiliser /autant de
% fois que nécessaire/ :
\laboratory[
logo=,
telephone=,
fax=,
email=,
url=
]{}{%
  \\
  \\
  \\
  \\
  \\
  }
%
% Directeur(s) de thèse et membres du jury, saisis au moyen des commandes
% \supervisor, \cosupervisor, \comonitor, \referee, \committeepresident,
% \examiner, \guest, à utiliser /autant de fois que nécessaire/ et /seulement
% si nécessaire/. Toutes basées sur le même modèle, ces commandes ont
% 2 arguments obligatoires, successivement les prénom et nom de chaque
% personne. Si besoin est, on peut apporter certaines précisions en argument
% optionnel, essentiellement au moyen des clés suivantes :
% - « professor », « seniorresearcher », « mcf », « mcf* »,
%   « juniorresearcher », « juniorresearcher* » (qui peuvent ne pas prendre de
%   valeur) pour stipuler le corps auquel appartient la personne ;
% - « affiliation » pour stipuler l'institut auquel est affiliée la personne ;
% - « female » pour stipuler que la personne est une femme pour que certains
%   mots clés soient accordés en genre.
%
\supervisor[,affiliation=]{}{}
% \cosupervisor[,affiliation=]{}{}
% \comonitor[,affiliation=]{}{}
\referee[,affiliation=]{}{}
\referee[,affiliation=]{}{}
\committeepresident[,affiliation=]{}{}
\examiner[,affiliation=]{}{}
\examiner[,affiliation=]{}{}
\examiner[,affiliation=]{}{}
% \guest{}{}
%
% (Facultatif) Mention du numéro d'ordre de la thèse (s'il est connu, ce numéro
% est à spécifier en argument optionnel) :
% \ordernumber[]
%
% Préparation des mots clés dans la langue principale (1er argument) et dans la
% langue secondaire (2e argument)
%%%%%%%%%%%%%%%%%%%%%%%%%%%%%%%%%%%%%%%%%%%%%%%%%%%%%%%%%%%%%%%%%%%%%%%%%%%%%%%
\keywords{}{}
