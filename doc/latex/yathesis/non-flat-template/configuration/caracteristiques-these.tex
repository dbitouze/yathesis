% Auteur de la thèse : prénom (1er argument), nom (2e argument) et courriel (3e
% argument). Les éventuels accents devront figurer et le nom /ne/ sera /pas/
% saisi en capitales.
\author{Alphonse}{Allais}{aa@zygo.fr}
%
% Titre de la thèse dans la langue principale (argument obligatoire) et dans la
% langue secondaire (argument optionnel).
\title[Laugh's Chaos]{Le chaos du rire}
%
% Sous-titre de la thèse dans la langue principale (argument obligatoire) et
% dans la langue secondaire (argument optionnel), si souhaité. Sinon, il suffit
% de commenter ou supprimer la ligne suivante.
\subtitle[Chaos' laugh]{Le rire du chaos}
%
% Spécialité dans la langue principale (argument obligatoire) et dans la langue
% secondaire (argument optionnel) si souhaitée. Sinon, il suffit de commenter
% ou supprimer la ligne suivante.
\speciality[Mathematics]{Mathématiques}
%
% Date de la soutenance, au format {jour}{mois}{année} donnés sous forme de
% nombres, si souhaitée. Sinon, il suffit de commenter ou supprimer la ligne
% suivante.
\date{1}{1}{2014}
%
% Sujet pour les méta-données du PDF
\subject{Rire chaotique}
%
% (Tous les fichiers images stipulés ci-dessous comme valeurs des clés « logo »
% se trouvent dans le sous-dossier « images ». Il est conseillé de disposer
% d'images à un format vectoriel, par exemple PDF.)
%
% Logo du PRES, si souhaité. Sinon, il suffit de commenter ou de supprimer la
% ligne suivante.
\pres[logo=images/pres]{Université Lille Nord de France}
%
% Institut (principal en cas de cotutelle).
\institute[logo=images/ulco,url=http://www.univ-littoral.fr/]{ULCO}
%
% En cas de cotutelle (normalement, seulement dans le cas de cotutelle
% internationale), second institut. Sinon, il suffit de commenter ou de
% supprimer la ligne suivante.
\coinstitute[logo=images/paris13,url=http://www.univ-paris13.fr/]{Université de Paris~13}
%
% Nom de l'école doctorale, si souhaité. Sinon, il suffit de commenter ou
% supprimer la ligne suivante.
\doctoralschool[url=http://edspi.univ-lille1.fr/]{EDSPI}
%
% Laboratoire (ou de l'unité) où la thèse a été préparée.
\laboratory[
logo=images/labo,
logoheight=1.25cm,
address=
  {Maison de la Recherche Blaise Pascal\newline
  50, rue Ferdinand Buisson\newline
  CS 80699 - 62228 Calais Cedex - France},
telephone=(33)(0)3 21 46 55 86,
fax=(33)(0)3 21 46 55 75,
email=secretariat@lmpa.univ-littoral.fr,
url=http://www-lmpa.univ-littoral.fr/
]{LMPA Joseph Liouville}
%
% Les membres du jury sont saisis au moyen des commandes \supervisor,
% \cosupervisor, \comonitor, \referee, \committeepresident, \examiner, \guest,
% utilisées /autant de fois que nécessaire/. Toutes basées sur le même modèle,
% ces commandes ont 2 arguments obligatoires, successivement les prénom et nom
% de chaque personne. Si besoin est, on peut apporter certaines précisions en
% argument optionnel, au moyen des clés suivantes :
%
% -- « professor », « seniorresearcher », « mcf », « mcfhdr »,
% « juniorresearcher », « juniorresearcherhdr » (qui ne prennent pas de valeur)
% pour stipuler que la personne appartient au corps des professeurs
% d'université ;
%
% -- « from » pour stipuler l'institut auquel est rattachée la personne.
%
\supervisor[professor,from=ULCO]{Michel}{de Montaigne}
\cosupervisor[mcfhdr,from=ULCO]{Charles}{Baudelaire}
\comonitor[mcf,from=ULCO]{Joseph}{Fourier}
\referee[professor,from=IHP]{René}{Descartes}
\referee[juniorresearcher,from=CNRS]{Denis}{Diderot}
\committeepresident[seniorresearcher,from=CNRS]{Victor}{Hugo}
\examiner[mcfhdr,from=Université de Paris~13]{Émile}{Zola}
\examiner[professor,from=ENS Lyon]{Sophie}{Germain}
\examiner[juniorresearcherhdr,from=ULCO]{Paul}{Verlaine}
\guest[seniorresearcher,from=CNRS]{Étienne}{de la Boétie}
%
% Mots clés dans la langue principale (argument obligatoire) et dans
% la langue secondaire (argument optionnel), si souhaités. Sinon, il
% suffit de commenter ou supprimer la ligne suivante.
\keywords[chaos, laugh]{chaos, rire}
%
% Numéro d'ordre de la thèse
\ordernumber{9999}
%
% Résumés (de 1700 caractères maximum, espaces compris) dans la
% langue principale (1re occurrence de l'environnement « abstract »)
% et, facultativement, dans la langue secondaire (2e occurrence de
% l'environnement « abstract »), si souhaités. Sinon, il suffit
% de commenter ou supprimer les lignes suivantes.
\begin{abstract}
  \lipsum[1-2]
\end{abstract}
\begin{abstract}
  \lipsum[3-4]
\end{abstract}
