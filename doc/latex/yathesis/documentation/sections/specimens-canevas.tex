\chapter{Specimens et canevas de thèse}\label{cha-specimen-canevas}%

% Pour aider à se figurer ses fonctionnalités et pour faciliter sa mise
% en œuvre, l
% La \yatcl fournit :% sous forme d'archives \gls{zip} :
% \begin{enumerate}
% \item un spécimen ;
% \item un canevas ;
% \end{enumerate}
 % sous forme d'archives \gls{zip} :
Un spécimen et un canevas de mémoires de thèse créés avec la \yatcl sont
fournis, chacun en deux versions selon la façon dont le source \file{.tex} est
organisé :
% \begin{enumerate}
% \item une \enquote{à plat} dans laquelle le source \file{.tex} est tout entier
%   dans un unique fichier, situé dans le même dossier que les fichiers annexes
%   (bibliographie et images) ;
% \item une \enquote{en arborescence} dans laquelle le source \file{.tex} est
%   scindé en fichiers maître et esclaves\footnote{Comme cela est en général
%     recommandé, cf. \vref{sec-repart-du-memo}.}, situés (ainsi que l'ensemble
%   des fichiers annexes) dans différents (sous-)dossiers.
% \end{enumerate}
\begin{description}
\item[une \enquote{à plat} :] le source est tout entier dans un unique fichier,
  situé dans le même dossier que les fichiers annexes (bibliographie et
  images) ;
\item[une \enquote{en arborescence} :] le source est scindé en fichiers maître
  et esclaves\footnote{Comme cela est en général recommandé,
    cf. \vref{sec-repart-du-memo}.}, situés (ainsi que l'ensemble des fichiers
  annexes) dans différents (sous-)dossiers.
\end{description}
Les deux specimens et deux canevas ainsi proposés ont pour but :
\begin{itemize}
\item d'illustrer les fonctionnalités de la classe ;
\item d'aider à la mise en œuvre de la classe en fournissant une base de départ
  que chacun peut progressivement adapter à ses propres
  besoins.% Pour ce faire, \emph{on copiera dans un
  %   répertoire personnel} son dossier (éventuellement en désarchivant son
  % archive \gls{zip} fournie) et on adaptera son contenu.
%   \begin{dbwarning}{Spécimens et canevas : à copier dans un répertoire
%       personnel}{}
%     Si on souhaite utiliser l'un de ces spécimens ou canevas, il est
%     \emph{essentiel} de \emph{ne pas} travailler directement avec (ou dans) le
%     dossier fourni par la distribution \TeX{} utilisée (\TeX{} Live ou
%     MiK\TeX{}) : toute adaptation personnelle risquerait en effet écrasée lors
%     d'une prochaine mise à jour de la \yatcl{}. Avant d'utiliser l'un de ces
%     dossiers (ou archives \gls{zip}), il est donc nécessaire de le copier
%     préalablement dans un répertoire personnel.
% \end{dbwarning}
\end{itemize}

La version électronique (\pdf{}) de la présente
documentation\footnote{Disponible à l'adresse
  \url{http://ctan.org/pkg/yathesis}, si besoin est.} intègre ces spécimens et
canevas par le biais d'archives \gls{zip} % qui sont alors
accessibles par simple clic sur les liens figurant au
\vref{tab-specimens-canevas}.

% \begin{figure}[ht]
%   \centering
%   \chapter{Canevas et spécimens de thèse}\label{cha-specimen-canevas}%

Un canevas et un spécimen de mémoires de thèse créés avec la \yatCl sont
fournis, chacun en deux versions selon la façon dont le source \file{.tex} est
organisé :
\begin{description}
\item[une \enquote{à plat} :] le source est tout entier dans un unique fichier,
  situé dans le même dossier que les fichiers annexes (bibliographie et
  images) ;
\item[une \enquote{en arborescence} :] le source est scindé en fichiers maître
  et esclaves\footnote{Comme cela est en général recommandé,
    cf. \vref{sec-repart-du-memo}.}, situés (ainsi que l'ensemble des fichiers
  annexes) dans différents (sous-)dossiers.
\end{description}
Les deux canevas et deux spécimens ainsi proposés ont pour but :
\begin{itemize}
\item d'aider à la mise en œuvre de la classe en fournissant une base de départ
  que chacun peut progressivement adapter à ses propres
  besoins ;
\item d'illustrer les fonctionnalités de la classe.
\end{itemize}

La version électronique (\pdf{}) de la présente
documentation\footnote{Disponible à l'adresse
  \url{http://ctan.org/pkg/yathesis}, si besoin est.} intègre ces canevas et
spécimens par le biais d'une archive \gls{zip}, normalement accessible par
simple clic sur le lien suivant :
\textattachfile{../exemples/canevas-specimen.zip}{\file{canevas-specimen.zip}}\footnote{En
  tous cas avec les afficheurs \pdf:{} \program{Evince} sous Linux et
  \href{http://www.sumatrapdfreader.org/free-pdf-reader-fr.html}{\program{SumatraPDF}}
  sous Windows.}. L'extraction
de cette archive fournit un dossier nommé \folder{exemples} dont l'arborescence
est la suivante :

\setlength{\DTbaselineskip}{15pt}
\begin{tcolorbox}
  \dirtree{%
  .1 \folder{exemples}/.
  .2 \folder{canevas}/.
  .3 \folder{a-plat}/.
  .3 \folder{en-arborescence}/.
  .2 \folder{specimen}/.
  .3 \folder{a-plat}/.
  .3 \folder{en-arborescence}/.
}%
\end{tcolorbox}

% La version électronique (\pdf{}) de la présente
% documentation\footnote{Disponible à l'adresse
%   \url{http://ctan.org/pkg/yathesis}, si besoin est.} intègre ces spécimens et
% canevas par le biais d'archives \gls{zip} % qui sont alors
% accessibles par simple clic sur les liens figurant au
% \vref{tab-canevas-specimen}.
% %
% \changes{v0.99m}{2016-05-22}{Réorganisation et changement de noms des spécimens
%   et canevas}%
% \changes{v0.99m}{2016-05-22}{(Archives \gls{zip} des)Spécimens et canevas
%   intégrés au \acrshort{pdf} de la documentation}%
% %
%   \begin{table}[ht]
%     \centering
%     \caption[Archives \gls{zip} des spécimens et canevas fournis]{(Liens vers
%       les )Archives \gls{zip} des spécimens et canevas fournis avec la
%       \yatCl{}}
%     \label{tab-canevas-specimen}
%     \input{tableaux/canevas-specimen-tab}
%   \end{table}
%
\changes{v0.99n}{2016-06-11}{Réorganisation des spécimens et canevas}%
\changes{v0.99m}{2016-05-22}{Réorganisation et changement de noms des spécimens
  et canevas}%
\changes{v0.99m}{2016-05-22}{Spécimens et canevas intégrés au \acrshort{pdf} de
  la documentation sous la forme d'archives \gls{zip}}%
\changes{v0.99l}{2014-10-23}{Réorganisation et changement de noms des spécimens
  et canevas}%^^A
\changes{v0.99c}{2014-06-06}{Spécimens et canevas fournis sous forme d'archives
  \file{.zip}}%^^A
\changes{v0.99b}{2014-06-02}{Réorganisation des spécimens et canevas}%^^A
\changes{v0.99a}{2014-06-02}{Spécimens et canevas enrichis}%^^A
\begin{comment}
  \begin{itemize}
  \item pour la distribution \TeX{} Live\versiontl, sur les systèmes :
    \begin{itemize}
    \item Linux et Mac OS X :
      \href{./.}{\folder{\unixtldirectory\tldistdirectory\jobdocdirectory/}} ;
    \item Windows :
      \href{./.}{\folder{\wintldirectory\tldistdirectory\jobdocdirectory/}} ;
    \end{itemize}
  \item pour la distribution MiK\TeX{} : \folder{\miktexdistdirectory}.
  \end{itemize}
\end{comment}

\begin{dbwarning}{Archive à extraire avant toute chose !}{}
  Pour pouvoir consulter et surtout tester sans problème les canevas et
  spécimens de l'archive \file{canevas-specimen.zip}, celle-ci \emph{doit} être
  extraite avant toute chose !
\end{dbwarning}

Les \vref{sec-canevas,sec-specimens} détaillent les fichiers qui constituent
chacun de ces canevas et spécimens.

% Parmi eux, \file{latexmkrc}, fichier de configuration du programme
% \program{latexmk} qui permet d'automatiser le processus de compilation
% complète de la thèse (cf. \vref{sec-autom-des-comp} pour plus de détails).

% Les \vref{sec-specimens,sec-canevas} détaillent davantage ces spécimens et
% canevas, en indiquant notamment tous les fichiers qui les constituent. Chacun
% des spécimens et canevas fournit un fichier, nommé \file{latexmkrc}, de
% configuration du programme \program{latexmk} qui permet d'automatiser le
% processus de compilation complète de la thèse (cf. \vref{sec-autom-des-comp}
% pour plus de détails).

% La commande à utiliser pour lister le contenu du répertoire est :%
% tree --charset=ascii -F -I \ %
% "*aux|*idx|*ind|*lof|*lot|*out|*toc|*acn|*acr|*alg|*bcf|*glg|*glo|*gls|*glg2|*gls2|*glo2|*ist|*run.xml|*xdy|*lol|*fls|*slg|*slo|*sls|*unq|*synctex.gz|*mw|*bbl|*blg|*fdb_latexmk|*log|*auto"

\section{Canevas}
\label{sec-canevas}

Les \emph{canevas} fournis (regroupés dans le dossier \folder{canevas}) ne sont
rien d'autre que les (quasi-)répliques des \emph{spécimens} correspondants dont
les données ont été vidées : pour les exploiter, il suffit donc de remplir les
\enquote{cases} vides.

\subsection{Canevas \enquote{à plat}}
\label{sec-canevas-a-plat}

Le dossier (\folder{canevas/a-plat}) de ce canevas ne contient que trois fichiers :
\begin{enumerate}
\item \file{these.tex}, source \file{.tex} (unique) de la thèse  ;
\item \file{these.pdf} produit par compilation du \File{these.tex} ;
\item \file{latexmkrc}.
\end{enumerate}

[TODO]

\subsection{Canevas \enquote{en arborescence}}
\label{sec-canevas-arborescence}

Le dossier (\folder{canevas/en-arborescence}) de ce canevas contient les fichiers :
\begin{enumerate}
\item ...
\item \file{latexmkrc}.
\end{enumerate}

[TODO]

\section{Spécimens}
\label{sec-specimens}

Sur la base de données plus ou moins fictives, de textes arbitraires et de
\gls{fauxtexte}, les spécimens (regroupés dans le dossier \folder{specimen})
mettent en évidence l'ensemble des possibilités offertes par la \yatCl{}.

\subsection{Spécimen \enquote{à plat}}
\label{sec-specimen-a-plat}

Le dossier (\folder{specimen/a-plat}) de ce spécimen contient les fichiers :
\begin{enumerate}
\item \file{these.tex} qui est le source \file{.tex} (unique) de la thèse ;
\item \file{bibliographie.bib}, contenant les références bibliographiques de
  la thèse ;
\item \file{these.pdf} produit par compilation du \File{these.tex} ;
\item \file{labo.pdf}, \file{paris13.pdf}, \file{pres.pdf}, \file{tiger.pdf},
  \file{ulco.pdf} (images : logos, etc.) ;
\item \file{latexmkrc}.
\end{enumerate}

[TODO]

\subsection{Spécimen \enquote{en arborescence}}
\label{sec-specimen-arborescence}

Le dossier (\folder{specimen/en-arborescence}) de ce spécimen contient les
fichiers :
\begin{enumerate}
\item ...
\item \file{latexmkrc}.
\end{enumerate}

[TODO]

%
\iffalse
%%% Local Variables:
%%% mode: latex
%%% eval: (latex-mode)
%%% ispell-local-dictionary: "fr_FR"
%%% TeX-master: "../yathesis-fr"
%%% End:
\fi

%   \caption{Dossiers, fichiers \acrshort{pdf} et archives \gls{zip} (et
%     hyperliens) des spécimens et canevas fournis}
%   \label{fig-specimens-canevas}
% \end{figure}

% ^^A\footnote{La scission du mémoire de thèse en fichiers maître et esclaves
% ^^Aest hautement recommandée : cf. \vref{sec-repart-du-memo}.}

% \begin{comment}
  \begin{table}[ht]
    \centering
    \caption[Archives \textsc{zip} des spécimens et canevas fournis]{(Liens vers
      les )Archives \textsc{zip} des spécimens et canevas fournis avec la
      \yatcl{}}
    \label{tab-specimens-canevas}
    \chapter{Canevas et spécimens de thèse}\label{cha-specimen-canevas}%

Un canevas et un spécimen de mémoires de thèse créés avec la \yatCl sont
fournis, chacun en deux versions selon la façon dont le source \file{.tex} est
organisé :
\begin{description}
\item[une \enquote{à plat} :] le source est tout entier dans un unique fichier,
  situé dans le même dossier que les fichiers annexes (bibliographie et
  images) ;
\item[une \enquote{en arborescence} :] le source est scindé en fichiers maître
  et esclaves\footnote{Comme cela est en général recommandé,
    cf. \vref{sec-repart-du-memo}.}, situés (ainsi que l'ensemble des fichiers
  annexes) dans différents (sous-)dossiers.
\end{description}
Les deux canevas et deux spécimens ainsi proposés ont pour but :
\begin{itemize}
\item d'aider à la mise en œuvre de la classe en fournissant une base de départ
  que chacun peut progressivement adapter à ses propres
  besoins ;
\item d'illustrer les fonctionnalités de la classe.
\end{itemize}

La version électronique (\pdf{}) de la présente
documentation\footnote{Disponible à l'adresse
  \url{http://ctan.org/pkg/yathesis}, si besoin est.} intègre ces canevas et
spécimens par le biais d'une archive \gls{zip}, normalement accessible par
simple clic sur le lien suivant :
\textattachfile{../exemples/canevas-specimen.zip}{\file{canevas-specimen.zip}}\footnote{En
  tous cas avec les afficheurs \pdf:{} \program{Evince} sous Linux et
  \href{http://www.sumatrapdfreader.org/free-pdf-reader-fr.html}{\program{SumatraPDF}}
  sous Windows.}. L'extraction
de cette archive fournit un dossier nommé \folder{exemples} dont l'arborescence
est la suivante :

\setlength{\DTbaselineskip}{15pt}
\begin{tcolorbox}
  \dirtree{%
  .1 \folder{exemples}/.
  .2 \folder{canevas}/.
  .3 \folder{a-plat}/.
  .3 \folder{en-arborescence}/.
  .2 \folder{specimen}/.
  .3 \folder{a-plat}/.
  .3 \folder{en-arborescence}/.
}%
\end{tcolorbox}

% La version électronique (\pdf{}) de la présente
% documentation\footnote{Disponible à l'adresse
%   \url{http://ctan.org/pkg/yathesis}, si besoin est.} intègre ces spécimens et
% canevas par le biais d'archives \gls{zip} % qui sont alors
% accessibles par simple clic sur les liens figurant au
% \vref{tab-canevas-specimen}.
% %
% \changes{v0.99m}{2016-05-22}{Réorganisation et changement de noms des spécimens
%   et canevas}%
% \changes{v0.99m}{2016-05-22}{(Archives \gls{zip} des)Spécimens et canevas
%   intégrés au \acrshort{pdf} de la documentation}%
% %
%   \begin{table}[ht]
%     \centering
%     \caption[Archives \gls{zip} des spécimens et canevas fournis]{(Liens vers
%       les )Archives \gls{zip} des spécimens et canevas fournis avec la
%       \yatCl{}}
%     \label{tab-canevas-specimen}
%     \input{tableaux/canevas-specimen-tab}
%   \end{table}
%
\changes{v0.99n}{2016-06-11}{Réorganisation des spécimens et canevas}%
\changes{v0.99m}{2016-05-22}{Réorganisation et changement de noms des spécimens
  et canevas}%
\changes{v0.99m}{2016-05-22}{Spécimens et canevas intégrés au \acrshort{pdf} de
  la documentation sous la forme d'archives \gls{zip}}%
\changes{v0.99l}{2014-10-23}{Réorganisation et changement de noms des spécimens
  et canevas}%^^A
\changes{v0.99c}{2014-06-06}{Spécimens et canevas fournis sous forme d'archives
  \file{.zip}}%^^A
\changes{v0.99b}{2014-06-02}{Réorganisation des spécimens et canevas}%^^A
\changes{v0.99a}{2014-06-02}{Spécimens et canevas enrichis}%^^A
\begin{comment}
  \begin{itemize}
  \item pour la distribution \TeX{} Live\versiontl, sur les systèmes :
    \begin{itemize}
    \item Linux et Mac OS X :
      \href{./.}{\folder{\unixtldirectory\tldistdirectory\jobdocdirectory/}} ;
    \item Windows :
      \href{./.}{\folder{\wintldirectory\tldistdirectory\jobdocdirectory/}} ;
    \end{itemize}
  \item pour la distribution MiK\TeX{} : \folder{\miktexdistdirectory}.
  \end{itemize}
\end{comment}

\begin{dbwarning}{Archive à extraire avant toute chose !}{}
  Pour pouvoir consulter et surtout tester sans problème les canevas et
  spécimens de l'archive \file{canevas-specimen.zip}, celle-ci \emph{doit} être
  extraite avant toute chose !
\end{dbwarning}

Les \vref{sec-canevas,sec-specimens} détaillent les fichiers qui constituent
chacun de ces canevas et spécimens.

% Parmi eux, \file{latexmkrc}, fichier de configuration du programme
% \program{latexmk} qui permet d'automatiser le processus de compilation
% complète de la thèse (cf. \vref{sec-autom-des-comp} pour plus de détails).

% Les \vref{sec-specimens,sec-canevas} détaillent davantage ces spécimens et
% canevas, en indiquant notamment tous les fichiers qui les constituent. Chacun
% des spécimens et canevas fournit un fichier, nommé \file{latexmkrc}, de
% configuration du programme \program{latexmk} qui permet d'automatiser le
% processus de compilation complète de la thèse (cf. \vref{sec-autom-des-comp}
% pour plus de détails).

% La commande à utiliser pour lister le contenu du répertoire est :%
% tree --charset=ascii -F -I \ %
% "*aux|*idx|*ind|*lof|*lot|*out|*toc|*acn|*acr|*alg|*bcf|*glg|*glo|*gls|*glg2|*gls2|*glo2|*ist|*run.xml|*xdy|*lol|*fls|*slg|*slo|*sls|*unq|*synctex.gz|*mw|*bbl|*blg|*fdb_latexmk|*log|*auto"

\section{Canevas}
\label{sec-canevas}

Les \emph{canevas} fournis (regroupés dans le dossier \folder{canevas}) ne sont
rien d'autre que les (quasi-)répliques des \emph{spécimens} correspondants dont
les données ont été vidées : pour les exploiter, il suffit donc de remplir les
\enquote{cases} vides.

\subsection{Canevas \enquote{à plat}}
\label{sec-canevas-a-plat}

Le dossier (\folder{canevas/a-plat}) de ce canevas ne contient que trois fichiers :
\begin{enumerate}
\item \file{these.tex}, source \file{.tex} (unique) de la thèse  ;
\item \file{these.pdf} produit par compilation du \File{these.tex} ;
\item \file{latexmkrc}.
\end{enumerate}

[TODO]

\subsection{Canevas \enquote{en arborescence}}
\label{sec-canevas-arborescence}

Le dossier (\folder{canevas/en-arborescence}) de ce canevas contient les fichiers :
\begin{enumerate}
\item ...
\item \file{latexmkrc}.
\end{enumerate}

[TODO]

\section{Spécimens}
\label{sec-specimens}

Sur la base de données plus ou moins fictives, de textes arbitraires et de
\gls{fauxtexte}, les spécimens (regroupés dans le dossier \folder{specimen})
mettent en évidence l'ensemble des possibilités offertes par la \yatCl{}.

\subsection{Spécimen \enquote{à plat}}
\label{sec-specimen-a-plat}

Le dossier (\folder{specimen/a-plat}) de ce spécimen contient les fichiers :
\begin{enumerate}
\item \file{these.tex} qui est le source \file{.tex} (unique) de la thèse ;
\item \file{bibliographie.bib}, contenant les références bibliographiques de
  la thèse ;
\item \file{these.pdf} produit par compilation du \File{these.tex} ;
\item \file{labo.pdf}, \file{paris13.pdf}, \file{pres.pdf}, \file{tiger.pdf},
  \file{ulco.pdf} (images : logos, etc.) ;
\item \file{latexmkrc}.
\end{enumerate}

[TODO]

\subsection{Spécimen \enquote{en arborescence}}
\label{sec-specimen-arborescence}

Le dossier (\folder{specimen/en-arborescence}) de ce spécimen contient les
fichiers :
\begin{enumerate}
\item ...
\item \file{latexmkrc}.
\end{enumerate}

[TODO]

%
\iffalse
%%% Local Variables:
%%% mode: latex
%%% eval: (latex-mode)
%%% ispell-local-dictionary: "fr_FR"
%%% TeX-master: "../yathesis-fr"
%%% End:
\fi

  \end{table}
% \end{comment}

%
\changes{v0.99l}{2014/10/23}{Réorganisation et changement de noms des spécimens
  et canevas}%^^A
\changes{v0.99c}{2014/06/06}{Spécimens et canevas fournis sous forme d'archives
  \file{.zip}}%^^A
\changes{v0.99b}{2014/06/02}{Réorganisation des spécimens et canevas}%^^A
\changes{v0.99a}{2014/06/02}{Spécimens et canevas enrichis}%^^A
\begin{comment}
  \begin{itemize}
  \item pour la distribution \TeX{} Live\versiontl, sur les systèmes :
    \begin{itemize}
    \item Linux et Mac OS X :
      \href{./.}{\folder{\unixtldirectory\tldistdirectory\jobdocdirectory/}} ;
    \item Windows :
      \href{./.}{\folder{\wintldirectory\tldistdirectory\jobdocdirectory/}} ;
    \end{itemize}
  \item pour la distribution MiK\TeX{} : \folder{\miktexdistdirectory}.
  \end{itemize}
\end{comment}

Les \vref{sec-specimens,sec-canevas} détaillent les fichiers qui constituent
chacun de ces specimens et canevas.

% ^^AParmi eux, \file{latexmkrc}, fichier de configuration du programme
% ^^A\program{latexmk} qui permet d'automatiser le processus de compilation
% ^^Acomplète de la thèse (cf. \vref{sec-autom-des-comp} pour plus de détails).

% ^^ALes \vref{sec-specimens,sec-canevas} détaillent davantage ces specimens et
% ^^Acanevas, en indiquant notamment tous les fichiers qui les
% ^^Aconstituent. Chacun des spécimens et canevas fournit un fichier, nommé
% ^^A\file{latexmkrc}, de configuration du programme \program{latexmk} qui
% ^^Apermet d'automatiser le processus de compilation complète de la thèse
% ^^A(cf. \vref{sec-autom-des-comp} pour plus de détails).

% ^^A La commande à utiliser pour lister le contenu du répertoire est :%
% ^^A tree -A -F -I \ %
% ^^A "*aux|*idx|*ind|*lof|*lot|*out|*toc|*acn|*acr|*alg|*bcf|*glg|*glo|*gls|*glg2|*gls2|*glo2|*ist|*run.xml|*xdy|*lol|*fls|*slg|*slo|*sls|*unq|*synctex.gz|*mw|*bbl|*blg|*fdb_latexmk|*log|*auto"

\section{Spécimens}
\label{sec-specimens}

Sur la base de données plus ou moins fictives, de textes arbitraires et de
\gls{fauxtexte}, les spécimens mettent en évidence l'ensemble des possibilités
offertes par la \yatcl{}.

\subsection[Spécimen \enquote{à plat}]{Spécimen \enquote{à plat} \attachfile{../exemples/specimen-a-plat.zip}}
\label{sec-specimen-a-plat}

% \def\repertoire{specimens}
% \def\subrepertoire{a-plat}

Le dossier de ce spécimen % (\folder{.../\repertoire/}\subdirectorytree{})
contient les fichiers :
\begin{enumerate}
\item \file{these.tex} qui est le source \file{.tex} (unique) de la thèse ;
\item \file{bibliographie.bib}, contenant les références bibliographiques de
  la thèse ;
\item \file{these.pdf} produit par compilation du \File{these.tex} ;
\item \file{labo.pdf}, \file{paris13.pdf}, \file{pres.pdf}, \file{tiger.pdf},
  \file{ulco.pdf} (images : logos, etc.) ;
\item \file{latexmkrc}.
\end{enumerate}

[TODO]

\subsection[Spécimen en \enquote{arborescence}]{Spécimen en \enquote{arborescence} \attachfile{../exemples/specimen-en-arborescence.zip}}
\label{sec-specimen-arborescence}

% \def\subdirectory{en-arborescence}

Le dossier de ce spécimen % (\folder{.../\directory/}\subdirectorytree)
contient les fichiers :
\begin{enumerate}
\item ...
\item \file{latexmkrc}.
\end{enumerate}

[TODO]

\section{Canevas}
\label{sec-canevas}

Les \emph{canevas} fournis ne sont rien d'autre que les (quasi-)répliques des
\emph{spécimens} correspondants dont les données ont été vidées : pour les
exploiter, il suffit donc de remplir les \enquote{cases} vides.

\subsection[Canevas \enquote{à plat}]{Canevas \enquote{à plat} \attachfile{../exemples/canevas-a-plat.zip}}
\label{sec-canevas-a-plat}

% \def\directory{canevas}
% \def\subdirectory{a-plat}

Le dossier de ce canevas % (\folder{.../\directory/}\subdirectorytree)
ne contient que trois fichiers :
\begin{enumerate}
\item \file{these.tex}, source \file{.tex} (unique) de la thèse  ;
\item \file{these.pdf} produit par compilation du \File{these.tex} ;
\item \file{latexmkrc}.
\end{enumerate}

[TODO]

\subsection[Canevas en \enquote{arborescence}]{Canevas en \enquote{arborescence} \attachfile{../exemples/canevas-en-arborescence.zip}}
\label{sec-canevas-arborescence}

% \def\subdirectory{en-arborescence}

Le dossier de ce spécimen % (\folder{.../\directory/}\subdirectorytree)
contient les fichiers :
\begin{enumerate}
\item ...
\item \file{latexmkrc}.
\end{enumerate}

[TODO]

%
\iffalse
%%% Local Variables:
%%% mode: latex
%%% eval: (latex-mode)
%%% ispell-local-dictionary: "fr_FR"
%%% TeX-master: "../yathesis"
%%% End:
\fi
