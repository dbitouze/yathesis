\chapter{Introduction}

\section{Objet de la \yatcl}
\label{sec-objet-de-la}

\LaTeX{} est un système particulièrement performant de préparation et de
production de toutes sortes de documents : rapports de stage, mémoires de
\emph{master} et de thèses, polycopiés de cours, rapports d'activité, etc.

Les outils standards ou généralistes de \LaTeX{} tels que les classes
\class{book} ou \class*{memoir} n'étant pas calibrés pour répondre aux exigences
particulières des mémoires de thèse, de nombreuses classes spécifiques ont été
créées\footnote{Cf. \url{http://ctan.org/topic/dissertation}.}  et sont livrées
avec toute distribution \TeX{} moderne. Toutefois, la plupart d'entre elles ne
sont pas destinées aux thèses préparées en France et sont souvent propres à une
université donnée.

Parmi les exceptions notables figurent :
\begin{itemize}
\item la \Class*{droit-fr}, destinée aux thèses en droit préparées en France ;
\item la \Class*{ulthese}, destinée aux thèses francophones préparées
  à l'Université Laval (Canada) ;
\item la \Class*[http://www.loria.fr/~roegel/TeX/TUL.html]{thesul}, destinée
  initialement aux thèses en informatique préparées à l'Université de Lorraine,
  mais aisément adaptable à tout autre champ disciplinaire et institut en
  France. Cette classe n'est toutefois pas fournie par les distributions \TeX{}
  et nécessite d'être installée manuellement.
\end{itemize}

La présente classe, \yat, a pour objet de faciliter la composition de mémoires
de thèses préparées en France, quels que soient les champs disciplinaires et
instituts. Elle implémente notamment l'essentiel des recommandations émanant du
\citeauthor{guidoct} \autocite{guidoct} et ce, de façon transparente pour
l'utilisateur.  Elle a en outre été conçue pour (facultativement) tirer profit
de plusieurs outils récents et puissants disponibles sous \LaTeX{}, notamment :
\begin{itemize}
\item la bibliographie avec le \Package*{biblatex} ;
\item les glossaire, liste d'acronymes et liste de symboles avec le
  \Package*{glossaries}.
\end{itemize}
% ^^A, mais peut éventuellement être adaptée à d'autres pays.

La \yatcl{}, basée sur la \Class{book}, se veut à la fois simple d'emploi et,
dans une certaine mesure, (aisément) personnalisable.

% ^^AMais sur ce dernier point, ce que fait observer
% ^^A\citefirstlastauthor{thesul} au sujet de sa \Class{thesul}, s'applique
% ^^Aégalement à la \yatcl{} : ^^A %
% ^^A\blockcquote{thesul}{%
% ^^ALa \Class{thesul} fait partie des classes de la gamme
% ^^A\enquote{prêt-à-porter}.  Elle satisfait un certain nombre de besoins, mais
% ^^Apas tous les besoins. C'est une classe faite pour ceux qui veulent
% ^^Autiliser un outil au prix d'un nombre très restreint (voire nul) de
% ^^Amodifications. Celui ou celle qui souhaiterait une classe très
% ^^Aparticulière, différant en de nombreux points de ce qu'offre la
% ^^A\Class{thesul}, pourrait bien sûr redéfinir les parties concernées de la
% ^^Aclasse mais gagnerait bien plus à se construire sa propre classe. Le
% ^^A\enquote{prêt-à-porter} ne vaudra jamais le \enquote{sur mesure}.%
% ^^A }

\section{Comment lire la présente documentation ?}
\label{sec-comment-lire-cette}

La présente documentation est divisée en deux parties : une principale dédiée
à l'usage courant de la \yatcl{} et une annexe concernant les aspects moins
courants, pouvant n'être consultés qu'occasionnellement.

\subsection{Partie principale}
\label{sec-partie-principale}

La partie principale de la documentation commence par présenter les commandes et
environnements fournis par la \yatcl{} et ce, dans l'ordre dans lequel on
rencontre les objets correspondants dans un mémoire de thèse :
\begin{enumerate}
\item en page(s) de titre (cf. \vref{cha-caract-du-docum,cha-pages-de-titre}) ;
\item en \glspl{liminaire} (cf. \vref{cha-liminaires}) ;
\item dans le corps de la thèse (cf. \vref{cha-corps}) ;
\item en pages annexes (cf. \vref{cha-annexes}) ;
\item en pages finales (cf. \vref{cha-pages-finales}).
\end{enumerate}
Elle indique enfin comment personnaliser la \yatcl{}
(cf. \vref{cha-configuration}).
% ^^A, par exemple pour redéfinir les expressions automatiquement insérées dans
% ^^Ales documents.

\subsection{Partie annexe}
\label{sec-partie-annexe}

L'installation de la \yatcl{} est décrite à l'\vref{cha-installation}.

L'\vref{cha-specimen-canevas} est dédiée à deux spécimens et deux canevas de
thèse produits par la \yatcl{}. On pourra :
\begin{itemize}
\item visualiser leurs \acrshortpl{pdf} pour se faire une idée du genre de
  mémoire qu'on peut obtenir ;
\item consulter et compiler leurs fichiers sources, et s'en servir de base pour
  les adapter à son propre mémoire de thèse.
\end{itemize}

L'\vref{cha-recomm-et-astuc} fournit quelques recommandations, trucs et astuces.

Les questions fréquemment posées au sujet de la \yatcl{} sont répertoriées
à l'\vref{cha-faq}.

L'\vref{cha-fichiers-importes-par} documente deux fichiers que la \yatcl{}
importe automatiquement.

L'\vref{cha-packages-charges} répertorie les packages chargés par la \yatcl{} et
qu'il est du coup préférable de \emph{ne pas} charger manuellement. Elle donne
également une liste non exhaustive de packages qu'elle ne charge pas mais
pouvant se révéler très utiles, notamment aux doctorants.

L'\vref{cha-incomp-conn} liste les incompatibilités connues de la \yatcl{}.

Si nécessaire, on pourra consulter l'\vref{cha-pagination} pour avoir une vue
d'ensemble de la \gls{pagination}, des \glspl{titrecourant} et de la
numérotation des chapitres par défaut avec la \yatcl{}.

Les notations, syntaxe, terminologie et codes couleurs de la présente
documentation se veulent intuitifs mais, en cas de doute, on se reportera
à l'\vref{cha-synt-term-notat}. De même, certains des termes employés ici sont
définis dans le glossaire \vpageref{glossaire}.

L'\vref{cha-add-ons} signale quelques \emph{add-ons} destinés à faciliter
l'usage de la \yatcl{} avec différents éditeurs de texte.

L'\vref{cha-usage-avance}, à ne pas mettre entre toutes les mains, indique
comment s'affranchir d'erreurs propres à la \yatcl{}. \emph{Elle n'est
  à consulter que si l'on est sûr de ce que l'on fait et qu'on pourra en gérer
  \emph{seul} les conséquences.}

L'\vref{cha-devel-futurs} est une \emph{TODO list} des fonctionnalités que
l'auteur de \yat{} doit encore mettre en œuvre, que ce soit pour la classe
elle-même ou pour sa documentation.

Enfin, l'historique des changements de la classe se trouve \vref{cha-history}.
Les changements les plus importants, notamment ceux qui rompent la compatibilité
ascendante, y figurent en rouge.

\section{Ressources Internet}
\label{sec-ressources-internet}

Cette classe est \phrase{ou sera disponible} à l'adresse :
\begin{itemize}
\item \url{http://www.ctan.org/pkg/\jobname}\footnote{Les distributions \TeX{}
    devraient alors soit la fournir d'emblée (à partir de la version
    \texttt{2014} pour la \TeX~Live et de la version 2.10 pour la MiK\TeX{})
    soit l'installer par simple mise à jour (à partir de la version
    \texttt{2014} pour la \TeX~Live).}  pour sa version stable ;
\item \url{https://github.com/dbitouze/\jobname} pour sa version de
  développement.
\end{itemize}

\section{Remerciements}
\label{sec-remerciements}

L'auteur de la \yatcl{} remercie tous les doctorants que, depuis plusieurs
années, il a formés à \LaTeX{} : les questions qu'ils ont soulevées et les
demandes de fonctionnalités qu'ils ont formulées sont à l'origine du présent
travail.

Il remercie en outre tous les auteurs de packages à qui il a soumis \aside{à un
  rythme parfois effréné} des questions, demandes de fonctionnalités et rapports
de bogues. Ils ont eu la gentillesse de répondre rapidement, clairement et
savamment, en acceptant souvent les suggestions formulées. Parmi eux, Nicola
Talbot pour \package*{datatool} et \package*{glossaries}, et Thomas F. Sturm
pour \package*{tcolorbox}.

L'auteur adresse des remerciements chaleureux à ceux qui ont accepté de
bêta-tester la \yatcl{}, notamment Cécile Barbet, Coralie Escande, Mathieu
Leroy-Lerêtre, Mathieu Bardoux, Yvon Henel et Jérôme Champavère.

Enfin, l'auteur sait gré de leur patience tous ceux à qui il avait promis une
version stable de la présente classe... pour la semaine dernière !

%
\iffalse
%%% Local Variables:
%%% mode: latex
%%% eval: (latex-mode)
%%% ispell-local-dictionary: "fr_FR"
%%% TeX-engine: xetex
%%% TeX-master: "../yathesis.dtx"
%%% End:
\fi
