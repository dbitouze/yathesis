\chapter{Fichiers automatiquement importés par la \yatcl{}}
\label{cha:fichiers-importes-par}

Pour faciliter son utilisation, la \yatcl{} importe automatiquement certains
fichiers (s'ils existent et sont situés dans le répertoire \emph{ad hoc},
cf. \vref{wa:import-sous-cond}) :
\begin{itemize}
\item un fichier nommé \file{\titlefile} dédié aux données amenées à figurer
  sur les pages de titre ;
\item un fichier nommé \file{\configurationfile} dédié à la configuration du
  document, où stocker notamment les réglages :
  \begin{itemize}
  \item de la classe \yatcl (cf. \vref{cha:configuration}) ;
  \item des différents packages chargés soit par la classe, soit manuellement
    (cf. \vref{cha:packages-charges}) ;
  \end{itemize}
\item un fichier nommé \file{\macrosfile} dédié aux macros personnelles créées
  pour le document.
\end{itemize}
\begin{dbwarning}{Fichiers automatiquement importés sous conditions}{import-sous-cond}
  Pour que ces fichiers soient automatiquement importés, il est donc
  nécessaire :
  \begin{enumerate}
  \item qu'ils existent ;
  \item qu'ils soient situés dans le répertoire \emph{ad hoc}, à savoir un
    sous-répertoire nommé \directory{\configurationdirectory} du répertoire où
    se trouve le fichier (maître) du document.
  \end{enumerate}
\end{dbwarning}
Ces fichiers et sous-répertoire sont à créer au besoin mais le canevas de thèse
\enquote{en relief} livré avec la classe, décrit \vref{sec:canevas-relief}, les
fournit d'emblée.
%^^A
\begin{dbwarning}{Fichiers à ne pas importer}{}
  Si les fichiers ci-dessus vérifient les conditions de
  l'\vref{wa:import-sous-cond}, la \yatcl{} les importe
  \emph{automatiquement} : ils doivent donc \emph{ne pas} être explicitement
  importés \phrase*{au moyen d'une commande \docAuxCommand{input} ou
    assimilée}.
\end{dbwarning}
