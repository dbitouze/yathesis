\chapter{Développements futurs}\label{cha:devel-futurs}

\section{Pour la prochaine version}
\label{sec:pour-la-prochaine}

\subsection{Classe}
\label{sec:classe}

\begin{enumerate}
\item S'assurer que les termes anglais choisis pour les noms de commandes sont
  judicieux.
\item Mettre le bon \docAuxCommand*{CheckSum}.
%^^A \item Clarifier les codes devant se trouver en préambule.
%^^A \item Fournir, en plus du \File{title.tex}, les fichiers
%^^A   \file{preliminary.tex}, \file{main.tex}, \file{appendix.tex}.
\end{enumerate}

\subsection{Documentation de la classe}
\label{sec:documentation-de-la}

\begin{enumerate}
\item \foreignquote{english}{Sample}.
\item Canevas.
\item \foreignquote{english}{Quick tour}.
\item Revoir les instructions d'installation de la classe et de production de
  sa documentation.
\item Réduire la profondeur de la table des matières.
\item Prévoir une version imprimée.
\item Insérer un graphique du \Package{pgfplots} dans le \foreignquote{english}{sample}.
\end{enumerate}

\section{Pour les versions ultérieures}
\label{sec:pour-les-versions}

\subsection{Classe}
\label{sec:classe-ult}

\begin{enumerate}
\item Vérifier que toutes les macros (privées) sont en anglais.
\item Factoriser, nettoyer et documenter correctement le code.
\item Éviter la duplication des warnings.
\item Créer un \enquote{type} de thèse \docAuxKey{hdr}.
\item Remplacer \refCom{coinstitute}, et peut-être aussi \refCom{company}, par des
  occurrences multiples de \refCom{institute}, distinguables par l'ordre de saisie
  et/ou par des  options.
%^^A \item Voir si les noms \foreignquote{english}{flat-template} et
%^^A   \foreignquote{english}{non-flat-template} ne devraient pas être changés en
%^^A   \foreignquote{english}{single-file} et
%^^A   \foreignquote{english}{master-slaves-files}.
%^^A \item Faire des pseudo-chapitres de la partie liminaire (\refCom{acknowledgements},
%^^A   \refCom{caution}, \refCom{frenchabstract}, \refCom{foreword}, \refCom{preface}) des objets analogues
%^^A   à \refEnv{abstract} (c-à-d des environnements pour les préparer et des commandes
%^^A   \docAuxCommand*{make...} pour les produire).
\item Finir d'implémenter et documenter \docAuxKey*{affiliationsecondary} et
  assimilés.
\item Options pour les polices.
\item Faire figurer la discipline sur la 4\ieme{} de couverture.
\item Permettre de choisir l'ordre dans les lignes et dans les colonnes du
  tableau des membres du jury.
\item Permettre de choisir l'ordre des éléments de la page de titre.
\item Augmenter le nombre de métadonnées du \File{.pdf} au moyen du
  \Package{hyperxmp}.
\item Fournir une commande \docAuxCommand*{includeall} permettant de
  neutraliser les effets de la commande \docAuxCommand*{includeonly}.
\item Fournir une commande \docAuxCommand*{phrase} pour les incises telles que
  \phrase{celle-ci} ou \phrase*{celle-là}.
\item Donner la possibilité de préciser des styles (par exemple pour la façon
  dont est composée la liste des membres du jury).
\item Faire écrire les \foreignquote{english}{warnings} propres à la \yatcl{}
  dans un fichier auxiliaire (disons \file{.yad}) lu avant le \File{.aux} de sorte que ceux-ci
  soient les premiers à figurer dans le fichier de
  \foreignquote{english}{log}. Faire alors usage du \Package*{rerunfilecheck}
  pour s'assurer que le \File{.yad} est à jour.
\item Répartir les \docAuxKey*{moretexcs} et \docAuxKey*{morekeywords} du
  \File{lstlang0.sty} selon leurs packages ou classes.
\item Faire en sorte que, optionnellement, il soit possible de centrer
  verticalement le titre (en fait la boîte le contenant) dans les pages de
  titre.
\end{enumerate}

\subsection{Documentation de la classe}
\label{sec:documentation-de-la-ult}

\begin{enumerate}
\item Pour les 2 précédents, indiquer la présence du \File{.latexmkrc} et
  expliquer l'usage de \program{latexmk}.
\item Utiliser le \Package{tcolorbox} pour s'affranchir des raccourcis
  % ^^A
  \lstDeleteShortInline×%^^A
  % ^^A
  \lstinline|×|
  %^^A
  \lstMakeShortInline[style=dbtex]×%^^A
  %^^A
  et ×÷× ainsi pouvoir compiler la documentation avec \program{pdflatex} et non
  plus \program{xelatex} (il faudra alors renoncer au \Package*{fontawesome}
  qui fournit l'icône en forme de canevas).
\item Prévoir un index des concepts en plus de celui des commandes.
\item Documenter la production des pages de titres et les macros publiques
  (\docAuxCommand*{print...}) qui permettent de faire apparaître les éléments qui
  les constituent.
\item Prévoir un \File{.el} (pour \program{Emacs+AUCTeX}) et voir le format
  pour \program{TeXworks}.
\item Documenter les dossiers et fichiers connus de la \yatcl{} :
  \begin{itemize}
  \item \directory{\configurationdirectory} ;
  \item \file{\configurationfile} ;
  \end{itemize}
  ainsi que les macros définissant leurs noms :
  \begin{itemize}
  \item \docAuxCommand*{configurationdirectory} ;
  \item etc.
  \end{itemize}
\item Est-il opportun de prévoir des fichiers automatiquement chargés par la
  \yatcl{}, par exemple :
  \begin{itemize}
  \item \file{\acronymsfile} ;
  \item \file{macros.tex} ;
  \item etc.
  \end{itemize}
  qui permettrait de ne pas avoir à les charger manuellement au moyen de
  \docAuxCommand{input} ?
\item Documenter \docAuxCommand{yatsetup}.
\end{enumerate}

%
\iffalse
%%% Local Variables:
%%% mode: latex
%%% eval: (latex-mode)
%%% ispell-local-dictionary: "fr_FR"
%%% TeX-engine: xetex
%%% TeX-master: "../yathesis.dtx"
%%% End:
\fi
