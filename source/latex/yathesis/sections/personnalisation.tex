\chapter{Personnalisation}\label{cha:configuration}

Cette section passe en revue les outils de personnalisation propres ou pas à la
\yatcl{} :
\begin{enumerate}
\item options de classe ;
\item options de préambule ;
\item commandes (et options de commandes) de la \yatcl;
\item packages chargés par la \yatcl ;
\item packages chargés manuellement.
\end{enumerate}

\section{Options de classe}\label{options-classe}

Les \meta{options} de classe de la \yatcl sont à passer selon la syntaxe
usuelle :
\begin{preamblecode}
\documentclass[÷\meta{options}÷]{yathesis}
\end{preamblecode}
% ^^A Tester et documenter la commande ×\yasetup×.

% ^^A La \yatcl accepte, en sus des options qui lui sont propres, celles de la
% ^^A \Class{book} sur laquelle est elle basée.

\subsection{Options de la classe \textsf{book}}\label{sec:options-usuelles-de}

Parmi les \meta{options} de la \yatcl figurent celles de la \Class{book},
notamment :
\begin{itemize}
\item \docAuxKey{10pt} (défaut), \docAuxKey{11pt}, \docAuxKey{12pt}, pour fixer
  la taille de base des caractères ;
\item éventuellement :
  \begin{itemize}
  \item \docAuxKey{leqno} pour afficher les numéros d'équations à gauche ;
  \item \docAuxKey{fleqn} pour que les équations hors texte soient toutes
    alignées à gauche avec un même retrait d'alinéa ;
  \item \docAuxKey{oneside} pour une pagination en recto seulement.
  \end{itemize}
\end{itemize}
\begin{dbwarning}{Options usuelles de la \Class{book} : à utiliser avec
    discernement}{}
  Dans le cadre d'un usage de la \yatcl, il est \emph{fortement} déconseillé de
  recourir à d'autres options usuelles de la \Class{book} que celles
  ci-dessus : cela risquerait de produire des résultats non souhaités.
\end{dbwarning}

% ^^A \subsection{Options de la \yatcl}\label{sec:options-yatcl}
%
% ^^A Les \meta{options} discutées dans cette section, propres à la \yatcl{},
% ^^A permettent de contrôler les grandes lignes du document.

\subsection{Langues (principale, secondaire,
  supplémentaires)}\label{sec:langues}

Par défaut, un mémoire créé avec la \yatcl est composé :
\begin{itemize}
\item en français comme langue principale;
\item en anglais comme langue secondaire\footnote{Utilisée ponctuellement pour
    des éléments supplémentaires tels qu'une page de titre, un résumé ou des
    mots clés.}.
\end{itemize}
%
\begin{docKey}{mainlanguage}{=\docValue{french}\textbar\docValue{english}}{pas
    de valeur par défaut, initialement \docValue{french}}
  Pour que la langue principale \phrase{et activée par défaut} du mémoire soit
  l'anglais, il suffit de le stipuler au moyen de l'option
  ×mainlanguage=english×. Le français devient alors automatiquement la langue
  secondaire de la thèse.
\end{docKey}

\begin{dbwarning}{Langues principales et secondaires prises en charge}{}
  Les seules langues \emph{principale} et \emph{secondaire} prises en charge
  par la \yatcl sont le français (\docValue{french}) et l'anglais
  (\docValue{english}).
\end{dbwarning}

\begin{dbremark}{Langues supplémentaires}{languessupplementaires}
  Il est cependant possible de faire usage de langues \emph{supplémentaires},
  autres que le français et l'anglais, en les stipulant en option de
  \docAuxCommand{documentclass}\footnote{Ces langues doivent être l'une de
    celles supportées par le \Package{babel}.} et en les employant selon la
  syntaxe du \Package*{babel}.
\end{dbremark}

\begin{dbexample}{Langue supplémentaire pour thèse
    multilingue principalement en français}{}
  Pour composer un mémoire ayant pour langue principale le français et
  supplémentaire l'espagnol \phrase{cas par exemple d'une thèse en linguistique
    espagnole}, il suffit de passer l'option suivante à la \yatcl{}.
\begin{preamblecode}
\documentclass[spanish]{yathesis}
\end{preamblecode}
\end{dbexample}

\begin{dbexample}{Langue supplémentaire pour thèse
    multilingue principalement en anglais}{}
  Pour composer un mémoire ayant pour langue principale l'anglais (donc
  secondaire le français) et supplémentaire l'espagnol \phrase{cas par exemple
    d'une thèse en linguistique espagnole}, il suffit de passer les options
  suivantes à la \yatcl{}.
\begin{preamblecode}
\documentclass[mainlanguage=english,spanish]{yathesis}
\end{preamblecode}
\end{dbexample}

\subsection{Profondeur de la numérotation}\label{sec:profondeur-de-la}

Par défaut, la numérotation des paragraphes a pour \enquote{niveau de
  profondeur} les sous-sections. Autrement dit, seuls les titres des parties
(éventuelles), chapitres, sections et sous-sections sont numérotés.  L'option
\refKey{secnumdepth} suivante permet de spécifier un autre niveau de
profondeur.
%
{%
  \tcbset{before lower=\vspace*{\baselineskip}\par}
  \begin{docKey}{secnumdepth}{=\docValue{part}\textbar\docValue{chapter}\textbar\docValue{section}\textbar\docValue{subsection}\textbar\docValue{subsubsection}\textbar\docValue{paragraph}\textbar\docValue{subparagraph}}{pas
      de valeur par défaut, initialement \docValue{subsection}}
    Cette clé permet de modifier le \enquote{niveau de profondeur} de la
    numérotation des paragraphes jusqu'aux, respectivement : parties,
    chapitres, sections, sous-sections, sous-sous-sections, paragraphes,
    sous-paragraphes.
  \end{docKey}
}

\subsection{Espace interligne}\label{sec:interligne}

L'interligne du document est par défaut \enquote{simple} mais, au moyen de
l'option \refKey{space} suivante, il est possible de spécifier un interligne
\enquote{un et demi} ou \enquote{double}.

\begin{docKey}{space}{=\docValue{single}\textbar\docValue{onehalf}\textbar\docValue{double}}{pas de valeur par défaut,
    initialement \docValue{single}}
  Cette clé permet de spécifier un interligne \docValue{single} (simple),
  \docValue{onehalf} (un et demi) ou \docValue{double} (double).
\end{docKey}

\begin{dbwarning}{Option d'interligne : seulement dans la partie
    principale}{}
  L'effet de l'option \refKey{space} ne débute qu'avec la partie principale du
  document (cf. \vref{cha:corps}) et se termine avec elle, avant la partie
  annexe (cf. \vref{cha:annexes}). Si on souhaite changer d'interligne ailleurs
  dans le mémoire, on recourra aux commandes du \Package*{setspace}
  \phrase*{chargé par la \yatcl}.
\end{dbwarning}

\subsection{Style des têtes de chapitres}\label{sec:style-des-tetes}

Pour gérer les têtes de chapitres, la \yatcl{} s'appuie sur le
\Package*{fncychap}, par défaut chargé avec le style \docValue{PetersLenny}. La
clé \refKey{chap-style} suivante permet de spécifier un autre style de ce
package.
%
{%
  \tcbset{before lower=\vspace*{\baselineskip}\par}
  \begin{docKey}{chap-style}{=\docValue{Sonny}\textbar\docValue{Lenny}\textbar\docValue{Glenn}\textbar\docValue{Conny}\textbar\docValue{Rejne}\textbar\docValue{Bjarne}\textbar\docValue{PetersLenny}\textbar\docValue{Bjornstrup}\textbar\docValue{none}}{pas
      de valeur par défaut, initialement \docValue{PetersLenny}}
    Cette clé permet de spécifier un autre style du \Package{fncychap}.

    Le \enquote{style} supplémentaire \docValue{none} permet de désactiver le
    chargement de \package{fncychap} pour retrouver les têtes de chapitres
    usuelles de la \Class{book}.
  \end{docKey}
}

\subsection{(Non-)Production de la page de première de
  couverture}\label{sec:non-production-de}

Par défaut, la commande \refCom{maketitle} produit une page de 1\iere{} de
couverture \phrase*{en plus de la ou des pages de titre}. La clé
\refKey{nofrontcover} suivante permet de s'en affranchir.

\begin{docKey}{nofrontcover}{=\docValue{true}\textbar\docValue{false}}{par défaut
    \docValue{true}, initialement \docValue{false}}
  Cette clé permet de désactiver la production de la page de 1\iere{} de
  couverture.
\end{docKey}

\subsection{Versions du mémoire}\label{sec:versions}

Au moyen de la clé \refKey{version}, la \yatcl{} permet de facilement produire
différentes versions du document : \enquote{intermédiaire} (par défaut),
\enquote{à soumettre}, \enquote{finale} et \enquote{brouillon}.

{\tcbset{before lower=\vspace*{\baselineskip}\par}
\begin{docKey}{version}{=\docValue{inprogress}\textbar\docValue{inprogress*}\textbar\docValue{submitted}\textbar\docValue{submitted*}\textbar\docValue{final}\textbar\docValue{draft}}{pas
      de valeur par défaut, initialement \docValue{inprogress}}
    Cette clé permet de spécifier la version du document à produire, au moyen
    des valeurs suivantes.
    \begin{description}
    \item[\docValue{inprogress}.] Cette valeur produit une version
      \enquote{intermédiaire} du document\footnote{Une telle version est
        éventuellement destinée à être diffusée à des relecteurs.}. Ses
      caractéristiques sont les suivantes.
      \begin{enumerate}
      \item\label{item:inprogress:1} Pour indiquer clairement qu'il s'agit
        d'une version \enquote{intermédiaire}, la mention \enquote{Version
          intermédiaire en date du \meta{date}} ou \foreignquote{english}{Work
          in progress as of \meta{date}}\selonlangue{}, est affichée en petites
        capitales sur (presque) tous les pieds de page.
      \item\label{item:inprogress:2} Aucun élément \enquote{obligatoire}
        (cf. \vref{sec:comm-oblig}) manquant n'est signalé.
      \end{enumerate}
    \item[\docValue{inprogress*}.] Cette valeur produit le même effet que la
      valeur \docValue{inprogress} sauf que le caractère non définitif de la
      version est renforcé par la mention \enquote{travail en cours} ou
      \foreignquote{english}{work in progress}\selonlangue{}, figurant en
      filigrane et en capitales sur toutes les pages.
    \item[\docValue{submitted}.] Cette valeur produit une version du document
      destinée à être \enquote{soumise} aux rapporteurs. \emph{Contrairement à}
      la version par défaut :
      \begin{enumerate}
      \item l'affichage en pied de page de la mention \enquote{Version
          intermédiaire en date du \meta{date}} ou \foreignquote{english}{Work
          in progress as of \meta{date}} est désactivé ;
      \item sur les pages de titre, la composition du jury est masquée et la
        date de soutenance est supprimée\footnote{En versions soumises aux
          rapporteurs, le doctorant ne peut préjuger ni d'un jury ni d'une date
          de soutenance, ne sachant pas encore s'il va être autorisé
          à soutenir.} ;
      \item tout élément \enquote{obligatoire} (cf. \vref{sec:comm-oblig})
        manquant est signalé par une erreur de compilation\footnote{La date de
          soutenance est normalement \enquote{obligatoire}, sauf dans les
          versions soumises aux rapporteurs où elle ne figure nulle part.}.
      \end{enumerate}
    \item[\docValue{submitted*}.] Cette valeur produit le même effet que la
      valeur \docValue{submitted} sauf que le caractère \enquote{à soumettre}
      de la version est renforcé par l'affichage en petites capitales, sur
      (presque) tous les pieds de pages, de la mention \enquote{Version soumise
        en date du} ou \foreignquote{english}{Submitted work as
        of}\selonlangue{}.
    \item[\docValue{final}.] Cette valeur produit une version \enquote{finale}
      du document. \emph{Contrairement à} la version par défaut :
      \begin{enumerate}
      \item l'affichage en pied de page de la mention \enquote{Version
          intermédiaire en date du \meta{date}} ou \foreignquote{english}{Work
          in progress as of \meta{date}} est désactivé ;
      \item si un élément \enquote{obligatoire} (cf. \vref{sec:comm-oblig})
        manque, une erreur de compilation signale l'omission.
      \end{enumerate}
    \item[\docValue{draft}.] Cette valeur produit une version
      \enquote{brouillon} du document\footnote{Une telle version est \emph{a
          priori} à usage exclusif de l'utilisateur et n'est en particulier pas
        destinée à être diffusée.}. Ses caractéristiques sont les suivantes :
      \begin{itemize}
      \item \emph{comme} la version par défaut, si un élément \enquote{obligatoire}
        (cf. \vref{sec:comm-oblig}) manque, aucune erreur de compilation ne
        signale l'omission ;
      \item \emph{contrairement à} la version par défaut, la mention \enquote{Version
          intermédiaire en date du \meta{date}} ou \foreignquote{english}{Work
          in progress as of \meta{date}} ne figure pas ;
      \item \emph{en plus de} la version par défaut :
        \begin{enumerate}
        \item Les différentes zones de la page, notamment celle allouée au
          texte, sont matérialisées et les dépassements de marges sont signalés
          par une barre verticale noire dans la marge.
        \item La mention \enquote{brouillon} ou
          \foreignquote{english}{draft}\selonlangue{} figure en filigrane (et
          en capitales) sur toutes les pages du document.
        \item Sur certaines pages, notamment celles de titre :
          \begin{enumerate}
          \item les données caractéristiques de la thèse\footnote{Auteur,
              (sous-)titre, institut(s), directeurs, rapporteurs, examinateurs,
              etc.} sont des hyperliens vers le fichier de configuration de la
            thèse\footnote{Cf. \vref{sec:lieu-de-saisie}.} où il est possible
            de les (re)définir (cf. \vref{sec:expressions-cles});
          \item\label{item-expression} les expressions fournies par la
            \yatcl\footnote{\enquote{Thèse présentée par},
              \foreignquote{english}{In order to become Doctor from},
              \foreignquote{english}{draft}, \enquote{Version intermédiaire en
                date du}, etc. insérées de façon automatique sur certaines
              pages du mémoire.} sont :
            \begin{itemize}
            \item estampillées du label qui les identifie;
            \item des hyperliens vers le fichier de configuration de la thèse
              (cf.  \vref{rq:configurationfile}) où il est possible de les
              (re)définir (cf. \vref{sec:expressions-cles}).
            \end{itemize}
          \end{enumerate}
          Si le système d'exploitation est correctement configuré, un simple
          clic sur ces hyperliens ouvre le fichier correspondant dans l'éditeur
          de texte \LaTeX{} par défaut.
        \end{enumerate}
      \end{itemize}
    \end{description}
  \end{docKey}
}

Les versions \enquote{à soumettre} et \enquote{finale} d'un mémoire de thèse ne
sont à produire qu'exceptionnellement, en toute fin de rédaction. De ce fait :
\begin{dbwarning}{Par défaut, documents en version intermédiaire}{}
  Un document composé avec la \yatcl{} est par défaut en version
  \emph{intermédiaire}. Autrement dit, la clé \refKey{version} a pour valeur
  initiale \docValue*{inprogress}.
\end{dbwarning}

\subsection{Formats de sortie}
\label{sec:formats-de-sortie}

Les documents composés avec la \yatcl{} peuvent avoir deux formats de sortie :
\enquote{écran} (par défaut) et \enquote{papier}, stipulés au moyen de la clé
\refKey{output}.

\begin{docKey}{output}{=\docValue{screen}\textbar\docValue{paper}\textbar\docValue{paper*}}{pas
    de valeur par défaut, initialement \docValue{screen}}
  Cette clé permet de spécifier le format de sortie du document, au moyen des
  valeurs suivantes.
  \begin{description}
  \item[\docValue{screen}.] Avec cette valeur, le document a un format de
    sortie destiné à être visualisé à l'écran. Ce format ne présente pas de
    spécificités particulières, sauf que des liens hypertextes émaillent le
    \acrshort{pdf} produit\footnote{Car la \yatcl{} charge automatiquement le
      \Package{hyperref}.}.
  \item[\docValue{paper}.] Avec cette valeur, le document a un format de sortie
    destiné à être imprimé sur papier. Les différences par rapport au format
    \enquote{écran} sont les suivantes :
    \begin{enumerate}
    \item Les liens hypertextes sont matérialisés comme ils le sont par défaut
      avec le \Package{hyperref} : ils sont encadrés par des rectangles de
      couleurs (différentes selon la nature de ces hyperliens) qui ne sont pas
      imprimés\footnote{Ainsi, si l'utilisateur recourt en préambule à la
        commande \protect\lstinline|\\hypersetup\{colorlinks=true\}| pour que
        les hyperliens soient en couleur, il n'a pas besoin de modifier ce
        choix pour que, en sortie \enquote{papier}, cette coloration soit
        désactivée.} ;
    \item\label{item:paper:1} la commande
      ×\href{×\meta{\acrshort*{url}}×}{×\meta{texte}×}×\footnote{Fournie par le
        \Package{hyperref}.} est automatiquement remplacée par :
      \begin{itemize}
      \item
        \meta{texte}\lstinline[deletekeywords={[2]url}]+\footnote{\url{+\meta{\acrshort*{url}}×}}×
        si elle figure dans le texte ordinaire ;
      \item \meta{texte}
        \lstinline[deletekeywords={[2]url}]+(\url{+\meta{\acrshort*{url}}×})×
        si elle figure en note de bas de page ;
      \end{itemize}
    \item\label{item:paper:2} les barres de navigation affichées par certains
      styles de glossaires\footnote{Telles qu'on peut en voir
        \vref{fig:printacronyms,fig:printglossary}.} \emph{sont} masquées.
    \end{enumerate}
  \item[\docValue{paper*}.] Cette valeur produit le même effet que la valeur
    \docValue{paper} sauf que son \vref{item:paper:2} est inversé : les barres
    de navigation \emph{ne} sont \emph{pas} masquées.
  \end{description}
\end{docKey}

\begin{dbwarning}{Mises en page éventuellement différentes en sortie
    \enquote{écran} et \enquote{papier}}{}
  Du fait des \cref{item:paper:1,item:paper:2} précédents, les mises en page
  des sorties \enquote{écran} et \enquote{papier} peuvent être différentes. Il
  pourra être opportun de les comparer, par exemple à l'aide d'un logiciel
  comparateur de fichiers \gls{pdf}.
\end{dbwarning}

\subsection{Expressions séparant les corps et affiliations des membres du jury}
\label{sec:expr-separ-les}

Sur les pages de titre, chaque membre du jury peut être précisé notamment par :
\begin{itemize}
\item son corps, cf. \refKey{professor}, \refKey{mcf}, \refKey{mcf*},
  \refKey{seniorresearcher}, \refKey{juniorresearcher} et
  \refKey{juniorresearcher*} ;
\item son affiliation, cf. \refKey{affiliation}.
\end{itemize}
Comme illustré \vref{fig:maketitle}, si ces deux précisions sont présentes,
elles sont par défaut séparées :
\begin{description}
\item[en français] par l'une des deux expressions contextuelles suivantes :
  \begin{itemize}
  \item \enquote{\textvisiblespace{}à l'}\footnote{Le symbole
      \enquote{\textvisiblespace{}} matérialise une espace.} ;
  \item \enquote{\textvisiblespace{}au\textvisiblespace{}} ;
  \end{itemize}
  où l'article défini est automatiquement élidé selon l'initiale (voyelle ou
  consonne) de l'affiliation ;
\item[en anglais] par l'expression fixe (non contextuelle)
  \enquote{\textvisiblespace{}at\textvisiblespace{}}.
\end{description}

\begin{dbwarning}{Élision automatique non robuste}{elision-separateurs}
  L'élision automatique des expressions contextuelles en français n'est pas
  robuste : elle peut en effet ne pas donner le résultat escompté si la valeur
  de la clé \refKey{affiliation}, définissant l'affiliation, a pour initiale :
  \begin{itemize}
  \item une consonne, mais est de genre féminin ;
  \item une voyelle, mais par le truchement d'une commande\commandeacronyme, et
    non pas \enquote{directement}.
  \end{itemize}
\end{dbwarning}

Au moyen des clés \refKey{sepcorpaffilfrench} et \refKey{sepcorpaffilenglish}
suivantes, les expressions séparatrices en français et en anglais peuvent être
redéfinies, globalement ou localement.

\begin{docKey}{sepcorpaffilfrench}{=\meta{expression}}{pas de valeur par
    défaut, initialement \lstinline[showspaces]+\ à l'+ ou \lstinline[showspaces]+\ au\ +}
  Cette option permet de redéfinir l'\meta{expression} employée en français
  pour séparer les corps et affiliations des membres du jury. Elle peut
  être employée :
  \begin{description}
  \item[globalement:] elle est alors à spécifier en option de la classe de
    document ;
  \item[localement:] elle est alors à spécifier en option de l'une des
    commandes de définition des membres du jury (cf.
    \vref{sec:definition-jury}).
  \end{description}
\end{docKey}

\begin{docKey}{sepcorpaffilenglish}{=\meta{expression}}{pas valeur par
    défaut, initialement \lstinline[showspaces]+\ at\ +}
  Cette option, analogue à \refKey{sepcorpaffilfrench}, permet de redéfinir
  l'\meta{expression} employée en anglais pour séparer les corps et
  affiliations des membres du jury.
\end{docKey}

\begin{dbwarning}{Expressions séparatrices débutant ou finissant par un espace}{}
  Si les valeurs des clés \refKey{sepcorpaffilfrench} ou
  \refKey{sepcorpaffilenglish} doivent \emph{débuter} ou \emph{finir} par un
  espace, celui-ci doit être saisi au moyen de
  % ^^A
  \lstinline[showspaces]+\ +
  % ^^A
  % ^^A ou de
  % ^^A
  % ^^A \lstinline[deletekeywords={[2]space}]+\space+,
  % ^^A
  et non pas seulement de
  % ^^A
  \lstinline[showspaces]+ +.
  % ^^A
\end{dbwarning}

\begin{dbexample}{Redéfinition (globale) de l'expression séparant corps et
    affiliations}{}
  L'exemple suivant montre comment remplacer l'expression (par défaut) séparant
  corps et affiliations par une virgule, et ce :
  \begin{itemize}
  \item globalement pour tous les membres du jury;
  \item en anglais.
  \end{itemize}
\begin{preamblecode}[listing options={showspaces}]
\documentclass[sepcorpaffilenglish={,\ }]{yathesis}
\end{preamblecode}
\end{dbexample}

\begin{dbexample}{Redéfinition (locale) de l'expression séparant corps et
    affiliation}{}
  L'exemple suivant montre comment remplacer l'expression séparant corps et
  affiliation par \enquote{\textvisiblespace{}à la\textvisiblespace{}}, et ce :
  \begin{itemize}
  \item localement (pour un membre du jury particulier);
  \item en français.
  \end{itemize}
\begin{bodycode}[listing options={showspaces}]
\referee[professor,sepcorpaffilfrench=\ à la\ ,affiliation=Cité des sciences]{René}{Descartes}
\end{bodycode}
\end{dbexample}

\section{Options de préambule}
\label{sec:options-de-preambule}

Pour des raisons techniques, certaines options de la \yatcl ne peuvent pas être
passées en argument optionnel de \docAuxCommand{documentclass} et doivent
l'être en préambule, en argument de la commande \refCom{yadsetup}.  C'est le
cas des options étudiées dans cette section.

\begin{docCommand}{yadsetup}{\marg{options}}
  Cette commande permet de spécifier les \meta{options} de la \yatcl ne pouvant
  pas être passées en argument optionnel de \docAuxCommand{documentclass}.
\end{docCommand}

\begin{dbwarning}{Commande \protect\refCom{yadsetup} : seulement en préambule}{}
  La commande \refCom{yadsetup} ne peut être utilisée qu'en préambule.
\end{dbwarning}


\subsection{Cadre autour du titre de la thèse}
\label{sec:cadre-autour-du}
\changes{v0.99c}{2014/06/06}{Personnalisation possible (p. ex. suppression) du
  cadre autour du titre}

{%
  \tcbset{before lower=\vspace*{\baselineskip}\par}
\begin{docKey}{frametitle}{=\docValue{fbox}\textbar\docValue{shadowbox}\textbar\docValue{ovalbox}\textbar\docValue{none}\textbar\marg{autre}}{pas de valeur par défaut, initialement \docValue{fbox}}
  Cette clé permet de personnaliser le cadre figurant par défaut autour du
  titre de la thèse sur les pages de titre :
  \begin{itemize}
  \item sa valeur \docValue{fbox} produit un cadre rectangulaire ;
  \item sa valeur \docValue{shadowbox} produit un cadre ombré ;
  \item sa valeur \docValue{ovalbox} produit un cadre dont les sommets sont
    arrondis ;
  \item sa valeur \docValue{none} permet de supprimer ce cadre. L'affichage des
    mentions \enquote{Titre de la thèse} et \foreignquote{english}{Thesis
      Title} est alors désactivé ;
  \item toute \meta{autre} valeur lui étant passée doit être :
    \begin{enumerate}
    \item une liste de clés/valeurs propres à l'environnement
      \docAuxEnvironment{tcolorbox} du \Package{tcolorbox} (cf. la
      documentation de ce package) ;
    \item passée entre paire d'accolades :
\begin{preamblecode}
\yadsetup{frametitle={÷\meta{autre}÷}}
\end{preamblecode}
    \end{enumerate}
  \end{itemize}
\end{docKey}
}

\begin{dbexample}{Cadre personnalisé autour du titre de la thèse}{}
  Pour que le cadre entourant le tire de la thèse soit ombré, il suffit de
  saisir :
\begin{preamblecode}
\yadsetup{frametitle=shadowbox}
\end{preamblecode}
\end{dbexample}

\begin{dbexample}{Cadre \enquote{fantaisie} autour du titre de la thèse}{}
  Cet exemple, certainement déconseillé, montre comment exploiter les
  fonctionnalités du \Package{tcolorbox} pour obtenir un cadre
  \enquote{fantaisie} autour du titre de la thèse.%
  \NoAutoSpacing%
\begin{preamblecode}
\tcbuselibrary{skins}
\yadsetup{frametitle={colback=red!50!white,beamer}}
\end{preamblecode}
\end{dbexample}

\section{Commandes de la \yatcl}

\begin{dbremark}{Lieu des commandes de personnalisations}{configurationfile}
  Les commandes de personnalisation listées dans cette section (et donc propres
  à \yatcl{}) ou fournies par les packages chargés manuellement peuvent être
  saisies :
  \begin{itemize}
  \item soit directement dans le (préambule du) fichier (maître) de la thèse ;
  \item soit dans un fichier (prévu à cet effet) à nommer
    \file{\configurationfile} et à placer dans un sous-dossier (prévu à cet
    effet) à nommer \directory{\configurationdirectory}\footnote{Ces fichier et
      sous-dossier sont à créer au besoin mais le canevas de thèse \enquote{en
        relief} livré avec la \yatcl, décrit \vref{sec:canevas-relief}, les
      fournit.}.
  \end{itemize}
\end{dbremark}

\begin{dbwarning}{Fichier de configuration à ne pas importer manuellement}{}
  Le \File{\configurationfile} est \emph{automatiquement} importé par la
  \yatcl{} et il doit donc \emph{ne pas} être explicitement importé : on
  \emph{ne} recourra donc \emph{pas} à la commande
  ×\input{×\file{\configurationfile}×}× (ou autre commande d'importation
  similaire à \docAuxCommand{input}).
\end{dbwarning}

\subsection{(Re)Définition des expressions de la
  thèse}\label{sec:expressions-cles}

Un mémoire de thèse composé avec la \yatcl est émaillé d'expressions insérées
de façon automatique sur certaines pages (titre, mots clés, laboratoire,
résumés, etc.). Que ces expressions soient définies par la \yatcl ou bien
standard, il est possible de les redéfinir.

\subsubsection{Expressions définies par la classe}
\label{sec:expr-defin-par}

Les expressions \meta{en français} et \meta{en anglais} définies par la \yatcl
sont listées dans le \vref{tab:expressions-cles} et y sont identifiées par un
\meta{label} permettant de les redéfinir (voire de les définir, cf.
\vref{ex:doctor}) au moyen de la commande \refCom{expression} suivante.
%
\begin{docCommand}{expression}{\marg{label}\marg{en français}\marg{en anglais}}
  Cette commande permet de (re)définir les valeurs \meta{en français} et
  \meta{en anglais} de l'expression identifiée par \meta{label}.
\end{docCommand}

\begin{table}
  \centering
  \lstset{%
  morekeywords=[5]{% Translation labels
    coinstitute,company,institute,cosupervisor,cosupervisor*,comonitor,%
    comonitor*,supervisor,supervisor*,academicfield,doctoralschool,keywords,%
    ordernumber,committeepresident,committeepresident*,speciality%
  }%
}
%
\newcommand{\expression}[3]{%^^A
  \ifthenelse{\isempty{#2}}{%^^A
    \meta{vide}%^^A
  }{%
    #2%^^A
  }%
  &
  \ifthenelse{\isempty{#3}}{%^^A
    \meta{vide}%^^A
  }{%
    #3%^^A
  }%
  &
  #1%^^A
  \\%^^A
}
\bgroup%
\footnotesize%
\begin{tabulary}{\linewidth}{LL>{\ttfamily\color{keyword5}}l}
  Valeur en français & Valeur en anglais & \textcolor{black}{Label} \\\toprule
  \expression{lbl-aim}{En vue de l'obtention du grade de docteur de l'}{In order to
  become Doctor from }%
\expression{lbl-aimand}{ et de l'}{ and from }%
\expression{lbl-caution}{Avertissement}{Caution}%
\expression{lbl-coinstitute}{}{}%
\expression{lbl-committeemembers}{Composition du jury}{Committee members}%
\expression{lbl-committeepresident}{pr\'esident}{President}%
\expression{lbl-comonitor}{co-encadrant}{Co-Monitor}%
\expression{lbl-company}{}{}%
\expression{lbl-conclusion}{Conclusion}{Conclusion}%
\expression{lbl-cosupervisor}{co-directeur}{Co-Supervisor}%
\expression{lbl-defendedon}{Soutenue le}{Defended on}%
\expression{lbl-doctoralschool}{\'Ecole doctorale}{Doctoral School}%
\expression{lbl-draft}{brouillon}{draft}%
\expression{lbl-email}{\Email}{\Email}%
\expression{lbl-examiners}{Examinateur}{Examiner}%
\expression{lbl-examiners-pl}{Examinateurs}{Examiners}%
\expression{lbl-fax}{\fax}{\fax}%
\expression{lbl-guests}{Invit\'e}{Guest}%
\expression{lbl-guests-pl}{Invit\'es}{Guests}%
\expression{lbl-institute}{}{}%
\expression{lbl-inprogress}{travail en cours}{work in progress}%
\expression{lbl-introduction}{Introduction}{Introduction}%
\expression{lbl-juniorresearcher}{charg\'e de recherche}{Junior Researcher}%
\expression{lbl-juniorresearcher*}{charg\'e de recherche
  \textsc{hdr}}{\textsc{hdr} Junior Researcher}%
\expression{lbl-keywords}{Mots cl\'es}{Keywords}%
\expression{lbl-mcf}{\textsc{mcf}}{Lecturer}%
\expression{lbl-mcf*}{\textsc{mcf} \textsc{hdr}}{\textsc{hdr} Lecturer}%
\expression{lbl-ordernumber}{Num\'ero d'ordre}{Order Number}%
\expression{lbl-phdthesis}{th\`ese}{Ph. D. Thesis}%
\expression{lbl-phone}{\Telefon}{\Telefon}%
\expression{lbl-preface}{Pr\'eface}{Preface}%
\expression{lbl-prepared-at}{Cette th\`ese a \'et\'e pr\'epar\'ee au}{This
  thesis has been prepared at}%
\expression{lbl-prepared-at-pl}{Cette th\`ese a \'et\'e pr\'epar\'ee dans les
  laboratoires suivants.}{This thesis has been prepared at the following
  research units.}%
\expression{lbl-professor}{professeur}{Professor}%
\expression{lbl-referees}{Rapporteur}{Referee}%
\expression{lbl-referees-pl}{Rapporteurs}{Referees}%
\expression{lbl-seniorresearcher}{directeur de recherche}{Senior Researcher}%
\expression{lbl-academicfield}{Discipline}{Academic Field}%
\expression{lbl-speciality}{Sp\'ecialit\'e}{Speciality}%
\expression{lbl-supervisor}{directeur}{Supervisor}%
\expression{lbl-supervisors}{Directeur de th\`ese}{Supervisor}%
\expression{lbl-supervisors-pl}{Directeurs de th\`ese}{Supervisors}%
\expression{lbl-thesisdefendedby}{Th\`ese pr\'esent\'ee par}{Thesis defended
  by}%
\expression{lbl-thesistitle}{Titre de la th\`ese}{Thesis Title}%
\expression{lbl-universitydepartment}{Unit\'e de recherche}{University
  Department}%
\expression{lbl-versiondate}{Version interm\'ediaire en date du}{Work in
  progress as of}%
\expression{lbl-website}{Site}{Web Site}%

\end{tabulary}
\egroup%

  \caption{Expressions de la \yatcl et labels correspondants (classés par ordre
    alphabétique des valeurs en français)}
  \label{tab:expressions-cles}
\end{table}

\begin{dbexample}{Modification d'expression définie par la classe}{}
  Pour remplacer l'expression en français \enquote{Unit\'e de recherche} (dont le label est
  ×lbl-universitydepartment×) par \enquote{Laboratoire}, il suffit de
  saisir :
  %
\begin{preamblecode}[title=Par exemple dans le \File{\configurationfile}]
\expression{lbl-universitydepartment}{Laboratoire}{University Department}
\end{preamblecode}
\end{dbexample}
%
\begin{dbexample}{Suppression d'expression définie par la classe}{}
  Si on souhaite supprimer des pages de titre les mentions \enquote{Titre de la
    thèse} et \foreignquote{english}{Thesis Title} (expressions dont le label
  est ×lbl-thesistitle×), il suffit de saisir :
\begin{preamblecode}[title=Par exemple dans le \File{\configurationfile}]
\expression{lbl-thesistitle}{}{}
\end{preamblecode}
\end{dbexample}

\begin{dbremark}{Modification d'expressions facilitée par la version
    \enquote{brouillon}}{}
  On a vu \vref{sec:versions} que l'option ×version=draft× permet de facilement
  retrouver les labels des expressions et atteindre le \File{\configurationfile}
  pour y modifier celles-ci.
\end{dbremark}

\subsubsection{Expressions standard}
\label{sec:expressions-standard}

Les commandes ×\addto×, ×\captionsfrench× et ×\captionsenglish× du
\Package{babel} permettent de redéfinir les expressions standard listées
\vref{tab:expressions-standard} au moyen de la syntaxe suivante.
\begin{preamblecode}[title=Par exemple dans le \File{\configurationfile}]
\addto\captionsfrench{\def\÷\meta{commande}÷{÷\meta{en français}÷}}
\addto\captionsenglish{\def\÷\meta{commande}÷{÷\meta{en anglais}÷}}
\end{preamblecode}
\begin{table}[hb]
  \centering
  \begin{tabular}{lll}
 Commande                      & Valeur en français & Valeur en anglais \\\toprule
 \lstinline+\abstractname+     & Résumé             & Abstract          \\
 \lstinline+\alsoname+         & voir aussi         & see also          \\
 \lstinline+\appendixname+     & Annexe             & Appendix          \\
 \lstinline+\bibname+          & Bibliographie      & Bibliography      \\
% \lstinline+\ccname+      & Copie à            & cc                \\
 \lstinline+\chaptername+      & Chapitre           & Chapter           \\
 \lstinline+\contentsname+     & Table des matières & Contents          \\
% \lstinline+\enclname+    & P.J.               & encl              \\
 \lstinline+\figurename+       & Figure             & Figure            \\
 \lstinline+\glossaryname+     & Glossaire          & Glossary          \\
 \lstinline+\indexname+        & Index              & Index             \\
 \lstinline+\listfigurename+   & Table des figures  & List of Figures   \\
 \lstinline+\listtablename+    & Liste des tableaux & List of Tables    \\
 \lstinline+\pagename+         & page               & Page              \\
 \lstinline+\partname+         & partie             & Part              \\
% \lstinline+\prefacename+ & Préface            & Preface           \\
 \lstinline+\proofname+        & Démonstration      & Proof             \\
 \lstinline+\refname+          & Références         & References        \\
 \lstinline+\seename+          & voir               & see               \\
 \lstinline+\tablename+        & Tableau            & Table
\end{tabular}

  \caption{Valeurs et commandes d'expressions standard du \Package{babel}}
  \label{tab:expressions-standard}
\end{table}
\begin{dbexample}{Redéfinition d'expressions du \Package{babel}}{}
\begin{preamblecode}[title=Par exemple dans le \File{\configurationfile}]
\addto\captionsfrench{\def\abstractname{Aperçu de notre travail}}
\addto\captionsenglish{\def\abstractname{Overview of our work}}
\end{preamblecode}
\end{dbexample}

En cas d'usage des packages \package{glossaries} et \package{biblatex}, la
syntaxe précédente est inopérante avec les commandes ×\glossaryname× et
×\bibname× (ainsi que ×\refname×). Dans ce cas, pour donner un \meta{titre
  alternatif} :
\begin{itemize}
\item aux glossaire, liste d'acronymes et liste de symboles, on recourra
  à l'une ou l'autre des instructions suivantes :
\begin{bodycode}
\printglossary[title=÷\meta{titre alternatif}÷]
\printglossaries[title=÷\meta{titre alternatif}÷]
\printacronyms[title=÷\meta{titre alternatif}÷]
\printsymbols[title=÷\meta{titre alternatif}÷]
\end{bodycode}
\item à la bibliographie, on recourra à :
\begin{bodycode}
\printbibliography[title=÷\meta{titre alternatif}÷]
\end{bodycode}
\end{itemize}

En outre, en cas d'usage du \Package*{listings}, un \meta{titre alternatif}
pourra être donné à la liste des listings, au moyen de:
\begin{preamblecode}[title=Par exemple dans le \File{\configurationfile}]
\renewcommand\lstlistingname{÷\meta{titre alternatif}÷}
\end{preamblecode}

\subsection{Nouveaux corps}\label{sec:nouveaux-corps}

On a vu \vref{sec:jury} que les commandes définissant les membres du jury
permettent de spécifier si ceux-ci appartiennent aux corps \emph{prédéfinis}
des professeurs ou des maîtres de conférences (\gls{hdr} ou pas) des
universités et des directeurs de recherche ou des chargé(e)s de recherche
(\gls{hdr} ou pas) du \gls{cnrs}. La clé \refKey{corps} suivante permet de
spécifier de \emph{nouveaux} corps à \emph{définir} au moyen de la commande
\refCom{expression}.

\changes{v0.99e}{2014/06/15}{La clé \protect\refAux{corporation} est remplacée
  par (et devient un alias de) la clé \protect\refKey{corps}}
% ^^A \marginpar{Nouveau (2014/06/15) : la clé \protect\refAux{corporation} est
% ^^A remplacée par (et devient un alias de) la clé \protect\refKey{corps}}
%
\begin{docKey}{corps}{=\meta{label}}{pas de
    valeur par défaut, initialement vide}
  L'option ×corps=×\meta{label} permet de stipuler un \meta{corps en français}
  et une \meta{corps en anglais} où \meta{label} identifie une expression
  à définir au moyen de :
\begin{preamblecode}[title=Par exemple dans le \File{\configurationfile}]
\expression{÷\meta{label}÷}{÷\meta{corps en français}÷}{÷\meta{corps en anglais}÷}
\end{preamblecode}
\end{docKey}

\begin{dbexample}{Nouveau corps}{doctor}
  Si on souhaite spécifier que certains membres du jury sont docteurs, il
  suffit de définir \phrase{une seule fois} l'expression suivante de label (par
  exemple) ×lbl-doctor× :
\begin{preamblecode}[title=Par exemple dans le \File{\configurationfile}]
\expression{lbl-doctor}{docteur}{Doctor}
\end{preamblecode}
  pour pouvoir ensuite l'utiliser \phrase{autant de fois que souhaité}, par
  exemple ainsi :
\begin{bodycode}
\examiner[corps=doctor]{Joseph}{Fourier}
\examiner[corps=doctor]{Paul}{Verlaine}
\end{bodycode}
\end{dbexample}

\section{Packages chargés par la \yatcl}

Le comportement par défaut de la \yatcl{} est aussi gouverné par le
comportement par défaut des packages qu'elle charge
automatiquement\footnote{ On pourra le cas échéant consulter
  \vref{cha:packages-charges} la liste des packages chargés par la \yatcl{}.}.
La personnalisation de \yat{} peut donc aussi passer par celle de ces packages.

Deux exemples parmi d'autres sont traités ci-après.

\subsection{Bibliographie absente de la table des matières}

La \yatcl{} fait par défaut figurer la bibliographie dans les sommaire, table
des matières et signets du document. Si cela n'est pas souhaité, il suffit de
passer à la commande \docAuxCommand{printbibliography} l'option
×heading=×\meta{entête}, où \meta{entête} vaut par exemple
\docValue*{bibliography} (cf. la documentation du \Package{biblatex} pour plus
de détails).

\subsection{Profondeurs différentes pour les signets et la table des matières}

Par défaut, les signets du fichier \acrshort{pdf} ont le même niveau de
profondeur que la table des matières. Mais l'option \docAuxKey{bookmarksdepth}
du \Package*{hyperref} permet de leur affecter un \meta{autre niveau}.
\begin{preamblecode}[title=Par exemple dans le \File{\configurationfile}]
\hypersetup{bookmarksdepth=÷\meta{autre niveau}÷}
\end{preamblecode}
où \meta{autre niveau} est l'une des valeurs possibles de la clé
\refKey{depth}.

\section{Packages chargés manuellement}
\label{sec:options-de-classes}
Si on souhaite recourir à des packages qui ne sont pas appelés par la \yatcl{},
on les chargera manuellement, par exemple en préambule du fichier (maître) de
la thèse.
%
\iffalse
%%% Local Variables:
%%% mode: latex
%%% eval: (latex-mode)
%%% ispell-local-dictionary: "fr_FR"
%%% TeX-engine: xetex
%%% TeX-master: "../yathesis.dtx"
%%% End:
\fi
