\chapter{Personnalisation}\label{cha:configuration}

Cette section passe en revue les outils de personnalisation propres
ou pas à la \yatcl{} :
\begin{enumerate}
\item options de classe ;
\item commandes (et options de commandes) de la \yatcl;
\item packages chargés par  la \yatcl ;
\item packages chargés manuellement.
\end{enumerate}

\section{Options de classe}\label{options-classe}

Les \meta{options} de la \yatcl sont à passer (exclusivement) selon la syntaxe
usuelle :
\begin{preamblecode}
\documentclass[÷\meta{options}÷]{yathesis}
\end{preamblecode}

La \yatcl accepte, en sus des options qui lui sont propres, celles de la
\Class{book} sur laquelle est elle basée.

\subsection{Options de la classe \textsf{book}}\label{sec:options-usuelles-de}

Parmi les \meta{options} de la \yatcl figurent celles de la
\Class{book}, notamment :
\begin{itemize}
\item \docAuxKey{10pt} (défaut), \docAuxKey{11pt}, \docAuxKey{12pt}, pour fixer la taille de base des
  caractères ;
\item éventuellement :
  \begin{itemize}
  \item \docAuxKey{leqno} pour afficher les numéros d'équations à gauche ;
  \item \docAuxKey{fleqn} pour afficher les équations toutes alignées à gauche
    avec un même retrait ;
  \item \docAuxKey{oneside} pour une pagination en recto seulement.
  \end{itemize}
\end{itemize}
\begin{dbwarning}{Options usuelles de la \Class{book} : à utiliser
    avec discernement}{}
  Dans le cadre d'un usage de la \yatcl, il est \emph{fortement}
  déconseillé de recourir à d'autres options usuelles de la
  \Class{book} que celles ci-dessus : cela risquerait de produire
  des résultats non souhaités.
\end{dbwarning}

\subsection{Options de la \yatcl}\label{sec:options-yatcl}

Les \meta{options} discutées dans cette section, propres à la \yatcl{},
permettent de contrôler les grandes lignes du document.
%^^A \begin{itemize}
%^^A \item \hyperref[sec:langues]{langues (principale, secondaire, supplémentaires)};
%^^A \item \hyperref[sec:profondeur-de-la]{profondeur de la numérotation des sections};
%^^A \item \hyperref[sec:interligne]{interligne};
%^^A \item \hyperref[sec:style-des-tetes]{style des têtes de chapitres};
%^^A \item \hyperref[sec:expr-separ-les]{expressions} séparant corporations et
%^^A   instituts des membres du jury figurant sur les pages de titre;
%^^A \item \hyperref[sec:versions]{versions}: \enquote{brouillon},
%^^A   \enquote{intermédiaire} ou \enquote{imprimée}.
%^^A \end{itemize}


\subsubsection{Langues (principale, secondaire, supplémentaires)}\label{sec:langues}

Par défaut, un mémoire créé avec la \yatcl est composé :
\begin{itemize}
\item en français comme langue principale;
\item en anglais comme langue secondaire\footnote{Utilisée ponctuellement pour
    des éléments supplémentaires tels qu'une page de titre, un résumé ou des
    mots clés.}.
\end{itemize}
%
\begin{docKey}{mainlanguage}{=\docValue{french}\textbar\docValue{english}}{pas de valeur par défaut,
    initialement \docValue{french}}
  Pour que la langue principale \phrase{et activée par défaut} du mémoire soit
  une langue autre que le français, il suffit de le stipuler au moyen de la clé
  \refKey{mainlanguage}. Le français devient alors automatiquement la langue
  secondaire de la thèse.
\end{docKey}

\begin{dbwarning}{Langues principales et secondaires prises en charge}{}
  Les seules langues \emph{principale} et \emph{secondaire} prises en charge
  par la \yatcl sont le français (\docValue{french}) et l'anglais
  (\docValue{english}).
\end{dbwarning}

\begin{dbremark}{Langues supplémentaires}{languessupplementaires}
  Il est cependant possible de faire usage de langues \emph{supplémentaires},
  autres que le français et l'anglais, en les stipulant en option de
  \docAuxCommand{documentclass}\footnote{Ces langues doivent être l'une de
    celles supportées par le \Package{babel}.} et en les employant selon la
  syntaxe du \Package*{babel}.
\end{dbremark}

\begin{dbexample}{Langue supplémentaire pour thèse
    multilingue}{these-multilingue}
  Pour composer un mémoire ayant pour langue principale l'anglais (donc
  secondaire le français) et supplémentaire l'espagnol \phrase{cas par exemple
    d'une thèse en linguistique espagnole}, on passera les options suivantes
  à la \yatcl{}.
\begin{preamblecode}
\documentclass[mainlanguage=english,spanish]{yathesis}
\end{preamblecode}
\end{dbexample}

\subsubsection{Profondeur de la numérotation}\label{sec:profondeur-de-la}

Par défaut, les numérotation des paragraphes a pour \enquote{niveau de
  profondeur} les sous-sections. Autrement dit, ne sont numérotés que les
titres des parties (éventuelles), chapitres, sections et sous-sections.
L'option \refKey{secnumdepth} suivante permet de spécifier un autre niveau de
profondeur.
%
{%
  \tcbset{before lower=\vspace*{\baselineskip}\par}
  \begin{docKey}{secnumdepth}{=\docValue{part}\textbar\docValue{chapter}\textbar\docValue{section}\textbar\docValue{subsection}\textbar\docValue{subsubsection}\textbar\docValue{paragraph}\textbar\docValue{subparagraph}}{pas
      de valeur par défaut, initialement \docValue{subsection}}
    Cette clé permet de modifier le \enquote{niveau de profondeur} de la
    numérotation des paragraphes jusqu'aux, respectivement : parties,
    chapitres, sections, sous-sections, sous-sous-sections, paragraphes,
    sous-paragraphes.
  \end{docKey}
}

\subsubsection{Espace interligne}\label{sec:interligne}

L'interligne du document est par défaut \enquote{simple} mais, au
moyen de l'option \refKey{space} suivante, il est possible de
spécifier un interligne \enquote{un et demi} ou
\enquote{double}.

\begin{docKey}{space}{=\docValue{single}\textbar\docValue{onehalf}\textbar\docValue{double}}{pas de valeur par défaut,
    initialement \docValue{single}}
  Cette clé permet de spécifier un \meta{interligne} \docValue{single}
  (simple), \docValue{onehalf} (un et demi) ou \docValue{double} (double).
\end{docKey}

\begin{dbwarning}{Option d'interligne : seulement dans la partie
    principale}{}
  L'effet de l'option \refKey{space} ne débute qu'avec la partie principale du
  document (cf. \vref{cha:corps}) et se termine avec elle, avant la partie
  annexe (cf. \vref{cha:annexes}). Si on souhaite changer d'interligne ailleurs
  dans le mémoire, on recourra aux commandes du \Package*{setspace}
  \phrase*{chargé par la \yatcl}.
\end{dbwarning}

\subsubsection{Style des têtes de chapitres}\label{sec:style-des-tetes}

Pour gérer les têtes de chapitres, la \yatcl{} s'appuie sur le
\Package*{fncychap}, par défaut chargé avec le style \docValue{PetersLenny}. La
clé \refKey{chap-style} suivante permet de spécifier un autre style de ce
package.
%
{%
  \tcbset{before lower=\vspace*{\baselineskip}\par}
  \begin{docKey}{chap-style}{=\docValue{Sonny}\textbar\docValue{Lenny}\textbar\docValue{Glenn}\textbar\docValue{Conny}\textbar\docValue{Rejne}\textbar\docValue{Bjarne}\textbar\docValue{PetersLenny}\textbar\docValue{Bjornstrup}\textbar\docValue{none}}{pas de valeur par défaut,
    initialement \docValue{PetersLenny}}
  Cette clé permet de spécifier un autre style du \Package{fncychap}.

  Le \enquote{style} supplémentaire \docValue{none} permet de désactiver le
  chargement de \package{fncychap} pour retrouver les têtes de
  chapitres usuelles de la \Class{book}.
\end{docKey}
}

\subsubsection{(Non-)Production de la page de 1\iere{} de couverture}\label{sec:non-production-de}

Par défaut, la commande \refCom{maketitle} produit une page de 1\iere{} de
couverture \phrase*{en plus de la ou des pages de titre}. La clé
\refKey{nofrontcover} suivante permet de s'en affranchir.

\begin{docKey}{nofrontcover}{=\docValue{true}\textbar\docValue{false}}{par défaut
    \docValue{true}, initialement \docValue{false}}
  Cette clé permet de désactiver la production de la page de 1\iere{} de
  couverture.
\end{docKey}

\subsubsection{Expressions séparant les corporations et instituts des membres du jury}
\label{sec:expr-separ-les}

Sur les pages de titre, chaque membre du jury peut être précisé notamment par :
\begin{itemize}
\item sa corporation, cf. \refKey{professor}, \refKey{mcf}, \refKey{mcf*},
  \refKey{seniorresearcher}, \refKey{juniorresearcher} et
  \refKey{juniorresearcher*} ;
\item son affiliation, cf. \refKey{affiliation}.
\end{itemize}
Comme illustré \vref{fig:maketitle}, si ces deux précisions sont présentes,
elles sont par défaut séparées :
\begin{description}
\item[en français] par l'une des deux expressions contextuelles suivantes,
  selon que l'initiale de l'affiliation est une voyelle ou une consonne:
  \begin{itemize}
  \item \enquote{\textvisiblespace{}à l'}\footnote{Le symbole
      \enquote{\textvisiblespace{}} matérialise une espace.} ;
  \item \enquote{\textvisiblespace{}au\textvisiblespace{}} ;
  \end{itemize}
\item[en anglais] par l'expression fixe (non contextuelle)
  \enquote{\textvisiblespace{}at\textvisiblespace{}}.
\end{description}

\begin{dbwarning}{Expressions contextuelles non robustes}{separateurs}
  Les expressions contextuelles en français ne sont pas robustes. Elles peuvent
  en effet ne pas donner le résultat escompté si la valeur de la clé
  \refKey{affiliation}, définissant l'affiliation, a pour initiale :
  \begin{itemize}
  \item une consonne, mais est de genre féminin ;
  \item une voyelle, mais par le truchement d'une commande\footnote{Notamment
      une commande d'acronyme \phrase*{telle que \docAuxCommand{gls} ou
        \docAuxCommand{acrshort}}.}, et non pas \enquote{directement}.
  \end{itemize}
\end{dbwarning}

Au moyen des clés \refKey{sepcorpaffilfrench} et \refKey{sepcorpaffilenglish}
suivantes, les expressions séparatrices en français et en anglais peuvent être
redéfinies, globalement ou localement.

\begin{docKey}{sepcorpaffilfrench}{=\meta{expression}}{pas de valeur par
    défaut, initialement \lstinline[showspaces]+\ à l'+ ou \lstinline[showspaces]+\ au\ +}
  Cette option permet de redéfinir l'\meta{expression} employée en français
  pour séparer les corporations et instituts des membres du jury. Elle peut
  être employée :
  \begin{description}
  \item[globalement:] elle est alors à spécifier en option de la classe de
    document ;
  \item[localement:] elle est alors à spécifier en option de l'une des
    commandes de définition des membres du jury (cf.
    \vref{sec:definition-jury}).
  \end{description}
\end{docKey}

\begin{docKey}{sepcorpaffilenglish}{=\meta{expression}}{pas valeur par
    défaut, initialement \lstinline[showspaces]+\ at\ +}
  Cette option, analogue à \refKey{sepcorpaffilfrench}, permet de redéfinir
  l'\meta{expression} employée en anglais pour séparer les corporations et
  instituts des membres du jury.
\end{docKey}

\begin{dbwarning}{Expressions séparatrices débutant ou finissant par un espace}{}
  Si les valeurs des clés \refKey{sepcorpaffilfrench} ou
  \refKey{sepcorpaffilenglish} doivent \emph{débuter} ou \emph{finir} par
  un espace, celui-ci doit être saisi au moyen de
  %^^A
  \lstinline[showspaces]+\ +
  %^^A
  %^^A ou de
  %^^A
  %^^A \lstinline[deletekeywords={[2]space}]+\space+,
  %^^A
  et non pas seulement de
  %^^A
  \lstinline[showspaces]+ +.
  %^^A
\end{dbwarning}

\begin{dbexample}{Redéfinition (globale) de l'expression séparant corporation et
    institut}{}
  L'exemple suivant montre comment remplacer l'expression (par défaut) séparant
  corporation et institut par une virgule, et ce :
  \begin{itemize}
  \item globalement pour tous les membres du jury;
  \item en anglais.
  \end{itemize}
\begin{preamblecode}[listing options={showspaces}]
\documentclass[sepcorpaffilenglish={,\ }]{yathesis}
\end{preamblecode}
\end{dbexample}

\begin{dbexample}{Redéfinition (locale) de l'expression séparant corporation et
    institut}{}
  L'exemple suivant montre comment remplacer l'expression séparant corporation
  et institut par \enquote{\textvisiblespace{}à la\textvisiblespace{}}, et ce :
  \begin{itemize}
  \item localement (pour un membre du jury particulier);
  \item en français.
  \end{itemize}
\begin{bodycode}
\referee[professor,sepcorpaffilfrench=\ à la\ ,affiliation=Cité des sciences]{René}{Descartes}
\end{bodycode}
\end{dbexample}

\subsubsection{(Non-)Affichage des  \foreignquote{english}{warnings} propres à la \yatcl{}}
\label{sec:non-affichage-des}

Dans certaines situations, par exemple lorsque des commandes ou environnements
\enquote{obligatoires} (cf. \vref{sec:comm-oblig}) sont omis, des
avertissements (\foreignquote{english}{warnings}) propres à la \yatcl{} sont
affichés dans le fichier de \enquote{log}. Il est possible de désactiver cet
affichage de façon :
\begin{description}
\item[globale] pour tous ces avertissements, au moyen de la clé
  \refKey{nowarning} ci-dessous.
\item[ciblée] pour chacun des commandes et environnements
  \enquote{obligatoires}, au moyen de clés indiquées seulement
  \vref{sec:non-affichage-cible} car celles-ci, pouvant altérer le bon
  fonctionnement de la \yatcl{}, ne sont à utiliser que dans le cadre d'un
  usage avancé.
\end{description}

\begin{docKey}{nowarning}{=\docValue{true}\textbar\docValue{false}}{par défaut \docValue{true},
    initialement \docValue{false}}
  Cette option désactive l'affichage de tous les avertissements propres à la
  \yatcl{}.
\end{docKey}

\begin{dbremark}{Avertissements masqués en versions \enquote{brouillon} et
    \enquote{intermédiaire}}{}
  Les versions \enquote{brouillon} et \enquote{intermédiaire} du document (clés
  \refKey{draft} et \refKey{intermediate}, cf. \vref{sec:versions}) désactivent
  l'affichage de tous les avertissements propres à la \yatcl{}.
\end{dbremark}

\subsubsection{Versions du mémoire}\label{sec:versions}

Grâce aux clés \refKey{draft}, \refKey{intermediate}, \refKey{printed} (et
\refKey{printed*}) suivantes, la \yatcl{} permet de facilement obtenir des
versions du document :
\begin{itemize}
\item \enquote{brouillon} ;
\item \enquote{intermédiaire}
\item \enquote{imprimée}, sous deux formes possibles ;
\end{itemize}
différentes de celle par défaut qui peut être considérée comme une version
\enquote{écran}.

\begin{docKey}{draft}{=\docValue{true}\textbar\docValue{false}}{par défaut \docValue{true},
    initialement \docValue{false}}
  Cette option produit une version \enquote{brouillon}\footnote{Une telle
    version est \emph{a priori} à usage exclusif de l'auteur et n'est en
    particulier pas destinée à être diffusée.} du document. Les différences par
  rapport à la version \enquote{écran} (par défaut) sont les suivantes :
  \begin{enumerate}
  \item Les dépassements de marges sont signalés par une barre
    verticale noire dans la marge.
  \item La mention \enquote{brouillon} ou
    \foreignquote{english}{draft}\selonlangue{} figure en filigrane (et en
    capitales) sur (presque) toutes les pages du document.
  \item Aucun avertissement propre à la \yatcl{}
    (cf. \vref{sec:non-affichage-des}) n'est affiché.
  \item Par exemple sur les pages de titre ou en intitulé de certains chapitres
    (des remerciements, introduction, conclusion, etc.) :
    \begin{enumerate}
    \item les données caractéristiques de la thèse\footnote{Auteur,
        (sous-)titre, institut(s), directeurs, rapporteurs,
        examinateurs, etc.} sont des hyperliens vers le fichier de
      configuration de la thèse\footnote{Cf. \vref{sec:lieu-de-saisie}.} où
      il est possible de les (re)définir (cf.
      \vref{sec:expressions-cles});
    \item\label{item-expression} les expressions fournies par
      la \yatcl\footnote{\enquote{Thèse présentée par},
        \foreignquote{english}{In order to become Doctor from},
        \foreignquote{english}{draft}, \enquote{Version
          intermédiaire en date du}, etc. insérées de façon
        automatique sur certaines pages du mémoire.} sont :
      \begin{itemize}
      \item estampillées du label qui les identifie;
      \item des hyperliens vers le fichier de configuration de la thèse (cf.
        \vref{rq:configurationfile}) où il est possible de les
        (re)définir (cf. \vref{sec:expressions-cles}).
    \end{itemize}
    \end{enumerate}
    Si le système d'exploitation est correctement configuré, un
    simple clic sur ces hyperliens ouvre le fichier correspondant
    dans l'éditeur de texte \LaTeX{} par défaut.
  \end{enumerate}
\end{docKey}

\begin{docKey}{intermediate}{=\docValue{true}\textbar\docValue{false}}{par défaut \docValue{true},
    initialement \docValue{false}}
  Cette option produit une version \enquote{intermédiaire}\footnote{Un telle
    version est éventuellement destinée à être diffusée à des relecteurs.} du
  document sur laquelle est clairement indiqué qu'il s'agit d'une version non
  définitive. Les différences par rapport à la version \enquote{écran} (par
  défaut) sont les suivantes :
  \begin{enumerate}
  \item Les mentions suivantes figurant sur (presque) toutes les pages du
    document :
    \begin{itemize}
    \item \enquote{travail en cours} ou \foreignquote{english}{work in
        progress}\selonlangue{}, en filigrane (et en capitales);
    \item \enquote{Version intermédiaire en date du \meta{date}} ou
      \foreignquote{english}{Intermediate version as of
        \meta{date}}\selonlangue{} en pied de page (et en petites
        capitales).
    \end{itemize}
  \item Aucun avertissement propre à la \yatcl{}
    (cf. \vref{sec:non-affichage-des}) n'est affiché.
  \end{enumerate}
\end{docKey}

\begin{docKey}{printed}{=\docValue{true}\textbar\docValue{false}}{par défaut \docValue{true},
    initialement \docValue{false}}
  Cette option produit une version du document destinée à être imprimée
  sur papier. Les différences par rapport à la version \enquote{écran} (par
  défaut) sont les suivantes :
  \begin{enumerate}
  \item la commande
    ×\href{×\meta{\acrshort*{url}}×}{×\meta{texte}×}×\footnote{Fournie par le
      \Package{hyperref}, chargé par la \yatcl{}.} est automatiquement
    remplacée par :
    \begin{itemize}
    \item \meta{texte}\lstinline[deletekeywords={[2]url}]+\footnote{\url{+\meta{\acrshort*{url}}×}}× si elle
      figure dans le texte ordinaire ;
    \item \meta{texte} \lstinline[deletekeywords={[2]url}]+(\url{+\meta{\acrshort*{url}}×})× si elle
      figure en note de bas de page ;
    \end{itemize}
  \item les liens hypertexte sont supprimés ;
  \item les barres de navigation affichées par certains styles de
    glossaires\footnote{Telles qu'on peut en voir
      \vref{fig:printacronyms,fig:printglossary}.} \emph{sont} masquées.
  \end{enumerate}
\end{docKey}

\begin{docKey}{printed*}{=\docValue{true}\textbar\docValue{false}}{par défaut \docValue{true},
    initialement \docValue{false}}
  Cette option produit le même effet que l'option \refKey{printed} sauf que les
  barres de navigation affichées par certains styles de glossaires \emph{ne}
  sont \emph{pas} masquées.
\end{docKey}

\begin{dbremark}{Versions compatibles}{}
  Les différentes versions ci-dessus du mémoire ne sont pas incompatibles mais
  seule la conjonction \refKey{intermediate} et \refKey{printed} (ou
  \refKey{printed*}) peut réellement être utile.
\end{dbremark}

\section{Commandes et options de commandes de la \yatcl}

\begin{dbremark}{Fichier de configuration}{configurationfile}
  Les commandes de personnalisation :
  \begin{itemize}
  \item listées dans cette section et donc propres à \yatcl{} ;
  \item fournies par les packages chargés manuellement ;
  \end{itemize}
  peuvent être saisies :
  \begin{itemize}
  \item soit directement dans le (préambule du) fichier (maître) de
    la thèse ;
  \item soit dans un fichier (prévu à cet effet) à nommer
    \file{\configurationfile} et à placer dans un sous-dossier
    (prévu à cet effet) à nommer
    \directory{\configurationdirectory}\footnote{Ces fichier et
      sous-dossier sont à créer au besoin mais le canevas de thèse
      \enquote{en relief} livré avec la \yatcl, décrit
      \vref{sec:canevas-relief}, les fournit.}.
  \end{itemize}
\end{dbremark}

\begin{dbwarning}{Fichier de configuration à ne pas importer}{}
  Le \File{\configurationfile} est \emph{automatiquement} importé par la
  \yatcl{} : il doit donc \emph{ne pas} être explicitement importé \phrase*{au
    moyen d'une commande \docAuxCommand{input} ou assimilée}.
\end{dbwarning}

\subsection{(Re)Définition des expressions de la thèse}\label{sec:expressions-cles}

Un mémoire de thèse composé avec la \yatcl est émaillé d'expressions insérées
de façon automatique sur certaines pages (titre, mots clés, laboratoire,
résumés, etc.). Que ces expressions soient définies par la \yatcl ou bien
standard, il est possible de les redéfinir.

\subsubsection{Expressions définies par la classe}
\label{sec:expr-defin-par}

Les expressions \meta{en français} et \meta{en anglais} définies par la \yatcl
sont listées dans le \vref{tab:expressions-cles} et y sont identifiées par un
\meta{label} permettant de les redéfinir (voire de les définir, cf.
\vref{ex:doctor}) au moyen de la commande \refCom{expression} suivante.
%
\begin{docCommand}{expression}{\marg{label}\marg{en français}\marg{en anglais}}
  Cette commande permet de (re)définir les valeurs \meta{en français} et
  \meta{en anglais} de l'expression identifiée par \meta{label}.
\end{docCommand}

\begin{table}
  \centering
  \lstset{%
  morekeywords=[5]{% Translation labels
    coinstitute,company,institute,cosupervisor,cosupervisor*,comonitor,%
    comonitor*,supervisor,supervisor*,academicfield,doctoralschool,keywords,%
    ordernumber,committeepresident,committeepresident*,speciality%
  }%
}
%
\newcommand{\expression}[3]{%^^A
  \ifthenelse{\isempty{#2}}{%^^A
    \meta{vide}%^^A
  }{%
    #2%^^A
  }%
  &
  \ifthenelse{\isempty{#3}}{%^^A
    \meta{vide}%^^A
  }{%
    #3%^^A
  }%
  &
  #1%^^A
  \\%^^A
}
\bgroup%
\footnotesize%
\begin{tabulary}{\linewidth}{LL>{\ttfamily\color{keyword5}}l}
  Valeur en français & Valeur en anglais & \textcolor{black}{Label} \\\toprule
  \expression{lbl-aim}{En vue de l'obtention du grade de docteur de l'}{In order to
  become Doctor from }%
\expression{lbl-aimand}{ et de l'}{ and from }%
\expression{lbl-caution}{Avertissement}{Caution}%
\expression{lbl-coinstitute}{}{}%
\expression{lbl-committeemembers}{Composition du jury}{Committee members}%
\expression{lbl-committeepresident}{pr\'esident}{President}%
\expression{lbl-comonitor}{co-encadrant}{Co-Monitor}%
\expression{lbl-company}{}{}%
\expression{lbl-conclusion}{Conclusion}{Conclusion}%
\expression{lbl-cosupervisor}{co-directeur}{Co-Supervisor}%
\expression{lbl-defendedon}{Soutenue le}{Defended on}%
\expression{lbl-doctoralschool}{\'Ecole doctorale}{Doctoral School}%
\expression{lbl-draft}{brouillon}{draft}%
\expression{lbl-email}{\Email}{\Email}%
\expression{lbl-examiners}{Examinateur}{Examiner}%
\expression{lbl-examiners-pl}{Examinateurs}{Examiners}%
\expression{lbl-fax}{\fax}{\fax}%
\expression{lbl-guests}{Invit\'e}{Guest}%
\expression{lbl-guests-pl}{Invit\'es}{Guests}%
\expression{lbl-institute}{}{}%
\expression{lbl-inprogress}{travail en cours}{work in progress}%
\expression{lbl-introduction}{Introduction}{Introduction}%
\expression{lbl-juniorresearcher}{charg\'e de recherche}{Junior Researcher}%
\expression{lbl-juniorresearcher*}{charg\'e de recherche
  \textsc{hdr}}{\textsc{hdr} Junior Researcher}%
\expression{lbl-keywords}{Mots cl\'es}{Keywords}%
\expression{lbl-mcf}{\textsc{mcf}}{Lecturer}%
\expression{lbl-mcf*}{\textsc{mcf} \textsc{hdr}}{\textsc{hdr} Lecturer}%
\expression{lbl-ordernumber}{Num\'ero d'ordre}{Order Number}%
\expression{lbl-phdthesis}{th\`ese}{Ph. D. Thesis}%
\expression{lbl-phone}{\Telefon}{\Telefon}%
\expression{lbl-preface}{Pr\'eface}{Preface}%
\expression{lbl-prepared-at}{Cette th\`ese a \'et\'e pr\'epar\'ee au}{This
  thesis has been prepared at}%
\expression{lbl-prepared-at-pl}{Cette th\`ese a \'et\'e pr\'epar\'ee dans les
  laboratoires suivants.}{This thesis has been prepared at the following
  research units.}%
\expression{lbl-professor}{professeur}{Professor}%
\expression{lbl-referees}{Rapporteur}{Referee}%
\expression{lbl-referees-pl}{Rapporteurs}{Referees}%
\expression{lbl-seniorresearcher}{directeur de recherche}{Senior Researcher}%
\expression{lbl-academicfield}{Discipline}{Academic Field}%
\expression{lbl-speciality}{Sp\'ecialit\'e}{Speciality}%
\expression{lbl-supervisor}{directeur}{Supervisor}%
\expression{lbl-supervisors}{Directeur de th\`ese}{Supervisor}%
\expression{lbl-supervisors-pl}{Directeurs de th\`ese}{Supervisors}%
\expression{lbl-thesisdefendedby}{Th\`ese pr\'esent\'ee par}{Thesis defended
  by}%
\expression{lbl-thesistitle}{Titre de la th\`ese}{Thesis Title}%
\expression{lbl-universitydepartment}{Unit\'e de recherche}{University
  Department}%
\expression{lbl-versiondate}{Version interm\'ediaire en date du}{Work in
  progress as of}%
\expression{lbl-website}{Site}{Web Site}%

\end{tabulary}
\egroup%

  \caption{Labels et valeurs des expressions de la \yatcl}
  \label{tab:expressions-cles}
\end{table}

\begin{dbexample}{Modification d'expression définie par la classe}{}
  L'expression \enquote{En vue de l'obtention du grade de docteur de l'} (dont
  le label est ×lbl-aim×) n'est pas appropriée si le nom de l'institut dans
  lequel a été préparée la thèse a pour initiale une consonne\footnote{Le choix
    de \enquote{de l'} plutôt que de \enquote{du} tient à ce que la plupart des
    thèses préparées en France le sont dans des \enquote{universités}, des
    \enquote{écoles} ou des \enquote{instituts}, termes qui commencent tous par
    une voyelle.}. Ainsi, si l'institut est par exemple la \enquote{Cité des
    sciences}, l'expression qui figure sur la page de titre en français :
  \enquote{En vue de l'obtention du grade de docteur de l'Cité des sciences}
  est inappropriée. On peut l'adapter en saisissant (notamment dans le
  \File{\configurationfile}) par exemple\footnote{La version anglaise de
    l'expression reste valable.} :
  %
\begin{preamblecode}[title=Par exemple dans le \File{\configurationfile}]
\expression{lbl-aim}{En vue de l'obtention du grade de docteur de la}{In order to become Doctor from}
\end{preamblecode}
\end{dbexample}
%
\begin{dbexample}{Suppression d'expression définie par la classe}{}
  Si on souhaite supprimer des pages de titre les mentions \enquote{Titre de la
    thèse} et \foreignquote{english}{Thesis Title} (expressions dont le label
  est ×lbl-thesistitle×), il suffit de saisir :
\begin{preamblecode}[title=Par exemple dans le \File{\configurationfile}]
\expression{lbl-thesistitle}{}{}
\end{preamblecode}
\end{dbexample}

\begin{dbremark}{Modification et suppression d'expressions facilitées par la
    version \enquote{brouillon}}{}
  On a vu que l'option \refKey{draft} permet de facilement retrouver les labels
  des expressions et atteindre \file{\configurationfile} pour y modifier
  celles-ci.
\end{dbremark}

\subsubsection{Expressions standard}
\label{sec:expressions-standard}

Les commandes ×\addto×, ×\captionsfrench× et ×\captionsenglish× du
\Package{babel} permettent de redéfinir les expressions standard listées
\vref{tab:expressions-standard} au moyen de la syntaxe suivante.
\begin{preamblecode}[title=Par exemple dans le \File{\configurationfile}]
\addto\captionsfrench{\def\÷\meta{commande}÷{÷\meta{en français}÷}}
\addto\captionsenglish{\def\÷\meta{commande}÷{÷\meta{en anglais}÷}}
\end{preamblecode}
\begin{table}[hb]
  \centering
  \begin{tabular}{lll}
 Commande                      & Valeur en français & Valeur en anglais \\\toprule
 \lstinline+\abstractname+     & Résumé             & Abstract          \\
 \lstinline+\alsoname+         & voir aussi         & see also          \\
 \lstinline+\appendixname+     & Annexe             & Appendix          \\
 \lstinline+\bibname+          & Bibliographie      & Bibliography      \\
% \lstinline+\ccname+      & Copie à            & cc                \\
 \lstinline+\chaptername+      & Chapitre           & Chapter           \\
 \lstinline+\contentsname+     & Table des matières & Contents          \\
% \lstinline+\enclname+    & P.J.               & encl              \\
 \lstinline+\figurename+       & Figure             & Figure            \\
 \lstinline+\glossaryname+     & Glossaire          & Glossary          \\
 \lstinline+\indexname+        & Index              & Index             \\
 \lstinline+\listfigurename+   & Table des figures  & List of Figures   \\
 \lstinline+\listtablename+    & Liste des tableaux & List of Tables    \\
 \lstinline+\pagename+         & page               & Page              \\
 \lstinline+\partname+         & partie             & Part              \\
% \lstinline+\prefacename+ & Préface            & Preface           \\
 \lstinline+\proofname+        & Démonstration      & Proof             \\
 \lstinline+\refname+          & Références         & References        \\
 \lstinline+\seename+          & voir               & see               \\
 \lstinline+\tablename+        & Tableau            & Table
\end{tabular}

  \caption{Valeurs et commandes d'expressions standard du \Package{babel}}
  \label{tab:expressions-standard}
\end{table}
\begin{dbexample}{Redéfinition d'expressions du \Package{babel}}{}
\begin{preamblecode}[title=Par exemple dans le \File{\configurationfile}]
\addto\captionsfrench{\def\abstractname{Aperçu de notre travail}}
\addto\captionsenglish{\def\abstractname{Overview of our work}}
\end{preamblecode}
\end{dbexample}

En cas d'usage des packages \package{glossaries} et \package{biblatex}, la
syntaxe précédente est inopérante avec les commandes ×\glossaryname× et
×\bibname× (ainsi que ×\refname×). Dans ce cas, pour donner un \meta{titre
  alternatif} :
\begin{itemize}
\item aux glossaire, liste d'acronymes et liste de symboles, on recourra à
  l'une ou l'autre des instructions suivantes :
\begin{bodycode}
\printglossary[title=÷\meta{titre alternatif}÷]
\printglossaries[title=÷\meta{titre alternatif}÷]
\printacronyms[title=÷\meta{titre alternatif}÷]
\printsymbols[title=÷\meta{titre alternatif}÷]
\end{bodycode}
\item à la bibliographie, on recourra à :
\begin{bodycode}
\printbibliography[title=÷\meta{titre alternatif}÷]
\end{bodycode}
\end{itemize}

En outre, en cas d'usage du \Package*{listings}, un \meta{titre alternatif}
pourra être donné à la liste des listings, au moyen de:
\begin{preamblecode}[title=Par exemple dans le \File{\configurationfile}]
\renewcommand\lstlistingname{÷\meta{titre alternatif}÷}
\end{preamblecode}

\subsection{Nouvelles corporations}\label{sec:corporations}

On a vu que les commandes définissant les membres du
jury\footnote{\refCom{supervisor}, \refCom{referee}, \refCom{examiner}, etc.,
  cf. \vref{sec:jury}.} permettent de préciser\footnote{Au moyen des clés
  \refKey{professor}, \refKey{mcf}, \refKey{mcf*}, \refKey{seniorresearcher},
  \refKey{juniorresearcher} et \refKey{juniorresearcher*}.} si ceux-ci
appartiennent aux corporations \emph{prédéfinies} des professeurs ou des
maîtres de conférences (\gls{hdr} ou pas) des universités et des directeurs de
recherche ou des chargé(e)s de recherche (\gls{hdr} ou pas) du \gls{cnrs}. La
clé \refKey{corporation} suivante permet de spécifier de \emph{nouvelles}
corporations.

\begin{docKey}{corporation}{=\meta{label}}{pas de
    valeur par défaut, initialement vide}
  L'option ×corporation=×\meta{label} permet de stipuler une
  \meta{corporation en français} et une \meta{corporation en
    anglais} où \meta{label} identifie une expression à
  définir au moyen de :
\begin{preamblecode}[title=Par exemple dans le \File{\configurationfile}]
\expression{÷\meta{label}÷}{÷\meta{corporation en français}÷}{÷\meta{corporation en anglais}÷}
\end{preamblecode}
\end{docKey}

\begin{dbexample}{Nouvelle corporation}{doctor}
  Si on souhaite spécifier que certains membres du jury sont docteurs, il
  suffit de définir \phrase{une seule fois} l'expression suivante de label (par
  exemple) ×lbl-doctor× :
\begin{preamblecode}[title=Par exemple dans le \File{\configurationfile}]
\expression{lbl-doctor}{docteur}{Doctor}
\end{preamblecode}
pour pouvoir ensuite l'utiliser \phrase{autant de fois que souhaité}, par
exemple ainsi :
\begin{bodycode}
\examiner[corporation=lbl-doctor]{Joseph}{Fourier}
\examiner[corporation=lbl-doctor]{Paul}{Verlaine}
\end{bodycode}
\end{dbexample}

\section{Packages chargés par la \yatcl}

%^^A \subsection{Glossaires absents de la table des matières}
%^^A
%^^A La \yatcl{} fait par défaut figurer les glossaires dans les sommaire, table des
%^^A matières et signets du document. Si cela n'est pas souhaité, il suffit
%^^A d'insérer la commande \docAuxCommand{glstocfalse} après le chargement du
%^^A \Package{glossaries}.

Le comportement par défaut de la \yatcl{} est aussi gouverné par le
comportement par défaut des packages qu'elle charge
automatiquement\footnote{ On pourra le cas échéant consulter
  \vref{cha:packages-charges} la liste des packages chargés par la \yatcl{}.}.
La personnalisation de \yat{} peut donc aussi passer par celle de ces
packages.

Deux exemples parmi d'autres sont traités ci-après.

\subsection{Bibliographie absente de la table des matières}

La \yatcl{} fait par défaut figurer la bibliographie dans les sommaire, table
des matières et signets du document. Si cela n'est pas souhaité, il suffit de
passer à la commande \docAuxCommand{printbibliography} l'option
×heading=×\meta{entête}, où \meta{entête} vaut par exemple
\docValue*{bibliography} (cf. la documentation du \Package{biblatex} pour plus
de détails).

\subsection{Profondeurs différentes pour les signets et la table des matières}

Par défaut, la table des matières et les signets ont le même niveau de
profondeur. Mais, grâce à l'option \docAuxKey{bookmarksdepth} du
\Package*{hyperref}, il est possible de spécifier un \meta{autre niveau} pour
ces derniers :
\begin{preamblecode}[title=Par exemple dans le \File{\configurationfile}]
\hypersetup{bookmarksdepth=÷\meta{autre niveau}÷}
\end{preamblecode}
où \meta{autre niveau} est l'une des valeurs possibles de la clé
\refKey{depth}.

\section{Packages chargés manuellement}
\label{sec:options-de-classes}
Si on souhaite recourir à des packages qui ne sont pas appelés par la \yatcl{},
on les chargera manuellement.

\begin{dbwarning}{Chargement de packages : en préambule du fichier maître}{}
  Le chargement manuel de packages doit se faire exclusivement dans le
  (préambule du) fichier (maître) de la thèse et notamment \emph{pas} dans le
  \File{\configurationfile} dont il est question
  \vref{rq:configurationfile}.
\end{dbwarning}

%
\iffalse
%%% Local Variables:
%%% mode: latex
%%% eval: (latex-mode)
%%% ispell-local-dictionary: "fr_FR"
%%% TeX-engine: xetex
%%% TeX-master: "../yathesis.dtx"
%%% End:
\fi
