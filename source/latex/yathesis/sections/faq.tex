\chapter{FAQ}\label{cha:faq}

Ce chapitre répertorie les questions fréquemment posées sur la \yatcl{}.

\begin{question}
  La \yatcl{} est vraiment formidable, mais je souhaite :
  \begin{enumerate}
  \item rapporter un bogue ;
  \item demander une nouvelle fonctionnalité ;
  \item communiquer avec l'auteur de la classe.
  \end{enumerate}
  Comment faire ?
\end{question}
\begin{solution}
  Pour les rapports de bogue et demandes de fonctionnalités, créer une
  \enquote{issue} à l'adresse
  \url{https://github.com/dbitouze/yathesis/issues/new}.

  Pour communiquer avec l'auteur de la classe, utiliser l'adresse indiquée en
  1\iere{} page de la présente documentation.
\end{solution}

\begin{question}
  Par défaut, les glossaire, liste d'acronymes et liste de symboles ne figurent
  pas dans les sommaire, table des matières et signets du document. Comment les
  faire apparaître ?
\end{question}
\begin{solution}
  Cette question ne concerne pas directement la \yatcl{} : il suffit de passer
  l'option \docAuxKey*{toc} au \Package{glossaries}.
\end{solution}

\begin{question}
  Je suis confronté à l'erreur \enquote{no room for a new \docAuxCommand{write}}.
\end{question}
\begin{solution}
  Il devrait suffire de charger le \Package{morewrites} (plutôt parmi les
  premiers packages).
\end{solution}

%
\iffalse
%%% Local Variables:
%%% mode: latex
%%% eval: (latex-mode)
%%% ispell-local-dictionary: "fr_FR"
%%% TeX-engine: xetex
%%% TeX-master: "../yathesis.dtx"
%%% End:
\fi
