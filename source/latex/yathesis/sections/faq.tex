\chapter{\texorpdfstring{\acrshort{faq}}{FAQ}}\label{cha:faq}

Ce chapitre répertorie les questions fréquemment posées sur la \yatcl{}.

\begin{dbfaq}{Comment communiquer avec l'auteur de la \yatcl{} ?}{bogues}
  La \yatcl{} est vraiment formidable, mais je souhaite :
  \begin{enumerate}
  \item signaler un dysfonctionnement (un bogue) ;
  \item demander une nouvelle fonctionnalité ;
  \item communiquer avec l'auteur de la classe.
  \end{enumerate}
  Comment faire ?
  %
  \tcblower
  %
  Pour les rapports de bogues et demandes de fonctionnalités, le mieux est de
  créer une \enquote{issue} à l'adresse
  \url{https://github.com/dbitouze/yathesis/issues/new} (pour les bogues, un
  \gls{ecm} est vivement souhaité).

  Pour communiquer avec l'auteur de la classe, il est possible d'utiliser l'adresse indiquée en
  page de titre de la présente documentation.
\end{dbfaq}

\begin{dbfaq}{Comment faire figurer le glossaire dans la table des matières ?}{}
  Par défaut, le glossaire (et les listes d'acronymes et de symboles) ne figure
  pas dans la table des matières (ni dans le sommaire, ni dans les signets) du
  document. Comment le faire apparaître ?
  %
  \tcblower
  %
  Cette question ne concerne pas directement la \yatcl{} : il suffit de passer
  l'option \docAuxKey*{toc} au \Package{glossaries}.
\end{dbfaq}

\begin{dbfaq}{Pourquoi mes signes de ponctuation ne sont pas précédés
    des espaces adéquates ?}{}
  J'ai saisi les caractéristiques du document (cf. \vref{cha:caract-du-docum})
  en préambule. Certaines d'entre elles contiennent des signes de ponctuation
  haute ({\NoAutoSpacing?;:!}) qui, dans le \pdf produit, ne sont pas précédés
  des espaces adéquates. Comment régler ce problème ?
  %
  \tcblower
  %
  Il suffit de saisir ces caractéristiques
  plutôt\footnote{Cf. \vref{sec:lieu-de-saisie}.} :
  \begin{itemize}
  \item soit dans le \emph{corps} du fichier (maître) de la thèse\footnote{Mais
      cf. alors \vref{wa:avant-maketitle}.} (et donc \emph{pas} dans son
    \emph{préambule}) ;
  \item soit dans le fichier \file{\characteristicsfile} prévu à cet effet.
  \end{itemize}
  Si on tient absolument à ce que ces caractéristiques soient saisies dans le
  préambule, il faut alors les entourer des commandes ×\shorthandon{;:!?}× et
  ×\shorthandoff{;:!?}×.
\end{dbfaq}

\begin{dbfaq}{Comment éviter l'erreur \enquote{Option clash for package babel} ?}{}
  Je suis confronté à l'erreur \enquote{Option clash for package
    babel}. Comment l'éviter ?
  %
  \tcblower
  %
  C'est probablement parce que le \Package{babel} a été explicitement chargé au
  moyen de la commande ×\usepackage[...]{babel}×, déconseillée avec la
  \yatcl{}. Supprimer cette commande devrait résoudre le problème
  (cf. \vref{sec:langues} pour plus de détails concernant la gestion des
  langues).
\end{dbfaq}

\begin{dbfaq}{Comment éviter l'erreur \enquote{No room for a new
      \protect\docAuxCommand{write}} ?}{}
  Je suis confronté à l'erreur \enquote{no room for a new
    \docAuxCommand{write}}. Comment l'éviter ?
  %
  \tcblower
  %
  Il devrait suffire de charger le \Package{morewrites} (plutôt parmi les
  premiers packages).
\end{dbfaq}

\begin{dbfaq}{Dans la table des matières, des numéros de pages débordent dans
    la marge de droite}{}
  Dans la table des matières, certains numéros de pages (en chiffres romains
  notamment) débordent dans la marge de droite. Comment l'éviter ?
  %
  \tcblower
  %
  Il suffit d'insérer, en préambule du fichier (maître) de la thèse ou dans le
  \File{\configurationfile}, les lignes :
\begin{preamblecode}
\makeatletter
\renewcommand*\@pnumwidth{÷\meta{distance}÷}
\makeatother
\end{preamblecode}
où \meta{distance}, à exprimer par exemple en points (par exemple ×27pt×), est
à déterminer par \enquote{essais/erreurs} de sorte que \meta{distance} soit :
\begin{enumerate}
\item suffisamment grande, pour empêcher les débordements de numéros de pages ;
\item aussi petite que possible, pour éviter les lignes de pointillés trop
  courtes.
\end{enumerate}
\end{dbfaq}

\begin{dbfaq}{Les sommaire et table des matières ne sont pas ceux attendus}{}
  Les sommaire et table des matières ne sont pas ceux attendus : par exemple,
  ils ne commencent pas sur une nouvelle page, le niveau de profondeur du
  sommaire spécifié par \refKey{depth} n'est pas pris en compte, etc. Que faire ?
  %
  \tcblower
  %
  Si le \Package{tocloft} a été chargé, le supprimer car il est incompatible
  avec la \yatcl{}. Sinon, faire un \hyperref[faq:bogues]{rapport de bogue}
  à l'auteur de la classe.
\end{dbfaq}


%
\iffalse
%%% Local Variables:
%%% mode: latex
%%% eval: (latex-mode)
%%% ispell-local-dictionary: "fr_FR"
%%% TeX-engine: xetex
%%% TeX-master: "../yathesis.dtx"
%%% End:
\fi
