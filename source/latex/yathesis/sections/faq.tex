\chapter{\texorpdfstring{\acrshort{faq}}{FAQ}}\label{cha:faq}

Ce chapitre répertorie les \glspl{faq} sur la \yatcl{}.

\begin{dbfaq}{Comment communiquer avec l'auteur de la \yatcl{} ?}{}
  La \yatcl{} est vraiment formidable, mais je souhaite :
  \begin{enumerate}
  \item rapporter un bogue ;
  \item demander une nouvelle fonctionnalité ;
  \item communiquer avec l'auteur de la classe.
  \end{enumerate}
  Comment faire ?
  %
  \tcblower
  %
  Pour les rapports de bogue et demandes de fonctionnalités, créer une
  \enquote{issue} à l'adresse
  \url{https://github.com/dbitouze/yathesis/issues/new}.

  Pour communiquer avec l'auteur de la classe, utiliser l'adresse indiquée en
  1\iere{} page de la présente documentation.
\end{dbfaq}

\begin{dbfaq}{Comment faire figurer les glossaire, liste d'acronymes et liste
    de symboles dans la table des matières ?}{}
  Par défaut, les glossaire, liste d'acronymes et liste de symboles ne figurent
  pas dans les sommaire, table des matières et signets du document. Comment les
  faire apparaître ?
  %
  \tcblower
  %
  Cette question ne concerne pas directement la \yatcl{} : il suffit de passer
  l'option \docAuxKey*{toc} au \Package{glossaries}.
\end{dbfaq}

\begin{dbfaq}{Pourquoi mes signes de ponctuation ne sont pas précédés
    des espaces adéquates ?}{}
  J'ai saisi les propriétés du document (cf. \vref{cha:propr-du-docum}) en
  préambule. Certaines d'entre elles contiennent des signes de ponctuation
  haute ({\NoAutoSpacing?;:!}) qui, dans le \pdf produit, ne sont pas précédés
  des espaces adéquates. Comment régler ce problème ?
  %
  \tcblower
  %
  Il suffit de saisir ces propriétés :
  \begin{itemize}
  \item soit dans le corps du fichier (maître) de la thèse (mais cf. alors
    \vref{wa:avant-maketitle}) ;
  \item soit dans le fichier \file{\propertiesfile} prévu à cet effet
    (cf. \vref{propertiesfile}).
  \end{itemize}
  Si on tient absolument à ce que ces propriétés soient saisies dans le
  préambule, il faut alors les entourer des commandes ×\shorthandon{;:!?}× et
  ×\shorthandoff{;:!?}×.
\end{dbfaq}

\begin{dbfaq}{Comment éviter l'erreur \enquote{no room for a new
      \protect\docAuxCommand{write}} ?}{}
  Je suis confronté à l'erreur \enquote{no room for a new
    \docAuxCommand{write}}. Comment l'éviter ?
  %
  \tcblower
  %
  Il devrait suffire de charger le \Package{morewrites} (plutôt parmi les
  premiers packages).
\end{dbfaq}

%
\iffalse
%%% Local Variables:
%%% mode: latex
%%% eval: (latex-mode)
%%% ispell-local-dictionary: "fr_FR"
%%% TeX-engine: xetex
%%% TeX-master: "../yathesis.dtx"
%%% End:
\fi
