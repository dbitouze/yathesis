\chapter{Questions fréquemment posées}\label{cha:faq}

Ce chapitre est une \gls{faq} \phrase{autrement dit une liste des questions
  fréquemment posées} sur la \yatcl{}.

\section{Communication}
\label{sec:communication}

\begin{dbfaq}{Comment communiquer avec l'auteur de la \yatcl{} ?}{bogues}
  La \yatcl{} est vraiment formidable, mais je souhaite :
  \begin{enumerate}
  \item signaler un dysfonctionnement (un bogue) ;
  \item demander une nouvelle fonctionnalité ;
  \item communiquer avec l'auteur de la classe.
  \end{enumerate}
  Comment faire ?
  %
  \tcblower
  %
  \begin{enumerate}
  \item Pour rapporter un dysfonctionnement :
    \begin{enumerate}
    \item s'assurer qu'il n'est pas déjà répertorié :
      \begin{enumerate}
      \item en lisant la suite du présent chapitre ;
      \item en lisant le \vref{cha:incomp-conn} ;
      \item en consultant la liste des \enquote{issues} à l'adresse
        \url{https://github.com/dbitouze/yathesis/issues/} ;
      \end{enumerate}
    \item s'il n'est pas déjà répertorié, créer une \enquote{issue} à l'adresse
      \url{https://github.com/dbitouze/yathesis/issues/new}\footnote{Un
        \gls{ecm} est vivement souhaité.}.
    \end{enumerate}
  \item Pour demander une fonctionnalité :
    \begin{enumerate}
    \item s'assurer qu'elle n'est pas déjà répertoriée en
      consultant la liste des \enquote{issues} à l'adresse
      \url{https://github.com/dbitouze/yathesis/issues/} ;
    \item si la fonctionnalité n'a pas déjà été demandée, créer une
      \enquote{issue} à l'adresse
      \url{https://github.com/dbitouze/yathesis/issues/new}.
    \end{enumerate}
  \item Pour communiquer avec l'auteur de la classe, il est possible d'utiliser
    l'adresse indiquée en page de titre de la présente documentation.
  \end{enumerate}
\end{dbfaq}

\section{Avertissements}
\label{sec:avertissements}

\begin{dbfaq}{Puis-je ignorer un avertissement signalant une version trop
    ancienne d'un package ?}{}
  Je suis confronté à un avertissement de la forme \enquote{You have requested,
    on input line \meta{numéro}, version `\meta{date plus récente}' of package
    \meta{nom d'un package}, but only version `\meta{date moins récente} ...'
    is available.}. Est-ce grave, docteur ?
  %
  \tcblower
  %
  Ça peut être grave. Cf. \vref{rq:packages-anciens} pour plus de précisions.
\end{dbfaq}

\section{Erreurs}
\label{sec:erreurs}

\begin{dbfaq}{Comment éviter l'erreur \enquote{Option clash for package
      \meta{package}} ?}{option-clash}
  Je suis confronté à l'erreur \enquote{Option clash for package
    \meta{package}} (notamment avec \meta{package}×=×\package{babel}). Comment
  l'éviter ?
  %
  \tcblower
  %
  Cette erreur est probablement due au fait que le \meta{package} a été
  manuellement chargé au moyen de la commande
  ×\usepackage[...]{×\meta{package}×}×, alors que la \yatcl{} le charge déjà
  automatiquement (cf. l'\vref{sec:packages-charges-par} pour la liste des
  packages automatiquement chargés). Supprimer cette commande devrait résoudre
  le problème (cf. également l'\vref{wa:packages-a-ne-pas-charger}).
\end{dbfaq}

% ^^A \begin{dbfaq}{Comment éviter l'erreur \enquote{No room for a new%
%   ^^A \protect\docAuxCommand*{write}} ?}{}%
%   ^^A Je suis confronté à l'erreur \enquote{no room for a new%
%   ^^A \docAuxCommand{write}}. Comment l'éviter ?%
%   ^^A%
%   ^^A \tcblower%
%   ^^A%
%   ^^A Il devrait suffire de charger le \Package{morewrites} (plutôt parmi%
%   ^^A les premiers packages).%
%   ^^A \end{dbfaq}

\section{Mise en page}
\label{sec:mise-en-page}

\begin{dbfaq}{Comment modifier l'apparence de la page de titre ?}{}
  L'apparence par défaut de la page de titre ne me convient pas et je voudrais
  la modifier. Comment faire ?
  %
  \tcblower
  %
  Il est prévu de permettre de modifier certains aspects de la mise en page de
  la page de titre, et même de fournir une documentation permettant d'obtenir
  une apparence complètement personnalisée, mais ce n'est pas encore
  implémenté.  En attendant que ça le soit, il faut composer cette page soit
  même, en y resaisissant manuellement toutes les caractéristiques nécessaires
  définies au \vref{cha:caract-du-docum}.
\end{dbfaq}

\begin{dbfaq}{Pourquoi les glossaire, listes d'acronymes et de symboles
    apparaissent en double dans la table des matières et dans les signets ?}{}
  Les glossaire, listes d'acronymes et de symboles apparaissent en double dans
  la table des matières et dans les signets. Comment éviter cela ?%
  \tcblower
  %
  La \yatcl{} fait d'elle-même figurer les glossaire, listes d'acronymes et de
  symboles à la fois dans la table des matières et dans les signets. Pour
  régler le problème, il devrait donc suffire de \emph{ne pas} explicitement
  demander que ce soit le cas, en \emph{ne} recourant \emph{ni} à l'option
  \docAuxKey*{toc} et \emph{ni} à la commande \docAuxCommand*{glstoctrue} du
  \Package{glossaries}.
\end{dbfaq}

\begin{dbfaq}{Pourquoi mes signes de ponctuation haute ne sont pas précédés des
    espaces adéquates ?}{}
  Certains éléments que j'ai saisis en préambule contiennent des signes de
  ponctuation haute ({\NoAutoSpacing?;:!}) mais, dans le \pdf produit, ces
  derniers ne sont pas précédés des espaces adéquates. Comment régler ce
  problème ?
  %
  \tcblower
  %
  Cette question ne concerne pas directement la \yatcl{} mais plutôt le
  \Package{babel} et les caractères actifs de son module \package{frenchb}. Si
  ces éléments concernent :
  \begin{enumerate}
  \item les caractéristiques du document (cf. \vref{cha:caract-du-docum}), il
    suffit de les saisir\footnote{Cf. \vref{sec:lieu-de-saisie}.} :
    \begin{itemize}
    \item soit dans le \emph{corps} du fichier (maître) de la
      thèse\footnote{Mais cf. alors \vref{wa:avant-maketitle}.} (et donc
      \emph{pas} dans son \emph{préambule}) ;
    \item soit dans le \File{\characteristicsfile} prévu à cet effet ;
    \item soit entre ×\shorthandon{;:!?}× et ×\shorthandoff{;:!?}× si on tient
      absolument à ce qu'ils soient saisis en préambule.
    \end{itemize}
  \item les termes du glossaire, des acronymes ou des symboles, il suffit de
    définir les entrées correspondantes ou d'utiliser la ou les commandes
    \docAuxCommand{loadglsentries} :
    \begin{itemize}
    \item soit dans le \File{\configurationfile}
      (cf. \vref{rq:configurationfile}) ;
    \item soit entre ×\shorthandon{;:!?}× et ×\shorthandoff{;:!?}×. Cette
      solution peut être préférée à la précédente pour ne pas perdre les
      fonctionnalités de complétion pour les labels des termes de glossaire
      fournies par certains éditeurs de texte orientés \LaTeX{}.
    \end{itemize}
  \end{enumerate}
\end{dbfaq}

\begin{dbfaq}{Dans la table des matières, des numéros de pages débordent dans
    la marge de droite}{}
  Dans la table des matières, certains numéros de pages (en chiffres romains
  notamment) débordent dans la marge de droite. Comment l'éviter ?
  %
  \tcblower
  %
  Il suffit d'insérer, en préambule du fichier (maître) de la thèse ou dans le
  \File{\configurationfile}, les lignes :
\begin{preamblecode}
\makeatletter
\renewcommand*\@pnumwidth{÷\meta{distance}÷}
\makeatother
\end{preamblecode}
  où \meta{distance}, à exprimer par exemple en points (par exemple ×27pt×),
  est à déterminer par \enquote{essais/erreurs} de sorte que \meta{distance}
  soit :
  \begin{enumerate}
  \item suffisamment grande, pour empêcher les débordements de numéros de
    pages ;
  \item aussi petite que possible, pour éviter les lignes de pointillés trop
    courtes.
  \end{enumerate}
\end{dbfaq}

\begin{dbfaq}{Pourquoi \protect\docAuxCommand*{setcounter} n'a-t-elle pas
    d'effet sur \protect\docAuxKey*{secnumdepth} ?}{}
  J'essaie de modifier la profondeur de numérotation de mon document en
  spécifiant la valeur du compteur \docAuxKey*{secnumdepth} au moyen de la
  commande :
\begin{preamblecode}
\setcounter{secnumdepth}{÷\meta{nombre}÷}
\end{preamblecode}
  mais cela n'a aucun effet. Pourquoi ?
  %
  \tcblower
  %
  La profondeur de numérotation d'un document composé avec la \yatcl{} est
  à spécifier au moyen de l'option de classe
  \refKey{secnumdepth}. Cf. \vref{sec:profondeur-de-la} pour plus de
  précisions.
\end{dbfaq}

\begin{dbfaq}{Comment faire en sorte que, dans la table des matières, seuls
    les numéros de page soient des liens hypertexte ?}{}
  Par défaut, les entrées de la table des matières sont toutes entières des
  liens hypertexte, ce qui est trop envahissant. Comment faire en sorte que
  seuls les numéros de page soient des liens hypertexte ?
  %
  \tcblower
  %
  Cette question ne concerne pas directement la \yatcl{} : il suffit de passer
  l'option ×linktoc=false× au \Package{hyperref}. Comme celui-ci est déjà
  chargé par la \yatcl{}, cette option est à passer (de préférence en
  préambule) en argument de \docAuxCommand{hypersetup} :
\begin{preamblecode}
\hypersetup{linktoc=false}
\end{preamblecode}
\end{dbfaq}

%
\iffalse
%%% Local Variables:
%%% mode: latex
%%% eval: (latex-mode)
%%% ispell-local-dictionary: "fr_FR"
%%% TeX-engine: xetex
%%% TeX-master: "../yathesis.dtx"
%%% End:
\fi
