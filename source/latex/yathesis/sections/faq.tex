\chapter{Questions fréquemment posées}\label{cha:faq}

Ce chapitre est une \gls{faq} \phrase{autrement dit une liste des questions
  fréquemment posées} sur la \yatcl{}.

\section{Communication}
\label{sec:communication}

\begin{dbfaq}{Comment communiquer avec l'auteur de la \yatcl{} ?}{bogues}
  La \yatcl{} est vraiment formidable, mais je souhaite :
  \begin{enumerate}
  \item signaler un dysfonctionnement (un bogue) ;
  \item demander une nouvelle fonctionnalité ;
  \item communiquer avec l'auteur de la classe.
  \end{enumerate}
  Comment faire ?
  %
  \tcblower
  %
  \begin{enumerate}
  \item Pour rapporter un dysfonctionnement :
    \begin{enumerate}
    \item s'assurer qu'il n'est pas déjà répertorié :
      \begin{enumerate}
      \item en lisant la suite du présent chapitre ;
      \item en lisant le \vref{cha:incomp-conn} ;
      \item en consultant la liste des \enquote{issues} à l'adresse
        \url{https://github.com/dbitouze/yathesis/issues/} ;
      \end{enumerate}
    \item s'il n'est pas déjà répertorié, créer une \enquote{issue} à l'adresse
      \url{https://github.com/dbitouze/yathesis/issues/new}\footnote{Un
        \gls{ecm} est vivement souhaité.}.
    \end{enumerate}
  \item Pour demander une fonctionnalité :
    \begin{enumerate}
    \item s'assurer qu'elle n'est pas déjà répertorié en
      consultant la liste des \enquote{issues} à l'adresse
      \url{https://github.com/dbitouze/yathesis/issues/} ;
    \item si la fonctionnalité n'a pas déjà été demandée, créer une
      \enquote{issue} à l'adresse
      \url{https://github.com/dbitouze/yathesis/issues/new}.
    \end{enumerate}
  \item Pour communiquer avec l'auteur de la classe, il est possible d'utiliser
    l'adresse indiquée en page de titre de la présente documentation.
  \end{enumerate}
\end{dbfaq}

\section{Avertissements}
\label{sec:avertissements}

\begin{dbfaq}{Pourquoi mes signes de ponctuation ne sont pas précédés
    des espaces adéquates ?}{}
  J'ai saisi les caractéristiques du document (cf. \vref{cha:caract-du-docum})
  en préambule. Certaines d'entre elles contiennent des signes de ponctuation
  haute ({\NoAutoSpacing?;:!}) qui, dans le \pdf produit, ne sont pas précédés
  des espaces adéquates. Comment régler ce problème ?
  %
  \tcblower
  %
  Il suffit de saisir ces caractéristiques
  plutôt\footnote{Cf. \vref{sec:lieu-de-saisie}.} :
  \begin{itemize}
  \item soit dans le \emph{corps} du fichier (maître) de la thèse\footnote{Mais
      cf. alors \vref{wa:avant-maketitle}.} (et donc \emph{pas} dans son
    \emph{préambule}) ;
  \item soit dans le fichier \file{\characteristicsfile} prévu à cet effet.
  \end{itemize}
  Si on tient absolument à ce que ces caractéristiques soient saisies dans le
  préambule, il faut alors les entourer des commandes ×\shorthandon{;:!?}× et
  ×\shorthandoff{;:!?}×.
\end{dbfaq}

\section{Erreurs}
\label{sec:erreurs}

\begin{dbfaq}{Comment éviter l'erreur \enquote{Option clash for package
      \meta{package}} ?}{option-clash}
  Je suis confronté à l'erreur \enquote{Option clash for package
    \meta{package}} (notamment avec \meta{package}×=×\package{babel}). Comment
  l'éviter ?
  %
  \tcblower
  %
  Cette erreur est probablement due au fait que le \meta{package} a été
  manuellement chargé au moyen de la commande
  ×\usepackage[...]{×\meta{package}×}×, alors que la \yatcl{} le charge déjà
  automatiquement (cf. l'\vref{sec:packages-charges-par} pour la liste des
  packages automatiquement chargés). Supprimer cette commande devrait résoudre
  le problème (cf. également l'\vref{wa:packages-a-ne-pas-charger}).
\end{dbfaq}

\begin{dbfaq}{Comment éviter l'erreur \enquote{No room for a new
      \protect\docAuxCommand*{write}} ?}{}
  Je suis confronté à l'erreur \enquote{no room for a new
    \docAuxCommand{write}}. Comment l'éviter ?
  %
  \tcblower
  %
  Il devrait suffire de charger le \Package{morewrites} (plutôt parmi les
  premiers packages).
\end{dbfaq}

\begin{dbfaq}{Dans la table des matières, des numéros de pages débordent dans
    la marge de droite}{}
  Dans la table des matières, certains numéros de pages (en chiffres romains
  notamment) débordent dans la marge de droite. Comment l'éviter ?
  %
  \tcblower
  %
  Il suffit d'insérer, en préambule du fichier (maître) de la thèse ou dans le
  \File{\configurationfile}, les lignes :
\begin{preamblecode}
\makeatletter
\renewcommand*\@pnumwidth{÷\meta{distance}÷}
\makeatother
\end{preamblecode}
où \meta{distance}, à exprimer par exemple en points (par exemple ×27pt×), est
à déterminer par \enquote{essais/erreurs} de sorte que \meta{distance} soit :
\begin{enumerate}
\item suffisamment grande, pour empêcher les débordements de numéros de pages ;
\item aussi petite que possible, pour éviter les lignes de pointillés trop
  courtes.
\end{enumerate}
\end{dbfaq}

%
\iffalse
%%% Local Variables:
%%% mode: latex
%%% eval: (latex-mode)
%%% ispell-local-dictionary: "fr_FR"
%%% TeX-engine: xetex
%%% TeX-master: "../yathesis.dtx"
%%% End:
\fi
