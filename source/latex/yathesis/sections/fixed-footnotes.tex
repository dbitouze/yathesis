\DeclareFixedFootnote{\pagededieelabo}{Produite au moyen de la commande
  facultative \protect\refCom{makelaboratory}.}
%
\DeclareFixedFootnote{\commandeacronyme}{Notamment une commande d'acronyme
  telle que \protect\docAuxCommand{gls} ou \protect\docAuxCommand{acrshort}.}
%
\DeclareFixedFootnote{\syntaxeoptions}{Le sens de la syntaxe décrivant les options est
  explicité \vref{sec:options}.}
%
\DeclareFixedFootnote{\versiontl}{L'année \directory{2013} est éventuellement à remplacer par celle de
  la version de la \TeX{}~Live effectivement utilisée.}
%
\DeclareFixedFootnote{\selonlangue}{Selon que la langue principale de la
  thèse est le français ou l'anglais.}
%
\DeclareFixedFootnote{\nofrontmatter}{Au contraire, la commande analogue
  \protect\docAuxCommand{frontmatter} pour les \protect\glspl{liminaire} ne
  doit pas être utilisée car elle l'est déjà en sous-main par la \yatcl{}.}
%
\DeclareFixedFootnote{\termesdefinisutilises}{Ne figurent dans ces listes que
  les termes, acronymes et symboles qui sont à la fois \emph{définis} et
  \emph{employés dans le texte}.}
%
\DeclareFixedFootnote{\redefexprcle}{Une autre manière de modifier cet intitulé
  est de recourir à la commande \protect\refCom{expression} pour redéfinir
  l'expression qui lui est attachée (cf. \vref{sec:expressions-standard}).}
%
\DeclareFixedFootnote{\hauteurpage}{Dans la limite de la hauteur de page.}
%
\DeclareFixedFootnote{\sepcorpaffil}{Selon l'initiale de l'institut :
  %
  \protect\lstinline[showspaces]+\ à l'+
  %
  ou
  %
  \protect\lstinline[showspaces]+\ au\ +.%
}
%
% \DeclareFixedFootnote{\noillustration}{Cette commande n'est pas illustrée car
%   elle est analogue aux commandes \protect\refCom{acknowledgements} et
%   \protect\refCom{caution}, illustrées
%   \vref{fig:acknowledgements,fig:caution}.}
%
\DeclareFixedFootnote{\nochapter}{Le contenu de ce chapitre doit donc \emph{ne
    pas} comporter d'occurrence de la commande \protect\docAuxCommand{chapter}.
  Il peut cependant contenir une ou plusieurs occurrences des autres commandes
  usuelles de structuration : \protect\docAuxCommand{section},
  \protect\docAuxCommand{subsection}, etc.}
%
\DeclareFixedFootnote{\fichierconfig}{Ceci peut être saisi directement dans le
  préambule du fichier (maître) de la thèse mais, pour optimiser l'usage de la
  \yatcl, il est conseillé de l'insérer dans un fichier nommé
  \file{\configurationfile} à placer dans un dossier nommé
  \directory{\configurationdirectory}. Le canevas de thèse livré avec la
  classe, décrit \vref{sec:canevas}, fournit ce dossier et ce fichier.}
%
\DeclareFixedFootnote{\ifscreenoutput}{Chargé seulement si le
  \Package{hyperref} l'est et si la clé \protect\refKey{output} n'a pour
  valeur ni \protect\docValue{paper}, ni \protect\docValue{paper*}.}
%
\DeclareFixedFootnote{\exceptoneside}{Sauf si l'option
  \protect\docAuxKey{oneside} est utilisée
  (cf. \vref{sec:options-usuelles-de}).}
%
\DeclareFixedFootnote{\noframe}{Sans le cadre.}
%
%
\iffalse
%%% Local Variables:
%%% mode: latex
%%% eval: (latex-mode)
%%% ispell-local-dictionary: "fr_FR"
%%% TeX-engine: xetex
%%% TeX-master: "../yathesis"
%%% End:
\fi
