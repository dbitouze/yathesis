\chapter{Packages chargés (ou pas) par la \yatcl{}}\label{cha:packages-charges}

\begin{dbremark*}{Chapitre à passer en 1\iere{} lecture}
  Ce chapitre est à passer en 1\iere{} lecture : il n'est utile qu'en cas de
  package chargé manuellement incompatible avec la \yatcl ou, éventuellement,
  pour s'épargner le chargement d'un package qui l'est déjà par la classe.
\end{dbremark*}

Ci-dessous, les packages qui peuvent être utiles dans le cadre d'un usage
standard de la \yatcl{} sont des hyperliens vers leur page sur le \gls{ctan}.

\section{Packages chargés par la \yatcl{}}

Pour plusieurs de ses fonctionnalités, la \yatcl s'appuie sur des packages
qu'elle charge explicitement. La liste suivante répertorie ces packages dans
l'ordre de chargement, en indiquant les raisons de leur emploi et les options
avec lesquelles ils sont appelés.

\begin{description}
\item[\package{xkvltxp} :] extension du \Package{xkeyval} ci-dessous ;
\item[\package{xkeyval} :] gestion d'options sous la forme
  \meta{clé}×=×\meta{valeur} ;
\item[\package{etoolbox} :] outils de programmation ;
\item[\package{xpatch} :] extension du package précédent ;
\item[\package{filehook} :] \enquote{hameçons} (\foreignquote{english}{hooks})
  pour fichiers importés ;
\item[\package{hopatch} :] emballage de \enquote{hameçons} pour packages et classes ;
\item[\package{xifthen} :] tests conditionnels ;
\item[\package*{geometry} :] gestion de la géométrie de la page.
  Option par défaut : \docAuxKey{a4paper} ;
\item[\package{textcomp} :] accès à certains caractères. Option par défaut :
  \docAuxKey{warn} ;
\item[\package*{graphicx} :] inclusion d'images, notamment des logos.
  Option par défaut : \docAuxKey{final} ;
\item[\package*{array} :] mise en forme automatique de colonnes (notamment) ;
\item[\package{xstring} :] manipulation de chaînes ;
\item[\package{translator} :] traduction d'expressions ;
\item[\package{fixltx2e} :] corrections de bogues de \hologo{LaTeX2e}
  ;
\item[\package*{epigraph} :] gestion des épigraphes ;
\item[\package*{marvosym} :] accès à des symboles spéciaux ;
\item[\package*{setspace} :] gestion de l'espace interligne ;
\item[\package{shorttoc} :] création de sommaire ;
\item[\package{tocvsec2} :] gestion des profondeurs de numérotation des
  sections et de la table des matières ;
\item[\package{tocbibind} :] table des matières et index dans la
  table des matières ;
\item[\package*{xcolor} :] gestion des couleurs ;
\item[\package{nonumonpart} :] suppression des numéros de pages sur
  les pages de garde des parties ;
\item[\package{datatool} :] gestion de bases de données (membres du
  jury, etc.) ;
\item[\package{fncychap} :] têtes de chapitres améliorées. Option par défaut :
  \docAuxKey{PetersLenny} ;
\item[\package{titleps} :] gestion des styles de pages ;
\item[\package{ifdraft} :] test conditionnel du mode brouillon ;
\item[\package{draftwatermark} :] texte en filigrane\footnote{Chargé seulement
    si l'une ou l'autre des valeurs \docValue{draft} ou \docValue{inprogress*}
    est passée à la clé \refKey{version}.} ;
\item[\package*{index} :] gestion du ou des index\footnote{Pour la gestion
    d'index, le \Package{makeidx} est plus courant mais le \Package*{index}
    l'améliore et offre des fonctionnalités supplémentaires, notamment pour
    produire des index multiples. Les deux ont des syntaxes très voisines. Le
    chargement du \Package{index} par la classe est nécessaire pour des raisons
    techniques.} ;
\item[\package*{idxlayout} :] correction d'un bogue affichant trop haut
  l'intitulé \enquote{Index} de l'index. Option par défaut
  \docAuxKey*{columns=1}\footnote{Si on souhaite un index non pas sur $1$ mais
    sur \meta{n} colonnes, il suffit de le spécifier au moyen de
    \protect\lstinline+\\idxlayout\{columns=+\meta{n}\protect\lstinline+\}+.} ;
\item[\package*{babel} :] gestion des langues ;
\item[\package{iflang} :] test de la langue en cours ;
\item[\package{datetime} :] gestion de la date. Option par défaut :
  \docAuxKey{nodayofweek} ;
\item[\package{datenumber} :] comparaison de dates ;
\item[\package*{hyperref} :] liens hypertextes. Options par défaut :
  \begin{itemize}
  \item \docAuxKey{final} ;
  \item \docAuxKey{unicode} ;
  \item \docAuxKey{breaklinks} ;
  \item ×hyperfootnotes=false× ;
  \item ×hyperindex=false×\footnote{Sans quoi certaines fonctionnalités sont
      ignorées, par exemple \protect\lstinline|see| pour les index.} ;
  \item ×plainpages=false× ;
  \item ×pdfpagemode=UseOutlines× ;
  \item ×pdfpagelayout=TwoPageRight× ;
  \end{itemize}
\item[\package{hypcap} :] liens hypertextes pointant au début des
  flottants\ifscreenoutput. Option par défaut : \docAuxKey{all} ;
\item[\package{bookmark} :] gestion des signets\ifscreenoutput. Option par
  défaut : \docAuxKey{numbered}.
\end{description}

\section{Packages non chargés par la \yatcl{}}

La liste suivante répertorie des packages non chargés par la \yatcl{} mais
pouvant se révéler très utiles, notamment aux doctorants.  Elle est loin d'être
exhaustive et ne mentionne notamment pas les packages nécessaires :
\begin{itemize}
\item \package{inputenc} et \package{fontenc}, si on utilise \hologo{LaTeX} ou
  \hologo{pdfLaTeX} ;
\item \package{fontspec} et \package{xunicode}, si on utilise \hologo{XeLaTeX}
  ou \hologo{LuaLaTeX}.
\end{itemize}
Elle ne mentionne pas non plus les packages de fontes PostScript tels que
\package{lmodern}, \package*{kpfonts}, \package*{fourier}, \package*{libertine},
etc. \phrase*{presque indispensables si on utilise \hologo{LaTeX} ou
  \hologo{pdfLaTeX}}. Des exemples de préambules complets figurent
\vref{cha:canevas}.

En outre, lorsqu'ils sont chargés manuellement par l'utilisateur, certains des
packages suivants se voient fixés par la \yatcl{} des options ou réglages dont
les plus notables sont précisés.

\begin{description}
\item[\package*{varioref} :] références croisées améliorées ;
\item[\package*{booktabs} :] tableaux plus professionnels ;
\item[\package*{siunitx} :] gestion des nombres, angles et unités. Réglages par
  défaut opérés par la \yatcl : \docAuxKey{detect-all} et \docAuxKey{locale=FR} ou
  \docAuxKey{locale=UK}\selonlangue{} ;
\item[\package*{pgfplots} :] graphiques plus professionnels,
  notamment de données expérimentales ;
\item[\package*{listings} :] insertion de listings informatiques ;
\item[\package*{microtype} :] raffinements typographiques
  automatiques (et subliminaux) ;
  % ^^A \footnote{Ce package peut poser problème s'il est déjà présent alors qu'une
  % ^^A fonte est utilisée pour la première fois. Il est donc à charger plutôt en
  % ^^A fin de rédaction, lors de la finition de la mise en page.}
\item[\package*{floatrow} :] gestion puissante (mais complexe) des
  flottants ;
\item[\package*{caption} :] personnalisation des légendes ;
\item[\package*{todonotes} :] insertion de
  \foreignquote{english}{TODOs}\footnote{Rappels de points qu'il ne
    faut pas oublier d'ajouter, de compléter, de réviser, etc.} ;
\item[\package*{csquotes} :] pour les citations informelles et formelles (avec
  citation des sources). Réglage par défaut opéré par la \yatcl (si le
  \package*{biblatex} est chargé) : ×\SetCiteCommand{\autocite}× ;
\item[\package*{biblatex} :] gestion puissante de la bibliographie ;
\item[\package*{glossaries} :] gestion puissante des glossaires,
  acronymes et liste de symboles ;
\item[\package*{cleveref} :] gestion puissante des références croisées.
\end{description}

%
\iffalse
%%% Local Variables:
%%% mode: latex
%%% eval: (latex-mode)
%%% ispell-local-dictionary: "fr_FR"
%%% TeX-engine: xetex
%%% TeX-master: "../yathesis.dtx"
%%% End:
\fi
