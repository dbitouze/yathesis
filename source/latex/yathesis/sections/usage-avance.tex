\chapter{Usage avancé}\label{cha:usage-avance}

\section{(Dés)Activation des erreurs ciblées propres aux éléments
  \enquote{obligatoires}}\label{sec:desact-des-erre}

On a vu \vref{sec:comm-oblig} que la \yatcl{} considère comme
\enquote{obligatoires} certains éléments (commandes et environnements). Leur
liste complète figure à la 1\iere{} colonne du \vref{tab:no-warnings}.
\begin{table}[ht]
  \centering
  \begin{tabular}{ll}
  Élément                    & Clé(s) de désactivation individuelle de l'erreur   \\\toprule
  \refCom{author}            & \refKey{noauthor}                                  \\
  \refCom{title}             & \refKey{notitle}                                   \\
  \refCom{academicfield}     & \refKey{noacademicfield}                           \\
  \refCom{date}              & \refKey{nodate}                                    \\
  \refCom{institute}         & \refKey{noinstitute}                               \\
  \refCom{doctoralschool}    & \refKey{nodoctoralschool}                          \\
  \refCom{laboratory}        & \refKey{nolaboratory}, \refKey{nolaboratoryadress} \\
  \refCom{supervisor}        & \refKey{nosupervisor}                              \\
  \refCom{maketitle}         & \refKey{nomaketitle}                               \\
  \refCom{keywords}          & \refKey{nokeywords}                                \\
  \refEnv{abstract}          & \refKey{noabstract}                                \\
  \refCom{makeabstract}      & \refKey{nomakeabstract}                            \\
  \refCom{tableofcontents}   & \refKey{notableofcontents}                         \\
  \refCom{printbibliography} & \refKey{noprintbibliography}                       \\\bottomrule
\end{tabular}

  \caption{Éléments \enquote{obligatoires} et options de désactivation des erreurs
    ciblées associées}
  \label{tab:no-warnings}
\end{table}

Dans le cadre d'un usage \emph{avancé} de la \yatcl{}, on peut décider de
passer outre le caractère \enquote{obligatoire} de tel ou tel élément.  Mais,
à chaque compilation, le fichier de \enquote{log} contient alors un
avertissement \enquote{personnalisé} rappelant que l'élément en question est
requis\footnote{Sauf si l'affichage des avertissements de la \yatcl{} est
  désactivé de façon \emph{globale} au moyen de la clé \refKey{nowarning} ou si
  on ne travaille qu'en version \enquote{brouillon} et \enquote{intermédiaire}
  du document (cf. clés \refKey{draft} et \refKey{intermediate}).}.
L'affichage de ce message \enquote{personnalisé} peut être désactivé de façon
ciblée au moyen d'une des clés figurant 2\ieme{} colonne du
\vref{tab:no-warnings} et dont les rôles sont précisés ci-après.

\begin{dbwarning}{Éléments \enquote{obligatoires} de la \yatcl{}
    fortement conseillés}{}
  Ne pas employer les éléments \enquote{obligatoires} de la \yatcl{} peut
  sérieusement altérer le bon fonctionnement de celle-ci. Cela est déconseillé,
  sauf dans le cadre d'un usage avancé \phrase*{si l'on est sûr de ce que l'on
    fait et qu'on pourra en gérer \emph{seul} les conséquences}.
\end{dbwarning}

\begin{docKey}{noauthor}{=\docValue{true}\textbar\docValue{false}}{par défaut \docValue{true},
    initialement \docValue{false}}
  Cette option désactive l'affichage de l'avertissement émis si la commande
  \refCom{author} est omise (ou à argument vide).
\end{docKey}
\begin{docKey}{notitle}{=\docValue{true}\textbar\docValue{false}}{par défaut \docValue{true},
    initialement \docValue{false}}
  Cette option désactive l'affichage de l'avertissement émis si la commande
  \refCom{title} est omise (ou à argument vide).
\end{docKey}
\begin{docKey}{noacademicfield}{=\docValue{true}\textbar\docValue{false}}{par défaut \docValue{true},
    initialement \docValue{false}}
  Cette option désactive l'affichage de l'avertissement émis si la commande
  \refCom{academicfield} est omise (ou à argument vide).
\end{docKey}
\begin{docKey}{nodate}{=\docValue{true}\textbar\docValue{false}}{par défaut \docValue{true},
    initialement \docValue{false}}
  Cette option désactive l'affichage de l'avertissement émis si la commande
  \refCom{date} est omise (ou à arguments vides).
\end{docKey}
\begin{docKey}{noinstitute}{=\docValue{true}\textbar\docValue{false}}{par défaut \docValue{true},
    initialement \docValue{false}}
  Cette option désactive l'affichage de l'avertissement émis si la commande
  \refCom{institute} est omise (ou à argument vide).
\end{docKey}
\begin{docKey}{nodoctoralschool}{=\docValue{true}\textbar\docValue{false}}{par défaut \docValue{true},
    initialement \docValue{false}}
  Cette option désactive l'affichage de l'avertissement émis si la commande
  \refCom{doctoralschool} est omise (ou à argument vide).
\end{docKey}
\begin{docKey}{nolaboratory}{=\docValue{true}\textbar\docValue{false}}{par défaut \docValue{true},
    initialement \docValue{false}}
  Cette option désactive l'affichage de l'avertissement émis si la commande
  \refCom{laboratory} est omise (ou à 1\ier{} argument vide).
\end{docKey}
\begin{docKey}{nolaboratoryadress}{=\docValue{true}\textbar\docValue{false}}{par défaut \docValue{true},
    initialement \docValue{false}}
  Cette option désactive l'affichage de l'avertissement émis si la commande
  \refCom{laboratory} est omise (ou à 2\ieme{} argument vide).
\end{docKey}
\begin{docKey}{nosupervisor}{=\docValue{true}\textbar\docValue{false}}{par défaut \docValue{true},
    initialement \docValue{false}}
  Cette option désactive l'affichage de l'avertissement émis si la commande
  \refCom{supervisor} est omise (ou à argument vide).
\end{docKey}
\begin{docKey}{nomaketitle}{=\docValue{true}\textbar\docValue{false}}{par défaut \docValue{true},
    initialement \docValue{false}}
  Cette option désactive l'affichage de l'avertissement émis si la commande
  \refCom{maketitle} est omise.
\end{docKey}
\begin{docKey}{nokeywords}{=\docValue{true}\textbar\docValue{false}}{par défaut \docValue{true},
    initialement \docValue{false}}
  Cette option désactive l'affichage de l'avertissement émis si la commande
  \refCom{keywords} est omise (ou à arguments vides).
\end{docKey}
\begin{docKey}{noabstract}{=\docValue{true}\textbar\docValue{false}}{par défaut \docValue{true},
    initialement \docValue{false}}
  Cette option désactive l'affichage de l'avertissement émis si l'environnement
  \refEnv{abstract} est omis.
\end{docKey}
\begin{docKey}{nomakeabstract}{=\docValue{true}\textbar\docValue{false}}{par défaut \docValue{true},
    initialement \docValue{false}}
  Cette option désactive l'affichage de l'avertissement émis si la commande
  \refCom{makeabstract} est omise.
\end{docKey}
\begin{docKey}{notableofcontents}{=\docValue{true}\textbar\docValue{false}}{par défaut \docValue{true},
    initialement \docValue{false}}
  Cette option désactive l'affichage de l'avertissement émis si la commande
  \refCom{tableofcontents} est omise.
\end{docKey}
% ^^A \begin{docKey}{nointroduction}{=\docValue{true}\textbar\docValue{false}}{par défaut \docValue{true},
% ^^A     initialement \docValue{false}}
% ^^A   Cette option désactive l'affichage de l'avertissement émis si l'environnement
% ^^A   \refEnv{introduction} est omis.
% ^^A \end{docKey}
% ^^A \begin{docKey}{noconclusion}{=\docValue{true}\textbar\docValue{false}}{par défaut \docValue{true},
% ^^A     initialement \docValue{false}}
% ^^A   Cette option désactive l'affichage de l'avertissement émis si l'environnement
% ^^A   \refEnv{conclusion} est omis.
% ^^A \end{docKey}
\begin{docKey}{noprintbibliography}{=\docValue{true}\textbar\docValue{false}}{par défaut \docValue{true},
    initialement \docValue{false}}
  Cette option désactive l'affichage de l'avertissement émis si la commande
  \refCom{printbibliography} est omise.
\end{docKey}

%
\iffalse
%%% Local Variables:
%%% mode: latex
%%% eval: (latex-mode)
%%% ispell-local-dictionary: "fr_FR"
%%% TeX-engine: xetex
%%% TeX-master: "../yathesis.dtx"
%%% End:
\fi
