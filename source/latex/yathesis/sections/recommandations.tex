
\chapter{Recommandations et astuces}
\label{cha:recomm-et-astuc}

\section{Images}

L'insertion d'images se fait au moyen des commandes du classique
\Package{graphicx}. On notera qu'il est conseillé, selon qu'il s'agit d'images
dont :
\begin{description}
\item[on \emph{n'}est \emph{pas} le créateur,] de disposer de celles-ci à un
  format (nativement) vectoriel, par exemple \pdf, afin de réduire la
  pixellisation ;
\item[on \emph{est} le créateur,] de :
  \begin{enumerate}
  \item si possible faire usage de packages \LaTeX{} spécialisés pour :
    \begin{itemize}
    \item des dessins (packages \package*{TikZ}, \package*{PSTricks}, etc.) ;
    \item des représentations graphiques de fonctions (packages
      \package*{tkz-fct}, \package*{pst-plot}, etc.) ;
    \item des données expérimentales (packages \package*{pgfplots},
      \package*{pst-plot}, etc.).
    \end{itemize}
  \item sinon :
    \begin{itemize}
    \item pour des dessins, de recourir à des logiciels de dessins vectoriels
      (par exemple \program{Inkscape}) ;
    \item de manière générale à enregistrer les images créées à un format
      (nativement) vectoriel, par exemple \pdf.
    \end{itemize}
  \end{enumerate}
\end{description}

\section{Acronymes}\label{acronymes}

On a vu \vref{rq:acronymes} que si un institut (par exemple) doit figurer
sous la forme d'un acronyme, on aura intérêt à ne pas le saisir tel
quel, mais à recourir aux fonctionnalités du
\Package{glossaries}\footnote{Cf. \vref{sec:sigl-gloss-nomencl} pour
  son usage avec la \yatcl.}. L'exemple suivant illustre la
procédure.
%
\begin{dbexample}{Institut sous forme d'acronymes}{acronyme}
  Si on créé l'acronyme suivant\footnote{Avec le canevas de thèse \enquote{en
      relief} fourni avec la présente classe, les acronymes peuvent être
    définis dans le \File{\acronymsfile} situé dans le
    \Directory{\auxiliarydirectory}.} :
\begin{preamblecode}
\newacronym{ulco}{ULCO}{université du Littoral Côte d'Opale}
\end{preamblecode}
on peut recourir, non pas à ×\institute{ULCO}×, mais à :
\begin{preamblecode}
\institute{\acrshort*{ulco}}
\end{preamblecode}
\end{dbexample}

\begin{dbremark}{Acronymes et élisions automatiques}{}
  Les \vref{wa:elision-disclaimer,wa:elision-separateurs} ont déjà signalé que,
  si de telles commandes d'acronymes sont employées pour spécifier les
  instituts (commandes \refCom{institute} et \refCom{coinstitute}) ou les
  affiliations des membres du jury (clé \refKey{affiliation}), les élisions
  automatiques de la clause de non-responsabilité ou des expressions
  contextuelles séparant corporations et affiliations ne donneront pas toujours
  le résultat escompté (en français notamment). On pourra alors le cas échéant
  faire usage :
  \begin{itemize}
  \item de la commande \refCom{disclaimer} ;
  \item des clés \refKey{sepcorpaffilfrench} ou \refKey{sepcorpaffilenglish} ;
  \end{itemize}
  pour redéfinir ces expressions.
\end{dbremark}

\section{Scission du mémoire en fichiers maître et esclaves}
\label{sec:repart-du-memo}

La scission du mémoire de thèse en différents fichiers maître et esclaves,
hautement recommandée, suppose de :
\begin{enumerate}
\item créer un fichier \enquote{maître}\footnote{Dans les spécimens et canevas
    de thèse fournis avec la classe, décrits \vref{cha:specimen-canevas}, le
    fichier maître est nommé \file{\thesismasterfile.tex}.};
\item stocker le contenu des chapitres, chacun dans un fichier
  \enquote{esclave}
  % ^^A \footnote{Dans les spécimens et canevas de thèse fournis avec la
  % ^^A classe, décrit \vref{cha:specimen-canevas}, l'inclusion des fichiers
  % ^^A esclaves, situés dans le \Directory{\dossiercorps}, est gérée dans
  % ^^A le \File{\fichiercorps} situé dans le même répertoire que le
  % ^^A fichier maître.}
  et d'inclure ceux-ci au moyen de la commande
  standard ×\include×\marg{fichier esclave}, le nom du \meta{fichier
    esclave} devant le cas échéant être précédé du chemin qui y
  conduit.
\end{enumerate}
%
Dans ce contexte, et de façon usuelle :
\begin{itemize}
\item sauf cas spécifique, chaque fichier de chapitre devrait débuter par une
  (unique) occurrence de la commande \docAuxCommand{chapter} et en général
  contenir une ou plusieurs occurrences des autres commandes usuelles de
  structuration (\docAuxCommand{section}, \docAuxCommand{subsection}, etc.);
\item si la thèse se présente en plusieurs grandes parties, chacune
  de celles-ci peut être stipulée au moyen de la commande
  \docAuxCommand{part} qu'il est alors recommandé de placer à
  l'extérieur des fichiers de chapitres (cf.
  \vref{ex:avec-parties}).
\end{itemize}
%
Les \vref{ex:sans-parties,ex:avec-parties} illustrent l'usage de ces commandes
pour la partie \enquote{corps} de la thèse et ce, dans l'hypothèse où les
fichiers de chapitres de la thèse sont tous placés dans un sous-répertoire,
nommé \directory{corps}, situé au même niveau que le fichier
maître\footnote{C'est-à-dire à la racine du répertoire contenant le fichier
  maître.}.
\begin{dbexample}{Structure d'une thèse en une seule partie}{sans-parties}
\begin{bodycode}
\include{corps/÷\meta{introduction}÷}
\include{corps/÷\meta{premier chapitre}÷}
...
\include{corps/÷\meta{dernier chapitre}÷}
\include{corps/÷\meta{conclusion}÷}
\end{bodycode}
\end{dbexample}
%
\begin{dbexample}{Structure d'une thèse en deux parties}{avec-parties}
  \lstset{keywordstyle=[3]\color{texcs}}%
\begin{bodycode}
\include{corps/÷\meta{introduction générale}÷}
%
\part{÷\meta{titre de la partie 1}÷}
\include{corps/÷\meta{introduction de la partie 1}÷}
\include{corps/÷\meta{premier chapitre de la partie 1}÷}
...
\include{corps/÷\meta{dernier chapitre de la partie 1}÷}
\include{corps/÷\meta{conclusion de la partie 1}÷}
%
\part{÷\meta{titre de la partie 2}÷}
\include{corps/÷\meta{introduction de la partie 2}÷}
\include{corps/÷\meta{premier chapitre de la partie 2}÷}
...
\include{corps/÷\meta{dernier chapitre de la partie 2}÷}
\include{corps/÷\meta{conclusion de la partie 2}÷}
%
\include{corps/÷\meta{conclusion générale}÷}
\end{bodycode}
\end{dbexample}
%
Le canevas \enquote{en relief}, détaillé \vref{sec:canevas-relief},
suit ce type d'organisation.

%
\iffalse
%%% Local Variables:
%%% mode: latex
%%% eval: (latex-mode)
%%% ispell-local-dictionary: "fr_FR"
%%% TeX-engine: xetex
%%% TeX-master: "../yathesis.dtx"
%%% End:
\fi
