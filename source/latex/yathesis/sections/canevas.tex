\chapter{Canevas de thèse}\label{cha:canevas}

Pour faciliter sa mise en œuvre, la \yatcl fournit deux canevas de thèse. Ils
se distinguent par le fait que la source \file{.tex} du mémoire de thèse est,
respectivement :
\begin{enumerate}
\item toute entière située dans un unique fichier (canevas \enquote{à plat}) ;
\item scindée en fichiers maître et esclaves, qui plus est répartis dans
  différents sous-dossiers (canevas \enquote{en relief}).
\end{enumerate}
Ils se trouvent dans les sous-dossiers\footnote{Une version de
  développement de ces canevas se trouve également à la page
  \url{https://github.com/dbitouze/yathesis/tree/master/doc/latex/yathesis/}.} :
\begin{enumerate}
\item \directory{\meta{racine}/\jobdocdirectory/single-file-template} ;
\item \directory{\meta{racine}/\jobdocdirectory/master-slaves-files-template} ;
\end{enumerate}
où, par défaut, \meta{racine} est avec la distribution :
\begin{description}
\item[\TeX{}~Live :]\
  \begin{description}
  \item[sous Linux et Mac OS X :] \unixtldirectory\tldistdirectory\versiontl ;
  \item[sous Windows :] \wintldirectory\tldistdirectory\versiontl ;
  \end{description}
\item[MiK\TeX{} :] \miktexdistdirectory.
\end{description}

\begin{dbwarning}{Ne pas travailer directement dans les dossiers de canevas
    fournis !}{}
  Si on souhaite utiliser l'un de ces canevas, il est \emph{essentiel} de
  copier dans un répertoire de travail habituel le dossier correspondant (qu'il
  est conseillé de renommer, par exemple en \directory{these}). On \emph{ne}
  travaillera \emph{surtout pas} directement dans le dossier fourni, sous peine
  que tout son travail soit écrasé lors d'une mise à jour de la classe.
\end{dbwarning}

\section{Canevas \enquote{à plat}}
\label{sec:canevas-a-plat}

Ce canevas ne contient que deux fichiers :
\begin{enumerate}
\item le fichier de la thèse nommé \file{these.tex} ;
\item le \File{latexmkrc} de configuration du programme \program{latexmk} qui
  permet d'automatiser le processus de compilation complète de la thèse.
\end{enumerate}

[TODO]

\section{Canevas \enquote{en relief}}
\label{sec:canevas-relief}

[TODO]

%
\iffalse
%%% Local Variables:
%%% mode: latex
%%% eval: (latex-mode)
%%% ispell-local-dictionary: "fr_FR"
%%% TeX-engine: xetex
%%% TeX-master: "../yathesis.dtx"
%%% End:
\fi
