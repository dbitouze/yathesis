\chapter{Canevas de thèse}\label{cha:canevas}

Pour faciliter sa mise en œuvre, la \yatcl fournit deux canevas de thèse. Ils
produisent deux fichiers \gls{pdf} identiques, mais se distinguent par le fait
que la source \file{.tex} du mémoire de thèse correspondant est,
respectivement :
\begin{enumerate}
\item toute entière située dans un unique fichier\footnote{Mis à part le
    fichier de bibliographie et les fichiers images.} (canevas \enquote{à
    plat}) ;
\item scindée en fichiers maître et esclaves, qui plus est répartis dans
  différents sous-dossiers (canevas \enquote{en relief}).
\end{enumerate}

Ces canevas peuvent être consultés dans les dossiers respectivement
\href{single-file-template/.}{single-file-template} et
\href{master-slaves-files-template/.}{master-slaves-files-template} du dossier
de documentation de la \yatcl{}\footnote{C'est-à-dire :
  \begin{itemize}
  \item pour la distribution \TeX{} Live, dans le dossier
    \tldistdirectory\jobdocdirectory{} situé dans le dossier :
    \begin{itemize}
    \item \unixtldirectory{} sous Linux et Mac OS X ;
    \item \wintldirectory{} sous Windows ;
    \end{itemize}
  \item pour la distribution MiK\TeX, dans le dossier \miktexdistdirectory.
  \end{itemize}
}.

\begin{dbwarning}{Ne pas travailler directement dans les dossiers de canevas !}{}
  Si on souhaite utiliser l'un de ces canevas, il est \emph{essentiel} de
  copier dans un répertoire de travail habituel le dossier correspondant (qu'il
  est conseillé de renommer, par exemple en \directory{these}). On \emph{ne}
  travaillera \emph{surtout pas} directement dans le dossier fourni, sous peine
  que tout son travail soit écrasé lors d'une mise à jour de la classe.
\end{dbwarning}

\section{Canevas \enquote{à plat}}
\label{sec:canevas-a-plat}

Ce canevas ne contient que deux fichiers :
\begin{enumerate}
\item le fichier de la thèse nommé \file{these.tex} ;
\item le \File{latexmkrc} de configuration du programme \program{latexmk} qui
  permet d'automatiser le processus de compilation complète de la thèse.
\end{enumerate}

[TODO]

\section{Canevas \enquote{en relief}}
\label{sec:canevas-relief}

[TODO]

%
\iffalse
%%% Local Variables:
%%% mode: latex
%%% eval: (latex-mode)
%%% ispell-local-dictionary: "fr_FR"
%%% TeX-engine: xetex
%%% TeX-master: "../yathesis.dtx"
%%% End:
\fi
