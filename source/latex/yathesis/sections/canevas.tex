\chapter{Canevas de thèse}\label{cha:canevas}

Pour faciliter son utilisation, la \yatcl fournit deux canevas de thèse :
\begin{enumerate}
\item un \enquote{à plat}, où la source \file{.tex} du mémoire de thèse est
  toute entière située dans un unique fichier ;
\item un \enquote{en relief}, avec scission de la source \file{.tex} du mémoire
  de thèse en fichiers maître et esclaves, qui plus est répartis dans
  différents sous-dossiers.
\end{enumerate}
Ceux-ci sont constitués des sous-dossiers
\begin{enumerate}
\item \directory{single-file-template}
\item \directory{master-slaves-files-template}
\end{enumerate}
du dossier \file{.../doc/latex/yathesis}. Ils sont également
disponibles à l'adresse TODO

Pour utiliser l'un ou l'autre de ces canevas, on copiera le dossier
correspondant dans un répertoire habituel de travail que, \emph{a
  priori}, on renommera par exemple en \directory{these}.

\section{Canevas  \enquote{à plat}}
\label{sec:canevas-a-plat}

TODO

\section{Canevas \enquote{en relief}}
\label{sec:canevas-relief}

TODO

%
\iffalse
%%% Local Variables:
%%% mode: latex
%%% eval: (latex-mode)
%%% ispell-local-dictionary: "fr_FR"
%%% TeX-engine: xetex
%%% TeX-master: "../yathesis.dtx"
%%% End:
\fi
