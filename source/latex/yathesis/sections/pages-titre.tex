\chapter{Pages de titre}\label{cha:pages-de-titre}

Ce chapitre documente la commande \refCom{maketitle} permettant de
\emph{produire}, à partir des données définies \vref{sec:proprietes-de-titre},
les pages de titre de la thèse.

\begin{docCommand}[doc description=\mandatory]{maketitle}{}
  Cette commande \emph{produit} :
  \begin{enumerate}
  \item
    \begin{enumerate}
    \item une page de 1\iere{} de couverture\footnote{Sauf s'il est
        explicitement demandé que celle-ci ne figure pas,
        cf. \refKey{nofrontcover}.} ;
    \item une page de titre.
    \end{enumerate}
    Ces deux pages sont composées dans la langue principale et
    identiques\footnote{À ceci près que le numéro d'ordre de la thèse ne figure
      que sur la page de 1\iere{} de couverture.} ;
  \item \emph{automatiquement}\footnote{Sans qu'il soit nécessaire de faire
      figurer une 2\ieme{} occurrence de la commande \refCom{maketitle}.} une
    seconde page de titre \emph{si} \phrase{et seulement si} l'une au moins des
    commandes \refCom{title}, \refCom{subtitle}, \refCom{academicfield} ou
    \refCom{speciality} est employée avec son argument optionnel
    (cf. \vref{rq:titre-supp}). Cette page est composée dans la langue
    secondaire.
  \end{enumerate}
\end{docCommand}

\section{Exemple complet de pages de titre}
\label{sec:exemple-complet}

\begin{dbexample}{Préparation et production des pages de titre}{}
  Avec les données caractéristiques suivantes, la commande
  \refCom{maketitle} produit :
  \begin{enumerate}
  \item en langue principale (ici le français),
    \begin{enumerate}
    \item une page de 1\iere{} de couverture illustrée \vref{fig:maketitle-fr} ;
    \item une page de titre ;
    \end{enumerate}
  \item en langue secondaire (ici l'anglais), une page de titre illustrée
    \vref{fig:maketitle-en}.
  \end{enumerate}
%
  \NoAutoSpacing%
  \lstset{morecomment=[is]{\%}{\^^M}}%
  \bodysample{%
    deletekeywords={[2]title},%
    rangeendsuffix={\^^M},%
    linerange={%
      author-42]
    }%
  }{title=Préparation du titre (par exemple dans le \File{\configurationfile})}
  %
  \lstset{deletecomment=[is]{\%}{\^^M}}%
  %
  \bodysample{rangesuffix=\^^M,linerange={maketitle}}{title=Production
    du titre}
\end{dbexample}

\begin{landscape}
  \begin{figure}[htb]
    \centering
    \begin{subfigure}[b]{.45\linewidth}
      \centering%
      \screenshot[1]{fr-title}
      \caption{Page de 1\iere{} de couverture en français}
      \label{fig:maketitle-fr}
    \end{subfigure}%
    \hspace{\stretch{1}}%
    \begin{subfigure}[b]{.45\linewidth}
      \centering%
      \screenshot[1]{en-title}
      \caption{Page de titre en anglais}
      \label{fig:maketitle-en}
    \end{subfigure}%
    \caption{Pages de 1\iere{} de couverture et de titre}
    \label{fig:maketitle}
  \end{figure}
\end{landscape}
%
\iffalse
%%% Local Variables:
%%% mode: latex
%%% eval: (latex-mode)
%%% ispell-local-dictionary: "fr_FR"
%%% TeX-engine: xetex
%%% TeX-master: "../yathesis.dtx"
%%% End:
\fi
