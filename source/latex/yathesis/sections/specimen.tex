\chapter{Specimen de thèse}\label{cha:specimen}

Pour montrer comment mettre en œuvre la \yatcl, un specimen de thèse composé
avec elle est fourni.  Celui-ci se trouve dans le sous-dossier\footnote{Une
  version de développement de ce specimen se trouve également à la page
  \url{https://github.com/dbitouze/yathesis/tree/master/doc/latex/yathesis/}.}
\directory{\meta{racine}/\jobdocdirectory/sample} où, par défaut, \meta{racine}
est avec la distribution :
\begin{description}
\item[\TeX{}~Live :]\
  \begin{description}
  \item[sous Linux et Mac OS X :] \unixtldirectory\tldistdirectory\versiontl ;
  \item[sous Windows :] \wintldirectory\tldistdirectory\versiontl ;
  \end{description}
\item[MiK\TeX{} :] \miktexdistdirectory.
\end{description}

\begin{dbwarning}{Ne pas travailler directement dans le dossier de spécimen
    fourni !}{}
  Si on souhaite utiliser ce spécimen, il est \emph{essentiel} de copier dans
  un répertoire de travail habituel le dossier \directory{sample}. On \emph{ne}
  travaillera \emph{surtout pas} directement dans le dossier fourni, sous peine
  que tout son travail soit écrasé lors d'une mise à jour de la classe.
\end{dbwarning}

% ^^A La commande à utiliser pour lister le contenu du répertoire est :
% ^^A tree -A -F -I
% ^^A "*aux|*idx|*ind|*lof|*lot|*out|*toc|*acn|*acr|*alg|*bcf|*glg|*glo|*gls|*glg2|*gls2|*glo2|*ist|*run.xml|*xdy|*lol|*fls|*slg|*slo|*sls|*unq|*synctex.gz|*mw|*bbl|*blg|*fdb_latexmk|*log|*auto"

[TODO]

%
\iffalse
%%% Local Variables:
%%% mode: latex
%%% eval: (latex-mode)
%%% ispell-local-dictionary: "fr_FR"
%%% TeX-engine: xetex
%%% TeX-master: "../yathesis.dtx"
%%% End:
\fi
