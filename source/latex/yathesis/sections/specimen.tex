\chapter{Specimen de thèse}\label{cha:specimen}

Pour montrer comment mettre en œuvre la \yatcl, un specimen de thèse composé
avec elle est fourni. Celui-ci se trouve dans le sous-dossier
\directory{sample} du dossier \file{.../doc/latex/yathesis} mais est également
disponible à l'adresse
\url{https://github.com/dbitouze/yathesis/tree/master/doc/latex/yathesis/sample}

% ^^A La commande à utiliser pour lister le contenu du répertoire est :
% ^^A tree -A -F -I
% ^^A "*aux|*idx|*ind|*lof|*lot|*out|*toc|*acn|*acr|*alg|*bcf|*glg|*glo|*gls|*glg2|*gls2|*glo2|*ist|*run.xml|*xdy|*lol|*fls|*slg|*slo|*sls|*unq|*synctex.gz|*mw|*bbl|*blg|*fdb_latexmk|*log|*auto"

TODO

%
\iffalse
%%% Local Variables:
%%% mode: latex
%%% eval: (latex-mode)
%%% ispell-local-dictionary: "fr_FR"
%%% TeX-engine: xetex
%%% TeX-master: "../yathesis.dtx"
%%% End:
\fi
