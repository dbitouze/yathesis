\chapter{Specimens de thèse}\label{cha:specimen}

Pour montrer comment mettre en œuvre la \yatcl, deux specimens de thèse
composés avec elle sont fournis.  Ils fournissent deux fichiers \gls{pdf}
identiques, mais se distinguent par le fait que la source \file{.tex} du
mémoire de thèse correspondant est, respectivement :
\begin{enumerate}
\item toute entière située dans un unique fichier (spécimen \enquote{à plat}) ;
\item scindée en fichiers maître et esclaves, qui plus est répartis dans
  différents sous-dossiers (spécimen \enquote{en relief}).
\end{enumerate}
Ils se trouvent dans les sous-dossiers\footnote{Une version de
  développement de ces canevas se trouve également à la page
  \url{https://github.com/dbitouze/yathesis/tree/master/doc/latex/yathesis/}.} :
\begin{enumerate}
\item \directory{\meta{racine}/\jobdocdirectory/single-file-sample} ;
\item \directory{\meta{racine}/\jobdocdirectory/master-slaves-files-sample} ;
\end{enumerate}
où, par défaut, \meta{racine} est avec la distribution :
\begin{description}
\item[\TeX{}~Live :]\
  \begin{description}
  \item[sous Linux et Mac OS X :] \unixtldirectory\tldistdirectory\versiontl ;
  \item[sous Windows :] \wintldirectory\tldistdirectory\versiontl ;
  \end{description}
\item[MiK\TeX{} :] \miktexdistdirectory.
\end{description}

\begin{dbwarning}{Ne pas travailler directement dans les dossiers de spécimens
    fournis !}{}
  Si on souhaite utiliser l'un de ces spécimens, il est \emph{essentiel} de
  copier dans un répertoire de travail habituel le dossier correspondant (qu'il
  est conseillé de renommer, par exemple en \directory{these}). On \emph{ne}
  travaillera \emph{surtout pas} directement dans le dossier fourni, sous peine
  que tout son travail soit écrasé lors d'une mise à jour de la classe.
\end{dbwarning}

% ^^A La commande à utiliser pour lister le contenu du répertoire est :
% ^^A tree -A -F -I
% ^^A "*aux|*idx|*ind|*lof|*lot|*out|*toc|*acn|*acr|*alg|*bcf|*glg|*glo|*gls|*glg2|*gls2|*glo2|*ist|*run.xml|*xdy|*lol|*fls|*slg|*slo|*sls|*unq|*synctex.gz|*mw|*bbl|*blg|*fdb_latexmk|*log|*auto"

[TODO]

%
\iffalse
%%% Local Variables:
%%% mode: latex
%%% eval: (latex-mode)
%%% ispell-local-dictionary: "fr_FR"
%%% TeX-engine: xetex
%%% TeX-master: "../yathesis.dtx"
%%% End:
\fi
