\chapter{Pages liminaires}\label{cha:liminaires}

Cette section détaille les commandes permettant de préparer et produire les
\glspl{liminaire}, à savoir :
\begin{enumerate}
\item la page (éventuelle) de clause de non-responsabilité ;
\item la page (éventuelle) des mots clés de la thèse ;
\item la page (éventuelle) du laboratoire où a été préparée la thèse ;
\item la page (éventuelle) des dédicaces ;
\item la page (éventuelle) des épigraphes ;
\item l'avertissement (éventuel) ;
\item les (éventuels) remerciements, préface, avant-propos, etc.
\item la page de résumés dans les langues principale et secondaire ;
\item le résumé (éventuel) substantiel en français ;
\item les listes (éventuelles), commune ou distinctes :
  \begin{itemize}
  \item des sigles et acronymes\footnote{Par commodité, nous ne parlerons plus
      dans la suite que d'acronymes mais ce qui les concernera s'appliquera de
      façon identique aux sigles.} ;
  \item des symboles ;
  \item des termes du glossaire ;
  \end{itemize}
\item le sommaire ou la table des matières ;
\item la liste (éventuelle) des tableaux ;
\item la liste (éventuelle) des figures ;
\item la liste (éventuelle) des listings informatiques.
\end{enumerate}

\begin{dbremark}{Commande \protect\docAuxCommand{frontmatter} à ne pas utiliser}{nofrontmatter}
  La commande \docAuxCommand{frontmatter} usuelle de la \Class{book}, employée
  habituellement pour entamer la partie liminaire d'un document, n'a pas besoin
  d'être utilisée avec la classe \yatcl{} car elle l'est déjà en sous-main. Au
  contraire les autres commandes analogues de la \Class{book} :
  \refCom{mainmatter}, \refCom{appendix} et \refCom{backmatter} doivent être
  explicitement employées pour entamer les parties respectivement principale,
  annexe et finale.
\end{dbremark}

\section{Clause de non-responsabilité}
\label{sec:clause-de-non}

La \yatcl{} permet de faire figurer une clause de non-responsabilité, telle
qu'exigée par certains instituts. Celle-ci apparaît sur une page dédiée et :
\begin{enumerate}
\item a pour contenu par défaut une phrase semblable
  à\selonlangue{} :
  \begin{itemize}
  \item \enquote{L'\meta{institut} n'entend donner aucune
      approbation ni improbation aux opinions \'emises dans les th\`eses : ces
      opinions devront \^etre consid\'er\'ees comme propres \`a leurs auteurs.}
  \item \foreignquote{english}{The \meta{institut} neither endorse
      nor censure authors' opinions expressed in the theses: these opinions
      must be considered to be those of their authors.}
  \end{itemize}
  où l'\meta{institut} est celui défini par la commande \refCom{institute}
  \phrase*{auquel est adjoint l'éventuel institut de cotutelle}.
\item peut être redéfinie au moyen de la commande \refCom{disclaimer}.
\end{enumerate}

La page dédiée à la clause de non-responsabilité est produite par la commande
\refCom{makedisclaimer}.

\begin{docCommand}{makedisclaimer}{}
  Cette commande produit une page où figure, seule et centrée
  verticalement, la clause de non-responsabilité.
\end{docCommand}

\begin{docCommand}{makedisclaimer*}{}
  Cette commande a le même effet que la commande
  \refCom{makedisclaimer} sauf que la clause de non-responsabilité est alignée
  sur le haut de la page et non centrée verticalement.
\end{docCommand}

\begin{dbexample}{Production de la page dédiée à la clause de
    non-responsabilité}{}
  \NoAutoSpacing%
%
  \bodysample{rangesuffix=\^^M,linerange={makedisclaimer}}{}
  Le résultat de ce code est illustré \vref{fig:disclaimerpage}.
\end{dbexample}

\begin{figure}[htbp]
  \centering
  \screenshot{disclaimer}%
  \caption{Page de clause de non-responsabilité}
  \label{fig:disclaimerpage}
\end{figure}

On peut modifier le contenu par défaut de la clause de non-responsabilité au
moyen de la commande \refCom{disclaimer} suivante.

\begin{docCommand}{disclaimer}{\marg{clause}}
  Cette commande permet de redéfinir le contenu par défaut de la
  \meta{clause} de non-responsabilité.
\end{docCommand}

\section{Mots clés}\label{sec:mots-cles}

Les mots clés de la thèse sont stipulés au moyen de la commande
\refCom{keywords} suivante.
%
\begin{docCommand}[doc description=\mandatory]{keywords}{\marg{mots clés}\marg{mots clés dans la langue
      secondaire}}
  Cette commande définit les \meta{mots clés} de la thèse dans
  les langues principale et secondaire. Ceux-ci :
  \begin{itemize}
  \item apparaissent comme propriété \enquote{Mots-clés} du fichier \pdf
    \phrase{dans la langue principale seulement} ;
  \item figurent, dans les deux langues principale et secondaire, précédés des
    expressions \enquote{Mots clés :} et
    \foreignquote{english}{Keywords:}\selonlangue :
    \begin{itemize}
    \item sur la page qui leur est dédiée (si la commande \refCom{makekeywords}
      est employée) ;
    \item sur la page dédiée au(x) résumé(s) de la thèse générée par la
      commande \refCom{makeabstract} ;
    \item sur la 4\ieme{} de couverture (si la commande \refCom{makebackcover}
      est employée).
    \end{itemize}
  \end{itemize}
\end{docCommand}
%
\begin{docCommand}{makekeywords}{}
  Cette commande produit une page où figurent, seuls et centrés
  verticalement, les mots clés de la thèse stipulés au moyen de la commande
  \refCom{keywords}.
\end{docCommand}
%
\begin{docCommand}{makekeywords*}{}
  Cette commande a le même effet que la commande
  \refCom{makekeywords} sauf que les mots clés sont alignés sur le haut de la
  page et non centrés verticalement.
\end{docCommand}

\begin{dbexample}{Préparation et production de la page dédiée aux mots clés}{}
  Les codes suivants produisent la page illustrée \vref{fig:makekeywords}.
  \bodysample{linerange={keywords-laugh}}{title=Préparation}
%
  \bodysample{rangesuffix=\^^M,linerange={makekeywords}}{title=Production}
\end{dbexample}

\begin{figure}[htbp]
  \centering \screenshot{keywords}%
  \caption{Page dédiée aux mots clés}
  \label{fig:makekeywords}
\end{figure}

\section{Laboratoire(s)}
\label{sec:laboratoires}

\begin{docCommand}{makelaboratory}{}
  Cette commande produit une page où figure, seul(s) et centré(s)
  verticalement, le ou les laboratoires où a été préparée la thèse, stipulé(s)
  au moyen de la commande \refCom{laboratory} et précisé(s) au moyen des
  options \refKey{logo}, \refKey{logoheight}, \refKey{telephone}, \refKey{fax}
  et \refKey{email}.
\end{docCommand}
%
\begin{docCommand}{makelaboratory*}{}
  Cette commande a le même effet que la commande
  \refCom{makelaboratory} sauf que le laboratoire est aligné sur le haut de la
  page et non centré verticalement.
\end{docCommand}

\begin{dbexample}{Préparation et production de la page dédiée au laboratoire}{}
  Les codes suivants produisent la page illustrée \vref{fig:makelaboratory}.
  \NoAutoSpacing%
  \preamblesample{linerange={laboratory-Liouville}}{title=Préparation}
%
  \bodysample{rangesuffix=\^^M,linerange={makelaboratory}}{title=Production}
\end{dbexample}

\begin{figure}[htbp]
  \centering \screenshot{laboratory}
  \caption{Page dédiée au laboratoire}
  \label{fig:makelaboratory}
\end{figure}

\section{Dédicaces}

\begin{docCommand}{dedication}{\marg{dédicace}}
  Cette commande, à employer autant de fois que
  souhaité\hauteurpage{}, permet de préparer une dédicace.
\end{docCommand}

\begin{docCommand}{makededications}{}
  Cette commande produit une page où figurent, seules,
  alignées à droite et centrées verticalement, la ou les dédicaces stipulées au
  moyen de la commande \refCom{dedication}.
\end{docCommand}
%
\begin{docCommand}{makededications*}{}
  Cette commande a le même effet que la commande
  \refCom{makededications} sauf que la ou les dédicaces sont alignées sur le
  haut de la page et non centrées verticalement.
\end{docCommand}

\begin{dbexample}{Préparation et production de la page dédiée aux dédicaces}{}
  \NoAutoSpacing%
  \preamblesample{linerange={dedication-{co-aimé\ !}}}{title=Préparation}
%
  \bodysample{rangesuffix=\^^M,linerange={makededications}}{title=Production}
  Le résultat de ce code est illustré \vref{fig:dedicationspage}.
\end{dbexample}

\begin{figure}[htbp]
  \centering \screenshot{dedications}%
  \caption{Page de dédicaces}
  \label{fig:dedicationspage}
\end{figure}

\section{Épigraphes liminaires}

\begin{docCommand}{frontepigraph}{\oarg{langue}\marg{épigraphe}\marg{auteur}}
  Cette commande, à employer autant de fois que
  souhaité\hauteurpage{}, permet de préparer une épigraphe destinée à
  apparaître sur une \gls{liminaire} dédiée.

  Si l'épigraphe est exprimée dans une \meta{langue} \phrase{connue du
    \Package{babel}} autre que la langue principale du document, on peut le
  spécifier en argument optionnel\footnote{Si cette \meta{langue} est autre que
    le français ou l'anglais, elle doit être explicitement chargée en option de
    la commande \docAuxCommand{documentclass} (cf.
    \vref{rq:languessupplementaires}).}.
\end{docCommand}

\begin{docCommand}{makefrontepigraphs}{}
  Cette commande produit une page où la ou les épigraphes stipulées au moyen de
  la commande \refCom{frontepigraph} figurent \phrase*{seules, alignées à droite et
  centrées verticalement}.
\end{docCommand}
%
\begin{docCommand}{makefrontepigraphs*}{}
  Cette commande a le même effet que la commande
  \refCom{makefrontepigraphs} sauf que la ou les épigraphes sont alignées sur
  le haut de la page et non centrées verticalement.
\end{docCommand}

\begin{dbexample}{Préparation et production de la page dédiée aux épigraphes
    liminaires}{}
  \NoAutoSpacing%
  Les codes suivants produisent la page illustrée \vref{fig:epigraphspage}.
  \preamblesample{linerange={frontepigraph-Einstein}}{title=Préparation}
  %
  \bodysample{rangesuffix=\^^M,linerange={makefrontepigraphs}}{title=Production}
\end{dbexample}

\begin{figure}[htbp]
  \centering \screenshot{frontepigraphs}
  \caption{Page d'épigraphes liminaires}
  \label{fig:epigraphspage}
\end{figure}

\begin{dbremark}{Épigraphes ailleurs dans le document}{}
  Pour gérer les épigraphes liminaires, la \yatcl{} exploite le
  \Package*{epigraph} \phrase*{qui est automatiquement chargé}. Il est bien sûr
  possible de recourir aux commandes de ce package pour faire figurer, ailleurs
  dans le mémoire, d'autres épigraphes.
\end{dbremark}

\section{Remerciements, avertissement, préface, avant-propos, etc.}

Les \glspl{liminaire} d'un mémoire de thèse peuvent contenir des remerciements,
un avertissement, une préface, un avant-propos, un résumé substantiel en
français (cf. \vref{wa:frenchabstract}), etc. Ceux-ci sont à considérer comme
des chapitres \enquote{ordinaires} et doivent donc être introduits au moyen de
la commande usuelle \docAuxCommand{chapter}, sous sa forme \emph{non} étoilée :
puisqu'ils seront situés dans la partie liminaire du mémoire, ces chapitres
seront automatiquement \emph{non} numérotés.

\begin{dbremark}{\protect\Glspl{titrecourant} des chapitres des \protect\glspl{liminaire}}{titrecourant}
  S'ils sont situés après la page dédiée aux résumés succincts en français et
  en anglais (cf. \vref{sec:abstract}), les chapitres \enquote{ordinaires} sont
  pourvus de \glspl{titrecourant}. Sinon, ils n'en sont pas pourvus.
\end{dbremark}

\section{Résumés succincts en français et en anglais}\label{sec:abstract}

Une page contenant de courts résumés en français et en anglais est requise.
L'environnement \refEnv{abstract} suivant permet de préparer une telle
page.
%
\begin{docEnvironment}[doclang/environment content=résumé,doc description=\mandatory]{abstract}{\oarg{intitulé alternatif}}
  Cet environnement, destiné à recevoir le ou les résumés de la thèse, est
  conçu pour être employé une ou deux fois :
  \begin{enumerate}
  \item sa 1\iere{} occurrence doit contenir le résumé dans la langue
    principale ;
  \item sa 2\ieme{} occurrence, si présente, doit contenir le résumé dans la
    langue secondaire.
  \end{enumerate}
  Ces résumés sont respectivement intitulés \enquote{Résumé} ou
  \foreignquote{english}{Abstract}\selonlangue{} mais l'argument optionnel
  permet de spécifier un \meta{intitulé alternatif}\redefexprcle et ils
  figurent, dans les langues principale et secondaire :
  \begin{itemize}
  \item sur la page dédiée au(x) résumé(s) de la thèse produite par la commande
    \refCom{makeabstract} ;
  \item sur la 4\ieme{} de couverture si la commande \refCom{makebackcover} est
    employée.
  \end{itemize}
\end{docEnvironment}

\begin{docCommand}[doc description=\mandatory]{makeabstract}{}
  Cette commande produit une page dédiée aux résumés en y faisant
  apparaître automatiquement :
  \begin{enumerate}
  \item dans les langues principale et secondaire :
    \begin{itemize}
    \item les titre, éventuel sous-titre et mots clés de la thèse, stipulés au
      moyen des commandes respectives \refCom{title}, \refCom{subtitle} et
      \refCom{keywords} ;
    \item les résumés saisis au moyen de l'environnement \refEnv{abstract} ;
    \end{itemize}
  \item le nom et l'adresse du laboratoire dans lequel la thèse a été
    principalement préparée, stipulés au moyen de la commande
    \refCom{laboratory}.
  \end{enumerate}
\end{docCommand}

\begin{dbexample}{Préparation et production de la page dédiée aux résumés}{}
  Les codes suivants produisent la page illustrée \vref{fig:resumes-succincts}.
%^^A \preamblesample{%
%^^A   includerangemarker=false,%
%^^A   rangebeginprefix={»).\^^M},%
%^^A   rangeendsuffix={\^^M\%\ Page},%
%^^A   linerange={\\begin\{abstract-\\end\{abstract\}}%
%^^A }{title=Préparation des résumés}
\begin{bodycode}
\begin{abstract}
  \lipsum[1-2]
\end{abstract}
\begin{abstract}
  \lipsum[3-4]
\end{abstract}
\end{bodycode}
  %
  \bodysample{rangesuffix=\^^M,linerange={makeabstract}}{title=Production
    des résumés}
\end{dbexample}

\begin{figure}[htbp]
  \centering \screenshot{abstract}%
  \caption{Résumés succincts en français et en anglais}
  \label{fig:resumes-succincts}
\end{figure}

\begin{dbwarning}{Résumés nécessairement courts dans l'environnement
    \protect\lstinline+abstract+}{}
  L'environnement \refEnv{abstract} est prévu pour des résumés courts, leurs
  versions dans les langues principale et secondaire devant tenir l'une sous
  l'autre sur une seule et même page. Cette limitation est en phase avec les
  recommandations du ministère stipulant que ces résumés doivent chacun
  contenir au maximum 1700~caractères, espaces compris\footnote{En cas de
    débordement sur plus d'une page, on pourra toujours recourir à un
    changement local de taille des caractères.}.
\end{dbwarning}

\begin{dbwarning}{Résumé en français nécessaire en cas de mémoire en langue
    étrangère}{frenchabstract}
  Un mémoire composé principalement en langue étrangère \phrase{notamment dans
    le cadre d'une cotutelle internationale} requiert, en sus de la page de
  résumé(s) ci-dessus, un résumé \emph{en français} de la thèse. Celui-ci doit
  être \emph{substantiel}, d'une dizaine de pages environ.
\end{dbwarning}

% ^^A \section[Résumé substantiel en français]{Résumé substantiel en français (éventuel)}
% ^^A
% ^^A Un mémoire composé principalement en langue étrangère \phrase{notamment dans le
% ^^A   cadre d'une cotutelle internationale} requiert, en sus de la page de
% ^^A résumé(s) ci-dessus, un résumé \emph{en français} de la thèse
% ^^A \emph{substantiel}\footnote{Une dizaine de pages suffisent.}.
% ^^A
% ^^A \begin{docCommand}{frenchabstract}{\oarg{intitulé alternatif}}
% ^^A   Cette commande débute le résumé substantiel en français de la
% ^^A   thèse. Elle crée un chapitre\nochapter{} ordinaire non numéroté, par défaut
% ^^A   intitulé \enquote{Résumé en français}\noillustration. Si on souhaite un
% ^^A   \meta{intitulé alternatif}, on recourra à son argument
% ^^A   optionnel\redefexprcle.
% ^^A \end{docCommand}
% ^^A
% ^^A \begin{dbexample}{Résumé substantiel en français}{}
% ^^A \begin{bodycode}
% ^^A \frenchabstract
% ^^A ÷\meta{résumé substantiel de la thèse en français}÷
% ^^A \end{bodycode}
% ^^A \end{dbexample}
% ^^A
% ^^A \begin{dbremark}{Absence/Présence fautive du résumé substantiel en
% ^^A     français}{}
% ^^A   La \yatcl émet un avertissement dans les cas où le mémoire est :
% ^^A   \begin{enumerate}
% ^^A   \item principalement composé en \emph{langue étrangère} et
% ^^A     \refCom{frenchabstract} \emph{n'}est \emph{pas} employée ;
% ^^A   \item principalement composé en \emph{français} et \refCom{frenchabstract}
% ^^A     \emph{est} employée.
% ^^A   \end{enumerate}
% ^^A \end{dbremark}

\section{Liste d'acronymes, liste de symboles,
  glossaire}\label{sec:sigl-gloss-nomencl}

\begin{dbremark*}{Section à passer en 1\iere{} lecture}
  Cette section est à passer en 1\iere{} lecture si on ne compte faire figurer
  ni listes d'acronymes, ni listes de symboles, ni glossaire.
\end{dbremark*}

Tout système de gestion de glossaire peut théoriquement être mis en œuvre avec
la \yatcl. Cependant, celle-ci fournit des fonctionnalités propres au
\Package{glossaries}\footnote{Dans ses versions à partir de la \texttt{4.0} en
  date du \formatdate{14}{11}{2013}. Dans cette section, le fonctionnement de
  ce package est supposé connu du lecteur (sinon, cf. par exemple
  \cite{en-ligne7}).} :
\begin{itemize}
\item une commande \refCom{newglssymbol}, destinée à faciliter la définition de
  symboles dans la base terminologique ;
\item un style de glossaire \docValue{yadsymbolstyle}, destiné à composer la
  liste des symboles sous forme de \enquote{nomenclature} (dans l'esprit du
  \Package*{nomencl}).
\end{itemize}

\begin{dbwarning}{Package \package{glossaries} non chargé par défaut}{}
  Le \Package{glossaries} \emph{n'étant pas} chargé par la \yatcl, on veillera
  à le charger manuellement si on souhaite l'utiliser.
\end{dbwarning}

% ^^A \begin{enumerate}
% ^^A \item les commandes de ce package produisant les liste des termes du ou des
% ^^A   glossaires sont légèrement modifiées (un style de pages propre leur étant
% ^^A   appliqué) :
% ^^A   \begin{itemize}
% ^^A   \item \docAuxCommand{printglossary} et \docAuxCommand{printglossaries} qui
% ^^A     produisent la liste des termes du ou des glossaires\termesdefinisutilises{}
% ^^A     (cf. \vref{fig:printglossary}) ;
% ^^A   \item \docAuxCommand{printacronyms}\footnote{L'usage de la commande
% ^^A       \docAuxCommand{printacronyms} nécessite que l'option \docAuxKey{acronyms}
% ^^A       soit passée au \Package{glossaries}.} qui produit la liste des
% ^^A     acronymes\termesdefinisutilises{} (cf. \vref{fig:printacronyms}) ;
% ^^A   \end{itemize}
% ^^A \item les commande et style propres \refCom{newglssymbol}, et
% ^^A   \docValue{yadsymbolstyle}, précisés ci-dessous, sont définis.
% ^^A \end{enumerate}

\begin{docCommand}{newglssymbol}{\oarg{classement}\marg{label}\marg{symbole}\marg{nom}\marg{description}}
  Cette commande définit un symbole au moyen :
  \begin{itemize}
  \item de son \meta{label}\footnote{Ce \meta{label}, qui identifie le symbole de
      manière unique dans la base terminologique, est notamment utilisé dans
      les commandes qui produisent celui-ci dans le texte \phrase*{par exemple
      \docAuxCommand{gls}\marg{label}}.} ;
  \item du \meta{symbole} proprement dit\footnote{Ce symbole peut être composé
      au moyen de la commande \docAuxCommand{ensuremath}\marg{symbole
        mathématique} ou de la commande \docAuxCommand{si}\marg{commande
        d'unité} du \Package*{siunitx} (à charger).} ;
  \item de son \meta{nom} ;
  \item de sa \meta{description}.
  \end{itemize}
  Dans la liste des symboles produite par la commande \refCom{printsymbols}, un
  symbole est par défaut classé selon l'ordre alphabétique de son \meta{label}
  mais peut optionnellement l'être selon celui d'une autre chaîne de
  \meta{classement}.
\end{docCommand}

\begin{dbwarning}{Option \texttt{symbols} nécessitée par la commande
    \protect\refCom*{newglssymbol}}{}
  L'usage de la commande \refCom{newglssymbol} nécessite que l'option
  \docAuxKey{symbols} soit passée au \Package{glossaries}.
\end{dbwarning}

\begin{docCommand}{printsymbols}{\oarg{options}}
  Cette commande, fournie par le \Package{glossaries}, produit la liste des
  symboles saisies (par exemple) au moyen de la \refCom{newglssymbol}. Mais
  elle a été légèrement redéfinie : sa clé \refKey{style} a pour valeur par
  défaut \docValue{yadsymbolstyle} (et non \docValue{list}) :
  \begin{docKey}{style}{=\docValue{yadsymbolstyle}\textbar\meta{style}}{pas de valeur
      par défaut, initialement \docValue{yadsymbolstyle}}
    Cette clé permet de spécifier le style appliqué à la liste des
    symboles. Tout \meta{style} spécifié, autre que \docValue{yadsymbolstyle},
    doit être l'un de ceux acceptés par la clé \refKey{style} du
    \Package{glossaries}.
  \end{docKey}
\end{docCommand}

\begin{dbexample}{Définitions et liste des symboles}{}
  Le code suivant définit certains symboles.
  \preamblesample[configuration/symboles.tex]{}{}%
  Le code suivant produit la liste de ces symboles \phrase*{composée avec le
    style \docValue{yadsymbolstyle}}.
  \bodysample{rangesuffix=\^^M,linerange={printsymbols}}{} Le résultat de ce
  code est illustré \vref{fig:printsymbols}.
\end{dbexample}

% \afterpage{%
  \begin{landscape}
    \begin{figure}[p]
      \centering
      \begin{subfigure}[b]{.45\linewidth}
        \centering
        \screenshot[1]{printacronyms}
        \caption{Acronymes}
        \label{fig:printacronyms}
      \end{subfigure}%
      \hspace{\stretch{1}}%
      \begin{subfigure}[b]{.45\linewidth}
        \centering
        \screenshot[1]{printsymbols}
        \caption{Symboles}
        \label{fig:printsymbols}
      \end{subfigure}%
      \caption{Listes des acronymes et des symboles}
      \label{fig:printacronyms-printsymbols}
    \end{figure}
  \end{landscape}
% }

% ^^A Si on souhaite faire figurer :
% ^^A \begin{enumerate}
% ^^A \item une liste \emph{commune} des acronymes et des termes du glossaire, on
% ^^A   chargera \package{glossaries} \emph{sans} l'option ×acronym× et on recourra à
% ^^A   la commande \docAuxCommand{printglossary} ;
% ^^A \item deux listes \emph{distinctes}, on chargera \package{glossaries}
% ^^A   \emph{avec} l'option ×acronym×. et on recourra à la commande
% ^^A   \begin{enumerate}
% ^^A   \item \refCom{printacronyms} pour celle des acronymes (cf.
% ^^A     \vref{fig:acronymes}) ;
% ^^A   \item\label{item:1} \docAuxCommand{printglossary} pour celle des termes du
% ^^A     glossaire (cf. \vref{fig:printglossary}).
% ^^A   \end{enumerate}
% ^^A \end{enumerate}

Dans un mémoire de thèse, les emplacements des listes des termes du glossaire,
des acronymes\footnote{Les commandes \docAuxCommand{printglossary} et
  \docAuxCommand{printacronyms} du \Package{glossaries}, produisant les listes
  des termes du glossaire et des acronymes, sont illustrées
  \vref{fig:printglossary,fig:printacronyms}.} et des symboles sont \emph{a
  priori} arbitraires. Il est cependant parfois conseillé de placer :
  \begin{itemize}
  \item si elles sont \emph{communes}, \emph{la} liste résultante en partie finale ;
  \item si elles sont \emph{distinctes} :
    \begin{enumerate}
    \item les listes des acronymes et des symboles avant qu'ils soient utilisés
      pour la première fois donc, \emph{a priori}, avant le ou les résumés ;
    \item la liste des termes du glossaire en partie finale.
    \end{enumerate}
  \end{itemize}

\section{Sommaire et/ou table des matières}\label{sec:table-des-matieres}

\begin{docCommand}[doc description=\mandatory]{tableofcontents}{\oarg{options}}
  Cette commande produit une table des matières dont le \enquote{niveau de
    profondeur} par défaut est celui des sous-sections : les intitulés des
  commandes de structuration qui y figurent sont (seulement) ceux des parties
  (éventuelles), des chapitres, des sections et des sous-sections.

  Son argument optionnel permet de stipuler des \meta{options} sous la forme
  d'une liste \meta{clé}×=×\meta{valeur} dont les clés disponibles sont les
  deux suivantes.
  %
  {%
    \tcbset{before lower=\vspace*{\baselineskip}\par}
    %
    \begin{docKey}{depth}{=\docValue{part}\textbar\docValue{chapter}\textbar\docValue{section}\textbar\docValue{subsection}\textbar\docValue{subsubsection}\textbar\docValue{paragraph}\textbar\docValue{subparagraph}}{pas
        de valeur par défaut, initialement \docValue{subsection}}
      Cette clé permet de modifier le \enquote{niveau de profondeur} de la
      table des matières jusqu'aux, respectivement : parties, chapitres,
      sections, sous-sections, sous-sous-sections, paragraphes,
      sous-paragraphes.
  \end{docKey}
  }
%
  \begin{docKey}{name}{=\meta{nom alternatif}}{pas de valeur par défaut,
      initialement \docAuxCommand{contentsname}}
    Par défaut, le nom de la table des matières est
    \docAuxCommand{contentsname}, c'est-à-dire \enquote{Table des matières} ou
    \foreignquote{english}{Contents}\selonlangue{}. Cette clé permet de
    spécifier un \meta{nom alternatif}.
  \end{docKey}
\end{docCommand}

\begin{dbremark}{Tables des matières multiples}{}
  Si la table des matières est longue, elle peut être placée en fin de document
  mais elle est alors à remplacer, en \glspl{liminaire}, par un sommaire
  c'est-à-dire par une table des matières allégée.

  À cet effet, la \yatcl{} permet de faire figurer, dans un même document,
  plusieurs tables des matières au moyen d'occurrences multiples de la commande
  \refCom{tableofcontents}, chacune d'elles étant sujette aux options
  précédentes.
\end{dbremark}

\begin{dbexample}{Sommaire et table des matières}{sommaire-table-des-matieres}
  Pour faire figurer, dans un même document :
  \begin{enumerate}
  \item un sommaire :
    \begin{itemize}
    \item ne faisant apparaître que les chapitres (et éventuelles parties) ;
    \item nommé \enquote{Sommaire} ;
    \end{itemize}
  \item la table des matières ;
  \end{enumerate}
  on insérera respectivement :
  \bodysample{rangeendsuffix=\],linerange={tableofcontents-Sommaire}}{}
  \bodysample{rangesuffix=\^^M,linerange={tableofcontents}}{} La
  \vref{fig:tableofcontentsto-tableofcontents} illustre ce code.
\end{dbexample}

\afterpage{%
  \begin{landscape}
    \begin{figure}[p]
      \centering
      \begin{subfigure}[b]{.45\linewidth}
        \centering \screenshot[1]{tableofcontents-withargument}
        \caption{Sommaire allant jusqu'aux chapitres}
        \label{fig:tableofcontentsto}
      \end{subfigure}%
      \hspace{\stretch{1}}%
      \begin{subfigure}[b]{.45\linewidth}
        \centering \screenshot[1]{tableofcontents-withoutargument}
        \caption{Table des matières allant jusqu'aux sous-sections}
        \label{fig:tableofcontents}
      \end{subfigure}%
      \caption[Sommaire et table des matières]{Sommaire et table des matières
        de profondeurs différentes dans un même document}
      \label{fig:tableofcontentsto-tableofcontents}
    \end{figure}
  \end{landscape}
}

\section{Tables et listes et usuelles}

Les commandes usuelles ×\listoftables× et ×\listoffigures× produisent les
listes respectivement des tableaux et des figures.
%
On peut faire figurer d'autres listes, par exemple celle des listings
informatiques au moyen de la commande ×\lstlistoflistings× du
\Package*{listings}.
%
Nous n'illustrons pas ces commandes, classiques.

%
\iffalse
%%% Local Variables:
%%% mode: latex
%%% eval: (latex-mode)
%%% ispell-local-dictionary: "fr_FR"
%%% TeX-engine: xetex
%%% TeX-master: "../yathesis.dtx"
%%% End:
\fi
