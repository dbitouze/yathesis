\chapter{Corps}\label{cha:corps}

Le corps de la thèse, c'est-à-dire sa partie principale, comprend :
\begin{enumerate}
\item l'introduction (\enquote{générale}) ;
\item les chapitres \enquote{ordinaires} ;
\item la conclusion (\enquote{générale}).
\end{enumerate}
Les introduction et conclusion peuvent éventuellement être
\enquote{générales} par exemple si la thèse comporte plusieurs
parties, chacune introduite par une introduction et conclue par
une conclusion \enquote{ordinaires}.

\begin{dbremark}{Scission du mémoire en fichiers maître et esclaves}{}
  Il est vivement recommandé de scinder le mémoire de thèse,
  notamment son corps, en fichiers maître et esclaves (ces derniers
  correspondants chacun à un chapitre). La procédure
  pour ce faire, standard, est rappelée \vref{sec:repart-du-memo}.
\end{dbremark}

\begin{docCommand}[doc description=\mandatory]{mainmatter}{}
  La partie principale de la thèse doit être manuellement introduite au moyen
  de la commande usuelle \docAuxCommand{mainmatter} de la
  \Class{book}\nofrontmatter.
\end{docCommand}

\section{Chapitres non numérotés}
\label{sec:chap-non-numer}

Si certains chapitres du corps de la thèse \phrase{notamment d'introduction de
  conclusion \enquote{générales}} doivent être \emph{non} numérotés, on recourra de
façon usuelle à la version étoilée de la commande
\docAuxCommand{chapter}. Celle-ci a toutefois été quelque peu modifiée afin
d'en simplifier l'usage.

% ^^A  : habituellement, si un chapitre non numéroté est créé
% ^^A \emph{dans la partie principale} (entre \docAuxCommand{mainmatter} et
% ^^A \docAuxCommand{backmatter}) avec la commande standard
% ^^A \docAuxCommand{chapter*} :
% ^^A \begin{enumerate}
% ^^A \item des précautions (assez techniques) doivent être prises pour que :
% ^^A   \begin{enumerate}
% ^^A   \item le titre correspondant figure dans la table des matières ;
% ^^A   \item les \glspl{titrecourant} correspondants soient corrects ;
% ^^A   \end{enumerate}
% ^^A \item toutes les (sous-(sous-))sections du chapitre, nécessairement non
% ^^A   numérotées elles aussi, doivent également être créées avec les versions
% ^^A   étoilées des commandes correspondantes : \docAuxCommand{section*},
% ^^A   \docAuxCommand{subsection*} et \docAuxCommand{subsubsection*}.
% ^^A \end{enumerate}

\begin{dbremark}{Variante étoilée de la commande \protect\refCom*{chapter} modifiée}{}
  La \yatcl{} modifie la commande \docAuxCommand{chapter*} de sorte que :
  \begin{enumerate}
  \item automatiquement, le titre du chapitre figure :
    \begin{enumerate}
    \item dans la table des matières ;
    \item dans les \glspl{titrecourant} ;
    \end{enumerate}
  \item les (sous-(sous-))sections du chapitre sont peuvent et même
    \emph{doivent} être créées avec les versions \emph{non} étoilées des
    commandes correspondantes : \docAuxCommand{section},
    \docAuxCommand{subsection} et \docAuxCommand{subsubsection}.
  \end{enumerate}
\end{dbremark}

\begin{dbexample}{Introduction}{}
  Le code suivant produit la \vref{fig:introduction} illustrant une
  introduction (générale) non numérotée. On constate que, bien que seule la
  commande \docAuxCommand{chapter} figure sous sa forme étoilée, aucun élément
  de structuration de ce chapitre n'est numéroté.
  %
  \bodysample[corps/introduction-generale]{%
    deletekeywords={[1]introduction},%
    deletekeywords={[3]section},%
    deletekeywords={[3]subsection},%
    deletekeywords={[3]subsubsection},%
    deletekeywords={[3]paragraph},%
    deletekeywords={[3]subparagraph}%
  }{}
\end{dbexample}

\begin{figure}[p]
  \centering
  \screenshot{introduction}
  \caption{Introduction (non numérotée)}
  \label{fig:introduction}
\end{figure}

\section{Chapitres numérotés}
\label{sec:chapitres-numerotes}

Les chapitres numérotés du corps de la thèse sont introduits par la commande
usuelle \docAuxCommand{chapter} (cf. \vref{fig:chapitre}).

\begin{figure}[ht]
  \centering
  \screenshot{chapter}
  \caption[Chapitre \enquote{ordinaire}]{(Première) Page de chapitre
    \enquote{ordinaire}}
  \label{fig:chapitre}
\end{figure}

\begin{dbremark}{Style des têtes de chapitres numérotés personnalisable}{}
  Les têtes de chapitres numérotés sont par défaut composées avec le style
  ×PetersLenny× du \Package*{fncychap}. La \vref{sec:style-des-tetes} explique
  comment ceci peut être modifié.
\end{dbremark}

\section{Références bibliographiques}

Les références bibliographiques font partie intégrante du corps de la thèse.

Tout système de gestion de bibliographie peut théoriquement être mis en œuvre
avec la \yatcl. Cependant, celle-ci a été conçue plus spécifiquement en vue
d'un usage du \Package{biblatex} et éventuellement de \package{biber},
remplaçant fortement conseillé de \hologo{BibTeX}\footnote{Dans cette section,
  leur fonctionnement est supposé connu du lecteur (sinon, cf. par exemple
  \cite{en-ligne6}).}.

\begin{docCommand}[doc description=\mandatory]{printbibliography}{\oarg{options}}
  Cette commande, fournie par \package{biblatex}, produit la liste des
  références bibliographiques saisies selon la syntaxe de ce package (cf.
  \vref{fig:printbibliography}). Mais elle a été légèrement redéfinie de sorte
  que la bibliographie figure automatiquement dans les sommaire, table des
  matières et signets du document.
\end{docCommand}

\begin{figure}[htbp]
  \centering
  \screenshot{printbibliography}
  \caption[Bibliographie]{Bibliographie (ici composée avec le style
    bibliographique par défaut)}
  \label{fig:printbibliography}
\end{figure}

\begin{dbwarning}{Package \package{biblatex} non chargé par défaut}{}
  Le \Package{biblatex} \emph{n'étant pas} chargé par la \yatcl, on veillera
  à le charger manuellement si on souhaite l'utiliser.
\end{dbwarning}

%
\iffalse
%%% Local Variables:
%%% mode: latex
%%% eval: (latex-mode)
%%% ispell-local-dictionary: "fr_FR"
%%% TeX-engine: xetex
%%% TeX-master: "../yathesis.dtx"
%%% End:
\fi
