\chapter{Specimens et canevas de thèse}\label{cha:specimen-canevas}%
\changes{v0.99c}{2014/06/06}{Documentation révisée}%


Pour aider à sa mise en œuvre, la \yatcl fournit deux specimens (produisant
des \gls{pdf} identiques) et deux canevas de thèse (produisant des \gls{pdf}
identiques) :
\begin{description}
\item[un spécimen et un canevas \enquote{à plat}] dont les sources \file{.tex}
  respectifs sont tout entier situés dans un unique fichier\footnote{Mis à part
    le fichier de bibliographie et les fichiers images pour le spécimen.} ;
\item[un spécimen et un canevas \enquote{en relief}] dont les sources
  \file{.tex} respectifs sont scindés en fichiers maître et esclaves, qui plus
  est répartis dans différents sous-dossiers.
\end{description}
Les canevas ne sont rien d'autre que les (quasi-)répliques des spécimens
correspondants dont les données ont été vidées : pour les exploiter, il suffit
donc de remplir \enquote{les cases vides}.

\changes{v0.99c}{2014/06/06}{Spécimens et canevas fournis sous forme d'archives
  \file{.zip}}%
Ces spécimens et canevas se trouvent dans les sous-dossiers du dossier de
documentation de la \yatcl{}\footnote{Le dossier de documentation de la
  \yatcl{} est, pour la distribution :
  \begin{itemize}
  \item \TeX{} Live :
    \begin{itemize}
    \item sous Linux et Mac OS X :
      \unixtldirectory\tldistdirectory\jobdocdirectory{} ;
    \item sous Windows :
      \wintldirectory\tldistdirectory\jobdocdirectory{} ;
    \end{itemize}
  \item MiK\TeX{} : \miktexdistdirectory.
  \end{itemize}
  \label{fn:yathesisdocfolder}} indiqués au \vref{tab:specimens-canevas}. Ils
sont également fournis sous forme d'archives \file{.zip} qui devraient être
accessibles par simples clics sur les liens hypertextes de ce tableau.
\begin{table}
  \centering
  \begin{tabular}{lcc}
 & \enquote{À plat} & \enquote{En relief} \\\toprule
 Spécimen & \href{single-file-sample.zip}{\directory{single-file-sample}} & \href{master-slaves-files-sample.zip}{\directory{master-slaves-files-sample}} \\\midrule
 Canevas & \href{single-file-template.zip}{\directory{single-file-template}} &
 \href{master-slaves-files-template.zip}{\directory{master-slaves-files-template}} \\\bottomrule
\end{tabular}

  \caption{Dossiers et archives \file{.zip} des spécimens et canevas fournis
    avec la \yatcl{}}
  \label{tab:specimens-canevas}
\end{table}

\begin{dbwarning}{Ne pas travailler directement dans les dossiers de
    spécimens et de canevas !}{}
  Si on souhaite utiliser l'un de ces spécimens ou canevas, il est
  \emph{essentiel} de \emph{ne pas} travailler directement dans le dossier
  fourni : toutes les modifications seraient en effet écrasées lors d'une mise
  à jour de la classe. Il faut donc copier le dossier ou l'archive \file{.zip}
  correspondant dans un répertoire de travail habituel.
\end{dbwarning}

% ^^A La commande à utiliser pour lister le contenu du répertoire est :%
% ^^A tree -A -F -I \ %
% ^^A "*aux|*idx|*ind|*lof|*lot|*out|*toc|*acn|*acr|*alg|*bcf|*glg|*glo|*gls|*glg2|*gls2|*glo2|*ist|*run.xml|*xdy|*lol|*fls|*slg|*slo|*sls|*unq|*synctex.gz|*mw|*bbl|*blg|*fdb_latexmk|*log|*auto"

\section{Spécimens}
\label{sec:specimens}

\subsection{Spécimen \enquote{à plat}}
\label{sec:specimen-a-plat}

Le dossier \directory{single-file-sample} de ce spécimen contient les
fichiers :
\begin{enumerate}
\item \file{these.tex}, source \file{.tex} (unique) de la thèse  ;
\item \file{bibliographie.bib}, contenant les références bibliographiques de
  la thèse ;
\item \file{these.pdf} produit par compilation du \File{these.tex} ;
\item \file{labo.pdf}, \file{paris13.pdf}, \file{pres.pdf}, \file{tiger.pdf},
  \file{ulco.pdf}, contenant des images (logos, etc.) ;
\item \file{latexmkrc}, fichier de configuration du programme \program{latexmk}
  qui permet d'automatiser le processus de compilation complète de la thèse.
\end{enumerate}

[TODO]

\subsection{Spécimen \enquote{en relief}}
\label{sec:specimen-relief}

[TODO]

\section{Canevas}
\label{sec:canevas}

\subsection{Canevas \enquote{à plat}}
\label{sec:canevas-a-plat}

Le dossier \directory{single-file-template} de ce canevas ne contient que trois
fichiers :
\begin{enumerate}
\item \file{these.tex}, source \file{.tex} (unique) de la thèse  ;
\item \file{these.pdf} produit par compilation du \File{these.tex} ;
\item \file{latexmkrc}, fichier de configuration du programme \program{latexmk}
  qui permet d'automatiser le processus de compilation complète de la thèse.
\end{enumerate}

[TODO]

\subsection{Canevas \enquote{en relief}}
\label{sec:canevas-relief}

[TODO]

%
\iffalse
%%% Local Variables:
%%% mode: latex
%%% eval: (latex-mode)
%%% ispell-local-dictionary: "fr_FR"
%%% TeX-engine: xetex
%%% TeX-master: "../yathesis.dtx"
%%% End:
\fi
