\chapter{\texorpdfstring{\Glsentryplural{titrecourant}}{Titres courants}, \glsentrytext{pagination} et numérotation}\label{cha:pagination}

Ce chapitre précise les \glspl{titrecourant}, la \gls{pagination} et la
numérotation des chapitres des documents composés avec la \yatcl{}.

\begin{enumerate}
\item La composition est en recto verso\exceptoneside.
\item À l'exception de la 4\ieme{} de couverture qui commence sur une page
  paire (et laisse son recto entièrement vide), les chapitres et objets
  analogues vus \vref{cha:pages-de-titre,cha:liminaires,cha:corps,cha:annexes}
  commencent systématiquement sur une page impaire\exceptoneside.
\item Les \glspl{titrecourant} sont activés sur toutes les pages sauf sur
  celles :
  \begin{itemize}
  \item de 1\iere{} de couverture et de titres (et leurs versos);
  \item dédiées :
    \begin{itemize}
    \item à la clause de non-responsabilité ;
    \item aux mots clés ;
    \item au(x) laboratoire(s) ;
    \item aux dédicaces ;
    \item aux épigraphes (et leurs versos) ;
    \end{itemize}
    % ^^A\item des chapitres ordinaires figurant avant la page dédiée aux
    % ^^Arésumés (cf. \vref{rq:titrecourant}) ;
  \item qui ouvrent les parties (et leurs versos);
  \item qui ouvrent les chapitres;
  \item de 4\ieme{} de couverture (et son recto).
  \end{itemize}
\item La \gls{pagination} commence dès la 1\iere{} page, de façon
  séquentielle, en chiffres :
  \begin{itemize}
  \item romains minuscules du début du mémoire jusqu'à la fin des
    \glspl{liminaire};
  \item arabes, avec remise à zéro, du début du corps jusqu'à la fin du
    mémoire.
  \end{itemize}
\item Les numéros de pages :
  \begin{itemize}
  \item sont imprimés sur (et seulement sur) les pages où les
    \glspl{titrecourant} sont activés et y figurent alors en haut, du côté des
    marges extérieures ;
  \item apparaissent tous dans le compteur de pages des afficheurs
    \pdf.
  \end{itemize}
\item Les chapitres numérotés sont les chapitres \enquote{ordinaires} :
  \begin{itemize}
  \item de la partie corps\footnote{Sauf ceux créés avec la forme étoilée de la
      commande \docAuxCommand{chapter} (cf. \vref{sec:chap-non-numer}).}, alors
    en chiffres arabes et précédés de la mention \enquote{Chapitre} ;
  \item de la partie annexe, alors en caractères latins majuscules (avec remise
    à zéro) et précédés de la mention \enquote{Annexe} (à la place de \enquote{Chapitre}).
  \end{itemize}
\end{enumerate}

%
\iffalse
%%% Local Variables:
%%% mode: latex
%%% eval: (latex-mode)
%%% ispell-local-dictionary: "fr_FR"
%%% TeX-engine: xetex
%%% TeX-master: "../yathesis.dtx"
%%% End:
\fi
