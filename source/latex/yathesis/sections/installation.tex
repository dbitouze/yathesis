\chapter{Installation}
\label{cha:installation}

La procédure d'installation de la \yatcl{} dépend de :
\begin{itemize}
\item sa version souhaitée : stable ou de développement ;
\item sa disponibilité par la distribution \TeX{} utilisée. On ne décrit ici la
  procédure qu'avec la \enquote{\TeX~Live}\footnote{La procédure avec la
    \enquote{MiK\TeX{}} sera décrite dès que possible.}.
\end{itemize}

\section{Version stable}
\label{sec:version-stable}

Si on souhaite utiliser la version stable de la \yatcl{}, deux cas se
présentent :
\begin{enumerate}
\item si la \yatcl{} est fournie par la distribution \TeX{}, il n'y a rien
  à faire ;
\item sinon, il faut télécharger l'archive \file{yathesis.tds.zip} à la page
  \url{http://www.ctan.org/pkg/yathesis} ou, à défaut, l'archive \file{.zip} ou
  \file{.tar.gz} le plus récent à la page
  \url{https://github.com/dbitouze/yathesis/releases}.

  Il faut ensuite décompresser l'archive téléchargée dans le dossier adéquat,
  appelé \enquote{arborescence \TeX{}} personnelle, et noté
  \directory{TEXMFHOME}.

  \begin{dbremark}{Arborescences \TeX{} personnelles par défaut}{}
    Par défaut, la distribution \enquote{\TeX~Live} considère comme dossier
    \directory{TEXMFHOME} :
    \begin{description}
    \item[sous Linux :] \directory{/home/\meta{nom}/texmf}, soit
      \path|~/texmf|
    \item[sous Mac OS X :] \directory{/Users/\meta{nom}/Library/texmf}
    \item[sous Windows :] \directory{C:/Users/\meta{nom}/texmf}
    \end{description}
    le dossier \directory{texmf} si ce n'est déjà fait.
  \end{dbremark}
\end{enumerate}

\section{Version de développement}
\label{sec:vers-de-devel}

Si on souhaite utiliser la version de développement\footnote{À ses risques et
  périls !} de la \yatcl{}, on clonera\footnote{Procédure non détaillée ici.}
son dépôt Git à la page \url{https://github.com/dbitouze/yathesis}.


%
\iffalse
%%% Local Variables:
%%% mode: latex
%%% eval: (latex-mode)
%%% ispell-local-dictionary: "fr_FR"
%%% TeX-engine: xetex
%%% TeX-master: "../yathesis.dtx"
%%% End:
\fi
