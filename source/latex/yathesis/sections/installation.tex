\chapter{Installation}
\label{cha:installation}

La procédure d'installation de la \yatcl{} dépend de :
\begin{itemize}
\item la version souhaitée (stable ou de développement) ;
\item la disponibilité \emph{via} la distribution \TeX{} utilisée.
\end{itemize}
On ne décrit ici la procédure que dans le cadre de la distribution
\TeX~Live\footnote{Le cas de la distribution MiK\TeX{} sera décrit dès que
  possible.}.

\section{Version stable}
\label{sec:version-stable}

Si on souhaite utiliser la classe sous sa version stable, deux cas se
présentent selon que la \TeX~Live la fournit d'emblée\footnote{Ce devrait être
  le cas à partir de la version \texttt{2014}.} ou pas :
\begin{description}
\item[classe fournie :] elle est alors installée en même temps que la
  distribution ;
\item[classe \emph{non} fournie :] il faut
  \begin{enumerate}
  \item télécharger l'archive \file{yathesis.tds.zip} à la page
    \url{http://www.ctan.org/pkg/yathesis} ou, à défaut, l'archive \file{.zip}
    ou \file{.tar.gz} la plus récente à la page
    \url{https://github.com/dbitouze/yathesis/releases} ;
  \item décompresser l'archive téléchargée dans le dossier adéquat, appelé
    \enquote{arborescence \TeX{}} personnelle, et noté \directory{TEXMFHOME}.
  \begin{dbremark}{Arborescences \TeX{} personnelles par défaut}{}
    Par défaut, la distribution \enquote{\TeX~Live} considère comme dossier
    \directory{TEXMFHOME} :
    \begin{description}
    \item[sous Linux :] \directory{/home/\meta{nom}/texmf}, soit \path|~/texmf|
    \item[sous Mac OS X :] \directory{/Users/\meta{nom}/Library/texmf}
    \item[sous Windows :] \directory{C:/Users/\meta{nom}/texmf}
    \end{description}
    le dossier \directory{texmf} si ce n'est déjà fait.
  \end{dbremark}
\end{enumerate}
\end{description}

\section{Version de développement}
\label{sec:vers-de-devel}

Si on souhaite utiliser la version de développement\footnote{À ses risques et
  périls !} de la \yatcl{}, on clonera\footnote{Procédure non détaillée ici.}
son dépôt Git à la page \url{https://github.com/dbitouze/yathesis}.


%
\iffalse
%%% Local Variables:
%%% mode: latex
%%% eval: (latex-mode)
%%% ispell-local-dictionary: "fr_FR"
%%% TeX-engine: xetex
%%% TeX-master: "../yathesis.dtx"
%%% End:
\fi
