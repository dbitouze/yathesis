\chapter{Installation}
\label{cha:installation}

\lstset{%
  basicstyle=\ttfamily\NoAutoSpacing,
  columns=flexible,
  frame=single
}

La procédure d'installation de la \yatcl{} dépend de la version souhaitée :
stable, de test ou de développement.

\section{Version stable}
\label{sec:version-stable}

La version stable de la classe devrait être fournie d'emblée par les
distributions \TeX{} \enquote{\TeX~Live\footnote{Ce devrait être le cas
    à partir de la version \texttt{2014} de la \TeX~Live dont la sortie est
    annoncée pour le début du mois de juillet 2014.}} et
\enquote{MiK\TeX{}}\footnote{Pour s'assurer que cette version stable est la
  plus récente, il est conseillé de mettre sa distribution \TeX{} à jour.}.

Si tel n'est pas le cas, on pourra installer sa version de test comme suit.

\section{Version de test}
\label{sec:version-de-test}

La procédure pour installer une version de test de la classe n'est décrite ici
que pour la distribution \TeX~Live\footnote{Bien qu'une procédure analogue
  existe certainement pour la distribution MiK\TeX{}, l'auteur ne la connaît
  pas : contributions bienvenues !}. Elle utilise des lignes de commandes,
méthode simple et automatisée:
% ^^A
% ^^A \begin{description}
% ^^A \item[Par lignes de commandes.]
  \begin{enumerate}
  \item Ouvrir un terminal\footnote{Sous Linux, c'est en général simple
      à trouver. Sous Mac OS X, il devrait suffire de visiter le menu
      \enquote{Applications} puis \enquote{Utilitaires} puis
      \enquote{Terminal}. Sous Windows, il devrait suffire de visiter le menu
      \enquote{Tous les programmes} puis \enquote{Accessoires} puis
      \enquote{Invite de commandes}.}.
  \item Dans ce terminal, lancer successivement les trois commandes
    suivantes\footnote{On évitera de les recopier manuellement : il est
      possible de les copier (\texttt{CTRL~+~C} ou assimilé) et de les coller
      (\texttt{CTRL~+~V} ou assimilé, ou clic droit).} (à faire éventuellement
    précéder de \enquote{\protect\lstinline|sudo|} sous Linux et Mac OS X) :
\begin{lstlisting}[numbers=left]
tlmgr repository add http://tlcontrib.metatex.org/2013 tlcontrib tlmgr
pinning add tlcontrib yathesis
tlmgr install yathesis
\end{lstlisting}
    Ne pas s'inquiéter pas de messages du type :
\begin{lstlisting}
TeX Live 2013 is frozen forever and will no longer be updated. This
happens in preparation for a new release. [...]
\end{lstlisting}
    et être patient lorsqu'apparaissent les lignes :
\begin{lstlisting}
running mktexlsr ...
done running mktexlsr.
running mtxrun --generate
...
\end{lstlisting}
\item Pour vérifier que la classe a été correctement installée, lancer la
  commande :
\begin{lstlisting}
kpsewhich yathesis.cls
\end{lstlisting}
    qui devrait renvoyer un message \emph{non} vide tel que, sous Linux et sous
    Mac OS X :
\begin{lstlisting}
/usr/local/texlive/2013/texmf-dist/tex/latex/yathesis/yathesis.cls
\end{lstlisting}
    et, sous Windows :
\begin{lstlisting}
c:/texlive/2013/texmf-dist/tex/latex/yathesis/yathesis.cls
\end{lstlisting}
  \end{enumerate}
% ^^A \item[Par extraction d'archives.] Cette méthode, moins simple à mettre en
% ^^A   œuvre, ne sera utilisée qu'en cas d'échec de la méthode précédente.
% ^^A   \begin{enumerate}
% ^^A   \item Créer, si ce n'est déjà fait, son \enquote{arborescence \TeX{}}
% ^^A     personnelle, notée \directory{TEXMFHOME}.
% ^^A     \begin{dbremark}{Arborescences \TeX{} personnelles par défaut}{}
% ^^A       Par défaut, la distribution \enquote{\TeX~Live} considère comme dossier
% ^^A       \directory{TEXMFHOME} :
% ^^A       \begin{description}
% ^^A       \item[sous Linux :] \directory{/home/\meta{nom}/texmf}, soit
% ^^A         \path|~/texmf|
% ^^A       \item[sous Mac OS X :] \directory{/Users/\meta{nom}/Library/texmf}
% ^^A       \item[sous Windows :] \directory{C:/Users/\meta{nom}/texmf}
% ^^A       \end{description}
% ^^A     \end{dbremark}
% ^^A   \item Télécharger l'archive \file{yathesis.tds.zip} à la page
% ^^A     \url{http://www.ctan.org/pkg/yathesis} ou, à défaut, l'archive \file{.zip}
% ^^A     ou \file{.tar.gz} la plus récente à la page
% ^^A     \url{https://github.com/dbitouze/yathesis/releases}.
% ^^A   \item \emph{Extraire} l'archive téléchargée.
% ^^A   \item Ouvrir le dossier \directory{yathesis-v}\meta{numéro de version}
% ^^A     résultant de l'extraction, ou dans un de ses sous-dossiers, qui contient
% ^^A     les dossiers :
% ^^A     \begin{itemize}
% ^^A     \item \directory{doc} ;
% ^^A     \item \directory{source} ;
% ^^A     \item \directory{tex}.
% ^^A     \end{itemize}
% ^^A   \item Déplacer les dossiers \directory{doc}, \directory{source} et
% ^^A     \directory{tex} dans le \directory{TEXMFHOME}.
% ^^A   \end{enumerate}
% ^^A \end{description}

\section{Version de développement}
\label{sec:vers-de-devel}

Si on souhaite utiliser la version de développement de la \yatcl{} (à ses
risques et périls !), on clonera son dépôt Git à la page
\url{https://github.com/dbitouze/yathesis}. La procédure pour ce faire, hors
sujet ici, n'est pas détaillée.

%
\iffalse
%%% Local Variables:
%%% mode: latex
%%% eval: (latex-mode)
%%% ispell-local-dictionary: "fr_FR"
%%% TeX-engine: xetex
%%% TeX-master: "../yathesis.dtx"
%%% End:
\fi
