\chapter{Installation}
\label{cha:installation}

La procédure d'installation de la \yatcl{} dépend de la version souhaitée
(stable ou de développement) et de la disponibilité \emph{via} la distribution
\TeX{} utilisée\footnote{Seul le cas de la distribution \TeX~Live est décrit
  ici. Celui de la distribution MiK\TeX{} le sera dès que possible.}.

\section{Version stable}
\label{sec:version-stable}

Si on souhaite utiliser la classe sous sa version stable, deux cas se
présentent selon qu'elle est ou pas fournie d'emblée par la
\TeX~Live\footnote{Ce devrait être le cas à partir de la version
  \texttt{2014}.}.
\begin{description}
\item[Classe \yat{} fournie :] elle est alors installée en même temps que la
  distribution ;
\item[Classe \yat{} \emph{non} fournie :] il faut
  \begin{enumerate}
  \item créer, si ce n'est déjà fait, son \enquote{arborescence \TeX{}}
    personnelle, notée \directory{TEXMFHOME}.
    \begin{dbremark}{Arborescences \TeX{} personnelles par défaut}{}
      Par défaut, la distribution \enquote{\TeX~Live} considère comme dossier
      \directory{TEXMFHOME} :
      \begin{description}
      \item[sous Linux :] \directory{/home/\meta{nom}/texmf}, soit
        \path|~/texmf|
      \item[sous Mac OS X :] \directory{/Users/\meta{nom}/Library/texmf}
      \item[sous Windows :] \directory{C:/Users/\meta{nom}/texmf}
      \end{description}
    \end{dbremark}
  \item télécharger l'archive \file{yathesis.tds.zip} à la page
    \url{http://www.ctan.org/pkg/yathesis} ou, à défaut, l'archive \file{.zip}
    ou \file{.tar.gz} la plus récente à la page
    \url{https://github.com/dbitouze/yathesis/releases} ;
  \item \emph{extraire} l'archive téléchargée ;
  \item ouvrir le dossier \directory{yathesis-v}\meta{numéro de version}
    résultant de l'extraction, ou dans un de ses sous-dossiers, qui contient
    les dossiers :
    \begin{itemize}
    \item \directory{doc} ;
    \item \directory{source} ;
    \item \directory{tex} ;
    \end{itemize}
  \item déplacer les dossiers \directory{doc}, \directory{source} et
    \directory{tex} dans le \directory{TEXMFHOME}.
  \end{enumerate}
\end{description}

\section{Version de développement}
\label{sec:vers-de-devel}

Si on souhaite utiliser la version de développement\footnote{À ses risques et
  périls !} de la \yatcl{}, on clonera\footnote{Procédure non détaillée ici.}
son dépôt Git à la page \url{https://github.com/dbitouze/yathesis}.


%
\iffalse
%%% Local Variables:
%%% mode: latex
%%% eval: (latex-mode)
%%% ispell-local-dictionary: "fr_FR"
%%% TeX-engine: xetex
%%% TeX-master: "../yathesis.dtx"
%%% End:
\fi
