\chapter{Installation}
\label{cha:installation}
\changes{v0.99}{2014/05/18}{Procédure d'installation précisée}%^^A
%^^A
\lstset{%
  basicstyle=\ttfamily\NoAutoSpacing,
  columns=flexible,
  frame=single
}%^^A
%^^A
La procédure d'installation de la \yatcl{} dépend de la version souhaitée :
stable
%^^A , de test
ou de développement.

\section{Version stable}
\label{sec:version-stable}

La version stable de la classe est normalement fournie par les distributions de
\TeX{}, notamment \program{\TeX~Live}\footnote{Par mise à jour de sa version
  \texttt{2014}, et d'emblée pour les versions suivantes.} et
\program{MiK\TeX{}\footnote{Par mise à jour de sa version \texttt{2.9}, et
    d'emblée pour les versions suivantes.}}. Pour s'assurer que cette version
stable est la plus récente, il est de toute façon conseillé de mettre à jour sa
distribution \TeX{}.

\section{Version de développement}
\label{sec:vers-de-devel}

Si on souhaite utiliser (à ses risques et périls !) la version de développement
de la \yatcl{}, on clonera son dépôt \program{Git} à la page
\url{https://github.com/dbitouze/yathesis}. La procédure pour ce faire, hors
sujet ici, n'est pas détaillée.

%
\iffalse
%%% Local Variables:
%%% mode: latex
%%% eval: (latex-mode)
%%% ispell-local-dictionary: "fr_FR"
%%% TeX-engine: xetex
%%% TeX-master: "../yathesis.dtx"
%%% End:
\fi
