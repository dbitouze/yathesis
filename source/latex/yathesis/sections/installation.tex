\chapter{Installation}
\label{cha:installation}

La procédure d'installation de la \yatcl{} dépend de la version souhaitée
(stable ou de développement) et de la disponibilité \emph{via} la distribution
\TeX{} utilisée\footnote{Seul le cas de la distribution \TeX~Live est décrit
  ici. Celui de la distribution MiK\TeX{} le sera dès que possible.}.

\section{Version stable}
\label{sec:version-stable}

Si on souhaite utiliser la classe sous sa version stable, deux cas se
présentent selon qu'elle est ou pas fournie d'emblée par la
\TeX~Live\footnote{Ce devrait être le cas à partir de la version
  \texttt{2014}.}.
\begin{description}
\item[Classe \yat{} fournie :] elle est alors installée en même temps que la
  distribution ;
\item[Classe \yat{} \emph{non} fournie :]\
  \begin{description}
  \item[Par lignes de commandes.] Cette méthode est la plus simple et la plus
    la plus automatisée.
    \begin{enumerate}
    \item Ouvrir un terminal\footnote{Sous Linux, c'est en général simple
        à trouver. Sous Mac OS X, il devrait suffire de visiter le menu
        \enquote{Applications} puis \enquote{Utilitaires} puis
        \enquote{Terminal}. Sous Windows, il devrait suffire de visiter le menu
        \enquote{Tous les programmes} puis \enquote{Accessoires} puis
        \enquote{Invite de commandes}.}.
    \item Y lancer successivement les commandes suivantes\footnote{Sous Linux
        et Mac OS X, en les faisant éventuellement précéder de
        \enquote{\protect\lstinline|sudo|}.} :
\begin{lstlisting}
tlmgr repository add http://tlcontrib.metatex.org/2013  tlcontrib
tlmgr pinning add tlcontrib yathesis
tlmgr install yathesis
\end{lstlisting}
      Ne pas s'inquiéter pas de messages du type :
\begin{lstlisting}
TeX Live 2013 is frozen forever and will no longer be updated.  This
happens in preparation for a new release. [...]
\end{lstlisting}
      et être patient lorsqu'apparaissent les lignes :
\begin{lstlisting}
running mktexlsr ...
done running mktexlsr.
running mtxrun --generate ...
\end{lstlisting}
    \item Pour vérifier que tout s'est bien passé, lancer la commande :
\begin{lstlisting}
kpsewhich yathesis.cls
\end{lstlisting}
      qui devrait renvoyer un message \emph{non} vide tel que, sous Linux et
      sous Mac OS X :
\begin{lstlisting}
/usr/local/texlive/2013/texmf-dist/tex/latex/yathesis/yathesis.cls
\end{lstlisting}
      et tel que, sous Windows :
\begin{lstlisting}
c:/texlive/2013/texmf-dist/tex/latex/yathesis/yathesis.cls
\end{lstlisting}
    \end{enumerate}
  \end{description}
\item[Par extraction d'archives.] Cette méthode, moins simple à mettre en
  œuvre, ne sera utilisée qu'en cas d'échec de la méthode précédente.
  \begin{enumerate}
  \item Créer, si ce n'est déjà fait, son \enquote{arborescence \TeX{}}
    personnelle, notée \directory{TEXMFHOME}.
    \begin{dbremark}{Arborescences \TeX{} personnelles par défaut}{}
      Par défaut, la distribution \enquote{\TeX~Live} considère comme dossier
      \directory{TEXMFHOME} :
      \begin{description}
      \item[sous Linux :] \directory{/home/\meta{nom}/texmf}, soit
        \path|~/texmf|
      \item[sous Mac OS X :] \directory{/Users/\meta{nom}/Library/texmf}
      \item[sous Windows :] \directory{C:/Users/\meta{nom}/texmf}
      \end{description}
    \end{dbremark}
  \item Télécharger l'archive \file{yathesis.tds.zip} à la page
    \url{http://www.ctan.org/pkg/yathesis} ou, à défaut, l'archive \file{.zip}
    ou \file{.tar.gz} la plus récente à la page
    \url{https://github.com/dbitouze/yathesis/releases}.
  \item \emph{Extraire} l'archive téléchargée.
  \item Ouvrir le dossier \directory{yathesis-v}\meta{numéro de version}
    résultant de l'extraction, ou dans un de ses sous-dossiers, qui contient
    les dossiers :
    \begin{itemize}
    \item \directory{doc} ;
    \item \directory{source} ;
    \item \directory{tex}.
    \end{itemize}
  \item Déplacer les dossiers \directory{doc}, \directory{source} et
    \directory{tex} dans le \directory{TEXMFHOME}.
  \end{enumerate}
\end{description}

\section{Version de développement}
\label{sec:vers-de-devel}

Si on souhaite utiliser la version de développement\footnote{À ses risques et
  périls !} de la \yatcl{}, on clonera\footnote{Procédure non détaillée ici.}
son dépôt Git à la page \url{https://github.com/dbitouze/yathesis}.


%
\iffalse
%%% Local Variables:
%%% mode: latex
%%% eval: (latex-mode)
%%% ispell-local-dictionary: "fr_FR"
%%% TeX-engine: xetex
%%% TeX-master: "../yathesis.dtx"
%%% End:
\fi
