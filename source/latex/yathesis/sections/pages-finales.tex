\chapter{Pages finales}
\label{cha:pages-finales}

Ce chapitre indique comment produire les pages finales de la thèse,
à savoir :
\begin{enumerate}
\item la liste éventuelle des acronymes et/ou
  termes du glossaire ;
\item l'éventuel index;
\item la table des matières, en cas de sommaire en \glspl{liminaire};
\item la quatrième de couverture (le dos de la thèse).
\end{enumerate}

\begin{docCommand}{backmatter}{}
  Les éventuelles pages finales de la thèse doivent être manuellement
  introduites au moyen de la commande usuelle
  \docAuxCommand{backmatter}\footnote{Cette commande n'est pas obligatoire en
    soi mais elle est fortement recommandée si la thèse contient des pages
    finales.} de la \Class{book}\nofrontmatter.
\end{docCommand}

\section{Glossaire}

Les commandes de production du glossaire \docAuxCommand{printglossary} (ou des
glossaires \docAuxCommand{printglossaries}) est détaillée et illustrée
\vref{sec:sigl-gloss-nomencl,fig:printglossary}.

\begin{figure}[htbp]
  \centering
  \screenshot{printglossary}
  \caption{Glossaire}
  \label{fig:printglossary}
\end{figure}

\section{Index}

\begin{dbremark*}{Section à passer en 1\iere{} lecture}
  Cette section est à passer en 1\iere{} lecture si on ne compte pas faire
  figurer d'index.
\end{dbremark*}

Bien que tout package de gestion d'index puisse théoriquement fonctionner avec
la \yatcl, celle-ci a été conçue plus spécifiquement en vue d'un usage du
\Package*{index}\footnote{Dans cette section, son fonctionnement est supposé
  connu du lecteur (sinon, cf. par exemple \cite{en-ligne7}).}.

La \yatcl{} ne définit rien de spécifique concernant l'index. Elle se contente
de charger le \Package{index}
\phrase{qu'il est donc inutile de charger manuellement} et de légèrement
modifier sa commande \docAuxCommand{printindex} (illustrée
\vref{fig:printindex}) :
\begin{itemize}
\item en lui appliquant un style de pages propre à l'index ;
\item pour que l'index figure automatiquement dans les
  sommaire, table des matières et signets du document.
\end{itemize}

\begin{figure}[htbp]
  \centering
  \screenshot{printindex}
  \caption{Index}
  \label{fig:printindex}
\end{figure}

\section{Table des matières}

Si la table des matières est longue, elle peut être placée en
annexe. Nous renvoyons ici à la \vref{sec:table-des-matieres} et à
la \vref{fig:tableofcontents} qui traite déjà cette question.

\section{Quatrième de couverture}\label{sec:quatr-de-couv}

La quatrième de couverture s'obtient au moyen de la commande
\refCom{makebackcover} suivante.

\begin{docCommand}{makebackcover}{}
  Cette commande a le même effet que la commande \refCom{makeabstract}
  % ^^A (elle affiche entre autres les résumés succincts en français et en
  % ^^A anglais),
  à ceci près que :
  \begin{enumerate}
  \item elle ne produit pas de titre courants (non souhaités au dos d'un
    document) ;
  \item la page est imprimée sur une page paire, son recto étant
    laissé entièrement vide.
  \end{enumerate}
\end{docCommand}

\begin{figure}[htbp]
  \centering
  \screenshot{makebackcover}
  \caption{Quatrième de couverture}
  \label{fig:makebackcover}
\end{figure}

%
\iffalse
%%% Local Variables:
%%% mode: latex
%%% eval: (latex-mode)
%%% ispell-local-dictionary: "fr_FR"
%%% TeX-engine: xetex
%%% TeX-master: "../yathesis.dtx"
%%% End:
\fi
